\documentclass[12pt, a4paper]{article}

\usepackage{lastpage}
\usepackage{mathtools}
\usepackage{xltxtra}
\usepackage{libertine}
\usepackage{amsmath}
\usepackage{amsthm}
\usepackage{amsfonts}
\usepackage{amssymb}
\usepackage{enumitem}
\usepackage{xcolor}
\usepackage[left=2.3cm, right=2.3cm, top=2.7cm, bottom=2.7cm, bindingoffset=0cm, headheight=15pt]{geometry}
\usepackage{fancyhdr}
\usepackage[russian]{babel}
% \usepackage{parindent}

\pagestyle{fancy}
\lfoot{M3137y2019}
\rhead{\thepage\ из \pageref{LastPage}}

\newcommand{\R}{\mathbb{R}}
\newcommand{\Q}{\mathbb{Q}}
\newcommand{\C}{\mathbb{C}}
\newcommand{\Z}{\mathbb{Z}}
\newcommand{\B}{\mathbb{B}}
\newcommand{\N}{\mathbb{N}}

\DeclareMathOperator*{\xor}{\oplus}
\DeclareMathOperator*{\equ}{\sim}
\DeclareMathOperator{\Ln}{\text{Ln}}
\DeclareMathOperator{\sign}{\text{sign}}
\DeclareMathOperator{\Sym}{\text{Sym}}
\DeclareMathOperator{\Asym}{\text{Asym}}
% \DeclareMathOperator{\sh}{\text{sh}}
% \DeclareMathOperator{\tg}{\text{tg}}
% \DeclareMathOperator{\arctg}{\text{arctg}}
% \DeclareMathOperator{\ch}{\text{ch}}

\DeclarePairedDelimiter{\ceil}{\lceil}{\rceil}

\setmainfont{Linux Libertine}

\theoremstyle{plain}
\newtheorem{theorem}{Теорема}
\newtheorem{axiom}{Аксиома}
\newtheorem{lemma}{Лемма}

\theoremstyle{remark}
\newtheorem*{remark}{Примечание}
\newtheorem*{exercise}{Упражнение}
\newtheorem*{consequence}{Следствие}
\newtheorem*{example}{Пример}
\newtheorem*{observation}{Наблюдение}

\theoremstyle{definition}
\newtheorem*{definition}{Определение}
\newtheorem*{obozn}{Обозначение}

\lhead{Конспект по матанализу}
\cfoot{}
\rfoot{December 23, 2019}

\usepackage{xcolor}

\begin{document}
\textcolor{red}{Первая половина лекции не была записана}

%<*разложениятейлора>
Некоторые разложения по Тейлору:

$$\begin{aligned}
e^x&=1+x+\frac{x^2}{2!}+\ldots+\frac{x^n}{n!} + o(x^n) \\
\sin x &= x-\frac{x^3}{3!}+\frac{x^5}{5!}+\ldots+(-1)^n\frac{x^{2n+1}}{(2n+1)!}+o(x^{2n+1}) \\
\cos x &= 1+\frac{x^2}{2!}+\frac{x^4}{4!}+\ldots+(-1)^n\frac{x^{2n}}{(2n)!}+o(x^{2n}) \\
\ln (1+x) &= x-\frac{x^2}{2}+\frac{x^3}{3}-\frac{x^4}{4}+\ldots+(-1)^{n+1}\frac{x^n}{n}+o(x^n) \\
(1+x)^\alpha &= 1+\alpha x + \frac{\alpha(\alpha-1)}{2!}x^2+\frac{\alpha(\alpha-1)(\alpha-2)}{3!}x^3+\ldots+\binom{\alpha}{n}x^n+o(x^n)
\end{aligned}$$
%</разложениятейлора>

\begin{lemma}
    %<*иррациональностьe2>
    $e^2$ --- ирр.
    %</иррациональностьe2>
\end{lemma}
%<*иррациональностьe2proof>
\begin{proof}
    $e^2=\frac{2k}{n}$

    $ne=2ke^{-1}$

    $n(2k-1)!e=(2k)!e^{-1}$
\end{proof}
%</иррациональностьe2proof>
\begin{lemma}
    Метод Ньютона

    %<*методньютона>
    $f:\langle a,b\rangle \to\R$ --- дважды дифф.

    $m:=\inf\limits_{\langle a,b\rangle}|f'|>0$

    $M:=\sup|f''|$
    
    $\xi \in(a,b):f(\xi)=0$

    $x_1\in(a,b) : |x_1-\xi|\frac{M}{2m}<1$

    Рассмотрим последовательность $x_{n+1}=x_n-\frac{f(x_n)}{f'(x_n)}$

    Тогда $\exists \lim x_n=\xi$ и при этом !. Кроме того, оно очень быстро сходится.

    $$|x_n-\xi|\leq \left(\frac{M}{2m}|x_1-\xi|\right)^{2n}$$
    %</методньютона>
\end{lemma}

$$x_{n+1}-\xi=x_n-\xi-\frac{f(x_n)}{f'(x_n)}=\frac{(x_n-\xi)f'(x_n)-f(x_n)}{f'(x_n)}$$

$$|x_{n+1}-\xi|=\frac{|f(x_n)+f'(x_n)(\xi-x_n)|}{|f'(x_n)|}=\frac{\frac{1}{2}|f''(c)||\xi-x_n|^2}{|f'(x_n)|}\leq \frac{2M}{m}|\xi-x_n|^2$$

\begin{theorem}
    О разложении рациональной дроби на простейшие.

    %<*оразложениирациональнойдроби>
    $P(x), Q(x)$ --- многочлен $\deg P<\deg Q=n$

    $Q(x)=(x-a_1)^{k_1}\ldots (x-a_m)^{k_m} \quad (k_1+\ldots+k_m=n; a_i\not= a_j)$

    Тогда $\exists$

    $$\frac{P(x)}{Q(x)}=\left( \frac{A_1}{(x-a_1)}+\frac{A_2}{(x-a_1)^2}+\ldots+\frac{A_{k_1}}{(x-a_1)^{k_1}} \right)+\left( \frac{B_1}{(x-a_2)}+\frac{B_2}{(x-a_2)^2}+\ldots+\frac{B_{k_2}}{(x-a_2)^{k_2}} \right)+$$
    $$+\ldots+\left( \frac{C_1}{(x-a_m)}+\frac{C_2}{(x-a_m)^2}+\ldots+\frac{C_{k_m}}{(x-a_m)^{k_m}} \right)$$
    %</оразложениирациональнойдроби>
\end{theorem}
%<*оразложениирациональнойдробиproof>
\begin{proof}
    $$\frac{P(x)}{(x-a_1)^{k_1}\ldots(x-a_m)^{k_m}}=\frac{1}{(x-a_1)^{k_1}}\frac{P(x)}{(x-a_2)^{k_2}\ldots(x-a_m)^{k_m}}=$$
    $$=\frac{1}{(x-a_1)^{k_1}}(A_{k_1}+A_{k_1+1}(x-a_1)+A_{k_1-2}(x-a_1)^2+\ldots+A_1(x-a_1)^{k_1}+o((x-a_1)^{k_1}))$$
    $$\frac{P}{Q}-\left(\frac{A_1}{x-a_1}+\ldots+\frac{A_{k_1}}{(x-a_1)^{k_1}} \right)=\frac{o((x-a_1)^{k_1})}{(x-a_1)^{k_1}}$$

    $\frac{P}{Q} - $ (Пр. часть) = знам. сократится $\Rightarrow$ многочлен $\equiv 0$
\end{proof}
%</оразложениирациональнойдробиproof>
\end{document}