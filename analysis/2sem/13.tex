\documentclass[12pt, a4paper]{article}

\usepackage{lastpage}
\usepackage{mathtools}
\usepackage{xltxtra}
\usepackage{libertine}
\usepackage{amsmath}
\usepackage{amsthm}
\usepackage{amsfonts}
\usepackage{amssymb}
\usepackage{enumitem}
\usepackage{xcolor}
\usepackage[left=2.3cm, right=2.3cm, top=2.7cm, bottom=2.7cm, bindingoffset=0cm, headheight=15pt]{geometry}
\usepackage{fancyhdr}
\usepackage[russian]{babel}
% \usepackage{parindent}

\pagestyle{fancy}
\lfoot{M3137y2019}
\rhead{\thepage\ из \pageref{LastPage}}

\newcommand{\R}{\mathbb{R}}
\newcommand{\Q}{\mathbb{Q}}
\newcommand{\C}{\mathbb{C}}
\newcommand{\Z}{\mathbb{Z}}
\newcommand{\B}{\mathbb{B}}
\newcommand{\N}{\mathbb{N}}

\DeclareMathOperator*{\xor}{\oplus}
\DeclareMathOperator*{\equ}{\sim}
\DeclareMathOperator{\Ln}{\text{Ln}}
\DeclareMathOperator{\sign}{\text{sign}}
\DeclareMathOperator{\Sym}{\text{Sym}}
\DeclareMathOperator{\Asym}{\text{Asym}}
% \DeclareMathOperator{\sh}{\text{sh}}
% \DeclareMathOperator{\tg}{\text{tg}}
% \DeclareMathOperator{\arctg}{\text{arctg}}
% \DeclareMathOperator{\ch}{\text{ch}}

\DeclarePairedDelimiter{\ceil}{\lceil}{\rceil}

\setmainfont{Linux Libertine}

\theoremstyle{plain}
\newtheorem{theorem}{Теорема}
\newtheorem{axiom}{Аксиома}
\newtheorem{lemma}{Лемма}

\theoremstyle{remark}
\newtheorem*{remark}{Примечание}
\newtheorem*{exercise}{Упражнение}
\newtheorem*{consequence}{Следствие}
\newtheorem*{example}{Пример}
\newtheorem*{observation}{Наблюдение}

\theoremstyle{definition}
\newtheorem*{definition}{Определение}
\newtheorem*{obozn}{Обозначение}

\lhead{Конспект по матанализу}
\cfoot{}
\rfoot{Лекция 12}

\renewcommand{\thesubsection}{\arabic{subsection}.}

\begin{document}

Рассмотрим такой ряд:

$$1-1+\frac{1}{2}+\frac{1}{2}-\frac{1}{2}+\frac{1}{4}+\frac{1}{4}-\frac{1}{2}+\frac{1}{4}+\frac{1}{4}\underbrace{-\frac{1}{4}+\frac{1}{8}+\frac{1}{8}}_{4 \text{ раза}}+\underbrace{-\frac{1}{8}+\frac{1}{16}+\frac{1}{16}}_{8 \text{ раз}}+\ldots$$

Сходится ли этот ряд? Да, потому что можно разбить на скобки из 3х слагаемых (кроме $1$), каждая из которых $=0$.

Рассмотрим похожий ряд:
$$-1+1-\frac{1}{2}-\frac{1}{2}+\frac{1}{2}-\frac{1}{4}-\frac{1}{4}+\frac{1}{2}-\frac{1}{4}-\frac{1}{4}\underbrace{+\frac{1}{4}-\frac{1}{8}-\frac{1}{8}}_{4 \text{ раза}}+\underbrace{\frac{1}{8}-\frac{1}{16}-\frac{1}{16}}_{8 \text{ раз}}+\ldots=-1$$
Произошла магия --- сумма ряда $=-1$, т.к. $b_n=-a_n$, где $b_n$ --- слагаемое этого ряда, $a_n$ --- прошлого ряда. Но мы просто переставили слагаемые предыдущего ряда $\Rightarrow$ перестановка бесконечного числа слагаемых меняет результат.

\begin{definition}
    $\sum a_k, w:\N\to\N$ --- биекция

    $b_k:=a_{w(k)}$, $\sum b_k$ называется \textbf{перестановкой} ряда $\sum a_k$
\end{definition}
\begin{theorem}
    Ряд $A$ абсолютно сходится, тогда его перестановка $B$ тоже абсолютно сходится и имеет ту же сумму.
\end{theorem}
\begin{proof}
    \begin{enumerate}
        \item $a_k\ge 0$
        $$S_n^{(b)}=b_1+\ldots +b_n=a_{w(1)}+\ldots+a_{w(n)}\le S^{(a)}_N, N=\max(w(1)\ldots w(n))$$
        Предельный переход: $S^{(b)}\le S^{(a)}$

        Т.к. $A$ --- перестановка $B$, то $S^{(a)}\le S^{(b)} \Rightarrow S^{(a)}=S^{(b)}$

        \item Общий случай
        
        $a_k^+=\max(a_k, 0), a_k^-=\max(-a_k, 0)$

        $\sum b_k^+$ --- перестановка $\sum a_k^+$; $\sum b_k^-$ --- перестановка $\sum a_k^-$

        Срезки сходятся по пункту 1., в силу абсолютной сходимости частичные суммы конечны $\Rightarrow S^{(a)}=S^{(b)}$
    \end{enumerate}
\end{proof}

\begin{theorem}
    Римана.

    $\sum a_k$ --- сходится неабсолютно. Тогда:
    \begin{enumerate}
        \item $\exists$ перестановка ряда $A$, которая не имеет предела частичной суммы
        \item $\forall S\in\overline\R$ $\exists$ перестанвка ряда $A$ с суммой $S$
    \end{enumerate}
\end{theorem}
\begin{proof}
    2. Т.к. $\sum a_k$ сходится неабсолютно, существует две кучи - одна из положительных $a_k$, другая из отрицательных. Обе кучи бесконечные и имеют бесконечную сумму. Тогда будем брать элементы из положительной кучи, пока частичная сумма $<S$, потом берем элементы из отрицательной кучи, пока сумма $>S$. Получаем ряд, осциллирующий вокруг $S$. Если есть нулевые элементы, то будем их добавлять в сумму, когда меняем направление.

    1. Будем осциллировать не вокруг $S$, а между $T$ и $S$.
\end{proof}

\begin{example}
    $$\sum_{n=1}^{+\infty}\frac{1}{n(2n-1)}=\sum_{n=1}^{+\infty}\left(\frac{2}{2n-1}-\frac{1}{n}\right)=2\left(1+\frac{1}{3}+\frac{1}{5}+\frac{1}{7}+\ldots\right)-1-\frac{1}{2}-\frac{1}{3}-\frac{1}{4}-\ldots=$$
    $$=2-1-\frac{1}{2}+\frac{2}{3}-\frac{1}{3}-\frac{1}{4}+\frac{2}{5}-\frac{1}{5}+\ldots=$$
    $$=1-\frac{1}{2}+\frac{1}{3}-\frac{1}{4}+\frac{1}{5}-\frac{1}{6}+\ldots$$
    Разложим $f(x)=\ln(1+x)$ по Тейлору:
    $$\ln(1+x)=x-\frac{x^2}{2}+\frac{x^3}{3}+\ldots+\frac{1}{(n+1)!}f^{(n+1)}(c)x^n$$
    $$f^{(n+1)}(c)=\frac{(-1)^n n!}{(1+c)^{n+1}}$$
    $$R_n\le \frac{1}{n+1}\frac{1}{(1+c)^{n+1}}\le \frac{1}{n+1}$$
    $$\ln 2 = 1-\frac{1}{2}+\frac{1}{3}-\ldots$$
    Проблема: сумма этого ряда должна быть $>1$, но мы получили обратное. Это произошло, потому что мы переставили слагаемые неабсолютно сходящегося ряда.
\end{example}

\section*{Произведение рядов}

$(a_1+\ldots+a_k)(b_1+\ldots+b_l)=\sum\sum a_ib_j$

\begin{definition}
    $\sum a_k, \sum b_k$

    $\gamma : \N\to\N\times\N$ --- биекция, $\gamma(k)=(\varphi(k),\psi(k))$

    \textbf{Произведение рядов} $A$ и $B$ --- ряд $\sum_{k=1}^{+\infty} a_{\varphi(k)}b_{\psi(k)}$
\end{definition}

\begin{theorem}
    Коши.

    Пусть ряды $\sum a_k, \sum b_k$ абсолютно сходятся. Тогда $\forall$ биекции $\gamma:\N\to\N\times\N$ произведение рядов абсолютно сходится и его сумма $=AB$ 
\end{theorem}
\begin{proof}
    $\sum |a_k|=A^*, \sum |b_k|=B^*, 0\le A^*,B^*<+\infty$

    $$\sum_{k=1}^N |a_{\varphi(x)}b_{\psi(x)}|\le \sum_{i=1}^{M}|a_i|\sum_{j=1}^L |b_j|\le A^*B^*$$
    $$M:=\max(\varphi(1)\ldots \varphi(N)) \quad N:=\max(\psi(1)\ldots \psi(N))$$
    Итого произведение сходится абсолютно $\Rightarrow \forall \gamma$ произведение рядов имеет одинаковую сумму.

    Возьмём $\gamma$ такое, что оно обходит точки $\N\times\N$ ``по квадратам'', т.е. не заходит в следующий квадрат, пока не обошло предыдущий. Тогда:
    $$\sum_{k=1}^{n^2}a_{\varphi(k)}b_{\psi(k)}=\sum_{i=1}^n a_i \sum_{j=1}^n b_j\xrightarrow[n\to+\infty]{}AB$$
\end{proof}

\begin{example}
    $x\in\R, x$ --- фиксированный
    $$\sum_{k=0}^{+\infty} a_kx^k \sum_{j=0}^{+\infty} b_jx^j = \sum_{n=0}^{+\infty} c_n x^n$$
    $$c_n=a_0b_n+a_1b_{n-1}+\ldots+a_nb_0$$
    Это называется произведение степенных рядов.
\end{example}

\end{document}