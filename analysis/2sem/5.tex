\documentclass[12pt, a4paper]{article}

\usepackage{lastpage}
\usepackage{mathtools}
\usepackage{xltxtra}
\usepackage{libertine}
\usepackage{amsmath}
\usepackage{amsthm}
\usepackage{amsfonts}
\usepackage{amssymb}
\usepackage{enumitem}
\usepackage{xcolor}
\usepackage[left=2.3cm, right=2.3cm, top=2.7cm, bottom=2.7cm, bindingoffset=0cm, headheight=15pt]{geometry}
\usepackage{fancyhdr}
\usepackage[russian]{babel}
% \usepackage{parindent}

\pagestyle{fancy}
\lfoot{M3137y2019}
\rhead{\thepage\ из \pageref{LastPage}}

\newcommand{\R}{\mathbb{R}}
\newcommand{\Q}{\mathbb{Q}}
\newcommand{\C}{\mathbb{C}}
\newcommand{\Z}{\mathbb{Z}}
\newcommand{\B}{\mathbb{B}}
\newcommand{\N}{\mathbb{N}}

\DeclareMathOperator*{\xor}{\oplus}
\DeclareMathOperator*{\equ}{\sim}
\DeclareMathOperator{\Ln}{\text{Ln}}
\DeclareMathOperator{\sign}{\text{sign}}
\DeclareMathOperator{\Sym}{\text{Sym}}
\DeclareMathOperator{\Asym}{\text{Asym}}
% \DeclareMathOperator{\sh}{\text{sh}}
% \DeclareMathOperator{\tg}{\text{tg}}
% \DeclareMathOperator{\arctg}{\text{arctg}}
% \DeclareMathOperator{\ch}{\text{ch}}

\DeclarePairedDelimiter{\ceil}{\lceil}{\rceil}

\setmainfont{Linux Libertine}

\theoremstyle{plain}
\newtheorem{theorem}{Теорема}
\newtheorem{axiom}{Аксиома}
\newtheorem{lemma}{Лемма}

\theoremstyle{remark}
\newtheorem*{remark}{Примечание}
\newtheorem*{exercise}{Упражнение}
\newtheorem*{consequence}{Следствие}
\newtheorem*{example}{Пример}
\newtheorem*{observation}{Наблюдение}

\theoremstyle{definition}
\newtheorem*{definition}{Определение}
\newtheorem*{obozn}{Обозначение}

\lhead{Конспект по матанализу}
\cfoot{}
\rfoot{Лекция 5}

\begin{document}

\begin{theorem}
    Обобщенная теорема о плотности.

    $\Phi:Segm\langle a,b\rangle\to\R$ --- аддитивная функция промежутка

    $f:\langle a,b\rangle\to\R$ --- непр.

    $\forall \Delta\in Segm\langle a,b\rangle\ \ \exists m_\Delta, M_\Delta$ --- не точный минимум/максимум

    \begin{enumerate}
        \item $m_\Delta l_\Delta\leq \Phi(\Delta)\leq M_\Delta l_\Delta$
        \item $m_\Delta\leq f(x)\leq M_\Delta$ при всех $x\in\Delta$
        \item $\forall$ фикс. $x\quad$ $M_\Delta-m_\Delta\xrightarrow[\text{``}\Delta\to x\text{''}]{} 0$
        
        $\forall \varepsilon>0 \ \ \exists \delta>0 \ \ \forall \Delta: l_\Delta<\delta \quad |M_\Delta-m_\Delta|<\varepsilon$
    \end{enumerate}

    Тогда $\forall [p,q]\in Segm\langle a,b] \quad \Phi([p,q])=\int\limits_p^q f$
\end{theorem}
\begin{proof}
    $F(x)=\begin{cases}
        0, & x=a \\
        \Phi[a,x], & x>a
    \end{cases}$

    Докажем, что $F$ --- первообразная $f$.

    Фиксируем $x$

    По 1.:
    $$m_\Delta\leq\frac{F(x+h)-F(x)}{h}=\frac{\Phi[x,x+h]}{h}\leq M_\Delta$$

    По 2.:
    $$m_\Delta\leq f(x)\leq M_\Delta$$

    $$\left|\frac{F(x+h)-F(x)}{h}-f(x)\right|\leq M_\Delta-m_\Delta\xrightarrow[\text{``}\Delta\to x\text{''}]{}0$$

    Мы не можем написать ``$\Delta\to x$'' без кавычек, т.к. $\Delta$ --- не число, но ``$\Delta\to x$''$\Leftrightarrow h\to0$

    Таким образом, $$\frac{F(x+h)-F(x)}{h}-f(x)\xrightarrow[h\to0]{}f(x)$$
\end{proof}

\subsection*{Объемы фигур вращения}

Объем это $V: Fig\to\R$:
\begin{enumerate}
    \item $V$ --- кон., адд.: $V(A_1\sqcup A_2)=V(A_1)+V(A_2)$
    \item $V(\text{ед. куб})=1$
    \item $V$ не меняется при движении
\end{enumerate}

Проблема: такой функции не существует.

Поэтому будем использовать объем для частных случаев фигур, а не для всех.

\begin{definition}
    $\sphericalangle A\in\R^2$ --- фигура в I квадранте.

    \textbf{Вращение} $A$:
    \begin{enumerate}
        \item по оси $x$ : $A_x=\{x,y,z\in\R^3 : (x,\sqrt{y^2+z^2})\in A\}$
        \item по оси $y$ : $A_y=\{x,y,z\in\R^3 : (\sqrt{x^2+z^2}, y)\in A\}$
    \end{enumerate}
\end{definition}

Для непр. $f:[a,b]\to\R, c\mapsto \tilde c\geq 0:$
$$\Phi(\Delta)=V(\text{ПГ}(f,\Delta)_x) \textit{(или $y$)}$$

\begin{theorem}
    $f:\langle a,b\rangle\to\R$ --- непр., $f\geq 0$

    $\Phi_x(\Delta)=$ ``объем фигуры вращения вокруг оси $OX$''

    $\Phi_y(\Delta)=$ ``объем фигуры вращения вокруг оси $OY$''

    Тогда : $\forall \Delta = [p,q]\in Segm\langle a,b\rangle:$
    \begin{enumerate}
        \item $\Phi_x[p,q]=\pi\int\limits_p^q f^2(x)dx$
        \item $\Phi_y[p,q]=2\pi\int\limits_p^q xf(x)dx$
    \end{enumerate}
\end{theorem}
\begin{proof}
    \begin{enumerate}
        \item Это --- упражнение, оно не использует ничего умного.
        \item Мы знаем, что объем цилиндра $=S(\text{основание})\cdot h$.
        
        Для оценки $\Phi(\Delta)$ найдем прямоугольник, который является минимальным по площади сечения и максимальный прямоугольники: $\text{П}_{min}$ и $\text{П}_{max}$.

        $$\pi m_\Delta(q-p)=\pi\min f\min 2x \cdot (q-p)\leq V((\text{П}_{min})_y)\leq \Phi(\Delta)\leq$$
        $$\leq V((\text{П}_{max})_y)\leq \pi\max\limits_{x\in[p,q]}f \max\limits_{x\in[p,q]} 2x\cdot (q-p)=\pi M_\Delta(q-p)$$

        Можем заметить, что $\Phi$ подходит под все три пункта теоремы об обобщенной плотности.
    \end{enumerate}
\end{proof}

\begin{example}
    Объем тора.

    Будем вращать окружность, центр которой лежат на оси $OX$ в точке $R$, с радиусом $r$, относительно оси $OY$.

    $$\frac{1}{2}V=2\pi \int\limits_{R-2}^{R+2}(x\mp R)\sqrt{r^2-(x-R)^2}dx=\pi\int\limits_{R_2}^{R+2} 2(x-R)\sqrt{r^2-(x-R)^2}dx+2\pi\int\limits_{R-2}^{R+2} R\sqrt{r^2-(x-r)^2}dx=$$
    $$=-\pi\frac{1}{3}(r^2-(x-R)^2)^{\frac{3}{2}}|_{x=R-2}^{x=R+2}+2\pi R\frac{\pi r^2}{2}=2\pi R\cdot \pi r^2$$
\end{example}

\subsection*{Длина гладкого пути}

$\gamma:[a,b]\to\R^m$ --- непр.

$\gamma(a)$ --- начало; $\gamma(b)$ --- конец

$\gamma:t\mapsto \begin{pmatrix}
    \gamma_1(t) \\
    \gamma_2(t) \\
    \vdots \\
    \gamma_m(t)
\end{pmatrix}$; $\gamma_i$ --- коорд. функции

Если все $\gamma_i\in C^1 [a,b]$, то $\gamma$ --- \textbf{гладкий путь}.

$$\gamma'(t)=\lim\limits_{\Delta t\to0}\frac{\gamma(t+\Delta t)-\gamma(t)}{\Delta t}=\lim \begin{pmatrix}
    \frac{\gamma_1(t+\Delta t)-\gamma_1(t)}{\Delta t} \\
    \vdots \\
    \frac{\gamma_m(t+\Delta t)-\gamma_m(t)}{\Delta t} \\
\end{pmatrix}=\begin{pmatrix}
    \gamma_1;(t) \\
    \gamma_2'(t) \\
    \vdots \\
    \gamma_m'(t)
\end{pmatrix}$$

Кривая Пеано: $[0,1]\to[0,1]\times[0,1]$ --- ломает длину пути, но она не гладкая, поэтому мы рассматриваем только гладкие пути.

\begin{definition}
    \textbf{Длина пути}
\end{definition}

\end{document}