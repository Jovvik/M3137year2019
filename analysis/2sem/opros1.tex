\documentclass[12pt, a4paper]{article}

\usepackage{lastpage}
\usepackage{mathtools}
\usepackage{xltxtra}
\usepackage{libertine}
\usepackage{amsmath}
\usepackage{amsthm}
\usepackage{amsfonts}
\usepackage{amssymb}
\usepackage{enumitem}
\usepackage{xcolor}
\usepackage[left=2.3cm, right=2.3cm, top=2.7cm, bottom=2.7cm, bindingoffset=0cm, headheight=15pt]{geometry}
\usepackage{fancyhdr}
\usepackage[russian]{babel}
% \usepackage{parindent}

\pagestyle{fancy}
\lfoot{M3137y2019}
\rhead{\thepage\ из \pageref{LastPage}}

\newcommand{\R}{\mathbb{R}}
\newcommand{\Q}{\mathbb{Q}}
\newcommand{\C}{\mathbb{C}}
\newcommand{\Z}{\mathbb{Z}}
\newcommand{\B}{\mathbb{B}}
\newcommand{\N}{\mathbb{N}}

\DeclareMathOperator*{\xor}{\oplus}
\DeclareMathOperator*{\equ}{\sim}
\DeclareMathOperator{\Ln}{\text{Ln}}
\DeclareMathOperator{\sign}{\text{sign}}
\DeclareMathOperator{\Sym}{\text{Sym}}
\DeclareMathOperator{\Asym}{\text{Asym}}
% \DeclareMathOperator{\sh}{\text{sh}}
% \DeclareMathOperator{\tg}{\text{tg}}
% \DeclareMathOperator{\arctg}{\text{arctg}}
% \DeclareMathOperator{\ch}{\text{ch}}

\DeclarePairedDelimiter{\ceil}{\lceil}{\rceil}

\setmainfont{Linux Libertine}

\theoremstyle{plain}
\newtheorem{theorem}{Теорема}
\newtheorem{axiom}{Аксиома}
\newtheorem{lemma}{Лемма}

\theoremstyle{remark}
\newtheorem*{remark}{Примечание}
\newtheorem*{exercise}{Упражнение}
\newtheorem*{consequence}{Следствие}
\newtheorem*{example}{Пример}
\newtheorem*{observation}{Наблюдение}

\theoremstyle{definition}
\newtheorem*{definition}{Определение}
\newtheorem*{obozn}{Обозначение}

\usepackage{sectsty}

\allsectionsfont{\raggedright}
\subsectionfont{\fontsize{14}{15}\selectfont}

\lhead{Конспект к опросу 1}
\cfoot{}
\rfoot{}

\begin{document}
%<*opros1>

\section{Определения}

\import{Локальный максимум, минимум, экстремум}{1.tex}{локальныймаксимум}

\import{\teormin Первообразная, неопределенный интеграл}{1.tex}{первообразная}
\get{1.tex}{неопределенныйинтеграл}

\import{Теорема о существовании первообразной}{1.tex}{осуществованиипервообразной}

\import{Таблица первообразных}{1.tex}{таблицапервообразных}

\import{Равномерная непрерывность}{1.tex}{равномернаянепрерывность}

\import{Площадь, аддитивность площади, ослабленная аддитивность}{2.tex}{фигуры}
\get{2.tex}{площадь}
\get{2.tex}{ослабленнаяплощадь}

\import{\teormin Определенный интеграл}{2.tex}{определенныйинтеграл}

\import{Положительная и отрицательная срезки}{2.tex}{срезки}

\import{Среднее значение функции на промежутке}{2.tex}{}
\textcolor{red}{Отсутствует}

\import{Кусочно-непрерывная функция}{3.tex}{кусочнонепрерывнаяфункция}

\section{Теоремы}

\import{Критерий монотонности функции. Следствия}{1.tex}{критериймонотонности}
\getwithproof{1.tex}{критериймонотонностиследствия1}
\getwithproof{1.tex}{критериймонотонностиследствия2}

\import{Теорема о необходимом и достаточном условиях экстремума}{1.tex}{необходимостьидостаточностьусловиялокальногоэкстремума}

\import{Теорема Кантора о равномерной непрерывности}{1.tex}{теоремакантора}

\import{Теорема Брауэра о неподвижной точке}{1.tex}{теоремаонеподвижнойточке}

\import{Теорема о свойствах неопределенного интеграла}{1.tex}{свойстванеопределенногоинтеграла}

\import{Интегрирование неравенств. Теорема о среднем}{2.tex}{интегрированиенеравенств}
\textcolor{red}{Кто такая теорема о среднем}

\import{Теорема Барроу}{2.tex}{теоремабарроу}

\import{Формула Ньютона-Лейбница, в том числе, для кусочно-непрерывных функций}{2.tex}{ньютоналейбница}
\textcolor{red}{Что с кусочно-непрерывными?}

\import{Лемма об ускоренной сходимости}{2.tex}{обускореннойсходимости}

\import{Правило Лопиталя}{2.tex}{правилолопиталя}

%</opros1>
\end{document}