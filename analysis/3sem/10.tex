\documentclass[12pt, a4paper]{article}

\usepackage{lastpage}
\usepackage{mathtools}
\usepackage{xltxtra}
\usepackage{libertine}
\usepackage{amsmath}
\usepackage{amsthm}
\usepackage{amsfonts}
\usepackage{amssymb}
\usepackage{enumitem}
\usepackage{xcolor}
\usepackage[left=2.3cm, right=2.3cm, top=2.7cm, bottom=2.7cm, bindingoffset=0cm, headheight=15pt]{geometry}
\usepackage{fancyhdr}
\usepackage[russian]{babel}
% \usepackage{parindent}

\pagestyle{fancy}
\lfoot{M3137y2019}
\rhead{\thepage\ из \pageref{LastPage}}

\newcommand{\R}{\mathbb{R}}
\newcommand{\Q}{\mathbb{Q}}
\newcommand{\C}{\mathbb{C}}
\newcommand{\Z}{\mathbb{Z}}
\newcommand{\B}{\mathbb{B}}
\newcommand{\N}{\mathbb{N}}

\DeclareMathOperator*{\xor}{\oplus}
\DeclareMathOperator*{\equ}{\sim}
\DeclareMathOperator{\Ln}{\text{Ln}}
\DeclareMathOperator{\sign}{\text{sign}}
\DeclareMathOperator{\Sym}{\text{Sym}}
\DeclareMathOperator{\Asym}{\text{Asym}}
% \DeclareMathOperator{\sh}{\text{sh}}
% \DeclareMathOperator{\tg}{\text{tg}}
% \DeclareMathOperator{\arctg}{\text{arctg}}
% \DeclareMathOperator{\ch}{\text{ch}}

\DeclarePairedDelimiter{\ceil}{\lceil}{\rceil}

\setmainfont{Linux Libertine}

\theoremstyle{plain}
\newtheorem{theorem}{Теорема}
\newtheorem{axiom}{Аксиома}
\newtheorem{lemma}{Лемма}

\theoremstyle{remark}
\newtheorem*{remark}{Примечание}
\newtheorem*{exercise}{Упражнение}
\newtheorem*{consequence}{Следствие}
\newtheorem*{example}{Пример}
\newtheorem*{observation}{Наблюдение}

\theoremstyle{definition}
\newtheorem*{definition}{Определение}
\newtheorem*{obozn}{Обозначение}

\lhead{Математический анализ}
\cfoot{}
\rfoot{16.11.2020}

\begin{document}

\subsection*{Гомотопия}

Неформально гомотопия --- непрерывная деформация объектов. У нас рассматриваемые объекты --- пути.

\begin{definition}
    \textbf{Гопотопия} двух (\textit{непрерывных}) путей \(\gamma_0, \gamma_1 : [a, b] \to O\subset \R^m\) это непрерывное отображение \(\Gamma : \underbrace{[a, b]}_{t} \times \underbrace{[0, 1]}_{u} \to O\), такое что:
    \begin{itemize}
        \item \(\Gamma(\circ , 0) = \gamma_0\)
        \item \(\Gamma(\circ , 1) = \gamma_1\)
    \end{itemize}

    Гопотопия \textbf{связанная} (\textit{не связная}), если:
    \begin{itemize}
        \item \(\gamma_0(a) = \gamma_1(a)\)
        \item \(\gamma_0(b) = \gamma_1(b)\)
        \item \(\forall u\in[0, 1]\ \Gamma(a, u) = \gamma_0(a), \Gamma(b, u) = \gamma_1(b)\)
    \end{itemize}

    \begin{figure}[h]
        \centering
        \includesvg[scale=1]{images/гомотопия_связанная.svg}
        \caption{Связанная гопотопия.\\ Пунктиром --- \(\Gamma(\circ, u)\) для различных \(u\)}
    \end{figure}

    Гопотопия \textbf{петельная}, если:
    \begin{itemize}
        \item \(\gamma_0(a) = \gamma_0(b)\)
        \item \(\gamma_1(a) = \gamma_1(b)\)
        \item \(\forall u\in[0, 1]\ \Gamma(a, u) = \Gamma(b, u)\)
    \end{itemize}

    \begin{figure}[h]
        \centering
        \includesvg[scale=1]{images/гомотопия_петельная.svg}
        \caption{Петельная гопотопия.\\ Пунктиром --- \(\Gamma(\circ, u)\) для различных \(u\)}
    \end{figure}

    \pagebreak
\end{definition}

\begin{theorem}\itemfix
    %<*интегралпосвязанногомотопнымпутям>
    \begin{itemize}
        \item \(V\) --- локально потенциальное векторное поле в \(O\subset\R^m\)
        \item \(\gamma_0, \gamma_1\) --- связанно гомотопные пути
    \end{itemize}
    Тогда \(\int_{\gamma_0} \sum V_i dx_i = \int_{\gamma_1} \sum V_idx_i\)
    %</интегралпосвязанногомотопнымпутям>
\end{theorem}

\begin{remark}
    То же самое верно для петельных гомотопий.
\end{remark}

%<*интегралпосвязанногомотопнымпутямproof>
\begin{proof}
    \(\gamma_u(t) : = \Gamma(t, u), t\in[a, b], u\in[0, 1]\)

    \[\Phi(u) = \int_{\gamma_u} \sum V_i dx_i\]

    Мы хотим доказать, что \(\Phi(u) = \const\). Докажем более простой факт, что \(\Phi\) --- локально постоянна, тогда в силу компактности отрезка \(\Phi\) будет постоянна.

    Определение локально постоянной функции:
    \[\forall u_0\in[0, 1]\ \ \exists W(u_0) : \forall u\in W(u_0)\cap [0, 1]\quad \Phi(u) = \Phi(u_0)\]

    \(\Gamma\) --- непр. на \([a, b] \times [0, 1]\) --- комп. \( \Rightarrow \Gamma\) равномерно непрерывна:
    \[\forall \delta > 0\ \exists \sigma > 0\ \forall t, t' : |t - t'| < \sigma\ \forall u,u': |u - u'|< \sigma\quad |\Gamma(t, u) - \Gamma(t', u')| < \frac{\delta}{2} \]

    Возьмём \(\delta\) из леммы о похожести близких путей (\ref{лемма 3, лекция 9}) для пути \(\gamma_{u_0}\).

    Если \(|u - u_0| < \delta \ \ |\Gamma(t, u) - \Gamma(t, u_0)|< \frac{\delta}{2}\) при \(t\in[a, b]\), т.е. \(\gamma_u\) и \(\gamma_{u_0}\) похожи по лемме о похожести близких путей. Хочется сказать, что интегралы по \(\gamma_u\) и \(\gamma_{u_0}\) таким образом равны, однако это не обосновано, для этого необходимо, чтобы пути были кусочно-гладкими.

    Построим кусочно-гладкий путь \(\tilde \gamma_{u_0}\), \(\frac{\delta}{4}\)-близкий к \(\gamma_{u_0}\), т.е.
    \[\forall t\in[a, b] \ \ |\gamma_{u_0}(t) - \tilde \gamma_{u_0}(t)| < \frac{\delta}{4}\]
    и кусочно-гладкий путь \(\tilde \gamma_u\), \(\frac{\delta}{4}\)-близкий к \(\gamma_{u}\). Тогда \(\tilde \gamma_{u_0}\) и \(\tilde \gamma_u\) - \(\delta\)-близкие к \(\gamma_{u_0}\) \( \Rightarrow \) они \(V\)-похожи \( \Rightarrow \) \[\int_{\gamma_u} \sum V_i dx_i\defeq \int_{\tilde \gamma_u} \dots = \int_{\tilde \gamma_{u_0}} \dots \defeq \int_{\gamma_{u_0}} \dots \]
    Таким образом, \(\Phi(u) = \Phi(u_0)\) при \(|u - u_0|< \delta\), т.е. \(\Phi\) --- локально постоянна.
\end{proof}
%</интегралпосвязанногомотопнымпутямproof>

\begin{definition}
    Область \(O\subset \R^m\) --- \textbf{односвязная}, если любой замкнутый путь в ней гомотопен постоянному пути.

    Простыми словами --- в \(O\) нет дырок, иначе путь вокруг дырки нельзя было бы стянуть.

    \begin{figure}[h]
        \centering
        \includesvg[scale=1]{images/односвязно.svg}
        \caption{Стягивание замкнутого пути (сплошной линией) к постоянному пути (точке)}
    \end{figure}
\end{definition}

\begin{remark}\itemfix
    \begin{enumerate}
        \item Выпуклая область --- односвязная.

              \begin{figure}[h]
                  \centering
                  \includesvg[scale=0.9]{images/гомотетия.svg}
                  \caption{Применение гомотетии с центром \(A\)}
              \end{figure}

              Это доказывается тем, что для любого пути можно применить гомотетию в качестве гомотопии: \(\Gamma(t, u) = F_{1 - u}(\gamma(t))\), где \(F_{\alpha}\) --- гомотетия с центром \(A\) (лежит внутри области, огр. путём \(\gamma\)) и коэффициентом \(\alpha\)
        \item Гомеоморфный образ односвязного множества --- односвязен.

              \(\Phi : O \to O'\) --- гомеоморфизм, \(\gamma\) --- петля в \(O'\), \(\Phi^{ - 1}(\gamma)\) --- петля в \(O\).

              \(\Gamma : [a, b] \times [0, 1] \to O\) --- гомотопия \(\Phi^{ - 1}(\gamma)\) и постоянного пути \(\tilde \gamma\equiv A\)

              \(\Phi \circ \Gamma\) --- гомотопия \(\gamma\) с постоянным путём \(\dbltilde \gamma \equiv \Phi(A)\)
    \end{enumerate}
\end{remark}

\begin{theorem}\itemfix
    \begin{itemize}
        \item \(O\subset \R^m\) --- односвязная область
        \item \(V\) --- локально потенциальное векторное поле в \(O\)
    \end{itemize}
    Тогда \(V\) --- потенциальное в \(O\)
\end{theorem}

\begin{proof}
    \(V\) --- локально потенциально, \(\gamma_0\) --- кусочно-гладкая петля, тогда \(\gamma_0\) гомотопна постоянному пути \(\gamma_1\) \( \Rightarrow \)
    \[\int_{\gamma_0} = \int_{\gamma_1} = \int_a^b \langle V(\gamma_1(t)), \underbrace{\gamma'_1(t)}_{\equiv0} \rangle dt = 0\]
    Тогда по теореме о характеризации потенциальных векторных полей в терминах интегралов \(V\) потенциально.
\end{proof}

\begin{corollary}
    Теорема Пуанкаре верна в односвязной области.
\end{corollary}

\end{document}