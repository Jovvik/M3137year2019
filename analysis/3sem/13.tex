\documentclass[12pt, a4paper]{article}

\usepackage{lastpage}
\usepackage{mathtools}
\usepackage{xltxtra}
\usepackage{libertine}
\usepackage{amsmath}
\usepackage{amsthm}
\usepackage{amsfonts}
\usepackage{amssymb}
\usepackage{enumitem}
\usepackage{xcolor}
\usepackage[left=2.3cm, right=2.3cm, top=2.7cm, bottom=2.7cm, bindingoffset=0cm, headheight=15pt]{geometry}
\usepackage{fancyhdr}
\usepackage[russian]{babel}
% \usepackage{parindent}

\pagestyle{fancy}
\lfoot{M3137y2019}
\rhead{\thepage\ из \pageref{LastPage}}

\newcommand{\R}{\mathbb{R}}
\newcommand{\Q}{\mathbb{Q}}
\newcommand{\C}{\mathbb{C}}
\newcommand{\Z}{\mathbb{Z}}
\newcommand{\B}{\mathbb{B}}
\newcommand{\N}{\mathbb{N}}

\DeclareMathOperator*{\xor}{\oplus}
\DeclareMathOperator*{\equ}{\sim}
\DeclareMathOperator{\Ln}{\text{Ln}}
\DeclareMathOperator{\sign}{\text{sign}}
\DeclareMathOperator{\Sym}{\text{Sym}}
\DeclareMathOperator{\Asym}{\text{Asym}}
% \DeclareMathOperator{\sh}{\text{sh}}
% \DeclareMathOperator{\tg}{\text{tg}}
% \DeclareMathOperator{\arctg}{\text{arctg}}
% \DeclareMathOperator{\ch}{\text{ch}}

\DeclarePairedDelimiter{\ceil}{\lceil}{\rceil}

\setmainfont{Linux Libertine}

\theoremstyle{plain}
\newtheorem{theorem}{Теорема}
\newtheorem{axiom}{Аксиома}
\newtheorem{lemma}{Лемма}

\theoremstyle{remark}
\newtheorem*{remark}{Примечание}
\newtheorem*{exercise}{Упражнение}
\newtheorem*{consequence}{Следствие}
\newtheorem*{example}{Пример}
\newtheorem*{observation}{Наблюдение}

\theoremstyle{definition}
\newtheorem*{definition}{Определение}
\newtheorem*{obozn}{Обозначение}

\lhead{Математический анализ}
\cfoot{}
\rfoot{7.12.2020}

\begin{document}

\section*{Ряды Тейлора}

\begin{example}
    \begin{align*}
        e^x             & = \sum_{n = 0}^{ +\infty},\ x\in\R                                             \\
        \sin x          & = \sum_{n = 1}^{ +\infty} ( - 1)^{n - 1} \frac{x^{2n - 1}}{(2n - 1)!},\ x\in\R \\
        \cos x          & = \sum ( - 1)^n \frac{x^{2n}}{(2n)!},\ x\in\R                                  \\
        \frac{1}{1 + x} & = \sum_{n = 0}^{ +\infty} ( -1)^nx^n,\ x\in( - 1, 1)                           \\
        \ln(1 + x)      & = \sum_{n = 0}^{ +\infty} ( - 1)^{n - 1}\frac{x^n}{n},\ x\in( - 1, 1)          \\
    \end{align*}
\end{example}

\begin{theorem}
    \(\forall \sigma\in\R \ \ \forall x\in( - 1, 1)\)
    \[(1 + x)^{\sigma} = 1 + \sigma x + \frac{\sigma(\sigma - 1)}{2} x^2 + \dots \]
\end{theorem}
\begin{proof}
    При \(|x|< 1\) ряд сходится по признаку Даламбера.

    \[\left|\frac{a_{n+1}}{a_n}\right|=\left|\frac{(\sigma - n)x}{n + 1} \right| \xrightarrow{n\to +\infty}|x|< 1\]

    Обозначим сумму ряда через \(S(x)\).

    Наблюдение: \(S'(x)(1 + x) = \sigma S(x)\)
    \[S'(x) = \dots + \frac{\sigma(\sigma - 1)\dots (\sigma - n)}{n!} x^n + \dots \]
    \[S(x) = \dots + \frac{\sigma(\sigma - 1)\dots (\sigma - n + 1)}{n!} x^n + \dots \]
    \begin{align*}
        (1 + x)S' & = \dots + \left( \frac{\sigma(\sigma - 1)\dots (\sigma - n)}{n!} + \frac{\sigma(\sigma - 1)\dots (\sigma - n + 1)}{n!}n \right)x^n + \dots \\
                  & = \dots + \frac{\sigma(\sigma - 1)\dots (\sigma - n + 1)}{n!}\sigma x^n + \dots
    \end{align*}

    \[f(x) = \frac{S(x)}{(1 + x)^\sigma} \quad f'(x) = \frac{S'(1 + x)^\sigma - \sigma(1 + x)^{\sigma - 1}S}{(1 + x)^{2\sigma}} = 0\]
    \(\Rightarrow f = \const, f(0) = 1\Rightarrow f\equiv 1\)
\end{proof}

\begin{corollary}
    \[\arcsin x = \sum {}^{\textcolor{red}{**}} \frac{(2n - 1)!!}{(2n)!!} \frac{x^{2n + 1}}{2n + 1},\ x\in( - 1, 1)\]
    \[(\arcsin x)' = \frac{1}{\sqrt{1 - x^2}} = \sum_{n = 0}^{ +\infty} \binom{\sigma}{n} ( - x^2)^n\Big|_{\sigma =- \frac{1}{2}} = \sum {}^{\textcolor{red}{*}} \frac{(2n - 1)!!}{(2n)!!}x^{2n}\]
    При \(n = 0\) \textcolor{red}{*} это \(1\), и тогда \textcolor{red}{**}: \(\arcsin x = x + \dots \)
\end{corollary}

\begin{corollary}
    \[\sum_{n = m}^{ +\infty} n(n - 1)\dots (n - m + 1)t^{n - m} = \frac{m!}{(1 - t)^{m + 1}},\ |t|< 1\]
\end{corollary}
\begin{proof}
    \[\sum_{n = 0}^{ +\infty} t^n = \frac{1}{1 - t}\]
    Дифференцируем \(m\) раз, получим искомое. Слагаемые с \(n < m\) пропадут, т.к. они \( = 0\)
\end{proof}

\begin{theorem}
    \(f\in C^{\infty} (x_0 - h, x_0 + h)\)

    Тогда \(f\) --- раскладывается в ряд Тейлора в окрестности \(x_0\) \(\iff\)
    \[\exists \delta, C, A > 0 \ \ \forall n \ \ \forall x : |x - x_0| < \delta \ \ |f^{(n)}(x)| < C A^n n!\]
\end{theorem}

\begin{remark}
    В ``Кошмарном сне'' \textit{(см. лекцию 12)} \(f^{(n)} \approx n! 2n! \Rightarrow f\) не раскладывается.
\end{remark}

\begin{proof}\itemfix
    \begin{enumerate}
        \item [ \( \Leftarrow \) ] формула Тейлора в \(x_0\) : \(f(x) = \sum\limits_{k = 0}^{n - 1} \frac{f^{(k)}(x_0)}{k!}(x - x_0)^k + \frac{f^{(n)}(c)}{n!} (x - x_0)^n\)

              \[\left|\frac{f^{(n)}(c)}{n!} (x - x_0)^n\right| \leq C|A(x - x_0)^n| \xrightarrow{n\to +\infty} 0\]

              Разложение имеет место при \(|x - x_0|<\min(\delta, \frac{1}{A})\)
        \item [ \( \Rightarrow \) ] \(f(x) = \sum \frac{f^{(n)}(x_0)}{n!}\)

              Возьмём \(x_1\neq x_0\), для которого это верно

              \begin{enumerate}
                  \item при \(x = x_1\), ряд сходится \(\Rightarrow\) слагаемые \(\to 0\) \(\Rightarrow\) огр.

                        \[\left|\frac{f^{(n)}(x_0)}{n!}(x_1 - x_0)^n \right| \leq C_1 \Leftrightarrow |f^{(n)}(x_0)| \leq C_1 n! B^n\]
                        , где \(B = \frac{1}{|x_1 - x_0|}\)
                  \item \begin{align*}
                            f^{(m)}(x_1) & = \sum \frac{f^{(n)}(x_0)}{n!}n(n - 1)\dots (n - m + 1)x^{n - m}          \\
                                         & = \sum_{n = m}^{ +\infty} \frac{f^{(n)}(x_0)}{(n - m)!} (x - x_0)^{n - m}
                        \end{align*}

                        Пусть \(|x - x_0|< \frac{1}{2B}\)

                        \begin{align}
                            |f^{(m)}(x)| & \leq \sum \left|\frac{f^{(n)}(x_0)}{(n - m)!} |x - x_0|^{n - m}\right|  \nonumber                 \\
                                         & \leq \sum \frac{C_1 n! B^n}{(n - m)!} |x - x_0|^{n - m}       \nonumber                           \\
                                         & = C_1 B^m \sum \frac{n!}{(n - m)!} (\underbrace{B|x - x_0|}_{\leq \frac{1}{2}})^{n - m} \nonumber \\
                                         & = \frac{C_1B^m m!}{(\underbrace{1 - B|x - x_0|}_{ > \frac{1}{2}})^{m + 1}}  \label{по сл2}        \\
                                         & < C_1 2^{m + 1} B^m m!           \nonumber                                                        \\
                                         & = \underbrace{(2C_1)}_{C}(\underbrace{2B}_{A})^m m!                        \nonumber
                        \end{align}

                        \ref{по сл2}: по следствию 2.

                        Эта оценка выполняется при \(|x - x_0|< \delta = \frac{1}{2B}\)
              \end{enumerate}
    \end{enumerate}
\end{proof}

\end{document}