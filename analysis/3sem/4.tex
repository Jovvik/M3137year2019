\documentclass[12pt, a4paper]{article}

\usepackage{lastpage}
\usepackage{mathtools}
\usepackage{xltxtra}
\usepackage{libertine}
\usepackage{amsmath}
\usepackage{amsthm}
\usepackage{amsfonts}
\usepackage{amssymb}
\usepackage{enumitem}
\usepackage{xcolor}
\usepackage[left=2.3cm, right=2.3cm, top=2.7cm, bottom=2.7cm, bindingoffset=0cm, headheight=15pt]{geometry}
\usepackage{fancyhdr}
\usepackage[russian]{babel}
% \usepackage{parindent}

\pagestyle{fancy}
\lfoot{M3137y2019}
\rhead{\thepage\ из \pageref{LastPage}}

\newcommand{\R}{\mathbb{R}}
\newcommand{\Q}{\mathbb{Q}}
\newcommand{\C}{\mathbb{C}}
\newcommand{\Z}{\mathbb{Z}}
\newcommand{\B}{\mathbb{B}}
\newcommand{\N}{\mathbb{N}}

\DeclareMathOperator*{\xor}{\oplus}
\DeclareMathOperator*{\equ}{\sim}
\DeclareMathOperator{\Ln}{\text{Ln}}
\DeclareMathOperator{\sign}{\text{sign}}
\DeclareMathOperator{\Sym}{\text{Sym}}
\DeclareMathOperator{\Asym}{\text{Asym}}
% \DeclareMathOperator{\sh}{\text{sh}}
% \DeclareMathOperator{\tg}{\text{tg}}
% \DeclareMathOperator{\arctg}{\text{arctg}}
% \DeclareMathOperator{\ch}{\text{ch}}

\DeclarePairedDelimiter{\ceil}{\lceil}{\rceil}

\setmainfont{Linux Libertine}

\theoremstyle{plain}
\newtheorem{theorem}{Теорема}
\newtheorem{axiom}{Аксиома}
\newtheorem{lemma}{Лемма}

\theoremstyle{remark}
\newtheorem*{remark}{Примечание}
\newtheorem*{exercise}{Упражнение}
\newtheorem*{consequence}{Следствие}
\newtheorem*{example}{Пример}
\newtheorem*{observation}{Наблюдение}

\theoremstyle{definition}
\newtheorem*{definition}{Определение}
\newtheorem*{obozn}{Обозначение}

\lhead{Математический анализ}
\cfoot{}
\rfoot{28.9.2020}

\begin{document}

$\sphericalangle \R^{m+n}, (x, y) \in\R^{m+n}, x=\begin{pmatrix}
        x_1 & \ldots & x_m
    \end{pmatrix}, y = \begin{pmatrix}
        y_1 & \ldots & y_n
    \end{pmatrix}$

$F : O\subset\R^{m+n} \to\R^m$

$$F' = \begin{pmatrix}
        \frac{\partial F_1}{\partial x_1} & \ldots & \frac{\partial F_1}{\partial x_m} & \frac{\partial F_1}{\partial y_1} & \ldots \frac{\partial F_1}{\partial y_n} \\
        \vdots                            &        & \vdots                            & \vdots                            & \vdots                                   \\
        \frac{\partial F_n}{\partial x_1} & \ldots & \frac{\partial F_n}{\partial x_m} & \frac{\partial F_n}{\partial y_1} & \ldots \frac{\partial F_n}{\partial y_n} \\
    \end{pmatrix}$$

\begin{theorem}[о неявном отображении]\itemfix
    \begin{itemize}
        \item $F: O\subset\R^{m+n}\to\R^n$
        \item $O$ откр.
        \item $F\in C^r (O, \R^n)$
        \item $(a, b)\in O$
        \item $F(a, b) = 0$
        \item $\det F'_y(a, b)\not=0$
    \end{itemize}
    Тогда:
    \begin{enumerate}
        \item $\exists$ откр. $P\subset\R^{m}, a\in P$

              $\exists$ откр. $Q\subset\R^{n}, b\in Q$

              $\exists!\ \Phi : P\to Q\in C^r : \forall x \in P \ \ F(x, \Phi(x)) = 0$
        \item $\Phi'(x) = -\left(F'_y(x, \Phi(x))\right)^{-1}\cdot F'_x(x, \Phi(x))$
    \end{enumerate}
\end{theorem}
\begin{proof}\itemfix
    \begin{itemize}
        \item [$1 \Rightarrow 2$:] $F(x, \Phi(x)) = 0 \Rightarrow F'_x(x, \Phi(x)) + F'_y(x, \Phi(x))\Phi'(x) = 0$
        \item [$1$:] $\tilde F: O\to\R^{m+n} : (x, y)\mapsto(x, F(x, y)), \tilde F(a, b) = (a, 0)$
              $$F' = \left(\begin{array}{c|c}
                          E_m  & 0    \\
                          \hline
                          F'_x & F'_y
                      \end{array}\right)$$

              Очевидно $\det \tilde F'\not=0$ в $(a, b)$, значит $\exists U(a, b) : \tilde F\Big|_{U}$ --- диффеоморфизм:
              \begin{enumerate}
                  \item $U = P_1 \times Q$
                  \item $V = \tilde F(U)$
                  \item $\tilde F$ --- диффеоморфизм на $U \Rightarrow \exists \Psi = \tilde F^{-1} : V\to U$
                  \item $\tilde F$ не меняет первые $m$ координат $\Rightarrow \Psi(u, v) = (u, H(u, v)),  H : V\to\R^n$.
                  \item ``Ось $x$'' $\Leftrightarrow$ ``ось $y$'', $P:=``\text{ось } u'' = \R^m\times{a}\cap V$, $P$ --- откр. в $\R^m$, $P=P_1$
                  \item $\Phi(x) := H(x, 0)$

                        $F\in C^r \Rightarrow \tilde F\in C^r \Rightarrow \Psi\in C^r \Rightarrow H\in C^r \Rightarrow \Phi\in C^r$

                        Единственность: $(x, y) = \Psi(\tilde F(x, y)) = \Psi(x, 0) = (x, H(x, 0)) = (x, \Phi(x))$
              \end{enumerate}
    \end{itemize}
\end{proof}

\begin{definition}\itemfix
    \begin{itemize}
        \item $M\subset\R^m$
        \item $x\in\{1 \ldots m\}$
    \end{itemize}
    $M$ --- \textbf{простое $k$-мерное \textit{(непрерывное)} многообразие} в $\R^m$, если оно гомеоморфно некоторому открытому множеству $O\subset\R^m$

    Т.е. $\exists \underbrace{\Phi}_{\text{параметризация}} : \underbrace{O}_{\text{откр.}} \subset\R^k \xrightarrow{\text{сюрьекция}} M$ --- непр., обратимо и $\Phi^{-1}$ непрерывно.
\end{definition}

\begin{definition}
    $M\subset\R^m$ --- \textbf{простое $k$-мерное $C^r$-гладкое многообразие} в $\R^m$, если

    $$\exists \Phi : O\subset\R^k \to\R^m, \Phi(O) = M, \Phi\in C^r \ \ \forall x\in O \ \ \rg \Phi'(x)=k$$
\end{definition}

\begin{example}\itemfix
    \begin{enumerate}
        \item Полусфера в $\R^3 = \{(x, y, z)\in\R^3 : r = 0, x^2 + y^2 + z^2 = r^2\}$

              $\Phi : (x, y)\mapsto (x, y, \sqrt{r^2 - x^2 - y^2})$

              $\Phi : B(0, r)\subset\R^2 \to\R^3$

              $\Phi\in C^{\infty}$

              $\Phi' = \begin{pmatrix}
                      1                             & 0                             \\
                      0                             & 1                             \\
                      \frac{-x}{\sqrt{r^2-x^2-y^2}} & \frac{-y}{\sqrt{r^2-x^2-y^2}}
                  \end{pmatrix}, \rg \Phi' = 2$
        \item Цилиндр $= \{(x, y, z)\in\R^3 : x^2 + y^2 = r^2, z\in(a, b)\}$

              $\Phi : [0, 2\pi]\times (0, h) \to\R^3$


              $(\varphi, z)\mapsto (r\cos \varphi, r\sin \varphi, z)$ --- не иньективно.

              Не существует $\Phi : \underbrace{O}_{\text{односвязн.}}\subset\R^2 \to$ цилиндр $\subset\R^3$, потому что топология: в цилиндре есть дырка, в $O$ --- нет.

              Если мы допускаем дырку в $O$, то $(x, y)\mapsto \left(\frac{rx}{\sqrt{x^2+y^2}}, \frac{ry}{\sqrt{x^2+y^2}}, \sqrt{x^2+y^2} - 1\right)$ --- параметризация.
        \item Сфера в $\R^3$ без точки
        
        $\Phi : (0, 2\pi)\times\left(-\frac{\pi}{2}, \frac{\pi}{2}\right) \to R^{3}$

        $(\varphi, \psi) \mapsto \begin{pmatrix}
            R\cos \varphi \cos \psi \\
            R\sin \varphi \cos \psi \\
            R\sin \psi
        \end{pmatrix}$
        
    \end{enumerate}
\end{example}

\end{document}