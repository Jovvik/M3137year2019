\documentclass[12pt, a4paper]{article}

\usepackage{lastpage}
\usepackage{mathtools}
\usepackage{xltxtra}
\usepackage{libertine}
\usepackage{amsmath}
\usepackage{amsthm}
\usepackage{amsfonts}
\usepackage{amssymb}
\usepackage{enumitem}
\usepackage{xcolor}
\usepackage[left=2.3cm, right=2.3cm, top=2.7cm, bottom=2.7cm, bindingoffset=0cm, headheight=15pt]{geometry}
\usepackage{fancyhdr}
\usepackage[russian]{babel}
% \usepackage{parindent}

\pagestyle{fancy}
\lfoot{M3137y2019}
\rhead{\thepage\ из \pageref{LastPage}}

\newcommand{\R}{\mathbb{R}}
\newcommand{\Q}{\mathbb{Q}}
\newcommand{\C}{\mathbb{C}}
\newcommand{\Z}{\mathbb{Z}}
\newcommand{\B}{\mathbb{B}}
\newcommand{\N}{\mathbb{N}}

\DeclareMathOperator*{\xor}{\oplus}
\DeclareMathOperator*{\equ}{\sim}
\DeclareMathOperator{\Ln}{\text{Ln}}
\DeclareMathOperator{\sign}{\text{sign}}
\DeclareMathOperator{\Sym}{\text{Sym}}
\DeclareMathOperator{\Asym}{\text{Asym}}
% \DeclareMathOperator{\sh}{\text{sh}}
% \DeclareMathOperator{\tg}{\text{tg}}
% \DeclareMathOperator{\arctg}{\text{arctg}}
% \DeclareMathOperator{\ch}{\text{ch}}

\DeclarePairedDelimiter{\ceil}{\lceil}{\rceil}

\setmainfont{Linux Libertine}

\theoremstyle{plain}
\newtheorem{theorem}{Теорема}
\newtheorem{axiom}{Аксиома}
\newtheorem{lemma}{Лемма}

\theoremstyle{remark}
\newtheorem*{remark}{Примечание}
\newtheorem*{exercise}{Упражнение}
\newtheorem*{consequence}{Следствие}
\newtheorem*{example}{Пример}
\newtheorem*{observation}{Наблюдение}

\theoremstyle{definition}
\newtheorem*{definition}{Определение}
\newtheorem*{obozn}{Обозначение}

\lhead{Математический анализ}
\cfoot{}
\rfoot{26.10.2020}

\begin{document}

\subsection*{Функциональные последовательности и ряды}

\begin{example}
    $\sum x^n, x\in (0, 1)$ --- нет равномерной сходимости

    $\exists \varepsilon = 0.1 \ \ \forall N \ \ \exists n>N$ --- подходит любое $>100$ $\exists p = 1 \ \ \exists x = 1 - \frac{1}{n+1} : |u_{n+1}(x)| \ge \varepsilon$, т.е. $\left(1 - \frac{1}{n+1}\right)^{n+1} \approx \frac{1}{e} > \frac{1}{10}$
\end{example}

\begin{theorem}[признак Вейерштрасса]\itemfix
    \begin{itemize}
        \item $\sum u_n(x)$
        \item $x\in X$
    \end{itemize}
    Пусть $\exists c_n$ --- вещественная:
    \begin{itemize}
        \item $|u_n(x)| \le c_n$ при $x\in E$
        \item $\sum c_n$ --- сходится
    \end{itemize}
    Тогда $\sum u_n(x)$ равномерно сходится на $E$
\end{theorem}
\begin{proof}
    $|u_{n+1}(x) + \ldots + u_{n+p}(x)| \le c_{n+1} + \ldots + c_{n+p}$ --- тривиально

    $\sum c_n$ --- сх. $\Rightarrow$ $c_n$ удовлетворяет критерию Больцано-Коши : $$\forall \varepsilon > 0 \ \ \exists N \ \ \forall n > N \ \ \forall p\in \N \ \ \forall x\in E \ \ c_{n+1} + \ldots c_{n+p} < \varepsilon$$

    Тогда $\sum u_n(x)$ удовлетворяет критерию Больцано-Коши равномерной сходимости.
\end{proof}

\begin{example}
    $\sum\limits_{n=1}^{+\infty} \frac{x}{1+n^2x^2}, x\in\R$. Попытаемся применить признак.

    $c_n := \sup\limits_{x\in \R} \left|\frac{x}{1+n^2x^2}\right|$ --- это минимальное возможное $c_n$, если для него не сработает признак, до ни для какого $c_n$ не сработает.

    $\sup$ достигается в точке $x_0 = \frac{1}{n}$, $\sup = \frac{1}{2n}$. $\sum \frac{1}{2n}$ расходится $\Rightarrow$ признак не сработал.

    Построим отрицание критерия Больцано-Коши:
    $$\exists \varepsilon = \frac{1}{6} \ \ \forall N \ \ \exists n>N \ \ p=n\in\N \ \ \exists x=\frac{1}{n} \ \ |u_{n+1}(x) + u_{2n}(x)| = \frac{\frac{1}{n}}{1+(n+1)^2\frac{1}{n^2}} + \ldots \frac{\frac{1}{n}}{1+(2n)^2\frac{1}{n^2}} \ge$$
    $$\ge n \frac{\frac{1}{n}}{1+(2n)^2\frac{1}{n^2}} = \frac{1}{5} > \frac{1}{6} = \varepsilon$$
\end{example}

\begin{example}
    $\sum \frac{x}{1+x^2n^2}, x\in\left(\frac{1}{2020}, 2020\right)$

    $$c_n := \sup \frac{x}{1+x^2n^2} \le \frac{2020}{1+\frac{1}{2020^2}n^2} \equ_{n\to+\infty} \frac{\?}{n^2}$$

    $\sum c_n$ сходится $\Rightarrow$ есть равномерная сходимость.
\end{example}

\subsection*{Приложения равномерной сходимости для рядов}

\begin{manualtheorem}{1'}[Стокса-Зайдля для рядов]\itemfix
    \begin{itemize}
        \item $u_n : X\to Y$
        \item $X$ --- метрическое пространство
        \item $Y$ --- нормированное пространство
        \item $x_0\in X$
        \item $u_n$ --- непрерывно в $x_0$
        \item $\sum u_n(x)$ \textbf{равномерно} сходится на $X$
        \item $S(x) := \sum u_n(x)$
    \end{itemize}
    Тогда $S(x)$ --- непрерывно в $x_0$.
\end{manualtheorem}
\begin{proof}
    По теореме 1 $S_n(x) \rightrightarrows S(x), S_n(x)$ --- непр. в $x_0 \xRightarrow{\text{т. 1}} S(x)$ непр. в $x_0$
\end{proof}

\begin{remark}
    Достаточно равномерной сходимости $u_n(x)$ на некоторой окрестности $x_0$
\end{remark}
\begin{remark}
    $u_n \in C(x), \sum u_n$ --- равномерно сходится на $X \Rightarrow S(x) \in C(x)$
\end{remark}

\begin{manualtheorem}{2'}{О почленном интегрировании ряда}\itemfix
    \begin{itemize}
        \item $u_n : [a, b] \to\R$
        \item $u_n$ --- непр. на $[a, b]$
        \item $\sum\limits_{n=0}^{+\infty} u_n(x)$ \textbf{равномерно} сходится на $[a, b]$
        \item $S(x) = \sum u_n(x)$
    \end{itemize}
    Тогда $\int_a^b S(x)dx = \sum_{n=0}^{+\infty} \int_a^b u_n(x)dx$

    Можно интегрировать, т.к. $S(x)$ --- непр. на $[a, b]$ по теореме 1'
\end{manualtheorem}
\begin{proof}
    По теореме 2

    $S_n\xrightrightarrows{[a, b]} S$

    По теореме 2 $\int_a^b S_n(x) dx \to \int_a^b S(x)dx$

    $\int_a^b \sum\limits_{k=0}^{n} u_k(x) dx = \sum\limits_{k=0}^{n} \int_a^b u_k(x) dx \to \sum\limits_{k=0}^{n} \?$
\end{proof}

\begin{example}
    $\sum\limits_{n=0}^{+\infty} (-1)^n x^n$ --- равномерно сходится при $|x| \le q < 1$ по Вейерштрассу: $|(-1)^n x^n|\le q^n, \sum q^n$ сходится.

    Проинтегрируем от $0$ до $t$ ($|t|\le q$)

    $$\sum_{n=0}^{+\infty} (-1)^n x^n = \frac{1}{1+x}$$
    $$\ln(1+t) = \sum_{n=0}^{+\infty} (-1)^n \frac{t^{n+1}}{n+1} = \sum_{k=1}^{+\infty} (-1)^{k+1} \frac{t^k}{k}$$
    Это верно при $t\in[-q, q]\ \ \forall q : 0<q<1$, т.е. верно при $t\in(-1, 1)$

    При $t=-1$ $\sum -\frac{1}{k}$ расходится

    При $t\to 1$ ряд $\sum (-1)^{k+1} \frac{t^k}{k}$ равномерно сходится на $[0, 1]$, слагаемые непрерывны в $t_0=1 \xRightarrow{\text{т. 1}}$ сумма ряда непрерывна в точке $t_0=1 \Rightarrow \ln 2 = \sum\limits_{k=1}^{+\infty} \frac{(-1)^{k+1}}{k}$
\end{example}

\end{document}