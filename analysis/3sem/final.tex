\documentclass[12pt, a4paper]{article}

\usepackage{lastpage}
\usepackage{mathtools}
\usepackage{xltxtra}
\usepackage{libertine}
\usepackage{amsmath}
\usepackage{amsthm}
\usepackage{amsfonts}
\usepackage{amssymb}
\usepackage{enumitem}
\usepackage{xcolor}
\usepackage[left=2.3cm, right=2.3cm, top=2.7cm, bottom=2.7cm, bindingoffset=0cm, headheight=15pt]{geometry}
\usepackage{fancyhdr}
\usepackage[russian]{babel}
% \usepackage{parindent}

\pagestyle{fancy}
\lfoot{M3137y2019}
\rhead{\thepage\ из \pageref{LastPage}}

\newcommand{\R}{\mathbb{R}}
\newcommand{\Q}{\mathbb{Q}}
\newcommand{\C}{\mathbb{C}}
\newcommand{\Z}{\mathbb{Z}}
\newcommand{\B}{\mathbb{B}}
\newcommand{\N}{\mathbb{N}}

\DeclareMathOperator*{\xor}{\oplus}
\DeclareMathOperator*{\equ}{\sim}
\DeclareMathOperator{\Ln}{\text{Ln}}
\DeclareMathOperator{\sign}{\text{sign}}
\DeclareMathOperator{\Sym}{\text{Sym}}
\DeclareMathOperator{\Asym}{\text{Asym}}
% \DeclareMathOperator{\sh}{\text{sh}}
% \DeclareMathOperator{\tg}{\text{tg}}
% \DeclareMathOperator{\arctg}{\text{arctg}}
% \DeclareMathOperator{\ch}{\text{ch}}

\DeclarePairedDelimiter{\ceil}{\lceil}{\rceil}

\setmainfont{Linux Libertine}

\theoremstyle{plain}
\newtheorem{theorem}{Теорема}
\newtheorem{axiom}{Аксиома}
\newtheorem{lemma}{Лемма}

\theoremstyle{remark}
\newtheorem*{remark}{Примечание}
\newtheorem*{exercise}{Упражнение}
\newtheorem*{consequence}{Следствие}
\newtheorem*{example}{Пример}
\newtheorem*{observation}{Наблюдение}

\theoremstyle{definition}
\newtheorem*{definition}{Определение}
\newtheorem*{obozn}{Обозначение}

\usepackage{sectsty}

\allsectionsfont{\raggedright}
\subsectionfont{\fontsize{14}{15}\selectfont}

\lhead{Итоговый конспект}
\cfoot{}
\rfoot{}

\settoggle{useproofs}{true}

\begin{document}

\section{Определения}

\import{Мультииндекс и обозначения с ним}{1.tex}{мультииндекс}

\import{\teormin Формула Тейлора (различные виды записи)}{1.tex}{формулатейлорадифференциал}
\get{1.tex}{формулатейлорамультииндекс}

\import{$n$-й дифференциал}{1.tex}{дифференциал}

\import{\teormin Норма линейного оператора}{1.tex}{нормалоп}

\import{Положительно-, отрицательно-, незнако- определенная квадратичная форма}{2.tex}{формы}

\import{Локальный максимум, минимум, экстремум}{2.tex}{экстремум}

\import{Диффеоморфизм}{3.tex}{диффеоморфизм}

\subsection{Формулировка теоремы о локальной обратимости}
\get{3.tex}{олокальнойобратимости}

\subsection{Формулировка теоремы о локальной обратимости в терминах систем уравнений}
\get{3.tex}{олокальнойобратимостисистема}

\import{Формулировка теоремы о неявном отображении в терминах систем уравнений}{4.tex}{}

\import{\teormin Простое $k$-мерное гладкое многообразие в $\R^m$}{4.tex}{простоеkмерноегладкоемногообразие}

\section{Теоремы}

\import{Лемма о дифференцировании ``сдвига''}{1.tex}{леммаодиффсдвига}

\import{\teormin Многомерная формула Тейлора (с остатком в форме Лагранжа и Пеано)}{1.tex}{}
\subsubsection{В форме Лагранжа}
\getwithproof{1.tex}{тейлорлагранж}
\subsubsection{В форме Пеано}
\getwithproof{1.tex}{тейлорпеано}

\import{Теорема о пространстве линейных отображений}{1.tex}{опространствелинейныхотображений}

\import{Лемма об условиях, эквивалентных непрерывности линейного оператора}{1.tex}{эквивалентностьнепрерывности}

\import{Теорема Лагранжа для отображений}{2.tex}{лагранжадляотображений}

\import{Теорема об обратимости линейного отображения, близкого к обратимому}{2.tex}{обобратимости}

\import{Теорема о непрерывно дифференцируемых отображениях}{2.tex}{онепрдифф}

\import{Теорема Ферма. Необходимое условие экстремума. Теорема Ролля}{2.tex}{ферма}
\get{2.tex}{необходимоеусловиеэкстремума}
\getwithproof{2.tex}{ролля}

\import{Лемма об оценке квадратичной формы и об эквивалентных нормах}{2.tex}{леммаобоценкенормы}

\import{Достаточное условие экстремума}{2.tex}{достаточноеусловиеэкстремума}

\import{Лемма о ``почти локальной инъективности''}{3.tex}{опочтилокальнойиньективности}

\import{Теорема о сохранении области}{3.tex}{осохраненииобласти}

\import{Следствие о сохранении области для отображений в пространство меньшей размерности}{3.tex}{следствиеосохраненииобласти}

\import{Теорема о гладкости обратного отображения}{3.tex}{огладкостиобратногоотображения}

\import{Теорема о неявном отображении}{4.tex}{онеявномотображении}

\import{Теорема о задании гладкого многообразия системой уравнений}{4.tex}{озаданиигладкогомногообразиясистемойуравнений}

\import{Следствие о двух параметризациях}{4.tex}{одвухпараметризациях}


\end{document}