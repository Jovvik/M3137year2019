\documentclass[12pt, a4paper]{article}

\usepackage{lastpage}
\usepackage{mathtools}
\usepackage{xltxtra}
\usepackage{libertine}
\usepackage{amsmath}
\usepackage{amsthm}
\usepackage{amsfonts}
\usepackage{amssymb}
\usepackage{enumitem}
\usepackage{xcolor}
\usepackage[left=2.3cm, right=2.3cm, top=2.7cm, bottom=2.7cm, bindingoffset=0cm, headheight=15pt]{geometry}
\usepackage{fancyhdr}
\usepackage[russian]{babel}
% \usepackage{parindent}

\pagestyle{fancy}
\lfoot{M3137y2019}
\rhead{\thepage\ из \pageref{LastPage}}

\newcommand{\R}{\mathbb{R}}
\newcommand{\Q}{\mathbb{Q}}
\newcommand{\C}{\mathbb{C}}
\newcommand{\Z}{\mathbb{Z}}
\newcommand{\B}{\mathbb{B}}
\newcommand{\N}{\mathbb{N}}

\DeclareMathOperator*{\xor}{\oplus}
\DeclareMathOperator*{\equ}{\sim}
\DeclareMathOperator{\Ln}{\text{Ln}}
\DeclareMathOperator{\sign}{\text{sign}}
\DeclareMathOperator{\Sym}{\text{Sym}}
\DeclareMathOperator{\Asym}{\text{Asym}}
% \DeclareMathOperator{\sh}{\text{sh}}
% \DeclareMathOperator{\tg}{\text{tg}}
% \DeclareMathOperator{\arctg}{\text{arctg}}
% \DeclareMathOperator{\ch}{\text{ch}}

\DeclarePairedDelimiter{\ceil}{\lceil}{\rceil}

\setmainfont{Linux Libertine}

\theoremstyle{plain}
\newtheorem{theorem}{Теорема}
\newtheorem{axiom}{Аксиома}
\newtheorem{lemma}{Лемма}

\theoremstyle{remark}
\newtheorem*{remark}{Примечание}
\newtheorem*{exercise}{Упражнение}
\newtheorem*{consequence}{Следствие}
\newtheorem*{example}{Пример}
\newtheorem*{observation}{Наблюдение}

\theoremstyle{definition}
\newtheorem*{definition}{Определение}
\newtheorem*{obozn}{Обозначение}

\lhead{Математический анализ}
\cfoot{}
\rfoot{1.3.2021}

\begin{document}

\begin{theorem}[об абсолютной непрерывности интеграла]\itemfix
    \begin{itemize}
        \item \((X, \mathfrak{A}, \mu)\) --- пространство с мерой
        \item \(f : X \to \overline \R\)
        \item \(f\) суммируемо
    \end{itemize}

    Тогда \(\forall \varepsilon > 0 \ \ \exists \delta > 0 \ \ \forall E \text{ --- изм., } \mu E < \delta : \left|\int_E f\right|< \varepsilon\)
\end{theorem}
\begin{corollary}
    \(f\) суммируемо на \(X\), \(E_n \subset X\), тогда \(\mu E_n \to 0 \Rightarrow \int_{E_n} f \to 0\)
\end{corollary}
\begin{proof}\footnote{Теоремы, не следствия}
    \[X_n : = X(|f| \geq n)\]
    \[X_n \supset X_{n+1} \supset \dots \Rightarrow \mu\left( \bigcap X_n \right) \symrefeq{почти везде конечна} 0\]
    \begin{equation}
        \forall \varepsilon > 0 \ \ \exists n_\varepsilon \ \ \int_{X_{n_\varepsilon}} |f| < \frac{\varepsilon}{2} \label{непрерывность сверху меры}
    \end{equation}

    Пусть \(\delta : = \frac{\varepsilon}{2n_\varepsilon} \). Тогда при \(\mu E < \delta\):
    \[\left|\int_E f\right| \leq \int_E |f| \symrefeq{оценка f} \int_{E\cap X_{n_\varepsilon}} |f| + \int_{E\cap X_{n_\varepsilon}^c} |f| \leq \int_{X_{n_\varepsilon}} |f| + \int_{E\cap X_{n_\varepsilon}^c} n_\varepsilon < \frac{\varepsilon}{2} + \underbrace{\mu E}_{\delta} \cdot n_\varepsilon \leq \varepsilon\]

    \begin{itemize}
        \item \eqref{почти везде конечна}: Т.к. \(f\) на \(\bigcap X_n\) бесконечна и \(f\) почти везде конечна.
        \item \eqref{непрерывность сверху меры}: По непрерывности сверху меры \(A \mapsto \int_A |f| d\mu\)
        \item \eqref{оценка f}: Т.к. \(|f|\) на \(E \cap X_{n_\varepsilon}^c\) не превосходит \(n_\varepsilon\) по построению \(X_{n_\varepsilon}\)
    \end{itemize}
\end{proof}

\begin{remark}
    Следующие два свойства не эквивалентны:
    \begin{enumerate}
        \item \(f_n \xRightarrow[\mu]{} f \xLeftrightarrow{def} \forall \varepsilon > 0 \ \ \mu X(|f_n - f| > \varepsilon) \to 0\)
        \item \(\int_X |f_n - f| d\mu \to 0\)
    \end{enumerate}

    Из 1 не следует 2: пусть \((X, \mathfrak{A}, \mu) = (\R, \mathfrak{M}, \lambda), f_n = \frac{1}{nx}\). Тогда \(f_n \xRightarrow{\lambda} 0\), но \(\int |f_n - f| = +\infty\) при всех \(n\).

    Из 2 следует 1, т.к.
    \[\mu \underbrace{X(|f_n - f|> \varepsilon)}_{X_n} = \int_{X_n} 1 \leq \int_{X_n} \frac{|f_n - f|}{\varepsilon} = \frac{1}{\varepsilon} \int_{X_n} |f_n - f| \leq \frac{1}{\varepsilon} \int_X |f_n - f| \xrightarrow{n \to +\infty} 0\]
\end{remark}

\begin{theorem}[Лебега о предельном переходе под знаком интеграла]\itemfix
    \begin{itemize}
        \item \((X, \mathfrak{A}, \mu)\) --- пространство с мерой
        \item \(f_n, f\) --- измеримо и почти везде конечно
        \item \(f_n \xRightarrow{\mu} f\)
        \item \(\exists g\), называемое ``суммируемая мажоранта'':
              \begin{enumerate}
                  \item \(\forall n \ \ |f_n| \symref{``1''}{\leq} g\) почти везде
                  \item \(g\) --- суммируемо на \(X\)
              \end{enumerate}
    \end{itemize}

    Тогда: \(f_n, f\) --- суммируемы и \(\int_X |f_n - f| d\mu \xrightarrow{n \to +\infty} 0\), и тем более \(\int_X f_n d\mu \to \int_X f d\mu\)
\end{theorem}
\begin{remark}
    Почти везде конечность \(f_n\) и \(f\) следует из \eqref{``1''}, поэтому в условии этого можно не требовать.
\end{remark}
\begin{proof}
    \(f_n\) --- суммируемы в силу неравенства \eqref{``1''}, \(f\) суммируемо в силу следствия теоремы Рисса, тем более \(|\int_X f_n - \int_X f| \leq  \int_X |f_n - f| \to 0\)

    \begin{enumerate}
        \item \(\mu X < +\infty\)

              Зафиксируем \(\varepsilon\). \(X_n : = X(|f_n - f| > \varepsilon)\)

              \(f_n \Rightarrow f\), т.е. \(\mu X_n \to 0\)

              \[|f_n - f| \leq |f_n| + |f| \leq 2g\]
              \[\int_X |f_n - f| = \int_{X_n} + \int_{X_n^c} = \underbrace{\int_{X_n} 2g}_{\xrightarrow[\text{сл. т. об абс. непр.}]{n \to +\infty} 0} + \int_{X_n^c} \varepsilon d\mu < \varepsilon + \varepsilon \mu X\]

        \item \(\mu X = +\infty\)

              Утверждение: \(\forall \varepsilon > 0 \ \ \exists A \subset X, \text{изм., конечной меры}, \mu A \text{ конечно : } \int_{X\setminus A} g < \varepsilon\). Докажем его.

              \[\int_X g = \sup \left\{\int g_n, 0 \leq g_n \leq g, g_n \text{ --- ступ.}\right\} \]
              \[A : = \{x : g_n(x) > 0\}\]
              \[0 \leq \int_X g - \int_X g_n = \int_A g - g_n + \int_{X\setminus A} g < \varepsilon\]
              \[\int_X |f_n - f| d\mu = \int_A + \int_{X\setminus A} \leq \underbrace{\int_A |f_n - f|}_{\substack{ \to 0 \\ \text{по случаю 1}}} + \underbrace{\int_{X\setminus A} 2g}_{ < 2\varepsilon} < 3 \varepsilon\]
    \end{enumerate}
\end{proof}

\begin{theorem}[Лебега]\itemfix
    \begin{itemize}
        \item \((X, \mathfrak{A}, \mu)\) --- пространство с мерой
        \item \(f_n, f\) --- измеримо
        \item \(f_n \to f\) почти везде
        \item \(\exists g\), называемое ``суммируемая мажоранта'':
              \begin{enumerate}
                  \item \(\forall n \ \ |f_n| \leq g\) почти везде
                  \item \(g\) --- суммируемо на \(X\)
              \end{enumerate}
    \end{itemize}

    Тогда \(f_n, f\) --- суммируемы, \(\int_X |f_n - f| d\mu \to 0\)
\end{theorem}

\textcolor{red}{Не дописано}

\end{document}