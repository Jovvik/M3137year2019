\documentclass[12pt, a4paper]{article}

\usepackage{lastpage}
\usepackage{mathtools}
\usepackage{xltxtra}
\usepackage{libertine}
\usepackage{amsmath}
\usepackage{amsthm}
\usepackage{amsfonts}
\usepackage{amssymb}
\usepackage{enumitem}
\usepackage{xcolor}
\usepackage[left=2.3cm, right=2.3cm, top=2.7cm, bottom=2.7cm, bindingoffset=0cm, headheight=15pt]{geometry}
\usepackage{fancyhdr}
\usepackage[russian]{babel}
% \usepackage{parindent}

\pagestyle{fancy}
\lfoot{M3137y2019}
\rhead{\thepage\ из \pageref{LastPage}}

\newcommand{\R}{\mathbb{R}}
\newcommand{\Q}{\mathbb{Q}}
\newcommand{\C}{\mathbb{C}}
\newcommand{\Z}{\mathbb{Z}}
\newcommand{\B}{\mathbb{B}}
\newcommand{\N}{\mathbb{N}}

\DeclareMathOperator*{\xor}{\oplus}
\DeclareMathOperator*{\equ}{\sim}
\DeclareMathOperator{\Ln}{\text{Ln}}
\DeclareMathOperator{\sign}{\text{sign}}
\DeclareMathOperator{\Sym}{\text{Sym}}
\DeclareMathOperator{\Asym}{\text{Asym}}
% \DeclareMathOperator{\sh}{\text{sh}}
% \DeclareMathOperator{\tg}{\text{tg}}
% \DeclareMathOperator{\arctg}{\text{arctg}}
% \DeclareMathOperator{\ch}{\text{ch}}

\DeclarePairedDelimiter{\ceil}{\lceil}{\rceil}

\setmainfont{Linux Libertine}

\theoremstyle{plain}
\newtheorem{theorem}{Теорема}
\newtheorem{axiom}{Аксиома}
\newtheorem{lemma}{Лемма}

\theoremstyle{remark}
\newtheorem*{remark}{Примечание}
\newtheorem*{exercise}{Упражнение}
\newtheorem*{consequence}{Следствие}
\newtheorem*{example}{Пример}
\newtheorem*{observation}{Наблюдение}

\theoremstyle{definition}
\newtheorem*{definition}{Определение}
\newtheorem*{obozn}{Обозначение}

\usepackage{sectsty}

\allsectionsfont{\raggedright}
\subsectionfont{\fontsize{14}{15}\selectfont}

\lhead{Итоговый конспект}
\rfoot{}

\settoggle{useproofs}{true}

\renewcommand{\import}[2]{
    \subsection{#1}
    \getwithproof{main.tex}{#2}
}

\begin{document}

\section{Определения}

\import{Ступенчатая функция}{ступенчатаяфункция}
\label{ступенчатая функция}

\import{Разбиение, допустимое для ступенчатой функции}{}
\given{ступенчатая функция}

\import{\teormin Измеримая функция}{измеримаяфункция}

\import{Свойство, выполняющееся почти везде}{свойствопочтивезде}

\import{Сходимость почти везде}{}

\import{Сходимость по мере}{сходимостьпомере}

\import{Теорема Егорова о сходиомсти почти везде и почти равномерной сходиомсти}{теоремаегорова}

\import{Интеграл ступенчатой функции}{интеграл1}

\import{Интеграл неотрицательной измеримой функции}{интеграл2}

\section{Теоремы}

\import{Лемма ``о структуре компактного оператора''}{оструктурекомпактногооператора}

\import{\teormin Теорема о преобразовании меры Лебега при линейном отображении}{опреобразованиимерылебегаподдействиемлинейногоотображения}

\import{Теорема об измеримости пределов и супремумов}{обизмеримостипределовисупрмемумов}

\import{Характеризация измеримых функций с помощью ступенчатых. Следствия}{характеризацияизмеримыхфункцийспомощьюступенчатых}

\import{Измеримость функции, непрерывной на множестве полной меры}{обизмеримостифункцийнепрерывныхнамножествеполноймеры}

\import{Теорема Лебега о сходимости почти везде и сходимости по мере}{лебега}

\import{Теорема Рисса о сходимости по мере и сходимости почти везде}{риссаproof}

\end{document}