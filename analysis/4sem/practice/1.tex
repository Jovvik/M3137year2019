\documentclass[12pt, a4paper]{article}

\usepackage{lastpage}
\usepackage{mathtools}
\usepackage{xltxtra}
\usepackage{libertine}
\usepackage{amsmath}
\usepackage{amsthm}
\usepackage{amsfonts}
\usepackage{amssymb}
\usepackage{enumitem}
\usepackage{xcolor}
\usepackage[left=2.3cm, right=2.3cm, top=2.7cm, bottom=2.7cm, bindingoffset=0cm, headheight=15pt]{geometry}
\usepackage{fancyhdr}
\usepackage[russian]{babel}
% \usepackage{parindent}

\pagestyle{fancy}
\lfoot{M3137y2019}
\rhead{\thepage\ из \pageref{LastPage}}

\newcommand{\R}{\mathbb{R}}
\newcommand{\Q}{\mathbb{Q}}
\newcommand{\C}{\mathbb{C}}
\newcommand{\Z}{\mathbb{Z}}
\newcommand{\B}{\mathbb{B}}
\newcommand{\N}{\mathbb{N}}

\DeclareMathOperator*{\xor}{\oplus}
\DeclareMathOperator*{\equ}{\sim}
\DeclareMathOperator{\Ln}{\text{Ln}}
\DeclareMathOperator{\sign}{\text{sign}}
\DeclareMathOperator{\Sym}{\text{Sym}}
\DeclareMathOperator{\Asym}{\text{Asym}}
% \DeclareMathOperator{\sh}{\text{sh}}
% \DeclareMathOperator{\tg}{\text{tg}}
% \DeclareMathOperator{\arctg}{\text{arctg}}
% \DeclareMathOperator{\ch}{\text{ch}}

\DeclarePairedDelimiter{\ceil}{\lceil}{\rceil}

\setmainfont{Linux Libertine}

\theoremstyle{plain}
\newtheorem{theorem}{Теорема}
\newtheorem{axiom}{Аксиома}
\newtheorem{lemma}{Лемма}

\theoremstyle{remark}
\newtheorem*{remark}{Примечание}
\newtheorem*{exercise}{Упражнение}
\newtheorem*{consequence}{Следствие}
\newtheorem*{example}{Пример}
\newtheorem*{observation}{Наблюдение}

\theoremstyle{definition}
\newtheorem*{definition}{Определение}
\newtheorem*{obozn}{Обозначение}

\lhead{Математический анализ \textit{(практика)}}
\cfoot{}
\rfoot{11.2.2021}

\begin{document}

Интеграл второго рода векторного поля по пути \(\gamma : [a, b] \to \R^3\), \(\gamma(t) = (x(t), y(t), z(t))\) обозначается следующим образом:
\[\int_\gamma P dx + Q dy + R dz\]
и вычисляется путём подстановки:
\[\int_a^b (P(x(t), y(t), z(t))x'(t) + Q(x(t), y(t), z(t))y'(t) + R(x(t), y(t), z(t))z'(t)) dt \]

Длина пути есть:
\[\int_a^b \sqrt{x'^2 + y'^2 + z'^2} dt\]

Тогда интеграл первого рода записывается следующим образом:
\[\int f dS : = \int_a^b f(x(t), y(t), z(t)) \cdot \sqrt{x'^2 + y'^2 + z'^2} dt\]

Можно заметить, что интеграл второго рода меняет знак от направления обхода пути.

\begin{exercise}[4221]
    Дан треугольник \(OAB\) с точками \((0, 0), (0, 1), (1, 0)\). Вычислить интеграл \(\int_\Delta (x + y) dS\).

    \[\int_\Delta (x + y) dS = \int_{OA} + \int_{AB} + \int_{OB}\]
    Здесь не важно направление.

    Параметризация \(OA\): \(x(t) = t, y(t) = 0\). Тогда \(\sqrt{x'^2 + y'^2} = 1\)

    \[\int_{OA} = \int_0^1 t\cdot 1 dt = \frac{1}{2}\]

    Параметризация \(AB\): \(x(t) = t, y(t) = 1 - t\). Тогда \(\sqrt{x'^2 + y'^2} = \sqrt{2}\)

    \[\int_{AB} = \int_0^1 1 \cdot \sqrt{2} dt = \sqrt{2}\]

    \[\int_{OB} = \int_{OA} = \frac{1}{2}\]
\end{exercise}

\begin{exercise}[4250]
    Здесь \(C\) есть часть параболы от \( - 1\) до \(1\)

    \[x(t) = t \quad y(t) = t^2\]
    \[\int_C (x^2 - 2xy)dx + (y^2 - 2xy)dy = \int_{ - 1}^1 (t^2 - 2t^3 + (t^4 - 2t^3) 2t)dt\] % TODO

    Это решение обречено на успех, но попробуем решить другим способом. Является ли наше поле потенциальным? Для этого нужно следующее: \(\begin{cases}
        P'_y = Q'_x \\
        P'_z = R'_x \\
        Q'_z = R'_y \\
    \end{cases}\). К сожалению, это не выполнено, поэтому такое решение не работает.
\end{exercise}

Если интеграл задан от точки до точки, то у него обязан быть потенциал, иначе он зависит от пути.

Нам важно, как устроена система координат. Наша выглядит так:

\begin{figure}[h]
    \centering
    \includesvg[scale=0.8]{images/оси.svg}
    \caption{Оси образуют правую тройку \(xyz\)}
\end{figure}

\end{document}