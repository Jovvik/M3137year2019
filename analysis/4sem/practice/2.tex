\documentclass[12pt, a4paper]{article}

\usepackage{lastpage}
\usepackage{mathtools}
\usepackage{xltxtra}
\usepackage{libertine}
\usepackage{amsmath}
\usepackage{amsthm}
\usepackage{amsfonts}
\usepackage{amssymb}
\usepackage{enumitem}
\usepackage{xcolor}
\usepackage[left=2.3cm, right=2.3cm, top=2.7cm, bottom=2.7cm, bindingoffset=0cm, headheight=15pt]{geometry}
\usepackage{fancyhdr}
\usepackage[russian]{babel}
% \usepackage{parindent}

\pagestyle{fancy}
\lfoot{M3137y2019}
\rhead{\thepage\ из \pageref{LastPage}}

\newcommand{\R}{\mathbb{R}}
\newcommand{\Q}{\mathbb{Q}}
\newcommand{\C}{\mathbb{C}}
\newcommand{\Z}{\mathbb{Z}}
\newcommand{\B}{\mathbb{B}}
\newcommand{\N}{\mathbb{N}}

\DeclareMathOperator*{\xor}{\oplus}
\DeclareMathOperator*{\equ}{\sim}
\DeclareMathOperator{\Ln}{\text{Ln}}
\DeclareMathOperator{\sign}{\text{sign}}
\DeclareMathOperator{\Sym}{\text{Sym}}
\DeclareMathOperator{\Asym}{\text{Asym}}
% \DeclareMathOperator{\sh}{\text{sh}}
% \DeclareMathOperator{\tg}{\text{tg}}
% \DeclareMathOperator{\arctg}{\text{arctg}}
% \DeclareMathOperator{\ch}{\text{ch}}

\DeclarePairedDelimiter{\ceil}{\lceil}{\rceil}

\setmainfont{Linux Libertine}

\theoremstyle{plain}
\newtheorem{theorem}{Теорема}
\newtheorem{axiom}{Аксиома}
\newtheorem{lemma}{Лемма}

\theoremstyle{remark}
\newtheorem*{remark}{Примечание}
\newtheorem*{exercise}{Упражнение}
\newtheorem*{consequence}{Следствие}
\newtheorem*{example}{Пример}
\newtheorem*{observation}{Наблюдение}

\theoremstyle{definition}
\newtheorem*{definition}{Определение}
\newtheorem*{obozn}{Обозначение}

\lhead{Математический анализ \textit{(практика)}}
\cfoot{}
\rfoot{18.2.2021}

\begin{document}

Вспомним два интеграла для \(\Gamma\):
\[\int_0^{+\infty} t^{x - 1} e^{ - t} dt = \Gamma(x) \quad x > 0\]
\[\int_0^1 t^{x - 1} (1 - t)^{y - 1} dt = B(x, y) \symrefeq{надо доказать} \frac{\Gamma(x)\Gamma(y)}{\Gamma(x + y)} \approx \frac{1}{C^x_{x + y}}\]
\ref{надо доказать} надо будет доказать.

\begin{example}
    \begin{align*}
        \int_0^{\frac{\pi}{2}} \sin^{2021} x dx                                                                                                                         \\
         & = \begin{bmatrix} t = \sin^2 x \\ dt = 2 \sin x \cos x dx  \\ dx = \frac{dt}{2t^{\frac{1}{2}} (1 - t^{\frac{1}{2}})}\end{bmatrix} \\
         & = \int_0^1 t^{1010.5} \frac{1}{2} t^{ - \frac{1}{2}} (1 - t)^{ - \frac{1}{2}} dt                                                                             \\
         & = \frac{1}{2}\int_0^1 t^{1010} (1 - t)^{ - \frac{1}{2}}dt                                                                                                    \\
         & = \frac{1}{2}B(1011, \frac{1}{2})                                                                                                                            \\
         & = \frac{1}{2} \frac{\Gamma(1011)\Gamma(\frac{1}{2})}{\Gamma(1011.5)}
    \end{align*}
\end{example}

Для вычисления \(\Gamma\) у нас есть несколько формул:
\begin{itemize}
    \item Формула понижения: \(\Gamma(x + 1) = x \cdot \Gamma(x)\)
    \item Формула дополнения: \(\Gamma(x)\Gamma(1 - x) = \frac{\pi}{\sin \pi x}\)
    \item \(\Gamma(0.5) = \sqrt{\pi}\)
\end{itemize}

Таким образом, мы можем вычислить все \(\Gamma\) в ответе.

\[\frac{1}{2} \frac{\Gamma(1011)\Gamma(\frac{1}{2})}{\Gamma(1011.5)} = \frac{1}{2} \frac{1010! \sqrt{\pi}}{\frac{2021}{2}\cdot \frac{2019}{2} \dots } = \frac{1010! 2^{1009}}{} \]

\begin{exercise}[3845]
    \[\int_0^{+\infty} \frac{\sqrt[4]{x}}{(1 + x)^2} dx\]

    Можно решить через \(t^4 = x\), но это сложно.

    \[t : = \frac{1}{1 + x} \quad x = \frac{1}{t} - 1\]

    \[\int_0^{+\infty} \frac{\sqrt[4]{x}}{(1 + x)^2} dx = \int_0^1 \left( \frac{1}{t} - 1 \right)^{\frac{1}{4}} dt = B\left( \frac{3}{4}, \frac{5}{4} \right) = \frac{\Gamma(\frac{3}{4})\Gamma(\frac{1}{4})}{1} = \frac{\pi}{2 \sqrt{2}} \]
\end{exercise}

\begin{exercise}[3849]
    \[\int_0^1 \frac{dx}{\sqrt[n]{1 - x^n}} = \int_0^1 \frac{dx}{(1 - x^n)^{\frac{1}{n}}} = \frac{1}{n} \int_0^1 \frac{dt}{(1 - t)^{\frac{1}{n}}} t^{\frac{1 - n}{n}} = \frac{1}{n} \int_0^1 t^{\frac{1 - n}{n}} (1 - t)^{ - \frac{1}{n}} dt = B\left( \frac{1}{n}, 1 - \frac{1}{n} \right) \frac{1}{n}\]
    \[ = \frac{\Gamma(\frac{1}{n})\Gamma(1 - \frac{1}{n})}{\Gamma(1)} = \frac{\pi}{\sin\left( \pi \frac{1}{n} \right)} \]
\end{exercise}

\begin{exercise}[3848]
    \[\int_0^{\frac{\pi}{2}} \sin^6 x \cos^4 x dx = [\sin^2 x = t] = \int_0^1 t^3 (1 - t)^2 \frac{dt}{2t^{\frac{1}{2}}(1 - t)^{\frac{1}{2}}} = \frac{1}{2}\int_0^1 t^{\frac{5}{2}} (1 - t)^{\frac{3}{2}} dt = \frac{1}{2} B\left( \frac{7}{2}, \frac{5}{2} \right) = \frac{7!!5!!\pi}{2^8 5!} \]
\end{exercise}

\section*{Кратные интегралы}

\begin{figure}[h]
    \centering
    \includesvg{\detokenize{images/штука.svg}}
\end{figure}

\[\int \int f dx dy = \int_a^b \left( \int_{c(x)}^{d(x)} f(x, y) dy \right) dx\]

\begin{exercise}
    \[\int_{ - 1}^2 dx \int_x^{x^2} f dy\]
\end{exercise}

\end{document}
