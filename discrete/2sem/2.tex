\documentclass[12pt, a4paper]{article}

\usepackage{lastpage}
\usepackage{mathtools}
\usepackage{xltxtra}
\usepackage{libertine}
\usepackage{amsmath}
\usepackage{amsthm}
\usepackage{amsfonts}
\usepackage{amssymb}
\usepackage{enumitem}
\usepackage{xcolor}
\usepackage[left=2.3cm, right=2.3cm, top=2.7cm, bottom=2.7cm, bindingoffset=0cm, headheight=15pt]{geometry}
\usepackage{fancyhdr}
\usepackage[russian]{babel}
% \usepackage{parindent}

\pagestyle{fancy}
\lfoot{M3137y2019}
\rhead{\thepage\ из \pageref{LastPage}}

\newcommand{\R}{\mathbb{R}}
\newcommand{\Q}{\mathbb{Q}}
\newcommand{\C}{\mathbb{C}}
\newcommand{\Z}{\mathbb{Z}}
\newcommand{\B}{\mathbb{B}}
\newcommand{\N}{\mathbb{N}}

\DeclareMathOperator*{\xor}{\oplus}
\DeclareMathOperator*{\equ}{\sim}
\DeclareMathOperator{\Ln}{\text{Ln}}
\DeclareMathOperator{\sign}{\text{sign}}
\DeclareMathOperator{\Sym}{\text{Sym}}
\DeclareMathOperator{\Asym}{\text{Asym}}
% \DeclareMathOperator{\sh}{\text{sh}}
% \DeclareMathOperator{\tg}{\text{tg}}
% \DeclareMathOperator{\arctg}{\text{arctg}}
% \DeclareMathOperator{\ch}{\text{ch}}

\DeclarePairedDelimiter{\ceil}{\lceil}{\rceil}

\setmainfont{Linux Libertine}

\theoremstyle{plain}
\newtheorem{theorem}{Теорема}
\newtheorem{axiom}{Аксиома}
\newtheorem{lemma}{Лемма}

\theoremstyle{remark}
\newtheorem*{remark}{Примечание}
\newtheorem*{exercise}{Упражнение}
\newtheorem*{consequence}{Следствие}
\newtheorem*{example}{Пример}
\newtheorem*{observation}{Наблюдение}

\theoremstyle{definition}
\newtheorem*{definition}{Определение}
\newtheorem*{obozn}{Обозначение}

\lhead{Дискретная математика}
\cfoot{}

\begin{document}
    
\section{Случайные величины}

\begin{definition}
    Случайная величина --- отображение $\xi : \Omega \to \R$
\end{definition}

Рассмотрим игру, где преподаватель бросает честную монетку, и если выпал 0, то он платит 1 рубль, иначе каждый ученик платит ему 1 рубль. Заметим, что игра не честная, т.к. $win(0) = -1, win(1)=100; \quad win : \Omega \to \R$

\begin{definition}
    \textbf{Функция распределения} случайной величины:
    $$F_\xi(a)=P(\xi\leq a)$$
\end{definition}

В выше рассмотренной игре $F_{win}(a)=\begin{cases}
    0 & a<-1 \\
    \frac{1}{2} & -1\leq a < 100 \\
    1 & a \geq 100
\end{cases}$

\begin{definition}
    \textbf{Плотность распределения} случайной величины $f_\xi(a)=P(\xi=a) \quad f_\xi(a)=F'_\xi(a)$
\end{definition}

График плотности распределения величины выигрыша для вышеуказанной игры $f_\xi(a)=\begin{cases}
    \frac{1}{2} & a=-1 \\
    \frac{1}{2} & a=100 \\
    0 & \text{иначе}
\end{cases}$

Вернемся к игре преподавателя. $\Omega = \{0, 1\} \quad win(0)=-1, win(1)=100 \quad P(win = -1) = \frac{1}{2}, P(win = 100) = \frac{1}{2}$

\begin{definition}
    \textbf{Математическое ожидание} --- среднее взвешенное значение случайной величины.
    $$E_\xi=M_\xi=\sum\limits_{\omega \in \Omega} p(\omega) \cdot \xi(\omega)$$
\end{definition}

В игре преподавателя $$E_\xi=\frac{1}{2}(-1)+\frac{1}{2}100 = 49.5$$

$$E_\xi=\sum\limits_{\omega \in \Omega} p(\omega) \cdot \xi(\omega)=\sum\limits_{a}\sum\limits_{\underset{\xi(\omega) = a}{w\in\Omega}}p(\omega)\cdot a=\sum\limits_{a}a\cdot P(\xi=a)$$

\subsection{Действия с случайными величинами}

$\xi, \eta$ --- случайные величины. Следующие операции возможны: $$\xi+\eta \quad \xi\cdot\eta \quad a\in\R \cdot \xi \quad \xi^2 \quad \arctg \xi$$

$$F_{c\xi}(a)=P(c\xi\leq a) = P(\xi\leq\frac{a}{c})=F_\xi\left(\frac{a}{c}\right)$$
$$f_{c\xi}(a)=f_\xi\left(\frac{a}{c}\right)$$
$$E(c\xi)=\sum\limits_{\omega\in\Omega} p(w)\cdot c\cdot \xi(\omega) = c\cdot E(\xi)$$
$$E(\xi+\eta)=\sum\limits_{\omega}p(\omega)(\xi(\omega)+\eta(\omega))=\sum\limits_{\Omega} (p(\omega)\xi(\omega)+p(\omega)\eta(\omega))=E\xi+E\eta$$

\begin{definition}
    \textbf{Неподвижной точкой} в перестановке называется такой индекс $i$, что $\pi_i=i$
\end{definition}

Рассмотрим $\xi(\pi)$ --- количество неподвижных точек в случайной перестановке.

Посчитать распределение $\xi$ --- сложная задача, в отличие от матожидания.

$$\xi_i(\pi):=\begin{cases}
    1 & \pi[i]=i \\
    0 & \text{иначе}
\end{cases}$$

Такие величины называются \textbf{индикаторными}.

$$\xi(\pi)=\sum\limits_{i=1}^n \xi_i(\pi)$$
$$P(\xi_i=1)=\frac{(n-1)!}{n!}=\frac{1}{n}$$
$$E_{\xi_i}=0\cdot\frac{n-1}{n}+1\cdot\frac{1}{n}=\frac{1}{n}$$
$$E_\xi = 1$$

Итого в случайной перестановке ожидается ровно 1 неподвижная точка.

Можно заметить, что $\xi_i$ вроде как влияют друг на друга --- если все, кроме одной $\xi_i=1$, то и оставшаяся $\xi_j=1$. Однако это не влияет на матожидание.

\begin{definition}
    $\xi$ и $\eta$ --- \textbf{независимые}, если $\forall a, b$ события $\xi\leq a$ и $\eta\leq b$ --- независимые.
\end{definition}

Матождиание линейно всегда, в том числе для зависимых величин.

\begin{theorem}
    $\xi$ и $\eta$ --- независимые, если $E(\xi, \eta) = E_\xi E_\eta$
\end{theorem}
\begin{lemma}
    $\xi$ и $\eta$ --- независимые $\Leftrightarrow [\xi=a], [\eta=b]$ независимые $\forall a$ и $b$
\end{lemma}
$$E(\xi\eta)=\sum\limits_{a}P(\xi\eta=a)a=\sum\limits_{a}\sum\limits_{x} P(\xi\cdot\eta=a | \xi=x)$$
$$P(\xi=x)\cdot a=\sum\limits_{a}\sum\limits_{x} P\left(\eta=\frac{a}{x}\right)P(\xi=x)\frac{a}{x}\cdot x$$
$$P(\xi\eta=a | \xi=x)=\frac{P(\xi\eta=a\cap\xi=x)}{P(\xi=x)}=\frac{P(\xi=x\cap \eta=\frac{a}{x})}{P(\xi=x)}=$$
$$\frac{P(\xi=x)P(\eta=\frac{a}{x})}{P(\xi=x)}=P(\eta=\frac{a}{x})$$

Матожидание --- не единственная оценка игры. Ещё есть \textbf{дисперсия} --- квадрат матожидания разности величины и её матожидания:
$$Var(x)=Dx=E(x-Ex)^2=E(x^2-2xEx+(Ex)^2)=Ex^2-2ExEx+(Ex)^2=Ex^2-(Ex)^2$$
$$D(\xi+\eta)=E(\xi+\eta)^2-(E\xi+E\eta)^2=E\xi^2+E\eta^2+2E(\xi\eta)-(E\xi)^2-(E\eta)^2-2E\xi E\eta=$$
$$=D\xi+D\eta+2(E\xi\eta - E\xi E\eta)=:D\xi+D\eta+Cov(\xi, \eta)$$

\end{document}