\documentclass[12pt, a4paper]{article}

\usepackage{lastpage}
\usepackage{mathtools}
\usepackage{xltxtra}
\usepackage{libertine}
\usepackage{amsmath}
\usepackage{amsthm}
\usepackage{amsfonts}
\usepackage{amssymb}
\usepackage{enumitem}
\usepackage{xcolor}
\usepackage[left=2.3cm, right=2.3cm, top=2.7cm, bottom=2.7cm, bindingoffset=0cm, headheight=15pt]{geometry}
\usepackage{fancyhdr}
\usepackage[russian]{babel}
% \usepackage{parindent}

\pagestyle{fancy}
\lfoot{M3137y2019}
\rhead{\thepage\ из \pageref{LastPage}}

\newcommand{\R}{\mathbb{R}}
\newcommand{\Q}{\mathbb{Q}}
\newcommand{\C}{\mathbb{C}}
\newcommand{\Z}{\mathbb{Z}}
\newcommand{\B}{\mathbb{B}}
\newcommand{\N}{\mathbb{N}}

\DeclareMathOperator*{\xor}{\oplus}
\DeclareMathOperator*{\equ}{\sim}
\DeclareMathOperator{\Ln}{\text{Ln}}
\DeclareMathOperator{\sign}{\text{sign}}
\DeclareMathOperator{\Sym}{\text{Sym}}
\DeclareMathOperator{\Asym}{\text{Asym}}
% \DeclareMathOperator{\sh}{\text{sh}}
% \DeclareMathOperator{\tg}{\text{tg}}
% \DeclareMathOperator{\arctg}{\text{arctg}}
% \DeclareMathOperator{\ch}{\text{ch}}

\DeclarePairedDelimiter{\ceil}{\lceil}{\rceil}

\setmainfont{Linux Libertine}

\theoremstyle{plain}
\newtheorem{theorem}{Теорема}
\newtheorem{axiom}{Аксиома}
\newtheorem{lemma}{Лемма}

\theoremstyle{remark}
\newtheorem*{remark}{Примечание}
\newtheorem*{exercise}{Упражнение}
\newtheorem*{consequence}{Следствие}
\newtheorem*{example}{Пример}
\newtheorem*{observation}{Наблюдение}

\theoremstyle{definition}
\newtheorem*{definition}{Определение}
\newtheorem*{obozn}{Обозначение}

\lhead{Дискретная математика}
\cfoot{}
\rfoot{Лекция 4}

\begin{document}

Информация $=-$ неопределенность

Рассмотрим вероятностное пространство $\Omega=\{\omega_1\ldots \omega_n\} \quad p_1\ldots p_n \quad \sum\limits_{i=1}^n p_i=1$

\begin{definition}
    $H$ --- \textbf{мера неопределенности случайного источника \textit{(энтропия)}}, если это отображение удовлетворяет следующему:
\end{definition}
$$H(p_1\ldots p_n) : \R^n\to\R_+$$
$$H(\frac{1}{n},\frac{1}{n}\ldots \frac{1}{n}) < H(\frac{1}{n+1}\ldots \frac{1}{n+1})$$
$H$ --- непр. от всех аргументов

Рассмотрим два эксперимента, где у первого исходы $p_i$, у второго $q_{ij}$.
$$H(p_1q_{11}, p_2q_{12}\ldots p_1q_{1m_1}, p_2q_{21}\ldots p_nq_{nm_n})=H(p_1\ldots p_n)+\sum\limits_{i=1}^n p_iH(q_{i1}\ldots q_{im_i})$$

$h(n)=H(\frac{1}{n}\ldots\frac{1}{n})$

\begin{lemma}
    $h(n)=c\log_2 n$
\end{lemma}

$p_i=\frac{1}{n} \ \ m_i=m \ \ a_{ij}=\frac{1}{m}$

$h(mn)=h(n)+h(m)$

$h(2)=c$ --- бит.

$$2^i\leq n^k < 2^{i+1}$$
$$i\leq k\log_2 n < i+1$$
$$\frac{i}{k}\leq\log_2 n<\frac{i+1}{k}$$
$$h(2^i)\leq h(n^k) < h(2^{i+1})$$
$$ci\leq h(n^k) < c(i+1)$$
$$ci\leq k\cdot h(n) < c(i+1)$$
$$\frac{i}{k}\leq \frac{h(n)}{c} < \frac{i+1}{k}$$
Т.к. $\frac{h(n)}{c}$ и $\log_2 n$ зажимаются $\frac{i}{k}$ и $\frac{i+1}{k} \Rightarrow \frac{h(n)}{c}=\log_2 n$

$] p_i=\frac{a_i}{b} \quad \sphericalangle q_{ij}=\frac{1}{a_i}, m_i=a_i$
$$h(b)=H(p_1\ldots p_n) + \sum\limits_{i=1}^n p_ih(a_i)$$
$$H(p_1\ldots p_n)=c\log_2 b - c\sum\limits_{i=1}^n p_i\log_2 a_i = -c\left(\sum\limits_{i=1}^n p_i(\log_2a_i-\log_2b)\right)=$$
$$=-c\sum\limits_{i=1}^n p_i\log_2p_i$$

Рассмотрим арифметическое кодирование:
$$H(p_1 \ldots p_n)=-\sum\limits_{i=1}^n p_i\log_2 p_i$$
$$L\geq \frac{1}{2^q} \Rightarrow -\log_2 L \leq q$$
$$\prod\limits_{i=1}^n\left(\frac{f_i}{m}\right)^{f_i}$$
$$-\log_2 \prod\limits_{i=1}^n\left(\frac{f_i}{m}\right)^{f_i}=-\sum\limits_{i=1}^n f_i\log_2 p_i = m\left(-\sum\limits_{i=1}^n p_i\log_2 p_i\right)=mH(p_1\ldots p_n)$$

\section{Симуляция одного распределения другим}

Рассмотрим распределение $1\ldots n$ с вероятностями $p_1\ldots p_n, n\geq 2, p_i>0$

Задача: сгенерировать распределение с вероятностями $q_1\ldots q_m$

Поделим отрезок $[0,1]$ в пропорциях $p_i$. Если отрезок $p_i$ не лежит полностью в одном отрезке, то делают зум по всем отрезкам $q_j$, которым соответствует какое-то число из $p_i$.

\end{document}