\documentclass[12pt, a4paper]{article}

\usepackage{lastpage}
\usepackage{mathtools}
\usepackage{xltxtra}
\usepackage{libertine}
\usepackage{amsmath}
\usepackage{amsthm}
\usepackage{amsfonts}
\usepackage{amssymb}
\usepackage{enumitem}
\usepackage{xcolor}
\usepackage[left=2.3cm, right=2.3cm, top=2.7cm, bottom=2.7cm, bindingoffset=0cm, headheight=15pt]{geometry}
\usepackage{fancyhdr}
\usepackage[russian]{babel}
% \usepackage{parindent}

\pagestyle{fancy}
\lfoot{M3137y2019}
\rhead{\thepage\ из \pageref{LastPage}}

\newcommand{\R}{\mathbb{R}}
\newcommand{\Q}{\mathbb{Q}}
\newcommand{\C}{\mathbb{C}}
\newcommand{\Z}{\mathbb{Z}}
\newcommand{\B}{\mathbb{B}}
\newcommand{\N}{\mathbb{N}}

\DeclareMathOperator*{\xor}{\oplus}
\DeclareMathOperator*{\equ}{\sim}
\DeclareMathOperator{\Ln}{\text{Ln}}
\DeclareMathOperator{\sign}{\text{sign}}
\DeclareMathOperator{\Sym}{\text{Sym}}
\DeclareMathOperator{\Asym}{\text{Asym}}
% \DeclareMathOperator{\sh}{\text{sh}}
% \DeclareMathOperator{\tg}{\text{tg}}
% \DeclareMathOperator{\arctg}{\text{arctg}}
% \DeclareMathOperator{\ch}{\text{ch}}

\DeclarePairedDelimiter{\ceil}{\lceil}{\rceil}

\setmainfont{Linux Libertine}

\theoremstyle{plain}
\newtheorem{theorem}{Теорема}
\newtheorem{axiom}{Аксиома}
\newtheorem{lemma}{Лемма}

\theoremstyle{remark}
\newtheorem*{remark}{Примечание}
\newtheorem*{exercise}{Упражнение}
\newtheorem*{consequence}{Следствие}
\newtheorem*{example}{Пример}
\newtheorem*{observation}{Наблюдение}

\theoremstyle{definition}
\newtheorem*{definition}{Определение}
\newtheorem*{obozn}{Обозначение}

\title{Конспект по дискретной математике}
\date{October 1, 2019}

\begin{document}
\maketitle

\section{Преобразование Мёбиуса}

$f(x_1\ldots x_n)\xor\limits_{\alpha_1\ldots\alpha_n\in A_f} x_1^{\alpha_1}\ldots x_n^{\alpha_n}$

$x_i^1=x_i, x_i^0=1$

$f(\ldots) = 0 \Leftrightarrow \exists x_i: x_i=0, \alpha_i=1$

$f(\ldots) = 1 \Leftrightarrow \forall i: x_i=1 \ or \ \alpha_i=0 \Leftrightarrow \overline x \succeq \overline \alpha$

$a_{\alpha_1\ldots\alpha_n} = \begin{cases} 1, \alpha_1\ldots\alpha_n\in A_f \\ 0, \text{иначе} \end{cases}$

\begin{definition}
    Доминирование - частничый порядок на $\mathbb{B}^n$, такой что $\forall i \ x_i\geq y_i$. Обозначается $x_1\ldots x_n \succeq y_1\ldots y_n$.
\end{definition}

$f(x_1\ldots x_n)=\xor\limits_{\alpha_1\ldots\alpha_n} a_{\alpha_1\ldots\alpha_n} x_1^{\alpha_1}\ldots x_n^{\alpha_n} = \xor\limits_{\alpha_1\ldots\alpha_n: X\succeq \alpha} a_{\alpha_1\ldots\alpha_n}$

$a_{\alpha_1\ldots\alpha_n} \mapsto f(\overline{x})$

$f(x_1\ldots x_n) \mapsto a_{\alpha_1\ldots\alpha_n}$

\begin{theorem}
    Преобразование Мёбиуса: $a_{\alpha_1\ldots\alpha_n}=\xor\limits_{\alpha_1\ldots\alpha_n: \alpha\succeq X}f(x_1\ldots x_n)$ --- $\mathbb{B}^{2^n}\to\mathbb{B}^{2^n}$
\end{theorem}

Пример ($x\vee y$): \begin{tabular}{ccc} $xy$&$x\vee y$& \\
    00&0&$\alpha_{00}=0$ \\
    01&1&$\alpha_{01}=1$ \\
    10&1&$\alpha_{10}=1$ \\
    11&0&$\alpha_{11}=1$ \\
\end{tabular}

\begin{proof}
Индукция по числу $1$ в $\alpha_1\ldots\alpha_n$.

База: $\alpha_{0\ldots 0}=f(0\ldots 0)$

Переход: $\xor\limits_{x:\alpha \succeq x}f(x_1\ldots x_n)=\xor\limits_{x:\alpha\succeq x}\ \xor\limits_{\beta:x\succeq \beta}a_{\beta_1\ldots\beta_n}=a_{\alpha_1\ldots \alpha_n}\xor \xor\limits_{\substack{x,\beta \\ \alpha\succeq x\succeq \beta \\ \text{неверно:}\alpha=x=\beta}}a_{\beta_1\ldots\beta_n}=a_{\alpha_1\ldots\alpha_n}$
\end{proof}

\begin{definition}
Инволюция --- преобразование, тождественное себе обратному.
\end{definition}

Преобразование Мёбиуса является инволюцией.

Можно заметить, что при записи через некий базис функций от множества переменных, формулы получаются длинными. Поэтому существуют другие формы записи --- схемы из функциональных элементов и линейные программы.

\section{Схемы из функциональных элементов}

Записывается в форме ациклического ориентированного графа --- direct acyclic graph --- dag.

\begin{theorem}
Если $G$ - ациклический ориентированный граф, тогда его вершинам можно присвоить номера так, что $uv\in E \Rightarrow \#u<\#v$. Это называется топологическая сортировка
\end{theorem}

\begin{lemma}
    Ациклический граф содержит вершину, из которой не выходят ребра.
\end{lemma}

\begin{definition}
    Схема из функциональных элементов над базисом $F$ --- ациклический ориентированный граф, у которого есть $n$ выделенных вершин, в которые ничего не входит, помеченные ``вход'' и есть одна вершина, из которой ничего не выходит, помеченная ``выход''. Остальные вершины называются внутренние, они --- функции $f\in F$, в которые входят $k$ ребер, каждое из которых помечено числом от $1$ до $k$.
\end{definition}

Чтобы построить таблицу истинности схемы из функциональных элементов, проводится слеюущий порядок действий:

\begin{enumerate}
\item Поместим во входные узлы значения аргументов схемы.
\item В остальные узлы в порядке топологической сортировки помещаются значения, равные значению функции, соответствующей этому узлы, если принять значения из входных узлов как аргументы данной функции.
\item Значение в выходном узле будет значением схемы.
\end{enumerate}

%! Вставить пикчу с графом

Самый правый узел - выход.

Глубина: $d(u)=\max\limits_{V:V\to u} (d(u)+1)$

Можно заметить, что с помощью схемы можно использовать промежуточные значения несколько раз, этим эта форма записи лучше формулы.

\begin{definition}
    Сложность функции $f$ в базисе $B$ --- минимальный размер схемы, которая вычисляет эту функцию, обозначается $size_B(f)=\min\{size(C)|C \text{вычисл.} f, C\in B\}$.
\end{definition}

$size_{\vee,\wedge,\neg}(\xor_2)\leq 5$

$size_{\wedge,\xor,1}(\xor_2)=1$

\begin{theorem}
    Пусть $B_1$ и $B_2$ - базисы. Тогда $\exists c>0$, которая зависит от этих двух базисов и больше не зависит ни от чего, что $\forall f : size_{B_1}(f)\leq c\cdot size_{B_2}(f)$
\end{theorem}

Таким образом, выбор базиса для сложности меняет только константу, т.е. ассимптотически сложность функции одинакова во всех базисах.

\begin{lemma}
    Если $B$ - базис, то $\forall f$ можно задать схемой из функциональных элементов над $B$.
\end{lemma}

\begin{proof}
    Построим схему через дерево обхода - поставим стрелки вверх и объединим все входы.
\end{proof}

\begin{proof}
    Построим для $f$ оптимальную схему в $B_2$. $B_2 =\{g_1,g_2\ldots g_t\}$. Рассмотрим схемы для функций из $B_2$ над базисом $B_1$. Для каждой этой функции построим такую схему $C_i$. $C:=\max size(C_i)$. Вместо каждой функции подставим соответствующую схему $C$.
\end{proof}

\begin{proof}
    Глубиной функции $f$ в базисе $B$ называется минимальная глубина схемы, которая вычисляет $f$, обозначается $depth_B(f)=\min\{depth(C)|C \text{вычисл.} f, C\in B\}$.
\end{proof}

$U_2$ - множество всех функций от $2$ аргументов.
$T_2=U_2\setminus \{\xor_,=\}$

\end{document}
