\documentclass[12pt, a4paper]{article}

\usepackage{lastpage}
\usepackage{mathtools}
\usepackage{xltxtra}
\usepackage{libertine}
\usepackage{amsmath}
\usepackage{amsthm}
\usepackage{amsfonts}
\usepackage{amssymb}
\usepackage{enumitem}
\usepackage{xcolor}
\usepackage[left=2.3cm, right=2.3cm, top=2.7cm, bottom=2.7cm, bindingoffset=0cm, headheight=15pt]{geometry}
\usepackage{fancyhdr}
\usepackage[russian]{babel}
% \usepackage{parindent}

\pagestyle{fancy}
\lfoot{M3137y2019}
\rhead{\thepage\ из \pageref{LastPage}}

\newcommand{\R}{\mathbb{R}}
\newcommand{\Q}{\mathbb{Q}}
\newcommand{\C}{\mathbb{C}}
\newcommand{\Z}{\mathbb{Z}}
\newcommand{\B}{\mathbb{B}}
\newcommand{\N}{\mathbb{N}}

\DeclareMathOperator*{\xor}{\oplus}
\DeclareMathOperator*{\equ}{\sim}
\DeclareMathOperator{\Ln}{\text{Ln}}
\DeclareMathOperator{\sign}{\text{sign}}
\DeclareMathOperator{\Sym}{\text{Sym}}
\DeclareMathOperator{\Asym}{\text{Asym}}
% \DeclareMathOperator{\sh}{\text{sh}}
% \DeclareMathOperator{\tg}{\text{tg}}
% \DeclareMathOperator{\arctg}{\text{arctg}}
% \DeclareMathOperator{\ch}{\text{ch}}

\DeclarePairedDelimiter{\ceil}{\lceil}{\rceil}

\setmainfont{Linux Libertine}

\theoremstyle{plain}
\newtheorem{theorem}{Теорема}
\newtheorem{axiom}{Аксиома}
\newtheorem{lemma}{Лемма}

\theoremstyle{remark}
\newtheorem*{remark}{Примечание}
\newtheorem*{exercise}{Упражнение}
\newtheorem*{consequence}{Следствие}
\newtheorem*{example}{Пример}
\newtheorem*{observation}{Наблюдение}

\theoremstyle{definition}
\newtheorem*{definition}{Определение}
\newtheorem*{obozn}{Обозначение}

\title{Конспект по дискретной математике}

\date{October 22, 2019}

\begin{document}
    
\maketitle

\section{Представление информации}

Направления развития:
\begin{enumerate}
    \item Сжатие
    \item Избыточное кодирование
    \item Криптографическое кодирование
\end{enumerate}

\begin{definition}
    \textbf{Алфавитом} $\Sigma$ называется непустое конечное множество. Множество из $n$ элементов $\Sigma$ обозначается $\Sigma^n$.
\end{definition}

$$\bigcup\limits_{i=0}^\infty \Sigma^i=\Sigma^* \quad \Sigma^0=\{\varepsilon\}$$

\begin{definition}
    Конкатенация:
    $$\alpha\in\Sigma^* \quad \beta\in\Sigma^* \mapsto \alpha\beta\in\Sigma^*$$
\end{definition}

Конкатенация транзитивна $(\alpha\beta)\gamma=\alpha(\beta\gamma) \Rightarrow$ алфавит --- полугруппа.

$\alpha\varepsilon=\varepsilon\alpha=\alpha \Rightarrow$ алфавит --- моноид.

Т.к. алфавит --- полугруппа и моноид, алфавит --- свободный моноид.

\begin{definition}
    \textbf{Гомоморфизм} $\varphi: \Sigma^*\to \Pi^*$
    
    $\varphi(\alpha\beta)=\varphi(\alpha)\varphi(\beta)$
\end{definition}

\begin{example}
    $$0\to a, 1\to ab$$
    $$\varphi: \{0,1\}^*\to\{a,b\}^*$$
    $$\varphi(001)=aaab$$
\end{example}

$\varphi$ --- гомоморфизм $\Rightarrow \varphi(c_1,c_2\ldots c_n)=\varphi(c_1)\varphi(c_2)\ldots\varphi(c_n)$

\begin{definition}
    Отображение из произвольного $\Sigma^*$ в $\Pi^*$ называется \textbf{кодом}.

    Если $\varphi$ --- гомоморфизм, $\varphi$ --- \textbf{разделяемый}.

    Если $\Pi=\mathbb{B}$, $\varphi$ --- \textbf{бинарный/двоичный}.
\end{definition}

\begin{example}
    $$\Sigma=\{a,b,c\} \quad \varphi(a)=0, \varphi(b)=01, \varphi(c)=1$$
    $$\varphi(abc)=0011 \quad \varphi(aacc)=0011$$
\end{example}

\begin{definition}
    Код называется \textbf{однозначно декодируемым}, если $\forall x,y\in\Sigma^* \quad \varphi(x)=\varphi(y) \Rightarrow x=y$
\end{definition}

\begin{definition}
    Кодом \textbf{постоянной длины} называется код, если $\varphi: \Sigma\to\Pi^k, k=const$
\end{definition}

\begin{lemma}
    $\varphi$ --- код постоянной длины

    $\forall c\not=d\in\Sigma \quad \varphi(c)\not=\varphi(d)$

    Тогда $\varphi$ --- однозначно декодируемый.
\end{lemma}

\begin{theorem}
    $\Sigma, \Pi, |\Sigma=s|, |\Pi|=p, \Sigma\to \Pi^k$
    
    $k=\ceil{\log_p s}$

    $p^k<s$
\end{theorem}

\begin{theorem}
    Крафта, Мак-Милана.

    $\exists$ двоичных разделяемый однозначно декодируемый код переменной длины с длинами кодовых слов $l_1,l_2\ldots l_s\Leftrightarrow\sum_{i=1}^s 2^{-l_i}\leq 1, S\geq 2$
\end{theorem}

\begin{proof}
    Докажем ``$\Rightarrow$''.

    Пусть $ab, abb, ab$ --- все члены $\Sigma$

    $$(ab+abb+bb)^2=abab+ababb+abbb+\ldots$$

    $$(ab+abb+bb)^k - S^k \text{ слов, при этом все слова разные} $$
    
    $$]a=\frac{1}{2}, b=\frac{1}{2}$$

    $$ab+abb+bb=\sum 2^{-l_i}$$

    $$(\sum 2^{-l_i})^k=\sum_{j=0}^{k\max l_i}(2^{-j}+2^{-j}+2^{-j}) \text{ --- всего $\leq 2^j$ слов}$$

    $$\sum_{j=0}^{k\max l_i}(2^{-j}+2^{-j}+2^{-j})\leq k\max l_i$$

    $$\forall k: x^k\leq k\max l_i \to x\leq 1$$
\end{proof}

\begin{definition}
    \textbf{Префиксный} код: $\forall c\not=d \quad \varphi(c)$ --- не префикс $\varphi(d)$
\end{definition}

\begin{lemma}
    Префиксный код --- однозначно декодируем.
\end{lemma}

\begin{proof}
    Докажем ``$\Leftarrow$''

    $\sum 2^{-l_i}\leq 1 \Rightarrow \exists$ префиксный код с длинами $l_1\ldots l_s$

    $$l_1\leq l_2\leq \ldots\leq l_s$$
    
    $2^{-l_1}$

    $2^{-l_1}+2^{-l_2}$

    $\vdots$

    $2^{-l_1}+2^{-l_2}+\ldots+2^{-l_s}$

    $$S=2 \quad 2^{-l_1}+2^{-l_2}\leq 1$$
\end{proof}

Тут автор сдох.

\begin{consequence}
    $\exists$ однозначно декодируемый код с длинами $l_1\ldots l_s \Rightarrow \exists$ префиксный код с длинами $l_1\ldots l_n$
\end{consequence}

\subsection{Код Хаффмана}

Дано: $f_1,f_2\ldots f_s$ --- как часто встречаются соответствующие слова. Найти $l_1\ldots l_s$, такие что $\sum 2^{-l_i}$ и $\sum l_if_i\to \min$

$S=2\Rightarrow l_1=l_2=1$

$S>2$ Возьмём два символа $x$ и $y$, такие что $f_x$ и $f_y\to\min$ \textit{($x$, $y$ -- самые редкие)}. Заменим их на $z, f_z=f_x+f_y$.

Возьмём в качестве кодового слова для $x$ слово для $z + 0$, а для $y$ возьмём $z + 1$.

\end{document}