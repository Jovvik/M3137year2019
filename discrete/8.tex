\documentclass[12pt, a4paper]{article}

\usepackage{lastpage}
\usepackage{mathtools}
\usepackage{xltxtra}
\usepackage{libertine}
\usepackage{amsmath}
\usepackage{amsthm}
\usepackage{amsfonts}
\usepackage{amssymb}
\usepackage{enumitem}
\usepackage{xcolor}
\usepackage[left=2.3cm, right=2.3cm, top=2.7cm, bottom=2.7cm, bindingoffset=0cm, headheight=15pt]{geometry}
\usepackage{fancyhdr}
\usepackage[russian]{babel}
% \usepackage{parindent}

\pagestyle{fancy}
\lfoot{M3137y2019}
\rhead{\thepage\ из \pageref{LastPage}}

\newcommand{\R}{\mathbb{R}}
\newcommand{\Q}{\mathbb{Q}}
\newcommand{\C}{\mathbb{C}}
\newcommand{\Z}{\mathbb{Z}}
\newcommand{\B}{\mathbb{B}}
\newcommand{\N}{\mathbb{N}}

\DeclareMathOperator*{\xor}{\oplus}
\DeclareMathOperator*{\equ}{\sim}
\DeclareMathOperator{\Ln}{\text{Ln}}
\DeclareMathOperator{\sign}{\text{sign}}
\DeclareMathOperator{\Sym}{\text{Sym}}
\DeclareMathOperator{\Asym}{\text{Asym}}
% \DeclareMathOperator{\sh}{\text{sh}}
% \DeclareMathOperator{\tg}{\text{tg}}
% \DeclareMathOperator{\arctg}{\text{arctg}}
% \DeclareMathOperator{\ch}{\text{ch}}

\DeclarePairedDelimiter{\ceil}{\lceil}{\rceil}

\setmainfont{Linux Libertine}

\theoremstyle{plain}
\newtheorem{theorem}{Теорема}
\newtheorem{axiom}{Аксиома}
\newtheorem{lemma}{Лемма}

\theoremstyle{remark}
\newtheorem*{remark}{Примечание}
\newtheorem*{exercise}{Упражнение}
\newtheorem*{consequence}{Следствие}
\newtheorem*{example}{Пример}
\newtheorem*{observation}{Наблюдение}

\theoremstyle{definition}
\newtheorem*{definition}{Определение}
\newtheorem*{obozn}{Обозначение}

\lhead{Конспект по дискретной математике}
\cfoot{}
\rfoot{October 29, 2019}

\usepackage{empheq}
\usepackage{xcolor}

\begin{document}

\subsection{Оптимальность кода Хаффмана}

\begin{lemma}
    Существует оптимальный префиксный код, для которого наиболее редко встречающиеся два символа $x$ и $y$:
    \begin{enumerate}
        \item Самые глубокие листья
        \item Братья
    \end{enumerate}
\end{lemma}
\begin{proof}
    Рассмотрим самый глубокий лист в дереве Хаффмана $a, f_a\leq x$.

    Т.к. дерево оптимально, у $a$ есть брат $b, f_b\leq y$.

    \noindent
    Поменяем местами $a, b$ с $x, y$. Тогда $\sum l_i f_i=A-l_af_a-l_bf_b-l_yf_y+l_af_x+l_xf_a+l_bf_y+l_y+f_b = A + (l_a-l_x)(f_x-f_a) + (l_b-l_y)(f_y-f_b)$. Первая и третья скобка $\geq 0$, вторая и четвертая $\leq 0 \Rightarrow \sum l_i f_i\leq A$ , т.е. новое дерево оптимально.
\end{proof}

Заменим $x$ и $y$ на $z$, так что $f_x+f_y=f_z$

$$\sum f_il_i=A-f_zl_z+f_xl_x+f_yl_y=A-f_xl_z-f_yl_z+f_x(l_z+1)+f_y(l_y+1)=A+f_x+f_y$$

Таким образом, оптимизация дерева с $z$ вместо $x$ и $y$ оптимизирует и дерево с $x$ и $y$, поэтому код Хаффмана оптимален.

Но это не значит, что нельзя сжимать лучше, чем Хаффман.

\section{Арифметическое кодирование}

Пусть $a$ встречается $1000$ раз, а $b$ --- $1$. Тогда скорее всего оптимально в первый бит писать $0$, если в строке только $a$, иначе $1$, а дальше --- по Хаффману.

\subsection{Кодирование}

Пусть $a$ встречается $3$ раз, $b - 2, c - 1$. Тогда закодируем строку $ababac$. Рассмотрим отрезок $[0, 1]$ и поделим его в отношении частот символов, т.е. в точках $\frac{1}{2}, \frac{5}{6}$. Теперь проведем то же самое для отрезка $[0, \frac{1}{2}]$, т.е. разделим его в точках $\frac{1}{4}$ и $\frac{5}{12}$. Теперь то же самое для $[\frac{1}{4}, \frac{5}{12}]$, т.к. он соответствует $b$, которая соответствует второму символу. В итоге получим некий отрезок $[l, r]$. Найдём в нем число вида $\frac{p}{2^a}$. Запишем $p$ в двоичном виде с дополнением слева нулями до длины $a$.

\subsection{Декодирование}

Из длины кодового слова можно понять $a$ --- длина слова, и $p$ --- перевод из двоичного представления кода. В таком случае известна дробь $\frac{p}{2^q}$ на отрезке $[0, 1]$. Как и в кодировании, делим этот отрезок на соответствующие части и спускаемся в часть, в которую попал $\frac{p}{2^q}$. Чтобы остановить построение, необходимо знать длину исходного слова.

$$\frac{f_a}{\sum f_i}\frac{f_b}{\sum f_i}\frac{f_c}{\sum f_i}=\frac{\prod\limits_{i=1}^L f_{s[i]}}{L^L} = \frac{\prod\limits_{i=1}^k f_i^{f_i}}{L^L}$$

Можно заметить, что алгоритм не работает, когда $$\frac{1}{2^q}>len=r-l$$
Поэтому возьмем $\frac{1}{2^q}\leq len$.

$$\frac{1}{2^q}\leq\frac{\prod\limits_{i=1}^k f_i^{f_i}}{L^L}$$

$$-q\leq \sum\limits_{i=1}^k f_i\log_2f_i-L\log_2 L=\sum\limits_{i=1}^k f_i(\log_2 f_i-\log_2L)$$

$$q\geq -L\sum\limits_{i=1}^k\frac{f_0}{L}\log_2\frac{f_i}{L}=-L\boxed{\sum\limits_{i=1}^k p_i\log_2 p_i} \text{ --- Энтропия Шеннона} =-LH(p_1,p_2\ldots p_k)$$

Оптимальность кодирования с учетом зависимостей между символами не определена, поэтому все рассматриваемые далее методы являются эвристиками.

\section{Словарное кодирование}

\subsection{LZ}

Используется zip.

Существует следующие виды токенов:

\begin{enumerate}
    \item символ
    \item ссылка: $abacaba\to abac(4,3)$ \textit{(сдивнуться на 4 символа назад и вывести три символа)}, $ab(2,6) = ababab$
\end{enumerate}

Оптимальное построение: Для каждого символа находим длиннейшую подстроку, начинающуюся  с него. Если записать ссылку на неё, то делаем это, иначе пишем символ без оптимизации. Для оптимального времени построения нужны суффиксные деревья. 

\subsection{LZW}

\textcolor{red}{Каво}

\subsection{BWT}

$\sphericalangle abacaba\$ $. Отсортируем все её циклические сдвиги.

\noindent
$\$ abacaba\\$
$a\$ bacaba\\$
$aba\$ caba\\$
$abacaba\$ \\$
$acaba\$ ab \\$
$ba\$ abaca \\$
$bacaba\$ a \\$
$caba\$ aba \\$

Последний столбец --- преобразование BWT.

Из того, что $x$ часто встречается как подстрока, получаем много одинаковых символов подряд.

\subsection{MTF --- move to front}

Исходно код символа равен его номеру в алфавите. Когда символ встречается, его код выводится и приравнивается к 0.

$aaaaaaaazzz\to ?00000000?00$

Полученную строку можно эффективно кодировать.

\subsection{bzip2}

\end{document}