\chapter{3 сентября}

Эта лекция обзорная и рукомахательная. Читать для общего развития.

В рамках этого курса мы будем изучать язык Haskell, названный в честь Хаскелла Карри, американского логика. Последний стандарт этого языка --- \texttt{Haskell2010} и его основной компилятор --- \texttt{ghc}. У него открыт исходный код, можно предлагать свои proposal'ы, которые фильтрует сообщество и комитет.

\subsection{История}

В двадцатых годах XX века рассматривались вопросы основ математики. Алонзо Чёрч предложил альтерантиву теории множеств --- \textbf{лямбда-исчисление}, которое основывается на понятии функции. Парадокс Клини-Россера показал, что начальная версия лямбда-исчисления противоречива, поэтому Чёрч ввел \textbf{типы} в лямбда-исчисление в сороковых годах.

Типы появились раньше\footnote{в 1910-х годах} в рамках теории типов Бертрана Рассела и Альфреда Норта Уайтхеда. После работ Чёрча, теория типов развивалась Хаскеллом Карри и Уильямом Ховардом как часть теории доказательств в 50-60 годах. В 70-х годах было создано полиморфное лямбда-исчисление, вместе с языком ML.

\subsection{Функции в теории множеств}

\begin{definition}
    Пусть \(X, Y\) --- непустые множества. Бинарное отношение \(R \subseteq X \times Y\) называется \textbf{функциональным}, если из \((x, y) \in R\) и \((x, z) \in R\) следует \(y = z\).
\end{definition}

\begin{definition}
    \textbf{Функция} \(f : X \to Y\) есть тройка \((X, Y, f)\), где \(f\) --- функциональное отношение. \(f(x) = y\) обозначает \((x, y) \in f\).
\end{definition}

Это определение отождествляет функцию с ее графиком, и это определение было предложено Бурбаки --- группой математиков. Однако, можно рассматривать функцию как примитив, что приведет нас к Тьюринг-полной модели вычислений.

\subsection{Лямбда-исчисление, базовые определения}

\begin{definition}
    Пусть \(Var = \{x_0, x_1, x_2 \dots\}\) --- счётное множество переменных. Множество \textbf{предтермов} есть множество, порожденное следующей контекстно-свободной грамматикой:
    \[M, N :: = x \mid (\lambda x.M) \mid (MN)\]
\end{definition}
\begin{remark}\itemfix
    % TODO
    \begin{itemize}
        \item \(\lambda x.M\) --- функция от \(x\), возвращающая частный случай терма \(M\). \?
        \item \(MN\) есть подстановка функций. \?
    \end{itemize}
\end{remark}

\begin{definition}
    \textbf{\(\alpha\)-конверсия} --- отношение, переписывающее связанные переменные. \textbf{\(\alpha\)-эквивалентность} есть рефлексивное, транзитивное и симметричное замыкание \(\alpha\)-конверсии.
\end{definition}
\begin{example}
    \(\lambda x.x + 3\) и \(\lambda y.y + 3\) \(\alpha\)-эквивалентны.
\end{example}
\begin{definition}
    \textbf{Лямбда-терм} есть претерм с точностью до \(\alpha\)-эквивалентности.
\end{definition}

Помимо грамматики, нам еще нужно определить операционную семантику.

\begin{definition}[\(\beta\)-редукция]
    Лямбда-терм \(M\) \(\beta\)-редуцируется к \(N\), если существует последовательность переписываний по следующим правилам:
    \[(\lambda x.M)N \to_\beta M[x \coloneqq N]\]
    \begin{center}
        \begin{prooftree}
            \hypo{M_1 \to_\beta M_2}
            \infer1{M_1 N \to_\beta M_2 N}
        \end{prooftree}
    \end{center}
    \begin{center}
        \begin{prooftree}
            \hypo{M_1 \to_\beta M_2}
            \infer1{NM_1 \to_\beta NM_2}
        \end{prooftree}
    \end{center}
\end{definition}

\begin{definition}
    Терм вида \((\lambda x.M)N\) называется \textbf{редексом}.
\end{definition}
\begin{definition}
    Терм в \textbf{нормальной форме}, если он не содержит редексов.
\end{definition}

\begin{obozn}
    \(M \twoheadrightarrow_\beta N\) обозначает ``\(M\) \(\beta\)-редуцируется к \(N\) в несколько шагов''.
\end{obozn}

\? % TODO

\subsection{Методы редукции}

\begin{remark}
    Применение левоассоциативно.
\end{remark}

\begin{example}
    Попробуем редуцировать терм \((\lambda xy.x)(\lambda z.z)((\lambda x.x x)(\lambda x.x x))\)

    С одной стороны:
    \begin{align*}
        (\lambda xy.x)(\lambda z.z)((\lambda x.x x)(\lambda x.x x))             & \to_\beta \\
        (\lambda y.[x \coloneqq (\lambda z.z)])((\lambda x.x x)(\lambda x.x x)) & \to_\beta \\
        (\lambda y.\lambda z.z)((\lambda x.x x)(\lambda x.x x))                 & \to_\beta \\
        (\lambda z.z)[y \coloneqq (\lambda x.x x)(\lambda x.x x)]               & \to_\beta \\
        \lambda z.z
    \end{align*}

    С другой стороны:
    \begin{align*}
        (\lambda xy.x)(\lambda z.z)((\lambda x.x x)(\lambda x.x x)) & \to_\beta \\
        (\lambda xy.x)(\lambda z.z)((x x)[x = \lambda x.x x])       & \to_\beta \\
        (\lambda xy.x)(\lambda z.z)((\lambda x.x x)(\lambda x.x x)) & \to_\beta \\
                                                                    & \vdots
    \end{align*}

    И так бесконечно.
\end{example}

Таким образом, одной стратегией мы привели результат к нормальной форме, а другой --- нет, мы зациклились. Несложно заметить, что порядок редукции важен. Проблема редукции относится только к применению функций.

Рассмотрим \((\lambda x_1\dots x_n.M)N_1 \dots N_n\). Есть два порядка редукций:
\begin{enumerate}
    \item \textbf{Аппликативный}: сначала редуцируем \(N_i\) для всех \(i \in \{1 \dots n\}\)
    \item \textbf{Нормальный}: сначала редуцируем \((\lambda x_1 \dots x_n.M)N_i\) для всех \(i \in \{1 \dots n\}\).
\end{enumerate}

Аппликативный порядок называется вызовом \textbf{по значению} и используется в традиционных ЯП: C, Java, Python .... Нормальный порядок называется вызовом \textbf{по имени}. В Haskell используется вызов \textbf{по необходимости}, который близок к вызову по имени с небольшими изменениями для возможности работы в реальном мире.

По следующей теореме нормальный порядок лучше:
\begin{theorem}
    Пусть \(M\) --- терм с нормальной формой \(M'\), тогда \(M\) можно редуцировать к \(M'\) с помощью нормального порядка редукции.
\end{theorem}

\subsection{Типы}

Типы были созданы как альтернатива теории множеств, однако в программировании они служат другим целям:
\begin{itemize}
    \item Частичная спецификация поведения программы
    \item Проверка типов позволяет отлавливать простые ошибки, т.е. сложение числа со строкой не скомпилируется, если нет приведения типов, как в JS.
\end{itemize}

Существуют следующие \textit{(грубые)} классификации систем типов:
\begin{itemize}
    \item Статическая или динамическая:
          \begin{itemize}
              \item C, C++, Java, Haskell
              \item JavaScript, Ruby, PHP
          \end{itemize}
    \item Неявная и явная:
          \begin{itemize}
              \item Ruby, JS
              \item Java, C++, C
          \end{itemize}
    \item Выведенная:
          \begin{itemize}
              \item Haskell, ML, OCaml
          \end{itemize}
\end{itemize}

\subsection{Типизированное лямбда-исчисление}

Это похоже на матлогику.

\begin{center}
    \begin{prooftree}
        \infer0{\Gamma, x : A \vdash x : A}
    \end{prooftree}
    \qquad
    \begin{prooftree}
        \hypo{\Gamma, x : A \vdash M : B}
        \infer1{\Gamma, \vdash \lambda x.M : A \to B}
    \end{prooftree}
    \qquad
    \begin{prooftree}
        \hypo{\Gamma \vdash M : A \to B}
        \hypo{\Gamma \vdash N : A}
        \infer2{\Gamma \vdash MN : B}
    \end{prooftree}
\end{center}

\(\Gamma\) есть конечный набор утверждений вида \(x : A\), обозначающих ``\(x\) имеет тип \(A\)''.

Второе правило значит, что написав код \textit{(терм \(M\))}, возвращающий переменную типа \(B\) и использующий некоторую переменную \(x\) типа \(A\), можно абстрагироваться по этой переменной \textit{(рефакторнуть)} и получить функцию \(A \to B\).

Третье правило значит, что если подставить в функцию \(A \to B\) переменную типа \(A\), мы получим переменную типа \(B\).

Функции высшего порядка --- функции, которые принимают другие функции как аргумент. В нетипизированном лямбда-исчислении все функции высшего порядка. В типизированном лямбда-исчислении функция высшего порядка, если она имеет вид \(A \to (B \to C)\).

\subsection{Полиморфизм}

Пример параметрического полиморфизма на типах в Haskell:

\begin{minted}{haskell}
changeTwiceBy :: (a -> a) -> a -> a
changeTwiceBy operation value =
    operation (operation value)
\end{minted}

Здесь \texttt{a} --- произвольный тип, т.е. спрятан квантор ``\(\forall\)''.

System F --- полиморфное лямбда-исчисление, которое было создано в 1970-х. Хотя оно создавалось для исследование арифметики второго порядка, но для программистов это формализованное представление параметрического полиморфизма. Полиморфизм System F реализован в Haskell через расширение \texttt{RankNTypes}.
