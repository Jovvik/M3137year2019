\documentclass[12pt, a4paper]{article}

\usepackage{lastpage}
\usepackage{mathtools}
\usepackage{xltxtra}
\usepackage{libertine}
\usepackage{amsmath}
\usepackage{amsthm}
\usepackage{amsfonts}
\usepackage{amssymb}
\usepackage{enumitem}
\usepackage{xcolor}
\usepackage[left=2.3cm, right=2.3cm, top=2.7cm, bottom=2.7cm, bindingoffset=0cm, headheight=15pt]{geometry}
\usepackage{fancyhdr}
\usepackage[russian]{babel}
% \usepackage{parindent}

\pagestyle{fancy}
\lfoot{M3137y2019}
\rhead{\thepage\ из \pageref{LastPage}}

\newcommand{\R}{\mathbb{R}}
\newcommand{\Q}{\mathbb{Q}}
\newcommand{\C}{\mathbb{C}}
\newcommand{\Z}{\mathbb{Z}}
\newcommand{\B}{\mathbb{B}}
\newcommand{\N}{\mathbb{N}}

\DeclareMathOperator*{\xor}{\oplus}
\DeclareMathOperator*{\equ}{\sim}
\DeclareMathOperator{\Ln}{\text{Ln}}
\DeclareMathOperator{\sign}{\text{sign}}
\DeclareMathOperator{\Sym}{\text{Sym}}
\DeclareMathOperator{\Asym}{\text{Asym}}
% \DeclareMathOperator{\sh}{\text{sh}}
% \DeclareMathOperator{\tg}{\text{tg}}
% \DeclareMathOperator{\arctg}{\text{arctg}}
% \DeclareMathOperator{\ch}{\text{ch}}

\DeclarePairedDelimiter{\ceil}{\lceil}{\rceil}

\setmainfont{Linux Libertine}

\theoremstyle{plain}
\newtheorem{theorem}{Теорема}
\newtheorem{axiom}{Аксиома}
\newtheorem{lemma}{Лемма}

\theoremstyle{remark}
\newtheorem*{remark}{Примечание}
\newtheorem*{exercise}{Упражнение}
\newtheorem*{consequence}{Следствие}
\newtheorem*{example}{Пример}
\newtheorem*{observation}{Наблюдение}

\theoremstyle{definition}
\newtheorem*{definition}{Определение}
\newtheorem*{obozn}{Обозначение}

\lhead{Линейная алгерба}
\cfoot{}
\rfoot{Лекция 3}

\begin{document}

\section{Линейный оператор}

\subsection{Основные определения}

%<*лоп>
$\sphericalangle \varphi : X\to Y$, $X, Y$ --- ЛП, $\dim X = n, \dim Y = m$

\begin{definition}
    Отображение $\varphi$ называется линейным, если
    $$\forall x_1,x_2\in X \quad \varphi(x_1+x_2)=\varphi(x_1)+\varphi(x_2)$$
    $$\forall \alpha\in K \quad \varphi(\alpha x) = \alpha \varphi (x)$$
\end{definition}

\begin{definition}
    Отображение $\varphi$, обладающее свойством линейности называется \textbf{линейным оператором} (ЛОп)
\end{definition}

\begin{example}
    \begin{itemize}
        \item $\Theta : \Theta x = 0_Y$ --- нулевой оператор
        \item $\mathcal I : \mathcal Ix = x$ --- единичный \textit{(тождественный)} оператор
        \item $X = L_1 \dot+ L_2 \stackrel{def}{\Leftrightarrow} \forall x\in X \ \ \exists! x_1\in L_1, x_2\in L_2 : x = x_1 + x_2$
        
        Проектор: $$\mathcal{P}_{L_1}^{||L_2} : X\to L_1 \quad \mathcal{P}_{L_1}^{||L_2} x = x_1$$

        $$\mathcal{P}_{L_2}^{||L_1} : X\to L_2 \quad \mathcal{P}_{L_2}^{||L_1} x = x_2$$
        
        \item $X = C^1[-1, 1]$ --- первая производная $\exists$ и непрерывна
        
        $$\forall f\in X \quad (\varphi f)(x) = \int\limits_{-1}^1 f(t) K(x, t) dt$$

        $K(x,t)$ --- интегральное ядро, например $x^2+tx$
    \end{itemize}
\end{example}

$\{e_j\}_{j=1}^n$ --- базис $X, \{h_k\}_{k=1}^m$ --- базис $Y, \varphi(e_j) = \sum\limits_{k=1}^m a_j^k h_k$

\begin{definition}
    Набор коэффициентов $||a_j^k||$ образует матрицу $m\times n$, которая называется \textbf{матрицей ЛОп} в паре базисов $\{e_j\}$ и $\{h_k\}$
\end{definition}
%</лоп>

\begin{itemize}
   \item $\Theta \to A_\Theta = 0_{m\times n}$
   \item $\mathcal I \to A_{\mathcal I} = E$
\end{itemize}

\begin{theorem}
    Задание ЛОп $\varphi$ эквивалентно заданию его матрицы в известной паре базисов пространств $X$ и $Y$
\end{theorem}
\begin{proof}
    ``$\Rightarrow$'' очевидно

    ``$\Leftarrow$'' $] A$ --- матрица ЛОп $\varphi \Rightarrow \sphericalangle x\in X, x=\sum\limits_{j=1}^n \xi^j e_j$
    $$\sphericalangle \varphi(x) = \varphi(\sum\limits_{j=1}^n \xi^j e_j) = \sum\limits_{j=1}^n \xi^j \varphi(e_j) = \sum\limits_{j=1}^n \sum\limits_{k=1}^m \xi^j a_j^k h_k = \sum\limits_{k=1}^m \left(\sum\limits_{j=1}^n \xi^j a_j^k\right) h_k$$
\end{proof}

%<*пространстволоп>
$\sphericalangle \varphi, \psi : X\to Y$ --- ЛОп

$\chi = \varphi + \psi$, если $\forall x\in X \quad \chi(x) = (\varphi + \psi) x = \varphi(x) + \psi(x)$

$\chi = \alpha\varphi$, если $\forall x\in X \quad \chi(x) = (\alpha\varphi) x = \alpha\varphi(x)$
%</пространстволоп>

\begin{remark}
    %<*размерностьпространствалоп>
    $$\dim \mathcal{L}(X, Y) = \dim X \cdot \dim Y = m\cdot n$$
    %</размерностьпространствалоп>
\end{remark}
\begin{remark}
    $] K_n^m$ --- множество матриц $m\times n$

    $K_n^m$ --- линейное пространство

    $\dim K_n^m=m\cdot n \Rightarrow \mathcal{L}(X, Y)\simeq K_n^m$
\end{remark}

\subsection{Алгебра операторов и матриц}

%<*алгебралоп>
$\sphericalangle \varphi : X \to Y \quad \psi : Y\to Z$

$X, Y, Z$ --- ЛП: $\dim X = n, \dim Y = m, \dim Z = k$

$\sphericalangle \sigma : X\to Z : \forall x \quad \sigma(x)=\psi(\varphi x)$

\begin{definition}
    Отображение $\sigma$ называется \textbf{произведением} \textit{(композицией)} $\psi$ и $\varphi$ 
\end{definition}

\begin{lemma}
    $\sigma$ --- ЛОп
\end{lemma}
\begin{proof}
    $\sphericalangle \sigma(x+y)=\psi(\varphi(x+y))=\psi(\varphi x + \varphi y) = \psi\varphi x + \psi \varphi y = \sigma x + \sigma y$
\end{proof}

$] \{e_i\}_{i=1}^n$ --- базис $X$, $\{h_j\}_{j=1}^m$ --- базис $Y$, $\{g_l\}_{l=1}^k$ --- базис $Z$

$\varphi\to A=||a^j_i|| \quad \psi \to B=||b^l_j|| \quad \sigma \to C=||c_i^l||$

\begin{theorem}
    $\sigma = \psi\varphi \Leftrightarrow C=BA$
\end{theorem}
\begin{proof}
    $\sigma(e_i)=\psi(\varphi(e_i))=\psi(\sum\limits_{j=1}^m a_i^j h_j) = ,\sum\limits_{j=1}^m a_i^j \psi(h_j)=\sum\limits_{j=1}^m a_i^j \sum\limits_{l=1}^k b_j^l g_l = \sum\limits_{l=1}^k \left(\sum\limits_{j=1}^m a_i^j b_j^l \right) g_l$

    $c_i^l=\sum\limits_{j=1}^m a_i^j b_j^l \Rightarrow C=BA$
\end{proof}

$\sphericalangle \varphi, \psi : X\to X \Rightarrow \sigma = \psi \varphi : X\to X$

Свойства композиции:
\begin{enumerate}
    \item $(\varphi+\psi)\sigma = \varphi\sigma + \psi\sigma$
    \begin{proof}
        $(\varphi+\psi)\sigma x = (\varphi + \psi)(\sigma x) = \varphi\sigma x + \psi \sigma x$
    \end{proof}
    \item $\varphi(\psi + \sigma)=\varphi\psi + \varphi\sigma$
    \begin{remark}
        $\varphi\psi\not=\psi\varphi$
    \end{remark}
    \item $\varphi(\alpha\psi) = (\alpha \varphi)\psi = \alpha \varphi \psi$
    \item $\varphi(\psi \sigma) = (\varphi\psi)\sigma = \psi \varphi \sigma$
\end{enumerate}

Таким образом, ЛОп --- алгебра, т.к. это мультипликативный моноид и аддитивная полугруппа.
%</алгебралоп>

\begin{definition}
    Множество, наделенное согласованными структурами линейного пространства и мультипликативного моноида, называется алгеброй.
\end{definition}

\begin{theorem}
    $\mathcal{L}(X, X) = X\times X$ --- алгебра.
\end{theorem}
\begin{proof}
    См. выше
\end{proof}

$] A$ --- алгебра

$] \{e_j\}_{j=1}^n$ --- базис $A$

$$x,y\in X \quad x=\sum\limits_{j=1}^n \xi^je_j \quad y=\sum\limits_{k=1}^n \eta^ke_k$$
$$\sphericalangle x\cdot y = \sum\limits_{j,k=1}^n\xi^j\eta^k(e_j\cdot e_k) = \sum\limits_{j,k=1}^n \xi^j\eta^k\sum\limits_{l=1}^n m^l_{jk}e_l=\sum\limits_{l=1}^n\left(\sum\limits_{j,k=1}^n m^l_{jk}\xi^j\eta^k \right)e_l$$

\begin{definition}
    Набор $m^l_{jk}$ называется структуной константой алгебры $A$.
\end{definition}

\begin{example}
    $\sphericalangle \mathbb{C} \quad \{1, i\}$ --- базис $\mathbb{C}$

    $$\begin{bmatrix}
        m^l_{jk} & 1 & i \\
        1 & 1 & i \\
        i & i & -1
    \end{bmatrix} \text{ --- таблица Кэли}$$
\end{example}
\begin{example}
    $\R^4 \quad \{1\ i\ j\ k\}$

    $$\begin{bmatrix}
        & 1 & i & j & k \\
        1 & 1 & i & j & k \\
        i & i & -1 & k & -j \\
        j & j & -k & -1 & i \\
        k & k & j & -i & -1
    \end{bmatrix}$$
\end{example}

$] A_1, A_2$ --- алгебры

\begin{definition}
    $A_1$ и $A_2$ называются \textbf{изоморфными}, если существует биекция, сохраняющая их алгебраическую структуру:
    $$\forall x_1, x_2\in A_1 \quad x_1\leftrightarrow y_1, x_2\leftrightarrow y_2 \Rightarrow (x_1+x_2)\leftrightarrow (y_1+y_2) \quad \alpha x_1 \leftrightarrow \alpha y_1 \quad x_1x_2 \leftrightarrow y_1y_2$$
\end{definition}

\end{document}