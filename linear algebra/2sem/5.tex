\documentclass[12pt, a4paper]{article}

\usepackage{lastpage}
\usepackage{mathtools}
\usepackage{xltxtra}
\usepackage{libertine}
\usepackage{amsmath}
\usepackage{amsthm}
\usepackage{amsfonts}
\usepackage{amssymb}
\usepackage{enumitem}
\usepackage{xcolor}
\usepackage[left=2.3cm, right=2.3cm, top=2.7cm, bottom=2.7cm, bindingoffset=0cm, headheight=15pt]{geometry}
\usepackage{fancyhdr}
\usepackage[russian]{babel}
% \usepackage{parindent}

\pagestyle{fancy}
\lfoot{M3137y2019}
\rhead{\thepage\ из \pageref{LastPage}}

\newcommand{\R}{\mathbb{R}}
\newcommand{\Q}{\mathbb{Q}}
\newcommand{\C}{\mathbb{C}}
\newcommand{\Z}{\mathbb{Z}}
\newcommand{\B}{\mathbb{B}}
\newcommand{\N}{\mathbb{N}}

\DeclareMathOperator*{\xor}{\oplus}
\DeclareMathOperator*{\equ}{\sim}
\DeclareMathOperator{\Ln}{\text{Ln}}
\DeclareMathOperator{\sign}{\text{sign}}
\DeclareMathOperator{\Sym}{\text{Sym}}
\DeclareMathOperator{\Asym}{\text{Asym}}
% \DeclareMathOperator{\sh}{\text{sh}}
% \DeclareMathOperator{\tg}{\text{tg}}
% \DeclareMathOperator{\arctg}{\text{arctg}}
% \DeclareMathOperator{\ch}{\text{ch}}

\DeclarePairedDelimiter{\ceil}{\lceil}{\rceil}

\setmainfont{Linux Libertine}

\theoremstyle{plain}
\newtheorem{theorem}{Теорема}
\newtheorem{axiom}{Аксиома}
\newtheorem{lemma}{Лемма}

\theoremstyle{remark}
\newtheorem*{remark}{Примечание}
\newtheorem*{exercise}{Упражнение}
\newtheorem*{consequence}{Следствие}
\newtheorem*{example}{Пример}
\newtheorem*{observation}{Наблюдение}

\theoremstyle{definition}
\newtheorem*{definition}{Определение}
\newtheorem*{obozn}{Обозначение}

\lhead{Линейная алгерба}
\cfoot{}
\rfoot{Лекция 5}

\begin{document}

\section{Спектральный анализ линейных операторов}

\subsection{Инвариантные пространства линейного оператора}

$\sphericalangle \varphi : X\to X$ --- автоморфизм

\begin{definition}
    Подпространство $L$ линейного пространства $X$ называется \textbf{инвариантным подпрострнаством} $\varphi$, если
    $$\forall x\in L \quad \varphi x \in L$$
\end{definition}
\begin{example}
    \begin{enumerate}
        \item $\varphi : X\to X$, тогда инвариантные подпрострнаства:
        \begin{itemize}
            \item $X$
            \item $\{0\}$
        \end{itemize}
        \item $\varphi = \Im, \quad \forall x\ \ \Im x = x \Rightarrow$ любое подпространство $X$ --- инвариантное
        \item $\varphi = \Theta, \quad \forall x\ \ \Theta x = 0 \Rightarrow$ любое подпространство $X$ --- инвариантное
        \item $\varphi : \R^n \to \R^n \Leftrightarrow A_\varphi = \begin{bmatrix}
            \lambda_1 & 0 & \ldots & 0 \\
            0 & \lambda_2 & \ldots & \vdots \\
            \vdots & \vdots & \ddots & \vdots \\
            0 & \ldots & \ldots & \lambda_n
        \end{bmatrix}\stackrel{\triangle}{=} diag\{\lambda_1\ldots \lambda_n\}$

        $\sphericalangle \{e_j\}$ --- базис $X \Rightarrow \forall j \quad A_\varphi e_j=\lambda_j e_j \quad e_j\to \mathcal{L}\{e_j\}$ --- инв.

        Всего $2^n$ инвариантных подпространств

        \item $] X = L_1 \dot{+} L_2$
        $$\forall x!=x_1+x_2 \quad \varphi x = \mathcal{P}_{L_1}^{\parallel L_2} x = x_1\in L_1$$
        $L_1$ --- инв., $\forall x\in L_1 \quad \mathcal{P}_{L_1}^{\parallel L_2} x = x \quad \forall$ подпространство $L_1$ инвариантно

        $L_2$ --- инв., $\forall x\in L_2 \quad \mathcal{P}_{L_1}^{\parallel L_2} x = 0 \quad \forall$ подпространство $L_2$ инвариантно
    \end{enumerate}
\end{example}

\begin{definition}
    \textbf{Инвариантном} линейного оператора $\varphi$ называется его числовая функция значений, которая не зависит от выбора базиса
\end{definition}

\begin{example}
    $\det \varphi$ --- инвариант

    $$\varphi^{\Lambda_n} z = \det \varphi \cdot z \quad \forall z\in\Lambda_n$$
    $$\det \varphi = \det A_\varphi \text{ --- в некотором фиксированном базисе}$$
    $$\tilde A_\varphi=T^{-1}A_\varphi T \quad \det \tilde A_\varphi = \det T^{-1}\det A_\varphi\det T = \det A_\varphi$$
\end{example}

\begin{definition}
    \textbf{Характеристическим полиномом} линейного оператора $\varphi$ называется определитель следующего вида:
    $$\chi_\varphi(\lambda)=\det (\varphi - \lambda \Im) \stackrel{\{e_j\}}{=} \det\{A_\varphi - \lambda E\}$$

    $$\chi_\varphi(\lambda)\stackrel{def \det}{=}\sum\limits_{(j_1\ldots j_n)}(-1)^{[j_1\ldots j_n]} \prod\limits_{i=1}^n (a_{ij_i}-\delta_{i, j_i}\lambda) = \det A_\varphi-\lambda\sum\limits_{i=1}^n \left(\prod\limits_{j=1}^n a_{ji}\right)(-1)^i+\ldots+\lambda^{n-1}\sum\limits_{i=1}^n a_{ii} + \lambda^n (-1)^n=$$
    $$=(-\lambda)^nZ^{(0)}+(-\lambda)^{n-1}Z^{(1)}+\ldots + (-\lambda)Z^{(n-1)}+Z^{(n)}$$
    $$Z^{(K)}=\sum\limits_{(1\leq i_1<\ldots <i_k\leq n)}M_{\vec i}^{\vec i}$$
\end{definition}

\begin{lemma}
    $$\chi_\varphi(\lambda)= inv$$
\end{lemma}
\begin{proof}
    $$\det (\tilde A_\varphi - \lambda E) = \det (T^{-1}A_\varphi T - \lambda T^{-1}T)=\det[S(A_\varphi - \lambda E)T]=\det T^{-1} \det (A_\varphi - \lambda E) \det T=\det (A_\varphi-\lambda E)$$
\end{proof}

\end{document}