\documentclass[12pt, a4paper]{article}

\usepackage{lastpage}
\usepackage{mathtools}
\usepackage{xltxtra}
\usepackage{libertine}
\usepackage{amsmath}
\usepackage{amsthm}
\usepackage{amsfonts}
\usepackage{amssymb}
\usepackage{enumitem}
\usepackage{xcolor}
\usepackage[left=2.3cm, right=2.3cm, top=2.7cm, bottom=2.7cm, bindingoffset=0cm, headheight=15pt]{geometry}
\usepackage{fancyhdr}
\usepackage[russian]{babel}
% \usepackage{parindent}

\pagestyle{fancy}
\lfoot{M3137y2019}
\rhead{\thepage\ из \pageref{LastPage}}

\newcommand{\R}{\mathbb{R}}
\newcommand{\Q}{\mathbb{Q}}
\newcommand{\C}{\mathbb{C}}
\newcommand{\Z}{\mathbb{Z}}
\newcommand{\B}{\mathbb{B}}
\newcommand{\N}{\mathbb{N}}

\DeclareMathOperator*{\xor}{\oplus}
\DeclareMathOperator*{\equ}{\sim}
\DeclareMathOperator{\Ln}{\text{Ln}}
\DeclareMathOperator{\sign}{\text{sign}}
\DeclareMathOperator{\Sym}{\text{Sym}}
\DeclareMathOperator{\Asym}{\text{Asym}}
% \DeclareMathOperator{\sh}{\text{sh}}
% \DeclareMathOperator{\tg}{\text{tg}}
% \DeclareMathOperator{\arctg}{\text{arctg}}
% \DeclareMathOperator{\ch}{\text{ch}}

\DeclarePairedDelimiter{\ceil}{\lceil}{\rceil}

\setmainfont{Linux Libertine}

\theoremstyle{plain}
\newtheorem{theorem}{Теорема}
\newtheorem{axiom}{Аксиома}
\newtheorem{lemma}{Лемма}

\theoremstyle{remark}
\newtheorem*{remark}{Примечание}
\newtheorem*{exercise}{Упражнение}
\newtheorem*{consequence}{Следствие}
\newtheorem*{example}{Пример}
\newtheorem*{observation}{Наблюдение}

\theoremstyle{definition}
\newtheorem*{definition}{Определение}
\newtheorem*{obozn}{Обозначение}

\lhead{Линейная алгерба}
\cfoot{}
\rfoot{Лекция 8}

\begin{document}

\section*{Спектральная теорема для оператора общего вида}

\begin{definition}
    Операторный полином $p\in \mathcal P_\infty [K]$ называется аннулирующим полиномом линейного оператора $\varphi$, если $p(\varphi)=0$
\end{definition}
\begin{remark}
    Множество аннулирующих полиномов операторов $\varphi$ --- ядро гомоморфизма $S_\varphi$ по определению.
\end{remark}

\begin{theorem}
    Аннулирующий полином существует.
\end{theorem}
\begin{proof}
    $\dim \mathcal P [\varphi] = n^2 \Rightarrow \exists n^2$ ЛНЗ элементов. Эти элементы : $\varphi, \varphi^2 \ldots \varphi^{n^2}$. Тогда $\{\mathcal I, \varphi, \varphi^2 \ldots \varphi^{n^2}\}$ --- ЛЗ
    $$\Rightarrow \exists p[\varphi] = \sum_{i=0}^{n^2} \alpha_i\varphi^i = 0 \Rightarrow \exists$$
\end{proof}

$] J_\varphi$ --- множество аннулирующих полиномов оператора $\varphi$
\begin{lemma}
    $J_\varphi$ --- идеал в $P_\infty [K]$
\end{lemma}
\begin{proof}
    $] p\in J_\varphi \Rightarrow p(\varphi)=0$

    $] q\in P_\infty[K]$

    $\sphericalangle p(\lambda)q(\lambda) \xrightarrow{S_\varphi} p(\varphi)q(\varphi)=0 \Rightarrow p(\lambda)q(\lambda)$ --- аннулирующий $\Rightarrow p(\lambda)q(\lambda)\in J_\varphi$
\end{proof}
\begin{definition}
    \textbf{Минимальным аннулирующим полиномом} оператора $\varphi$ назвыается мнимальнй полином $J_\varphi$
\end{definition}
\begin{remark}
    Обозначение минимального полинома: $p_\varphi(\lambda) \leftrightarrow p_\varphi(\varphi)=0$
\end{remark}

\begin{example}
    $] \varphi : X\to X$ --- оператор с простым спектром

    $] \chi_\varphi(\lambda)$ --- характеристический полином $\varphi \Rightarrow \chi_\varphi(\lambda)=p_\varphi(\lambda)$
\end{example}
\begin{proof}
    $$\varphi=\sum_{i=1}^n \lambda_i\mathcal P_i \Rightarrow \chi_\varphi(\varphi)=\sum_{i=1}^n \chi_\varphi(\lambda_i)\mathcal P_i=0$$
    Предположим обратное: $] p_\varphi(\lambda)$ --- минимальный полином, такой что $\deg p_\varphi < \deg \chi_\varphi$

    $] \chi_\varphi(\lambda)=(\lambda-\lambda_k)p_\varphi(\lambda)$

    $$\sphericalangle p_\varphi(\varphi)=\sum_{i=1}^n p_\varphi(\lambda_i)\mathcal P_i = p(\lambda_k)\mathcal P_k \Rightarrow p_\varphi(\varphi)\not=0 \Rightarrow \text{противоречие}$$
\end{proof}

\begin{lemma}
    $] p(\varphi)=q(\varphi) \Leftrightarrow [p(\lambda)-q(\lambda)] \mid p_\varphi(\lambda)$
\end{lemma}
\begin{proof}
    $\sphericalangle p(\lambda)-q(\lambda)=0 \Rightarrow p(\lambda)-q(\lambda)\in J_\varphi$
\end{proof}

\begin{lemma}
    $] p(\lambda)=q(\lambda)p_\varphi(\lambda)+r(\lambda) \Rightarrow p(\varphi)=r(\varphi)$
\end{lemma}

\begin{theorem}
    $\sphericalangle p_\varphi=p_1\ldots p_k$, $p_1\ldots p_k$ --- взаимно простые
    $$\Rightarrow \dot+\sum_{j=1}^k \ker p_j(\varphi)=X$$
\end{theorem}
\begin{proof}
    $$\ker p_\varphi(\varphi)=\dot+\sum_{j=1}^k \ker p_j(\varphi)$$
    $$\ker p_\varphi(\varphi)=\ker 0 = X$$
\end{proof}
\begin{theorem}
    О ядре и образе.

    $] p_\varphi(\lambda)=p_1(\lambda)p_2(\lambda) \Rightarrow \ker p_1(\varphi)=\im p_2(\varphi)$
\end{theorem}
\begin{proof}
    Покажем, что:
    \begin{enumerate}
        \item $\im p_2(\varphi)\subset \ker p_1(\varphi)$
        \item $\dim \im p_2(\varphi)=\dim \ker p_1(\varphi)$
    \end{enumerate}

    \begin{enumerate}
        \item $\im p_2(\varphi)\subset \ker p_1(\varphi)$
        
        $] y \in \im p_2(\varphi) \Rightarrow \exists x\in X : y=p_2(\varphi) x$

        $\sphericalangle p_1(\varphi) y = p_1(\varphi)p_2(\varphi) x = p_\varphi(\varphi) = 0$

        \item $\ker p_\varphi(\varphi) = \ker p_1(\varphi) \dot+ \ker p_2(\varphi) \Rightarrow$
        $$\dim X = \dim \ker p_1(\varphi) + \dim \ker p_2(\varphi)$$
        $$\dim X = \dim \ker p_2(\varphi) + \dim \im p_2(\varphi)$$
        $$\dim \ker p_1(\varphi)=\dim \im p_2(\varphi)$$
    \end{enumerate}
\end{proof}

\begin{theorem}
    $] p_\varphi(\lambda) = \prod\limits_{i=1}^k p_i(\lambda)$ --- минимальный аннулирующий полином $\varphi$, $p_1\ldots p_k$ --- взаимно простые делители

    $\Rightarrow$
    \begin{enumerate}
        \item $\sum\limits_{j=1}^k p_j'(\varphi)q_j(\varphi)=\mathcal I, \quad p_j'=\frac{p_\varphi}{p_j}$
        \item $p_j'(\varphi)q_j(\varphi)=\mathcal P_{L_j} \quad L_j=\ker p_j(\varphi)$
    \end{enumerate}
\end{theorem}
\begin{proof}
    $\sphericalangle p_\varphi(\lambda) = p_1(\lambda)p_2(\lambda)\ldots p_k(\lambda) \quad \exists q_1\ldots q_k :$
    $$\sum_{j=1}^k p_j'(\lambda)q_j(\lambda)=1 \xrightarrow{S_\varphi} \sum_{j=1}^n p_j'(\varphi)q_j(\varphi)=\mathcal I$$
    $] p_1(\lambda)=p_i(\lambda), p_2(\lambda)=p_i'(\lambda) \Rightarrow \im p_1(\varphi) = \ker p_2(\varphi)$

    $\sphericalangle \mathcal P_{L_1} x = p_i'(\varphi)q(\varphi) \in \ker p_i(\varphi)$, т.к.
    $$p_i(\varphi)[p_i'(\varphi)q_i(\varphi) x]=p_i(\varphi)p_i'(\varphi)q_i(\varphi)x=p_\varphi(\varphi)q_i(\varphi)x=0$$

    Осталось доказать, что $\mathcal P_{L_i}\mathcal P_{L_j}=\delta_i^j \mathcal P_{L_i}$

    $$] i\not=j \Rightarrow \mathcal P_{L_i}\mathcal P_{L_j}=p_i'(\varphi)q_i(\varphi)p_j'(\varphi)q_j(\varphi)=\frac{p_\varphi(\varphi)}{p_i(\varphi)p_j(\varphi)}q_i(\varphi)q_j(\varphi)p_\varphi(\varphi)=0$$
    $$] i = j \Rightarrow \mathcal P_{L_i} (x) = \mathcal P_{L_i} (\mathcal I \cdot x)=\mathcal P_{L_i} \left(\sum_{j=1}^n \mathcal P_{L_j}\right) x=\mathcal P_{L_i}\mathcal P_{L_i} x \quad \forall x$$
    $$\Rightarrow \mathcal P_{L_i}\mathcal P_{L_i}=\mathcal P_{L_i}$$
\end{proof}

\section*{Ультраинвариантные подпространства}

$] \varphi : X\to X, \dim X = n$

$L\subset X$ --- инвариантное подпространство $\varphi$, если $\varphi(L)\subset L$

\begin{definition}
    Инвариантное подпространство называется \textbf{ультраинвариантным подпространством}, если существует его дополнение $L'$, такое что:
    $$L\dot + L'=X \quad L' \text{ --- инвариантное подпространство }\varphi$$
\end{definition}

$L$ --- инвариантное подпространство оператора $\varphi$

\begin{definition}
    Оператор $\varphi_L : L\to L$, такой что:
    $$\varphi_L x =\varphi x \quad \forall x\in L$$
    называется \textbf{сужением} оператора $\varphi$ на $L$.

    Если $L$ --- ультраинвариантное подпространство, то $\varphi_L$ называется \textbf{компонетной} $\varphi$ в $L$ 
\end{definition}

\begin{lemma}
    Дополнение $L'$ ультраинвариантного подпространства $L$ является ультраинвариантным подпространством.
\end{lemma}
\begin{lemma}
    $] X = L \dot+ L' \quad L,L'$ --- ультраинвариантное подпространства $\Rightarrow$
    $$\varphi = \varphi_L \mathcal P_L^{\parallel L'} + \varphi_{L'} \mathcal P_{L'}^{\parallel L}$$
\end{lemma}
\begin{proof}
    $$X = L \dot+ L' \Rightarrow \forall x! = x_1+x_2 = \mathcal P_L^{\parallel L'} x + \mathcal P_{L'}^{\parallel L} x$$
    $$\varphi x = \varphi \mathcal P_L^{\parallel L'} x + \varphi \mathcal P_{L'}^{\parallel L} x \quad \forall x \quad \Rightarrow$$
    $$\Rightarrow \varphi=\varphi_L \mathcal P_L^{\parallel L'} + \varphi_{L'} \mathcal P_{L'}^{\parallel L} \quad (*)$$
\end{proof}

\begin{remark}
    Запись $(*)$ эквивалентна записи
    $$\varphi = \varphi_L \dot+ \varphi_{L'}$$
\end{remark}

\begin{definition}
    Инвариантное подпространство называется \textbf{минимальным}, если оно не содержит внутри себя нетривиальных инвариантных подпространств меньшей размерности.
\end{definition}

\end{document}