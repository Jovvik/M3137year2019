\documentclass[12pt, a4paper]{article}

\usepackage{lastpage}
\usepackage{mathtools}
\usepackage{xltxtra}
\usepackage{libertine}
\usepackage{amsmath}
\usepackage{amsthm}
\usepackage{amsfonts}
\usepackage{amssymb}
\usepackage{enumitem}
\usepackage{xcolor}
\usepackage[left=2.3cm, right=2.3cm, top=2.7cm, bottom=2.7cm, bindingoffset=0cm, headheight=15pt]{geometry}
\usepackage{fancyhdr}
\usepackage[russian]{babel}
% \usepackage{parindent}

\pagestyle{fancy}
\lfoot{M3137y2019}
\rhead{\thepage\ из \pageref{LastPage}}

\newcommand{\R}{\mathbb{R}}
\newcommand{\Q}{\mathbb{Q}}
\newcommand{\C}{\mathbb{C}}
\newcommand{\Z}{\mathbb{Z}}
\newcommand{\B}{\mathbb{B}}
\newcommand{\N}{\mathbb{N}}

\DeclareMathOperator*{\xor}{\oplus}
\DeclareMathOperator*{\equ}{\sim}
\DeclareMathOperator{\Ln}{\text{Ln}}
\DeclareMathOperator{\sign}{\text{sign}}
\DeclareMathOperator{\Sym}{\text{Sym}}
\DeclareMathOperator{\Asym}{\text{Asym}}
% \DeclareMathOperator{\sh}{\text{sh}}
% \DeclareMathOperator{\tg}{\text{tg}}
% \DeclareMathOperator{\arctg}{\text{arctg}}
% \DeclareMathOperator{\ch}{\text{ch}}

\DeclarePairedDelimiter{\ceil}{\lceil}{\rceil}

\setmainfont{Linux Libertine}

\theoremstyle{plain}
\newtheorem{theorem}{Теорема}
\newtheorem{axiom}{Аксиома}
\newtheorem{lemma}{Лемма}

\theoremstyle{remark}
\newtheorem*{remark}{Примечание}
\newtheorem*{exercise}{Упражнение}
\newtheorem*{consequence}{Следствие}
\newtheorem*{example}{Пример}
\newtheorem*{observation}{Наблюдение}

\theoremstyle{definition}
\newtheorem*{definition}{Определение}
\newtheorem*{obozn}{Обозначение}

\lhead{Линейная алгерба}
\cfoot{}
\rfoot{Лекция 9}

\begin{document}

\section*{Нильпотентный оператор. Базис Жордана}

$\sphericalangle \tau : L \to L$ --- нильпотентный оператор

$] \{e_j\}_{j=1}^n$ --- базис $L \Rightarrow \tau \leftrightarrow A_\tau$
$$\tau(e_i)=\sum_{j=1}^n a_i^j e_j, ||a_i^j||=A_\tau$$

$$\tau(\tau(e_i))=\sum_{j=1}^n a_i^j \sum_{k=1}^n a_j^k e_k = \sum_{j,k=1}^n a_i^j a_j^ke_k$$
$$0=\tau(\ldots \tau(\tau(e_i)))=\sum_{j,k,l\ldots = 1}^n a_i^j a_j^k\cdots a_k^l e_l$$
Если хотя бы один диагональный элемент $a_i^i\not=0$, то для $j=k=\ldots=i$ получается ненулевой коэффициент при $e_i \Rightarrow$ результат суммы не $0$, что противоречит нильпотентности $\Rightarrow a_i^i=0$

Канонический вид матрицы нильпотентного оператора: $\begin{bmatrix}
    0 & 1 & 0 & 0 & \ldots & 0 \\
    0 & 0 & 1 & 0 & \ldots & 0 \\
    0 & 0 & 0 & 1 & \ldots & 0 \\
    0 & 0 & 0 & 0 & \ldots & 0 \\
    \vdots & \vdots & \vdots & \vdots & \ddots & \vdots \\
    0 & 0 & 0 & 0 & \ldots & 0 
\end{bmatrix}$

\begin{definition}
    Линейный оператор, который в некотором базисе имеет своей матрицей (жорданову) клетку вида $A_\tau$ называется \textbf{одноклеточным} нильпотентным оператором.
\end{definition}

\begin{lemma}
    $p_\tau(\lambda)=\lambda^m$ --- минимальный многочлен для $\tau^m$
\end{lemma}

$] \{e_j\}_{j=1}^4=\begin{bmatrix}
    1 \\
    0 \\
    0 \\
    0
\end{bmatrix}\begin{bmatrix}
    0 \\
    1 \\
    0 \\
    0
\end{bmatrix}\begin{bmatrix}
    0 \\
    0 \\
    1 \\
    0
\end{bmatrix}\begin{bmatrix}
    0 \\
    0 \\
    0 \\
    1
\end{bmatrix}, A_\tau=\begin{bmatrix}
    0 & 1 & 0 & 0 \\
    0 & 0 & 1 & 0 \\
    0 & 0 & 0 & 1 \\
    0 & 0 & 0 & 0 \\
\end{bmatrix}$

$A_\tau(e_4)=\begin{bmatrix}
    0 \\
    0 \\
    1 \\
    0
\end{bmatrix}=e_3, A_\tau(e_3)=e_2, A_\tau(e_2)=e_1, A_\tau(e_1)=\begin{bmatrix}
    0 \\
    0 \\
    0 \\
    0
\end{bmatrix}$

$e_1$ --- единственный собственный вектор для $\tau$, собственное значение $=0$

$\sigma_\tau=\{0^{(m)}\}$

$\sphericalangle L_j=\mathcal L\{e_1\ldots e_j\}$ --- инвариантное подпространство $\forall j$, но не ультраинвариантное, т.к. $\mathcal L\{e_{j+1}\ldots e_n\}$ --- не инвариантное подпространство.

$e_2$ --- присоединенный вектор первого порядка, т.к. $\tau e_2=e_1$

$e_3$ --- присоединенный вектор второго порядка

$] \tau$ --- не одноклеточный, тогда $$\tau=\dot+ \sum_{i=1}^k \tau_i=\sum_{i=1}^k \tau_i\mathcal P_i$$

\begin{lemma}
    $\tau$ --- нильпотентный оператор порядка $m=\max\limits_{i=1\ldots k} m_i$
\end{lemma}
\begin{proof}
    $\tau\leftrightarrow A_\tau = diag \{A_\tau^1, A_\tau^2\ldots A_\tau^k\}$, где $A_\tau^j$ одноклеточная.
    $$A^l = diam \{(A_\tau^1)^{l}, (A_\tau^2)^l \ldots (A_\tau^k)^l\} = 0 \Leftrightarrow \forall j \ \ (A_\tau^j)^l = 0 \Leftrightarrow l=\max_{i=1\ldots k}m_i$$
\end{proof}

$] \tau : X\to X$ --- нильпотентный оператор порядка $m$, тогда в $x$ $\exists$ базис, в котором:
$$\tau = \dot+ \sum_{i=1}^k \tau_i$$
где $\tau_i$ --- одноклеточный оператор.
\begin{proof}
    $] \{L_j\}_{j=1}^k$ --- ультраинвариантные для $\tau$, $k$ --- число собственных векторов оператора $\tau$.

    $L_j\to \tau_j \to T_j$ --- одноклеточный оператор.
\end{proof}

Таким образом, в базисе Жордана $\varphi_j = \lambda_j \mathcal I + \tau_j$

$] \varphi : X\to X, X = \dot+\sum_{j=1}^k L_j$

$] \beta(X)$ --- базис $X \Rightarrow \beta(X) = \{\beta(L_j)\}_{j=1}^k$

$\beta(L_j)$ --- базис Жордана в $L_j$ (чтобы $\varphi_j$ выглядел как выше)

$\beta(X)$ --- базис Жордана в пространстве $X$

\begin{definition}
    Матрица оператора $\varphi$ в базисе $\beta(x)$ называется \textbf{жордановой нормальной формой} матрицы линейного оператора $\varphi$.
\end{definition}

$\varphi = diag \{\varphi_1 \ldots \varphi_k\} \quad \varphi_j=diag \{\lambda_j \mathcal I_1 + \tau_1, \lambda_j\mathcal I_2+\tau_2\ldots\}$

\begin{theorem}
    Гамильтона-Коши.

    $$\chi_\varphi(\lambda)\in J_\varphi$$
\end{theorem}
\begin{proof}
    Тривиально.
\end{proof}

$$p_\varphi(\lambda) = \prod_{j=1}^k (\lambda-\lambda_j)^{m_j} \quad \chi_\varphi(\lambda) = \prod_{j=1}^k (\lambda - \lambda_j)^{n_j}$$
\begin{definition}
    \begin{itemize}
        \item $n_j$ --- \textbf{полная кратность} собственных значений $\lambda_j$
        \item $n_j$ --- \textbf{алгебраическая кратность} собственных значений $\lambda_j$
    \end{itemize}
\end{definition}

$$n_j = \dim L_j \quad m_j = \dim \ker(\lambda-\lambda_j) \quad r_j = \text{чило жордановых блоков}$$

\begin{lemma}
    $$1\leq m_j\leq n_j, 1\leq r_j\leq n_j$$
\end{lemma}

Частные случаи:
\begin{enumerate}
    \item $n_i=1 \Rightarrow r_i=m_i=1 \Rightarrow$ оператор с простым спектром
    \item $r_i = n_i \Leftrightarrow m_i = 1 \Rightarrow$ оператор скалярного типа
    \item $r_i = 1 \Leftrightarrow m_i=n_i \Rightarrow$ один жорданов блок
\end{enumerate}

\section*{Функции от оператора}

$$\varphi : X\to X, f(x) = \sum_{m=0}^\infty c_m x^m$$
$\sphericalangle f(\varphi)-?$

$$\varphi = \dot+ \sum_{j=1}^k \varphi_j \Rightarrow A_\varphi = diag \{A_\varphi^{(1)}, A_\varphi^{(2)}\ldots A_\varphi^{(k)}\}$$
$$f(\varphi)= diag \{f(A_\varphi^{(1)}), f(A_\varphi^{(2)})\ldots f(A_\varphi^{(k)})\}$$
$$f(\varphi_j)-? \quad \varphi_j = \lambda_j\mathcal I + \tau_j, \tau_j^{m_j}=0$$
$$\sphericalangle (\lambda_j\mathcal I + \tau_j)^m = \sum_{r=1}^m c_m^r \tau_j^r \lambda_j^{m-r}$$
Если $r\geq m_j$, то слагаемое $=0$, т.к. $\tau_j^{m_j}=0$
$$diag_0 (\lambda_j\mathcal I + \tau_j)^m = \{c_m^0\lambda_j^m, c_m^0\lambda_j^m \ldots c_m^0\lambda_j^m\}$$
$$diag_{+1} (\lambda_j\mathcal I + \tau_j)^m = \{c_m^1\lambda_j^{m-1}, c_m^1\lambda_j^{m-1} \ldots c_m^1\lambda_j^{m-1}\}$$
$$\vdots$$
$$diag_{+m_{j-1}} (\lambda_j\mathcal I + \tau_j) = \{c_m^{m-1}\lambda_j, c_m^{m-1}\lambda_j \ldots c_m^{m-1}\lambda_j\}$$
\begin{remark}
    $$diag f(\lambda_j\mathcal I + \tau_j) = \{f(\lambda_j), f(\lambda_j) \ldots f(\lambda_j)\}$$
    $$diag_{+1} f(\lambda_j\mathcal I + \tau_j) = \{f'(\lambda_j), f'(\lambda_j) \ldots f'(\lambda_j)\}$$
    $$diag_{+2} f(\lambda_j\mathcal I + \tau_j) = \{\frac{1}{2!}f''(\lambda_j), \frac{1}{2!}f''(\lambda_j) \ldots \frac{1}{2!}f''(\lambda_j)\}$$
\end{remark}
\begin{remark}
    $$\tilde A_\varphi = SA_\varphi T \quad (\tilde A_\varphi)^p = SA_\varphi^p T$$
\end{remark}

\begin{example}
    $f(x) = \sin x \quad A_\varphi=\begin{bmatrix}
        \lambda & 1 & 0 & 0 \\
        0 & \lambda & 1 & 0 \\
        0 & 0 & \lambda & 1 \\
        0 & 0 & 0 & \lambda \\
    \end{bmatrix}$
    $$f(A_\varphi) = \begin{bmatrix}
        \sin \lambda & \cos \lambda & -\frac{1}{2}\sin \lambda & -\frac{1}{6}\cos \lambda \\
        0 & \sin \lambda & \cos \lambda & -\frac{1}{2}\sin \lambda \\
        0 & 0 & \sin \lambda & \cos \lambda \\
        0 &0 & 0 & \sin \lambda
    \end{bmatrix}$$
\end{example}

% \begin{remark}
%     $f(A_\varphi) = \begin{bmatrix}
%         f(\lambda) & f'(\lambda) & \frac{f''}{2!} & 0 \\
%         0 & f(\lambda) & f'(\lambda) & 0 \\
%         0 & 0 & f(\lambda) & 0 \\
%         0 & 0 & 0 & f(\lambda)
%     \end{bmatrix}$
% \end{remark}

\end{document}