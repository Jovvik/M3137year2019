\documentclass[12pt, a4paper]{article}

\usepackage{lastpage}
\usepackage{mathtools}
\usepackage{xltxtra}
\usepackage{libertine}
\usepackage{amsmath}
\usepackage{amsthm}
\usepackage{amsfonts}
\usepackage{amssymb}
\usepackage{enumitem}
\usepackage{xcolor}
\usepackage[left=2.3cm, right=2.3cm, top=2.7cm, bottom=2.7cm, bindingoffset=0cm, headheight=15pt]{geometry}
\usepackage{fancyhdr}
\usepackage[russian]{babel}
% \usepackage{parindent}

\pagestyle{fancy}
\lfoot{M3137y2019}
\rhead{\thepage\ из \pageref{LastPage}}

\newcommand{\R}{\mathbb{R}}
\newcommand{\Q}{\mathbb{Q}}
\newcommand{\C}{\mathbb{C}}
\newcommand{\Z}{\mathbb{Z}}
\newcommand{\B}{\mathbb{B}}
\newcommand{\N}{\mathbb{N}}

\DeclareMathOperator*{\xor}{\oplus}
\DeclareMathOperator*{\equ}{\sim}
\DeclareMathOperator{\Ln}{\text{Ln}}
\DeclareMathOperator{\sign}{\text{sign}}
\DeclareMathOperator{\Sym}{\text{Sym}}
\DeclareMathOperator{\Asym}{\text{Asym}}
% \DeclareMathOperator{\sh}{\text{sh}}
% \DeclareMathOperator{\tg}{\text{tg}}
% \DeclareMathOperator{\arctg}{\text{arctg}}
% \DeclareMathOperator{\ch}{\text{ch}}

\DeclarePairedDelimiter{\ceil}{\lceil}{\rceil}

\setmainfont{Linux Libertine}

\theoremstyle{plain}
\newtheorem{theorem}{Теорема}
\newtheorem{axiom}{Аксиома}
\newtheorem{lemma}{Лемма}

\theoremstyle{remark}
\newtheorem*{remark}{Примечание}
\newtheorem*{exercise}{Упражнение}
\newtheorem*{consequence}{Следствие}
\newtheorem*{example}{Пример}
\newtheorem*{observation}{Наблюдение}

\theoremstyle{definition}
\newtheorem*{definition}{Определение}
\newtheorem*{obozn}{Обозначение}

\lhead{Линейная алгерба}
\cfoot{}
\rfoot{Конспект к экзамену}

\newcommand{\yoba}[0]{
    \includegraphics[height=20pt]{images/yoba.png}
}

\begin{document}

\section{Линейные операторы}

\subsection{Линейные операторы и их матричная запись, примеры.}

\get{3.tex}{лоп}

\subsection{Пространство линейных операторов.}

\get{3.tex}{пространстволоп}
\get{3.tex}{размерностьпространствалоп}

\subsection{Алгебра. Примеры. Изоморфизм алгебр.}

\textbf{Алгебра} --- модуль над коммутативным кольцом с единицей, являющийся кольцом.

\textbf{Кольцо} --- множество, на котором заданы бинарные операции $+$ и $\cdot$ с следующими свойствами:
\begin{enumerate}
    \item $a+b=b+a$
    \item $a+(b+c)=(a+b)+c$
    \item $\exists 0\in R : \forall x\in R : x + 0 = 0 + x = x$
    \item $\forall x\in R : \exists (-x)\in R : x + (-x) = (-x) + x = 0$
    \item $a\cdot(b\cdot c)=(a\cdot b)\cdot c$
    \item $a\cdot (b + c) = a \cdot b + a \cdot c$
    \item $(b + c)\cdot a = b \cdot a + c \cdot a$
\end{enumerate}

\textbf{Коммутативное кольцо} --- кольцо с коммутативным умножением: $a\cdot b = b\cdot a$

\textbf{Кольцо с единицей} --- кольцо с нейтральным элементом по умножению: $\exists 1\in R : a\cdot 1 = a$

\textbf{Модуль над кольцом} \textit{(коммутативным, с единицей)} $R$ --- множество $M$ с операциями:
\begin{enumerate}
    \item $+:M\times M\to M$ \begin{enumerate}
        \item $a+b=b+a$
        \item $a+(b+c)=(a+b)+c$
        \item $\exists 0\in R : \forall x\in R : x + 0 = 0 + x = x$
        \item $\forall x\in R : \exists (-x)\in R : x + (-x) = (-x) + x = 0$
    \end{enumerate}
    \item $\cdot : M\times R\to M$ \begin{enumerate}
        \item $(r_1r_2)m=r_1(r_2m)$
        \item $1m=m$
        \item $r(m_1+m_2)=rm_1+rm_2$
        \item $(r_1+r_2)m=r_1m+r_2m$
    \end{enumerate}
\end{enumerate}

Примеры:
\begin{enumerate}
    \item $\R^3$ с векторным произведением --- алгебра над $\R$
    \item $\C$ --- алгебра над $\R$
    \item $\mathbb H$ \textit{(кватернионы)}
    \item Многочлены
\end{enumerate}

\textbf{Изоморфизм алгебр} --- биекция $F : A\to B$, где $A$ и $B$ --- алгебры, сохраняющая ``$+$'' и ``$\cdot$'':
\begin{enumerate}
    \item $F(kx)=kF(x)$
    \item $F(x + y) = F(x) + F(y)$
    \item $F(xy) = F(x)F(y)$
\end{enumerate}

Из этого следует, что $F(0_X)=0_Y$

\subsection{Алгебра операторов и матриц.}

Умножение ЛОП: $(\mathcal B \cdot \mathcal A)x = \mathcal B(\mathcal A x)$

Умножение матриц: $(A\cdot B)_{ik}=\sum\limits_{j} a_{ij}b_{jk}$

\begin{theorem}
    $$\underbrace{\mathcal C}_{C} = \underbrace{\mathcal B}_{B} \underbrace{\mathcal A}_{A} \Leftrightarrow C = BA$$
\end{theorem}
\begin{proof}
    $$\mathcal C e_i = \mathcal B(\mathcal A e_i) = \mathcal B\left(\sum_{j} a_{ji}e_j\right)=\sum_j a_{ji} \mathcal B e_j = \sum_j a_{ji} \sum_k b_{kj} e_k$$
    $$c_{il} = (\mathcal C e_i)_l = \sum_j a_{ji} b_{lj} \Rightarrow C = BA$$
\end{proof}

Пространство ЛОП $\mathcal F : X\to X$ --- алгебра, пространство квадратных матриц $\R_n^n$ --- алгебра.

\subsection{Обратная матрица: критерий обратимости, метод Гаусса вычисления обратной матрицы.}

В алгебре $A$ выполняется $a_1 \cdot a_2 = e$, где $e$ --- единичный элемент матрицы. Тогда:
\begin{enumerate}
    \item $a_1$ --- \textbf{левый обратный} элемент для $a_2$
    \item $a_2$ --- \textbf{правый обратный} элемент для $a_1$
\end{enumerate}

Если $a_1$ --- и левый, и правый обратный к $a_2$, то он называется \textbf{обратным} элементом к $a_2$.

\label{inversability}
\begin{theorem}
    $\exists A^{−1} \Leftrightarrow \det A \not= 0$
\end{theorem}
\begin{proof}
    ``$\Leftarrow$''
    $$\det A\not=0 \stackrel{?}{\Rightarrow} \exists A^{-1} : AA^{-1}=E, A^{-1}A=E$$
    $$\sum_j a_{ij}a^{-1}_{jk}=\delta_{ik}$$
    Это система Крамера, т.к. $\det A = 0 \xRightarrow{def}$ вектора $\in A$ ЛНЗ $\Rightarrow$ единственное решение.

    ``$\Rightarrow$'' то же самое, но наоборот. \yoba
\end{proof}

\get{4.tex}{методгаусса}

Здесь $T_i$ --- матрица элементарного преобразования.

\subsection{Обратная матрица: критерий обратимости, вычисление обратной матрицы методом присоединенной матрицы.}

Критерий обратимости: \given{inversability}

\get{4.tex}{присоединеннаяматрица}

$A(a_j \to b)$ --- матрица $A$, где заменили $j$-тый вектор на $b$

$$\det A(a_j\to b) = 0 \cdot M_j^1 + \ldots + 1 \cdot M_j^k + \ldots + 0 = M^k_j$$

$$b_{jk} = \frac{(\tilde A^T)^j_k}{\det A} \Rightarrow B = \frac{\tilde A_k^j}{\det A}$$

\subsection{Ядро и образ линейного оператора. Теорема о ядре и образе. Функции матриц и операторов.}

\get{4.tex}{ядроиобраз}

\subsection{Обратный оператор. Критерий существования обратного оператора.}

\get{4.tex}{обратныйоператор}
\get{4.tex}{другойкритерийобратимости}
\get{4.tex}{существованиеобратногоопреатора}

\section{Тензорная алгебра}

\subsection{Преобразование координат векторов $Х$ и $Х^*$ при замене базиса.}
\subsection{Преобразование матрицы линейного оператора при замене базиса. Преобразование подобия.}
\subsection{Тензоры \textit{(ковариантность, независимое от ПЛФ определение)}. Пространство тензоров.}
\subsection{Свертка тензора.}
\subsection{Транспонирование тензора.}
\subsection{Определитель линейного оператора. Внешняя степень оператора.}
\subsection{Независимость определителя оператора от базиса. Теорема умножения определителей.}

% \section{Спектральный анализ линейных операторов в конечномерных пространствах}

% \subsection{Инварианты линейного оператора. Инвариантные подпространства. }
% \subsection{Собственные векторы и собственные значения линейного оператора: основные определения и свойства. }
% \subsection{Собственные векторы и собственные значения линейного оператора: существование, вычисление. }
% \subsection{Спектральный анализ линейного оператора с простым спектром: спектр, диагональный вид матрицы, спектральные проекторы, спектральная теорема. }
% \subsection{Спектральный анализ скалярного оператора: спектр, диагональный вид матрицы, спектральные проекторы, спектральная теорема. }
% \subsection{Спектральная теорема и функциональное исчисление для скалярного оператора. }
% \subsection{Спектральная теорема и инварианты скалярного оператора. Тождество Кэли. }

% \section{Спектральный анализ линейных операторов в конечномерном пространстве: операторы общего вида}

% \subsection{Ультраинвариантные подпространства. }
% \subsection{Алгебра скалярных полиномов. Идеал. Минимальный полином. }
% \subsection{Алгебра операторных полиномов. Минимальный полином линейного оператора. }
% \subsection{Разложение линейного пространства в сумму подпространств. 2я теорема о ядре и образе. Теорема о проекторах. }
% \subsection{Минимальный полином и инвариантные подпространства. Спектральная теорема для линейного оператора произвольного вида. }
% \subsection{Нильпотентные операторы (определение, простейшие свойства). Жорданова клетка. }
% \subsection{Структура нильпотентного оператора. Базис Жордана (обзор). }
% \subsection{Жорданова форма матрицы линейного оператора. }
% \subsection{Кратности собственных чисел (алгебраическая, геометрическая, полная). Теорема Гамильтона-Кэли. }

% \section{Евклидово пространство}

% \subsection{Метрические, нормированные и евклидовы пространства. }
% \subsection{Вещественное евклидово и псевдоевклидово пространство. Основные неравенства. }
% \subsection{Комплексное евклидово пространство. Основные неравенства. }
% \subsection{Ортогональность. Ортогональный базис. Процесс ортогонализации Грама-Шмидта. }
% \subsection{Ортогональная сумма подпространств. Ортогональный проектор. }
% \subsection{Задача о перпендикуляре. }
% \subsection{Ортогональные системы векторов: коэффициенты Фурье, неравенства Бесселя и Парсеваля. }
% \subsection{Метрический тензор. Естественный изоморфизм евклидова и сопряженного ему пространств. }
% \subsection{Ковариантные и контравариантные координаты вектора. Операции поднятия и опускания индексов. }
% \subsection{Эрмитовски сопряженный и эрмитов оператор в евклидовом пространстве: основные определения и свойства. }
% \subsection{Эрмитов и самосопряженный операторы в евклидовом пространстве: теоремы о скалярном типе эрмитова и самосопряженного оператора. }
% \subsection{Эрмитов и самосопряженный операторы в евклидовом пространстве: спектральная теорема, минимальное свойство. }
% \subsection{Унитарный и ортогональный операторы: основные определения и свойства. }
% \subsection{Унитарный оператор: теорема о скалярном типе унитарного оператора, спектральная теорема. }
% \subsection{Приведение эрмитовой матрицы к диагональному виду унитарным преобразованием. }

% \section{Квадратичные формы}

% \subsection{Квадратичные формы: основные определения, приведение к каноническому виду методом Лагранжа. }
% \subsection{Квадратичные формы: приведение к каноническому виду унитарным преобразованием. }
% \subsection{Квадратичные формы: закон инерции квадратичной формы. }
% \subsection{Квадратичные формы: одновременное приведение пары квадратичных форм к сумме квадратов.}


\end{document}