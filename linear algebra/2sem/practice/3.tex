\documentclass[12pt, a4paper]{article}

\usepackage{lastpage}
\usepackage{mathtools}
\usepackage{xltxtra}
\usepackage{libertine}
\usepackage{amsmath}
\usepackage{amsthm}
\usepackage{amsfonts}
\usepackage{amssymb}
\usepackage{enumitem}
\usepackage{xcolor}
\usepackage[left=2.3cm, right=2.3cm, top=2.7cm, bottom=2.7cm, bindingoffset=0cm, headheight=15pt]{geometry}
\usepackage{fancyhdr}
\usepackage[russian]{babel}
% \usepackage{parindent}

\pagestyle{fancy}
\lfoot{M3137y2019}
\rhead{\thepage\ из \pageref{LastPage}}

\newcommand{\R}{\mathbb{R}}
\newcommand{\Q}{\mathbb{Q}}
\newcommand{\C}{\mathbb{C}}
\newcommand{\Z}{\mathbb{Z}}
\newcommand{\B}{\mathbb{B}}
\newcommand{\N}{\mathbb{N}}

\DeclareMathOperator*{\xor}{\oplus}
\DeclareMathOperator*{\equ}{\sim}
\DeclareMathOperator{\Ln}{\text{Ln}}
\DeclareMathOperator{\sign}{\text{sign}}
\DeclareMathOperator{\Sym}{\text{Sym}}
\DeclareMathOperator{\Asym}{\text{Asym}}
% \DeclareMathOperator{\sh}{\text{sh}}
% \DeclareMathOperator{\tg}{\text{tg}}
% \DeclareMathOperator{\arctg}{\text{arctg}}
% \DeclareMathOperator{\ch}{\text{ch}}

\DeclarePairedDelimiter{\ceil}{\lceil}{\rceil}

\setmainfont{Linux Libertine}

\theoremstyle{plain}
\newtheorem{theorem}{Теорема}
\newtheorem{axiom}{Аксиома}
\newtheorem{lemma}{Лемма}

\theoremstyle{remark}
\newtheorem*{remark}{Примечание}
\newtheorem*{exercise}{Упражнение}
\newtheorem*{consequence}{Следствие}
\newtheorem*{example}{Пример}
\newtheorem*{observation}{Наблюдение}

\theoremstyle{definition}
\newtheorem*{definition}{Определение}
\newtheorem*{obozn}{Обозначение}

\lhead{Линейная алгерба}
\cfoot{}
\rfoot{Практика 3}

\begin{document}

\section{Ранг матрицы}

$] A,B$ --- матрицы $n\times n$

$C = A\cdot B \quad c_{ij}=\sum\limits_{k=1}^n a_{ik}b_{kj}$

В матрице $C$ $j$-тый столбец является линейной комбинацией столбцов матрицы $A$ с коэффициентами из $j$-того столбца матрицы $B$ $\Rightarrow rg C \leq rg A, rg \leq rg B \quad rg C = \min(rg A, rg B)$.

\begin{definition}
    Назовем \textbf{элементарными} матрицы, которые получаются из $E$ элементарными преобразованиями её строк.
\end{definition}

$E\cdot B = B$

\begin{enumerate}
    \item Перестановка строк
    
    $E\to E'$

    Если у матрицы $E$ поменять строки $i$ и $j$ местами, то то же случится с матрицей $B$.

    \item Умножение строки на число $\not=0$ --- то же случится с матрицей $B$.
    \item Сложение строк --- то же самое.
\end{enumerate}

Это доказывает, что при решении СЛАУ методом Гаусса не изменяется $rg$ матрицы.

\begin{example}
    Сформулировать в терминах рангов критерий того, что три точки $A, B, C$ не лежат на одной прямой.
\end{example}

    $A(x_1, y_1) \quad B(x_2, y_2) \quad C(x_3, y_3)$

    Предположим, что существует такая прямая, на которой лежат эти точки, вида $Ax + By = C$
    $$\begin{cases}
        Ax_1 + By_1 + C = 0 \\
        Ax_2 + By_2 + C = 0 \\
        Ax_3 + By_3 + C = 0 \\
    \end{cases}$$
    $$\sphericalangle A = \begin{bmatrix}
        x_1 & y_1 & 1 \\
        x_2 & y_2 & 1 \\
        x_3 & y_3 & 1 \\
    \end{bmatrix}$$
    \begin{itemize}
        \item $rg A = 3$ --- не лежат
        \item $rg A = 2$ --- лежат
        \item $rg A = 1$ --- совпали
    \end{itemize}

    $\sphericalangle \R^3$
    $$\begin{cases}
        A_1x+B_1y+C_1z+D_1=0 \\
        A_2x+B_2y+C_2z+D_2=0
    \end{cases}$$
    $$A=\begin{bmatrix}
        x_1 & y_1 & z_1 & 1 \\
        x_2 & y_2 & z_2 & 1 \\
        x_3 & y_3 & z_3 & 1 \\
    \end{bmatrix}$$

    Докажем, что через 4 точки, не лежащие на одной плоскости, можно провести единственную сферу.
    $$(x-x_0)^2+(y-y_0)^2+(z-z_0)^2=R^2$$
    $$x^2-2xx_0+x_0^2 + y^2-2yy_0+y_0^2 + z^2-2zz_0+z_0^2 = R^2$$
    $$2xx_0 + 2yy_0 + 2zz_0 + b = x^2 + y^2 + z^2 \quad b = R^2-x_0^2-y_0^2-z_0^2$$
    $$\begin{bmatrix}
        2x_1 & 2y_1 & 2z_1 & 1 & x_1^2+y_1^2+z_1^2 \\
        2x_2 & 2y_2 & 2z_2 & 1 & x_2^2+y_2^2+z_2^2 \\
        2x_3 & 2y_3 & 2z_3 & 1 & x_3^2+y_3^2+z_3^2 \\
        2x_4 & 2y_4 & 2z_4 & 1 & x_4^2+y_4^2+z_4^2 \\
    \end{bmatrix}$$
    $rg A = 4 \quad \exists!$ сфера

    $$rg(AB) = min (rg A, rg B)$$
    $$rg(A^T)=rg A$$
    
\end{document}