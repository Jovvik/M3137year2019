\documentclass[12pt, a4paper]{article}

\usepackage{lastpage}
\usepackage{mathtools}
\usepackage{xltxtra}
\usepackage{libertine}
\usepackage{amsmath}
\usepackage{amsthm}
\usepackage{amsfonts}
\usepackage{amssymb}
\usepackage{enumitem}
\usepackage{xcolor}
\usepackage[left=2.3cm, right=2.3cm, top=2.7cm, bottom=2.7cm, bindingoffset=0cm, headheight=15pt]{geometry}
\usepackage{fancyhdr}
\usepackage[russian]{babel}
% \usepackage{parindent}

\pagestyle{fancy}
\lfoot{M3137y2019}
\rhead{\thepage\ из \pageref{LastPage}}

\newcommand{\R}{\mathbb{R}}
\newcommand{\Q}{\mathbb{Q}}
\newcommand{\C}{\mathbb{C}}
\newcommand{\Z}{\mathbb{Z}}
\newcommand{\B}{\mathbb{B}}
\newcommand{\N}{\mathbb{N}}

\DeclareMathOperator*{\xor}{\oplus}
\DeclareMathOperator*{\equ}{\sim}
\DeclareMathOperator{\Ln}{\text{Ln}}
\DeclareMathOperator{\sign}{\text{sign}}
\DeclareMathOperator{\Sym}{\text{Sym}}
\DeclareMathOperator{\Asym}{\text{Asym}}
% \DeclareMathOperator{\sh}{\text{sh}}
% \DeclareMathOperator{\tg}{\text{tg}}
% \DeclareMathOperator{\arctg}{\text{arctg}}
% \DeclareMathOperator{\ch}{\text{ch}}

\DeclarePairedDelimiter{\ceil}{\lceil}{\rceil}

\setmainfont{Linux Libertine}

\theoremstyle{plain}
\newtheorem{theorem}{Теорема}
\newtheorem{axiom}{Аксиома}
\newtheorem{lemma}{Лемма}

\theoremstyle{remark}
\newtheorem*{remark}{Примечание}
\newtheorem*{exercise}{Упражнение}
\newtheorem*{consequence}{Следствие}
\newtheorem*{example}{Пример}
\newtheorem*{observation}{Наблюдение}

\theoremstyle{definition}
\newtheorem*{definition}{Определение}
\newtheorem*{obozn}{Обозначение}

\lhead{Линейная алгерба}
\cfoot{}
\rfoot{Практика 4}

\begin{document}

$] x = \begin{pmatrix}
        \xi^1 & \xi^2 & \xi^3
    \end{pmatrix} \quad \varphi(x) = \begin{pmatrix}
        \xi^1+\xi^2, \xi^2, \xi^3
    \end{pmatrix}$ --- линеный оператор.

\begin{definition}
    \textbf{Ядро} ЛОп $\varphi$ называется множество
    $$Ker \varphi = \{ x\in X : \varphi x = O_Y\}$$
\end{definition}
\begin{remark}
    $Ker \varphi$ --- подпространство ЛП $X$
\end{remark}

\begin{definition}
    \textbf{Образом} ЛОп $\varphi$ называется множество:
    $$Im \varphi = \{y\in Y : \exists x \quad \varphi x = y\} = \varphi(X)$$
\end{definition}
\begin{remark}
    $Im \varphi$ --- подпространство ЛП $Y$
\end{remark}
\begin{proof}
    $y_1, y_2\in Im \varphi \Rightarrow \exists x_1, x_2 \in X : \varphi x_1=y_1 \ \ \varphi x_2 = y_2$

    $\sphericalangle y_1 + y_2 = \varphi(x_1) + \varphi(x_2) = \varphi(x_1+x_2) \Rightarrow y_1+y_2\in Im \varphi$
\end{proof}

\begin{example}
    $E_3$ --- евклидово пространство

    $\varphi E_3 \to E_3$

    $$\varphi(\vec x) = \vec x - \frac{(\vec x \vec n)}{(\vec n \vec n)}\vec n \quad \vec n\not=\vec 0$$

    \begin{enumerate}
        \item $Ker \varphi$
        \item $Im \varphi$
        \item Геометрический смысл
    \end{enumerate}

    $Ker \varphi:$
    $$\vec x - \frac{(\vec x \vec n)}{(\vec n \vec n)}\vec n=0$$
    $$\vec x = \frac{(\vec x \vec n)}{(\vec n \vec n)}\vec n \Rightarrow Ker \varphi = \mathcal{L}(\vec n)$$

    $Im \varphi:$
    $$\sphericalangle \varphi(\vec y) = \vec y - \frac{(\vec y \vec n)}{(\vec n \vec n)}\vec n$$
    $$y\in\mathcal{L}^\perp (\vec n) \Rightarrow \varphi(y) = y \Rightarrow Im \varphi = \mathcal{L}^\perp (\vec n)$$

    $$\mathcal{L}(\vec n)\cap\mathcal{L}^\perp(\vec n)=0 \Rightarrow E_3=\mathcal{L}(\vec n) + \mathcal{L}^\perp(\vec n)$$
\end{example}

\begin{example}
    $E_3$

    $\varphi : E_3\to E_3$
    $$\varphi(\vec x) = \vec x - \frac{(\vec x \vec n)}{(\vec a \vec n)}\vec a \quad (\vec a, \vec n)\not=0$$

    $Ker \varphi=\mathcal{L}(\vec a)$

    $\vec x \stackrel{!}{=} \vec y + \alpha \vec a$
\end{example}

$] \{e_j\}_{j=1}^n$ --- базис $X$, $\{h_k\}_{k=1}^m$ --- базис $Y$ $\Rightarrow \varphi(e_j)=\sum\limits_{k=1}^m a_j^kh_k$

$\sphericalangle A_\varphi=||a_j^k||$ --- матрица оператора $\varphi$ в паре выбранных базисов

\begin{example}
    Найти матрицу оператора $\varphi$ в стандартном базисе $E_3$

    $\vec n = \begin{pmatrix}
            1 & 2 & 3
        \end{pmatrix}\tran $

    $$\varphi(e_1)=\begin{bmatrix}
            1 \\ 0 \\ 0
        \end{bmatrix} - \frac{1}{14}\begin{bmatrix}
            1 \\
            2 \\
            3
        \end{bmatrix}=\begin{bmatrix}
            \frac{13}{14} \\
            \frac{-2}{14} \\
            \frac{-3}{14}
        \end{bmatrix}$$

    $$\varphi(e_2)=\begin{bmatrix}
            0 \\ 1 \\ 0
        \end{bmatrix} - \frac{w}{14}\begin{bmatrix}
            1 \\
            2 \\
            3
        \end{bmatrix}=\begin{bmatrix}
            \frac{-2}{14} \\
            \frac{10}{14} \\
            \frac{-6}{14}
        \end{bmatrix}$$
    То же самое для $e_3$ и собрать все в матрицу.
\end{example}

\begin{example}
    $\varphi_\theta$ --- оператор поворота вокруг $\vec S$ на $\theta$

    $] y = \beta\vec a + \gamma\vec b, \vec a\perp\vec b\perp \vec c, |\vec a|=|\vec b|=1$

    $]\vec x = \alpha\vec s + \vec y = \alpha\vec s + \beta\vec a + \gamma\vec b$

    $\varphi(x)=\varphi(\alpha\vec s + \beta\vec a + \gamma\vec b) = \alpha\vec s + (\beta\cos\theta-\gamma\sin\theta)\vec a + (\beta\sin\theta+\gamma\cos\theta)\vec b$

    $\varphi(e_1)=...$ и так далее
\end{example}

$] \varphi : X \to X$

$\{e_j\}_{j=1}^n$ --- базис $X \Rightarrow A_\varphi$

$\{\tilde e_k\}_{k=1}^n$ --- базис $X \Rightarrow \tilde A_\varphi$

$\{e\}\stackrel{T}{\to}\{\tilde e\}$

$$\sphericalangle \varphi(\tilde e_j)=\sum\limits_{k=1}^n\tilde a_j^k \tilde e_k=\sum\limits_{k=1}^n\tilde a_j^k \sum\limits_{l=1}^n t_k^le_l$$
$$\varphi(\tilde e_j) = \varphi\left(\sum\limits_{l=1}^n \tilde a_j^le_l\right)=\sum\limits_{l=1}^n \tilde a_j^l\varphi (e_l)=\sum\limits_{l=1}^n t_j^l\sum\limits_{k=1}^n a_l^k e_k$$
$$\sum\limits_{k=1}^n \tilde a_j^k t_k^l = \sum\limits_{k=1}^n t_j^k a_k^l$$
$$(T\tilde A_\varphi)_j^l=(A_\varphi T)_j^l \quad \forall l, j$$
$$T\tilde A_\varphi=A_\varphi T$$
$$\tilde A_\varphi=T^{-1}A_\varphi T$$

\end{document}
