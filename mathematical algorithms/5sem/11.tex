\chapter{13 ноября}

Пусть \(\{G_i\}\) --- группы. Рассмотрим объект \(\prod_i G_i\) --- декартово произведение этих групп как множеств.

Пусть \(G_i = \{x_i', x_i'' \dots \}, \prod_i G_i = \{(x_i, x_j \dots )\} = \{(x_i)\}\)

\begin{lemma}
    \(\prod_i G_i\) может быть наделено структурой группы.

    \(\sphericalangle (x_i), (y_i) \in \prod_i G_i\) и \((x_1, x_2 \dots x_n \dots ) * (y_1, y_2 \dots y_n \dots ) = (x_1y_1, x_2y_2 \dots x_ny_n \dots)\)
\end{lemma}
\begin{proof}
    Проверим аксиомы группы. Они все очевидны из аксиом групп \(G_i\).
\end{proof}

\(\sphericalangle \lambda_j : G_j \to \prod_i G_i \quad \lambda_j(x) = (e_1, e_2 \dots x \dots e_n \dots)\) и обратное к нему отображение \(\mathrm{proj}_j\)

\begin{lemma}
    \((\prod_i G_i, \{\mathrm{proj}_k\})\) --- произведение в \(\mathrm{Grp}\).
\end{lemma}
\begin{proof}
    Рассмотрим \(\tilde{G} \in \obj(\mathrm{Grp}), \{g_i : \tilde{G} \to G_i\}\). Нужно показать, что \(\exists! h : f_i \circ h = g_i\).

    \(\sphericalangle y \in \tilde{G}, \begin{tikzcd}
        y \arrow{r}{h} & (y_i) \arrow{r}{\mathrm{proj}_i} & y_i
    \end{tikzcd}\)

    \(g_i(y) = (y)_i\), поэтому \(f_i \circ \underbrace{h(y)}_{x_1 \dots x_i \dots } = \underbrace{g_i (y)}_{x_i}\). Тогда \(h(y)\) существует, и это может быть только \((g_1(y), g_2(y) \dots g_n(y) \dots)\), из этого следует единственность.

    \[\begin{tikzcd}
            & & \tilde{G} \arrow[bend right = 30, swap]{dd}{h} \arrow[swap]{d}\arrow{ld}{g_2}\arrow[swap]{lld}{g_1}\arrow[swap]{rd}{g_n}\arrow{rrd} & & \\
            G_1 & G_2 & \dots & G_n & \dots \\
            & & \prod_i G_i \arrow{u}\arrow[swap]{lu}{f_2}\arrow{llu}{f_1}\arrow{ru}{f_n}\arrow{rru} & & \\
        \end{tikzcd}\]
\end{proof}

\begin{lemma}[критерий прямого произведения]\itemfix
    \begin{itemize}
        \item \(G\) --- группа
        \item \(H, K\) --- подгруппы \(G\)
        \item \(H \cap K = \{e\}\)
        \item \(\forall x \in H \ \ y \in K \ \ xy = yx\)
        \item \(HK = G\)
    \end{itemize}

    Тогда и только тогда \(H \times K \cong G\)
\end{lemma}
\begin{proof}
    \(\sphericalangle \psi : (x, y) \mapsto xy, \psi \in \hom(H \times K, G)\)

    Сюръективность очевидна, т.к. \(HK = G\).

    Рассмотрим \((x, y)\), такие что \(\psi((x, y)) = e\). Тогда \(xy = e \Rightarrow x = y^{-1}\). \(y \in K \Rightarrow y^{-1} \in K \Rightarrow x \in K\), но кроме того \(x \in H \Rightarrow x \in H \cap K\), следовательно, \(x = e\). Аналогично \(y = e\).

    Т.к. \(\psi\) --- биективный гомоморфизм, \(\psi\) --- изоморфизм.
\end{proof}

Обобщение:
\[H_1 \times H_2 \times \dots \times H_n \cong G \Leftrightarrow \begin{cases}
        H_{j+1} \cap (H_1 H_2 \dots H_j) = \{e\} \\
        H_i H_j = H_j H_i \ \ \forall i, j
    \end{cases}\]

\section{Свободные группы}

Рассмотрим \(S\) --- множество.

\(\sphericalangle g : S \to g(S) \subset G\), где \(g(S)\) --- множество образующих группы \(G\).

\begin{definition}
    Отображение \(g : S \to G\) \textbf{порождает} группу \(G\), если образ \(g\) порождает \(G\).
\end{definition}

\begin{definition}
    \(S\) --- \textbf{множество образующих}\footnote{Также называется множеством порождающих.} группы \(G\), если \(\forall y \in G \ \ y = \prod_i x_i\), где \(x_i \in S\) или \(x_i^{-1} \in S\)
\end{definition}

Рассмотрим два отображения \(f : S \to F, g : S \to G\), где \(f(S)\) порождает \(F\) и \(g(S)\) порождает \(G\).
\[\begin{tikzcd}
        & F \arrow{dd}{g_*} \\
        S \arrow{ur}{f} \arrow{dr}{g} & \\
        & G
    \end{tikzcd}\]
По доказанной ранее лемме~\ref{проигранная} существует не более одного гомоморфизма \(g_*\).

Рассмотрим категорию \(\zeta\), объекты которой являются парами вида \((F, f)\). Рассмотрим гомоморфизм \(\varphi : F \to G\), тогда \(\varphi \in \arr(\zeta)\): \(\begin{tikzcd}
    (F, f) \arrow{r}{\varphi} & (G, g)
\end{tikzcd}\).

\begin{definition}
    \textbf{Свободная группа}, определяемая множеством \(S\) --- инициальный объект в категории \(\zeta\).
\end{definition}

\begin{theorem}
    Для всякого множества \(S\) существует определяемая им свободная группа \((F, f)\), при этом:
    \begin{enumerate}
        \item \(f\) инъективен.
        \item \(f\) порождает \(F\).
    \end{enumerate}
\end{theorem}
\begin{proof}
    Рассмотрим случай, когда \(S\) конечно.

    \(\sphericalangle T\) --- счётное множество. Пусть на нём мы можем заводить групповые структуры, множество всех таких структур назовём \(\Gamma\). Пусть тогда \(T_\gamma\), где \(\gamma \in \Gamma\) --- реализация группы на \(T\).

    Рассмотрим отображение \(\varphi\) между \(S\) и \(T_\gamma\). Т.к. множество образующих можно по--разному вкладывать в \(T_\gamma\). Чтобы это фиксировать, скажем, что \(\varphi : S \to T_{\gamma, \varphi}\). \(T_{\gamma, \varphi} \in \obj(\zeta)\), а \(T_\gamma \in \mathrm{Grp}\).

    Рассмотрим также множество \(M_\gamma = \{\varphi\}\) --- множество отображений \(S\) в \(T_\gamma\).\footnote{\(M_\gamma = \arr( - , T_\gamma)\)}
    \[F_0 \coloneqq \prod_{g \in \Gamma} \prod_{\varphi \in M_\gamma} T_{\gamma, \varphi}\]
    \(F_0 \in \mathrm{Grp}\)\footnote{Мы игнорируем тот факт, что произведение это тройка из объекта и двух морфизмов, нас интересует только объект.}, поскольку это произведение элементов \(\obj(\mathrm{Grp})\). Рассмотрим \(f_0 : S \to F_0 : S \mapsto (\varphi_1(S)_{\gamma_1}, \varphi_2(S)_{\gamma_1} \dots \varphi_1'(S)_{\gamma_2}, \varphi_2'(S)_{\gamma_2} \dots)\).

    Покажем, что \(\forall g : S \to G \ \ \exists! g_* : F_0 \to G\), такой что следующая диаграмма коммутативна:
    \[\begin{tikzcd}
            & F_0 \arrow{dd}{g_*} \\
            S \arrow{ur}{f_0} \arrow{dr}{g} \\
            & G
        \end{tikzcd}\]

    Пусть \(g(S)\) порождает \(G\). Т.к. \(S\) конечно, \(|G| \leq |T|\), т.к. \(T\) счётно.

    Рассмотрим \(\overline{G} = G \times \Z\). Надо, чтобы \(|G \times \Z| = |T|\) и тогда будет существовать биекция между \(G \times \Z\) и \(T\), тогда \(\exists \gamma \in \Gamma : G \times \Z \cong T_\gamma\) с изоморфизмом \(\lambda : G \times \Z \to T_\gamma\).

    \[\begin{tikzcd}
            S \arrow{r}{g} & G \arrow{r}{h} & G \times \Z \arrow{r}{\lambda} & T_\gamma
        \end{tikzcd}\]
    \[\psi \coloneqq \lambda \circ h \circ g \in M_\gamma \quad \begin{tikzcd}
            S \arrow{r}{\psi} & T_{\gamma, \psi}
        \end{tikzcd}\]
    \[\psi_* \coloneqq \mathrm{proj}_G \circ \lambda^{-1} \circ \mathrm{proj}_{\gamma, \psi}\]
    \[\begin{tikzcd}
            F_0 \arrow{r}{\mathrm{proj}_{\gamma, \psi}} \arrow{d}{\psi_*} & T_{\gamma, \psi} \arrow{d}{\lambda^{-1}} \\
            G & G \times \Z \arrow{l}{\mathrm{proj}_{G}}
        \end{tikzcd}\]
    \[\begin{tikzcd}
            & F_0 \arrow{d}{\mathrm{proj}_{\gamma, \psi}} \arrow[bend left = 75]{ddd}{\psi_*} \\
            S \arrow{ur}{f_0} \arrow{ddr}{g} & T_{\gamma, \psi} \arrow{d}{\lambda^{-1}} \\
            & G \times \Z \arrow{d}{\mathrm{proj}_{G}} \\
            & G
        \end{tikzcd}\]
    Рассмотрим \(f : S \to F, f(S) = F\). Для единственности \(g_*\) мы сужаем его на \(g_*\Big|_F\). Весь трюк заключается в том, что в \(F_0\) много лишнего, т.к. там много одинаковых элементов.

    \(f\) инъективно, т.к. всех \(\varphi \in M_\gamma\) найдётся инъектвиное.

    Если \(S\) несчётно, ты мы не знаем что делать.

    Если \(S\) счётно, то все то же самое, но за \(T\) возьмём \(S\) и тогда \(G \times \Z \cong T \underbrace{\times \Z \times \Z \times \dots}_{\text{нужное количество } \Z}\)
\end{proof}
