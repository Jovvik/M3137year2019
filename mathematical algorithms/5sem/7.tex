\chapter{16 октября}

Пусть \(G\) --- произвольная группа, \(\sphericalangle \sigma : \Z \to G, \sigma : z \longmapsto a^z\)

\(\im \sigma = \ev{a} \subset G\)

Есть два случая:
\begin{enumerate}
    \item \(\ker \sigma = \{0\} \Rightarrow \im \sigma \cong \Z\) и \(G\) содержит бесконечную циклическую подгруппу.
    \item \(\ker \sigma \neq \{0\} \Rightarrow \ker \sigma = H \subset \Z \Rightarrow H = \{nh\}_{n \in \Z} \Rightarrow \Z / H = \{[0], [1], [2] \dots [h - 1]\}\)
          \[\begin{tikzcd}
                  \Z \arrow[bend right]{rr}{\sigma} \arrow{r}{\varphi} & \Z / H \arrow{r}{\sigma^*} & G
              \end{tikzcd}\]
          Разложили \(\sigma = \sigma^* \circ \varphi\), где \(\varphi\) --- канонический гомоморфизм.

          Тогда \(\sigma^*\) отображает \(\Z / H\) в \(a^0, a^1, a^2 \dots a^{h - 1}\), где \(a^h = a^0 = e\).

          \begin{statement}
              Все элементы различны, т.е. \(\sphericalangle s, r : a^s = a^r\). Тогда \(s = r\).
          \end{statement}
          \begin{proof}
              \(a^{s - r} = e \Rightarrow s - r = kh = 0 \Rightarrow s = r\).
          \end{proof}
\end{enumerate}

\begin{definition}
    Пусть \(G\) --- циклическая группа \(a^0, a^1 \dots a^{h-1}\). Тогда \(h\) --- \textbf{период} элемента \(a\). Это не то же самое, что показатель: показатель имеет вид \(qh\).
\end{definition}

\begin{lemma}
    \(G\) - конечная \(\Rightarrow\) период \(\forall g \in G\) делит порядок группы.
\end{lemma}
\begin{proof}
    Пусть \(d\) --- период \(g \in G\), тогда \(g^d = e\).

    \(\sphericalangle H = \ev{g} \text{ --- подгруппа } G\) и \(|H| = d\)

    \[|G| = (G : 1) = (G : H) (H : 1) = (G : H)|H|\]
\end{proof}

\begin{lemma}
    Пусть \(|G| = p\) --- простое число, \(\sphericalangle g \in G, g \neq e\).

    Тогда \(G = \ev{g}\).
\end{lemma}
\begin{proof}
    \(\sphericalangle g \in G, g \neq e\)

    \(\sphericalangle H = \ev{g} \Rightarrow |H| \neq 1\), т.к. \(e \in H, g \in H\).

    \(p = (G : 1) = (G : H)(H : 1)\). Но тогда \((G : H) = 1\) по простоте \(p\), следовательно \(G = \ev{g}\)
\end{proof}

\begin{lemma}
    \(G\) --- циклическая группа. Тогда
    \begin{enumerate}
        \item \(H \subset G\) --- циклическая
        \item \(\sigma(G)\) --- циклическая, если \(\sigma \in \mathrm{Hom}(G)\)
    \end{enumerate}
\end{lemma}
\begin{proof}
    \(G\) --- циклическая группа

    \begin{enumerate}
        \item \begin{enumerate}
                  \item \(G\) --- бесконечная циклическая группа.

                        Тогда \(G \cong \Z\) --- знаем все подгруппы \textit{(они циклические)}.
                  \item \(G\) --- конечная циклическая группа.

                        \(\sphericalangle H \subset G\) --- подгруппа.

                        \(|G| \divided |H| \Rightarrow |H|\) конечна.

                        \(\sphericalangle a \in H \Rightarrow a = g^n \Rightarrow a^k = g^{k n} \Rightarrow H = \ev{a}\)
              \end{enumerate}
        \item Пусть \(G = \ev{g}\), тогда \(\sigma(g)\) --- образующая для \(\sigma(G)\) и значит \(\sigma(G) = \ev{\sigma(g)}\)
    \end{enumerate}
\end{proof}

\begin{lemma}
    \(G\) --- бесконечная циклическая группа. Тогда у \(G\) есть две образующие: \(g\) и \(g^{-1}\).
\end{lemma}

\section{Силовские группы}

\begin{definition}
    Группа называется \textbf{\(p\)-группой}, если ее порядок является степенью простого числа \(p\).
\end{definition}

\begin{definition}
    Подгруппа \(H\) называется \textbf{\(p\)-подгруппой группы} \(G\), если \(H \subset G\), \(H\) --- \(p\)-группа.
\end{definition}

\begin{definition}
    \(H\) называется \textbf{силовской} подгруппой \(G\), если \(H\) --- \(p\)-подгруппа \(G\) и \(|H| = p^n\), где \(p^n\) --- максимальный порядок в группе.
\end{definition}

Пусть \(n\) --- порядок группы \(G\). Мы знаем\footnote{Но докажем потом.}, что \(n = p_1^{n_1} p_2^{n_2} \dots\), где \(p_i\) --- простые. \(n_i\) --- максимальная степень \(p_i\), которая встречается в \(n\), т.е.  \(n \not\divided p_i^{n_{i+1}}\). Т.к. порядок подгруппы делит порядок группы, то найдутся подгруппы, порядки которых соответствуют этому разложению.

\begin{lemma}\itemfix
    \begin{itemize}
        \item \(|G| = m\)
        \item Показатель \(G = n\)
        \item \(G\) --- коммутативная группа
    \end{itemize}
    Тогда порядок \(G\) делит некоторую степень показателя:
    \[\exists k : n^k \divided m\]
\end{lemma}

\begin{proof}
    По индукции (по порядку группы)

    \(\sphericalangle H \vartriangleleft G, H = \ev{b}\). Т.к. показатель \(G = n, b^n = e\).

    \(\sphericalangle |G / H|\)

    Так как \(n \divided (H : 1)\) и по индукции \(n^k \divided (G : H)\), то \(n^{k + 1} \divided (G : 1) = (G : H)(H : 1)\)
\end{proof}

\begin{lemma}\itemfix
    \label{lm_абелева_группа_делит_простое}
    \begin{itemize}
        \item \(G\) --- конечная абелева группа
        \item \(|G| \divided p\) (\(p\) --- простое)
    \end{itemize}

    Тогда \(\exists H \subset G : |H| = p\).
\end{lemma}

\begin{proof}
    \(|G| \divided p\) по условию.

    \(\sphericalangle H = \ev{x}, x^n = e\)

    Пусть показатель группы \(G\) есть \(n\), \(m\) --- порядок группы.

    \[m \divided p \Rightarrow \exists s : m = sp\]
    Некоторая степень показателя делится на порядок группы: \(n^k \divided m \Rightarrow \exists z : n^k = z \cdot m = zsp\)
    \[x^{zs} \eqqcolon y,\ \ y^p = e \Rightarrow H' = \ev{y} \text{ --- искомая группа}\]
\end{proof}

\begin{lemma}\itemfix
    \begin{itemize}
        \item \(G\) --- конечная группа
        \item \(|G| \divided p\) (\(p\) --- простое)
    \end{itemize}

    Тогда в \(G\) \(\exists\) cиловская подгруппа.
\end{lemma}

\begin{proof}
    По индукции.

    Если \(|G| = p\), искомое очевидно.

    Пусть искомое доказано для всех порядков меньших \(G\).

    Пусть \(H \subset G \Rightarrow (G : 1) = (G : H)(H : 1)\)

    \begin{enumerate}
        \item Если \(|H| \divided p\), то силовская подгруппа для \(G\) будет силовской подгруппой для \(H\), которая существует по индукционному предположению.
        \item Если \((G : H) \divided p\)

              Пусть \(G\) действует на себя.
              \[(G : 1) = |Z_G| + \sum_x (G : G_x) \]
              Так как \((G : 1) \divided p\) и \(\forall x : (G : G_x) \divided p \Rightarrow |Z_G| \divided p\), т.е. центр нетривиальный. Кроме того, центр абелев, следовательно по лемме \ref{lm_абелева_группа_делит_простое} \(\exists H \subset Z_G\) - абелева подгруппа, такая что \(|H| = p\).

              Т.к. \(H \subset G\), \(H \vartriangleleft G \Rightarrow G/H\). В \(G / H\) существует силовская подгруппа \(p^{n-1}\) по индукционному предположению, назовём ее \(K'\).

              \(|K'|= p^{n-1}, |K'H| = p^{n-1} \cdot p = p^n\), при этом \(K'H\) --- подгруппа, т.к. \(H\) --- нормальная подгруппа. \(K'H\) --- искомая подгруппа.
    \end{enumerate}
\end{proof}
