\chapter{23 октября}

\subsection{Теоремы Силова}

\begin{remark}\itemfix
    \begin{itemize}
        \item \(G\) --- произвольная группа
        \item \(H, K\) --- подгруппы \(G\)
        \item \(H \subset N_K = \{g \in G : gKg^{-1} = K\}\)
    \end{itemize}

    Тогда:
    \begin{enumerate}
        \item \(HK\) --- подгруппа \(G\)
              \begin{proof}
                  \(\sphericalangle h_1k_1, h_2k_2 \in HK\)
                  \[(h_1k_1)(h_2k_2) = h_1k_1h_2k_2 = \underbrace{h_1h_2}_h\underbrace{k_1k_2}_k\]
              \end{proof}
        \item \(K \vartriangleleft HK \Rightarrow \exists HK / K\)

              \(\sphericalangle \varphi : HK \to HK / K\) --- канонический гомоморфизм

              \(\ker \varphi = K\), т.к. \(1 \cdot K \cdot K = K^2 = K\), что есть нейтральный элемент фактор-группы.

              Мы запутались, но каким-то образом \(HK / K \cong H / H \cap K\).

              \unfinished % https://youtu.be/G9huOHYZqbE?t=4842
    \end{enumerate}
\end{remark}

\begin{theorem}[первая теорема Силова]
    \label{силов1}
    Каждая \(p\)-подгруппа содержится в силовской \(p\)-подгруппе.
\end{theorem}
\begin{proof}
    Пусть \(G\) --- группа, \(S\) --- множество силовских \(p\)-подгрупп и \(G\) действует на \(S\) сопряжением.

    \(\sphericalangle \mathcal{P} \in S, S = S_G\)
    \[S_0 \coloneqq O_G(\mathcal{P}) \defeq \{g\mathcal{P}g^{-1}\}_{g \in G} = \{ \tilde{\mathcal{P}}_1, \tilde{\mathcal{P}}_2 \dots \tilde{\mathcal{P}}_m\}\]
    Сколько элементов в \(S_0\)? \((G : \mathcal{P}) \not\divided p \Rightarrow |S_0| \not\divided p\)

    Пусть \(H\) --- \(p\)-подгруппа \(G\), действующая на \(S_0\) сопряжением.

    \begin{remark}
        \(|H| = p^k \Rightarrow \forall \tilde{H} \subset H \ \ |\tilde{H}| \divided p\)
    \end{remark}
    \[|S_0| = \sum_{C} (H : \tilde{H}_x)\]
    Т.к. \(H\) --- \(p\)-подгруппа, остатки от деления \((H : \tilde{H}_x)\) либо \(\divided p\), либо \(= 1\). Т.к. \(|S_0| \not\divided p\), существуют слагаемые, не делящиеся на \(p\) и по предыдущему утверждению они равны единице. Рассмотрим одну из таких групп, \(\tilde{H}'\) Ей соответствует \(\mathcal{P}'\), причём \(O_{H}(\mathcal{P}') = \mathcal{P}', \forall h \in H \ \ h\mathcal{P}'h^{-1} = \mathcal{P}' \Rightarrow h\mathcal{P}' = \mathcal{P}'h\), а следовательно \(H \subset N_{\mathcal{P}'}\).

    Так как \(HK / K \cong H / H \cap K\), \(H\mathcal{P}' / \mathcal{P}' \cong H / (H \cap \mathcal{P}') \Rightarrow \mathcal{P}'H \cong \mathcal{P}' \Rightarrow H \subset \mathcal{P}'\)
\end{proof}

\begin{theorem}[вторая теорема Силова]
    \label{силов2}
    Силовские \(p\)-подгруппы сопряжены.
\end{theorem}

\begin{theorem}[третья теорема Силова]
    \label{силов3}
    Число силовских \(p\)-подгрупп \(\equiv 1 \mod p\).
\end{theorem}

\unfinished
