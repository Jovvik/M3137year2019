\documentclass[12pt, a4paper]{article}

\usepackage{lastpage}
\usepackage{mathtools}
\usepackage{xltxtra}
\usepackage{libertine}
\usepackage{amsmath}
\usepackage{amsthm}
\usepackage{amsfonts}
\usepackage{amssymb}
\usepackage{enumitem}
\usepackage{xcolor}
\usepackage[left=2.3cm, right=2.3cm, top=2.7cm, bottom=2.7cm, bindingoffset=0cm, headheight=15pt]{geometry}
\usepackage{fancyhdr}
\usepackage[russian]{babel}
% \usepackage{parindent}

\pagestyle{fancy}
\lfoot{M3137y2019}
\rhead{\thepage\ из \pageref{LastPage}}

\newcommand{\R}{\mathbb{R}}
\newcommand{\Q}{\mathbb{Q}}
\newcommand{\C}{\mathbb{C}}
\newcommand{\Z}{\mathbb{Z}}
\newcommand{\B}{\mathbb{B}}
\newcommand{\N}{\mathbb{N}}

\DeclareMathOperator*{\xor}{\oplus}
\DeclareMathOperator*{\equ}{\sim}
\DeclareMathOperator{\Ln}{\text{Ln}}
\DeclareMathOperator{\sign}{\text{sign}}
\DeclareMathOperator{\Sym}{\text{Sym}}
\DeclareMathOperator{\Asym}{\text{Asym}}
% \DeclareMathOperator{\sh}{\text{sh}}
% \DeclareMathOperator{\tg}{\text{tg}}
% \DeclareMathOperator{\arctg}{\text{arctg}}
% \DeclareMathOperator{\ch}{\text{ch}}

\DeclarePairedDelimiter{\ceil}{\lceil}{\rceil}

\setmainfont{Linux Libertine}

\theoremstyle{plain}
\newtheorem{theorem}{Теорема}
\newtheorem{axiom}{Аксиома}
\newtheorem{lemma}{Лемма}

\theoremstyle{remark}
\newtheorem*{remark}{Примечание}
\newtheorem*{exercise}{Упражнение}
\newtheorem*{consequence}{Следствие}
\newtheorem*{example}{Пример}
\newtheorem*{observation}{Наблюдение}

\theoremstyle{definition}
\newtheorem*{definition}{Определение}
\newtheorem*{obozn}{Обозначение}

\lhead{Алгоритмы в математике \textit{(практика)}}
\lfoot{Михайлов Максим}
\cfoot{}
\rfoot{18.12.2021}

\begin{document}

\begin{exercise}
    Доказать, что евклидово кольцо \(R\) есть область целостности.
\end{exercise}
\begin{solution}
    Это неверно, т.к. вырожденное кольцо \(\{0\}, 0 = 1\) является евклидовым, но не является целостным.

    Если в такое не верится, то кольцо \(\{0, 1, x, 1 + x\}, x^2 = 0\) не является целостным, т.к. \(x \cdot x = 0\), но \(x \neq 0\), при этом это кольцо является евклидовым:
    \[F(t) \coloneqq \begin{cases}
            0, & t = 0     \\
            1, & t = 1     \\
            1, & t = 1 + x \\
            2, & t = x     \\
        \end{cases}\]

    \begin{align*}
        a     & = q \cdot b + r                                                        \\
        0     & = 0 \cdot *\footnotemark + 0                                           \\
        1     & = 1 \cdot 1 + 0                                                        \\
        1     & = \underbrace{x \cdot \overbrace{x}^{F = 2}}_0 + \overbrace{1}^{F = 1} \\
        1     & = \underbrace{(1 + x) \cdot \overbrace{(1 + x)}^{F = 1}}_1 {}+ 0       \\
        x     & = x \cdot 1 + 0                                                        \\
        x     & = 1 \cdot x + 0                                                        \\
        x     & = \underbrace{x \cdot (1 + x)}_x {}+ 0                                 \\
        1 + x & = (1 + x) \cdot 1 + 0                                                  \\
        1 + x & = (1 + x) \cdot \overbrace{x}^{F = 2} + \overbrace{1}^{F = 1}          \\
        1 + x & = \underbrace{1 \cdot (1 + x)}_{1 + x} {}+ 0                           \\
    \end{align*}
    \footnotetext{Подразумевается любой элемент кольца.}
\end{solution}

\begin{exercise}
    Найти \(\gcd(9 + 12i, 5)\) в \(\Z[i]\).
\end{exercise}
\begin{solution}
    \begin{align*}
        9 + 12i  & = 5 \cdot (2 + 2i) + ( - 1 + 2i) \\
        2 + 2i   & = ( - 1 + 2i) \cdot ( -i) + i    \\
        - 1 + 2i & = i \cdot 2 - 1                  \\
        i        & = ( - 1) \cdot ( - i)            \\
    \end{align*}
    \textbf{Ответ:} \(- 1\).
\end{solution}

\begin{exercise}
    Рассмотрим кольцо \(R\):
    \[R = \{n + md \mid n,m \in \Z\}, \quad d^2 = 3\]
    Какими свойствами обладает данное кольцо? Является ли оно областью целостности? Евклидовым?
\end{exercise}
\begin{solution}\itemfix
    \begin{itemize}
        \item В этом кольце есть единица, это \(1 + 0 \cdot d\).
        \item Коммутативность тривиальна.
        \item \[\sphericalangle (a + bd) \neq 0, (x + yd) \neq 0 : (a + bd)(x + yd) = 0\]
              \begin{align*}
                  (a + bd)(x + yd)        & = 0 \\
                  ax + ayd + bdx + 3by    & = 0 \\
                  ax + 3by + ayd + bdx    & = 0 \\
                  (ax + 3by) + d(ay + bx) & = 0
              \end{align*}
              \[\begin{cases}
                      ax + 3by = 0 \\
                      ay + bx = 0
                  \end{cases}\]
              \[\begin{cases}
                      axy + 3by^2 = 0 \\
                      ayx + bx^2 = 0
                  \end{cases}\]
              \begin{align*}
                  3by^2      & = bx^2 \\
                  3y^2       & = b^2  \\
                  \sqrt{3} y & = b
              \end{align*}
              Решений нет, следовательно, это кольцо целостности.
    \end{itemize}
\end{solution}

\begin{exercise}
    Рассмотрим кольцо \(R\):
    \[R = \{n + md \mid n,m \in \Z\}, \quad d^2 = 0\]
    Какими свойствами обладает данное кольцо? Является ли оно областью целостности? Евклидовым?
\end{exercise}
\begin{solution}\itemfix
    \begin{itemize}
        \item В этом кольце есть единица, это \(1 + 0 \cdot d\).
        \item Коммутативность тривиальна.
        \item \(d \cdot d = 0, d\) --- делитель нуля.
        \item \(\sphericalangle n_1 + m_1d, n_2 + m_2d\)
              \begin{align*}
                  n_1 + m_1d & = q \cdot (n_2 + m_2d) + r                       \\
                  n_1 + m_1d & = (n_3 + m_3d) \cdot (n_2 + m_2d) + (n_4 + m_4d) \\
                  n_1 + m_1d & = n_2n_3 + n_2m_3d + n_3m_2d + n_4 + m_4d
              \end{align*}
              \[\begin{cases}
                      n_1 = n_2n_3 + n_4 \\
                      m_1 = n_2m_3 + n_3m_2 + m_4
                  \end{cases}\]
              Первое уравнение решается в \(\Z\), при этом \(|n_4| < |n_2|\) или \(n_4 = 0\). Т.к. \(n_3\) и \(m_2\) теперь фиксированы, то уравнение \(m_1 - n_3m_2 = n_2m_3 + m_4\) также решается и \(|m_4|< |m_2|\) или \(m_4 = 0\). Тогда \(N(n + md) = |n| + |m|\) подходит и кольцо евклидово.
    \end{itemize}
\end{solution}

\end{document}
