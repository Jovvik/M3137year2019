\documentclass[12pt, a4paper]{article}

\usepackage{lastpage}
\usepackage{mathtools}
\usepackage{xltxtra}
\usepackage{libertine}
\usepackage{amsmath}
\usepackage{amsthm}
\usepackage{amsfonts}
\usepackage{amssymb}
\usepackage{enumitem}
\usepackage{xcolor}
\usepackage[left=2.3cm, right=2.3cm, top=2.7cm, bottom=2.7cm, bindingoffset=0cm, headheight=15pt]{geometry}
\usepackage{fancyhdr}
\usepackage[russian]{babel}
% \usepackage{parindent}

\pagestyle{fancy}
\lfoot{M3137y2019}
\rhead{\thepage\ из \pageref{LastPage}}

\newcommand{\R}{\mathbb{R}}
\newcommand{\Q}{\mathbb{Q}}
\newcommand{\C}{\mathbb{C}}
\newcommand{\Z}{\mathbb{Z}}
\newcommand{\B}{\mathbb{B}}
\newcommand{\N}{\mathbb{N}}

\DeclareMathOperator*{\xor}{\oplus}
\DeclareMathOperator*{\equ}{\sim}
\DeclareMathOperator{\Ln}{\text{Ln}}
\DeclareMathOperator{\sign}{\text{sign}}
\DeclareMathOperator{\Sym}{\text{Sym}}
\DeclareMathOperator{\Asym}{\text{Asym}}
% \DeclareMathOperator{\sh}{\text{sh}}
% \DeclareMathOperator{\tg}{\text{tg}}
% \DeclareMathOperator{\arctg}{\text{arctg}}
% \DeclareMathOperator{\ch}{\text{ch}}

\DeclarePairedDelimiter{\ceil}{\lceil}{\rceil}

\setmainfont{Linux Libertine}

\theoremstyle{plain}
\newtheorem{theorem}{Теорема}
\newtheorem{axiom}{Аксиома}
\newtheorem{lemma}{Лемма}

\theoremstyle{remark}
\newtheorem*{remark}{Примечание}
\newtheorem*{exercise}{Упражнение}
\newtheorem*{consequence}{Следствие}
\newtheorem*{example}{Пример}
\newtheorem*{observation}{Наблюдение}

\theoremstyle{definition}
\newtheorem*{definition}{Определение}
\newtheorem*{obozn}{Обозначение}

\lhead{Алгоритмы в математике \textit{(практика)}}
\cfoot{}
\rfoot{2.10.2021}

\begin{document}

\begin{exercise}
    Доказать, что остаток квадрата нечётного числа на \(8\) равен \(1\).
\end{exercise}
\begin{solution}
    \[(2n + 1)^2 = 4n^2 + 4n + 1 = 4n(n + 1) + 1\]
    Либо \(n \equiv 0 \pmod 2\), либо \(n + 1 \equiv 0 \pmod 2 \Rightarrow 4n(n + 1) \equiv 0 \pmod 2\). Тогда \(4n(n + 1) + 1 \equiv 1 \pmod 2\).
\end{solution}

\begin{exercise}
    Доказать, что \(n(n^2 + 1)(n^2 + 4)\) делится на \(5\).
\end{exercise}
\begin{solution}\itemfix
    \begin{caseof}
        \case{\(n \equiv 0 \pmod 5\)}{Очевидно.}
        \case{\(n \equiv 1 \pmod 5\)}{\(n^2 + 4 \equiv 1 + 4 \equiv 0 \pmod 5\)}
        \case{\(n \equiv 2 \pmod 5\)}{\(n^2 + 1 \equiv 4 + 1 \equiv 0 \pmod 5\)}
        \case{\(n \equiv 3 \pmod 5\)}{\(n^2 + 1 \equiv 9 + 1 \equiv 0 \pmod 5\)}
        \case{\(n \equiv 4 \pmod 5\)}{\(n^2 + 4 \equiv 16 + 4 \equiv 0 \pmod 5\)}
    \end{caseof}
\end{solution}

\begin{exercise}
    Найти все натуральные \(n\) для которых \(n^2 + 1 \divided n + 1\)
\end{exercise}
\begin{solution}
    Очевидно для \(n = 0\) и \(n = 1\) искомое верно. \(\sphericalangle n > 1\).

    \[n^2 - 1 = (n - 1)(n + 1) \divided n + 1 \Rightarrow n^2 - 1 \equiv 0 \pmod{n + 1} \Rightarrow n^2 + 1 \equiv 2 \pmod{n + 1}\]

    Для \(n > 1\) выполнено \(2 \not \equiv 0 \pmod{n + 1}\), таким образом ответ \(n = 0\) и \(n = 1\).
\end{solution}

\begin{exercise}
    Доказать, что \(n^9 + 17n^3 - 18\) делится на \(3\).
\end{exercise}
\begin{solution}
    \begin{caseof}
        \case{\(n \equiv 0 \pmod 3\)}{\(n^9 + 17n^3 - 18 \equiv - 18 \equiv 0 \pmod 3\)}
        \case{\(n \equiv 1 \pmod 3\)}{\(n^9 + 17n^3 - 18 \equiv 1 + 17 - 18 \equiv 0 \pmod 3\)}
        \case{\(n \equiv 2 \pmod 3\)}{\(n^9 + 17n^3 - 18 \equiv 512 + 136 - 18 \equiv 630 \equiv 0 \pmod 3\)}
    \end{caseof}
\end{solution}

\begin{exercise}
    Доказать, что \(5ab\) делится на \(45\), если \(a^6 + b^6\) делится на \(3\).
\end{exercise}
\begin{solution}
    \(5ab \divided 45 \Leftrightarrow ab \divided 9\)

    \begin{center}
        \begin{tabular}{CC}
            \toprule
            a & a^6 \pmod 3 \\ \midrule
            0 & 0           \\
            1 & 1           \\
            2 & 1           \\ \bottomrule
        \end{tabular}
    \end{center}
    \begin{caseof}
        \case{\(a \equiv 0 \pmod 3\)}{Тогда \(b^6 \equiv 0 \pmod 3 \Rightarrow b \equiv 0 \pmod 3 \Rightarrow a = 3k, b = 3l \Rightarrow ab = 9l \divided 9\)}
        \case{\(a \equiv 1 \pmod 3\)}{Тогда \(b^6 \equiv 2 \pmod 3\), но \(\nexists b\). Таким образом, \(a \not \equiv \pmod 3\).}
        \case{\(a \equiv 2 \pmod 3\)}{Тогда \(b^6 \equiv - a^6 \equiv 2 \pmod 3\), см. случай 2.}
    \end{caseof}
\end{solution}

\end{document}
