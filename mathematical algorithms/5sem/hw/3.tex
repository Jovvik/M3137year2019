\documentclass[12pt, a4paper]{article}

\usepackage{lastpage}
\usepackage{mathtools}
\usepackage{xltxtra}
\usepackage{libertine}
\usepackage{amsmath}
\usepackage{amsthm}
\usepackage{amsfonts}
\usepackage{amssymb}
\usepackage{enumitem}
\usepackage{xcolor}
\usepackage[left=2.3cm, right=2.3cm, top=2.7cm, bottom=2.7cm, bindingoffset=0cm, headheight=15pt]{geometry}
\usepackage{fancyhdr}
\usepackage[russian]{babel}
% \usepackage{parindent}

\pagestyle{fancy}
\lfoot{M3137y2019}
\rhead{\thepage\ из \pageref{LastPage}}

\newcommand{\R}{\mathbb{R}}
\newcommand{\Q}{\mathbb{Q}}
\newcommand{\C}{\mathbb{C}}
\newcommand{\Z}{\mathbb{Z}}
\newcommand{\B}{\mathbb{B}}
\newcommand{\N}{\mathbb{N}}

\DeclareMathOperator*{\xor}{\oplus}
\DeclareMathOperator*{\equ}{\sim}
\DeclareMathOperator{\Ln}{\text{Ln}}
\DeclareMathOperator{\sign}{\text{sign}}
\DeclareMathOperator{\Sym}{\text{Sym}}
\DeclareMathOperator{\Asym}{\text{Asym}}
% \DeclareMathOperator{\sh}{\text{sh}}
% \DeclareMathOperator{\tg}{\text{tg}}
% \DeclareMathOperator{\arctg}{\text{arctg}}
% \DeclareMathOperator{\ch}{\text{ch}}

\DeclarePairedDelimiter{\ceil}{\lceil}{\rceil}

\setmainfont{Linux Libertine}

\theoremstyle{plain}
\newtheorem{theorem}{Теорема}
\newtheorem{axiom}{Аксиома}
\newtheorem{lemma}{Лемма}

\theoremstyle{remark}
\newtheorem*{remark}{Примечание}
\newtheorem*{exercise}{Упражнение}
\newtheorem*{consequence}{Следствие}
\newtheorem*{example}{Пример}
\newtheorem*{observation}{Наблюдение}

\theoremstyle{definition}
\newtheorem*{definition}{Определение}
\newtheorem*{obozn}{Обозначение}

\lhead{Алгоритмы в математике \textit{(практика)}}
\cfoot{}
\rfoot{9.10.2021}

\begin{document}

\begin{exercise}
    Решить сравнения:
    \begin{itemize}
        \item \(3x \equiv 1 \mod 7\)
        \item \(100x \equiv 21 \mod 23\)
        \item \(315x \equiv - 10 \mod 275\)
        \item \(76x \equiv 232 \mod 220\)
    \end{itemize}
\end{exercise}
\begin{solution}\itemfix
    \begin{itemize}
        \item \((3, 7) = 1 \Rightarrow\) решение единственно. Решение \(x \equiv 5 \mod 7\).
        \item \((100, 23) = 1 \Rightarrow\) решение единственно. Решение \(x \equiv 17 \mod 23\).
        \item \((315, 275) = 5 \Rightarrow 63x \equiv - 2 \mod 55\). Решение этого уравнения \textit{(алгоритмом Евклида)} \(x \equiv 41 \mod 55\). Решения исходного уравнения это \(x_1 \equiv 41 \mod 275, x_2 \equiv 96 \mod 275, x_3 \equiv 151 \mod 275, x_4 \equiv 206 \mod 275, x_5 \equiv 261 \mod 275\)
        \item \((76, 220) = 4 \Rightarrow 19x \equiv 3 \mod 55\). Решение этого уравнения \(x \equiv 32 \mod 55\). Решения исходного уравнения это \(x_1 \equiv 32 \mod 220, x_2 \equiv 87 \mod 220, x_3 \equiv 142 \mod 220, x_4 \equiv 197 \mod 220\)
    \end{itemize}
\end{solution}

\begin{exercise}
    Найти число целых точек на прямой \(8x - 13y + 6 = 0\), которые лежат между прямыми \(x = 100\) и \(x = -100\).
\end{exercise}
\begin{solution}
    Решим \(8x + 6 \equiv 0 \mod 13\). Т.к. \((8, 13) = 1\), решение единственно. Это решение \(x = 9\). Тогда для каждого \(\tilde{x} \equiv 9 \mod 13\) и какого-либо \(\tilde{y}\), являющегося решением исходного уравнения, \((\tilde{x}, \tilde{y})\) --- решение исходного уравнения. Кроме того, все решения уравнения записываются в таком виде. Всего \(\tilde{x} : \tilde{x} \equiv 9 \mod 13\) и \(\tilde{x} \in [ - 100, 100]\) существует \(\floor{\frac{100 + 95}{13}} + 1 = 16\) штук, т.к. ``крайние'' \(\tilde{x}\) это \( - 95\) и \(100\).
\end{solution}

\begin{exercise}
    Найти все решения в целых числах:
    \[34x + 26y = 6\]
\end{exercise}
\begin{solution}
    \[34x + 26y = 6\]
    Решим следующее уравнение:
    \[34a + 26b = (34, 26) = 2\]
    Расширенный алгоритм Евклида даёт частное решение:
    \begin{center}
        \begin{tabular}{LCCCC}
            \toprule
            i & r_i & s_i & t_i & q_i \\ \midrule
            0 & 34  & 1   & 0   &     \\
            1 & 26  & 0   & 1   &     \\
            2 & 8   & 1   & -1  & 1   \\
            3 & 2   & -3  & 4   & 3   \\
            4 & 0   & 13  & -17 & 4   \\
            \bottomrule
        \end{tabular}
    \end{center}
    \(a = s_3 = - 3, b = t_3 = 4\)

    Тогда \(x_0 = a \cdot \frac{6}{2} = - 9, y_0 = b \cdot \frac{6}{2} = 12\) --- частное решение исходного уравнения.

    Рассмотрим решение в общем виде:
    \[\begin{cases}
            x = x_0 + k \cdot n \\
            y = y_0 + k \cdot m
        \end{cases}, k \in \Z\]
    \begin{align*}
        34(x_0 + k \cdot n) + 26(y_0 + k \cdot m) & = 6     \\
        34 k \cdot n + 26 k \cdot m               & = 0     \\
        34 n                                      & = - 26m \\
        17 n                                      & = - 13m
    \end{align*}

    Т.к. \((17, 13) = 1\), \(n = \pm 13, m = \mp 17\)

    Ответ: \(\{\ev{ - 9 + k \cdot 13, 12 - k \cdot 17} \mid k \in \Z\}\), где \(\ev{\cdot,\cdot}\) обозначает пару.
\end{solution}

\begin{exercise}
    Решить сравнение \((a^2 + b^2)x \equiv a - b \mod ab, (a, b) = 1\)
\end{exercise}
\begin{solution}
    \begin{align*}
        (a^2 + b^2)x       & \equiv a - b \mod ab               \\
        (a^2 + b^2 - 2ab)x & \equiv a - b \mod ab               \\
        (a - b)^2 x        & \stackrel{*}{\equiv} a - b \mod ab \\
        (a - b) x          & \equiv 1 \mod ab
    \end{align*}

    Переход \(*\) верный, т.к. \((a - b, ab) = 1\):

    \begin{statement}
        \((a - b, ab) = 1\)
    \end{statement}
    \begin{proof}
        Предположим обратное: \((a - b, ab) = k \neq 1\). Возьмём \(\tilde{k}\) --- произвольное простое число, делящее \(k\). Такое существует, т.к. \(k > 1\).

        Тогда \(ab \divided \tilde{k}\) и \(a - b \divided \tilde{k} \Rightarrow a = \tilde{k}p + b\). Тогда \((\tilde{k}p + b)b \divided \tilde{k} \Rightarrow \tilde{k}pb + b^2 \divided \tilde{k} \Rightarrow b^2 \divided \tilde{k}\). По простоте \(\tilde{k}\) \(b = \tilde{k}n\), т.к. иначе \([b]\) делитель \(\tilde{k}\), отличный от него и единицы \textit{(случай \(b = 1\) тривиален, т.к. \((a - 1, a) = 1\) всегда выполнено)}. Аналогичными выкладками \(a \divided \tilde{k}\) и следовательно \((a, b) = \tilde{k} \cdot t\), что противоречит условию.
    \end{proof}

    \(x\) находится как мультипликативное обратное к \(a - b\) в кольце \(\Z_{ab}\).
\end{solution}

\end{document}
