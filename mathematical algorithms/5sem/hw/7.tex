\documentclass[12pt, a4paper]{article}

\usepackage{lastpage}
\usepackage{mathtools}
\usepackage{xltxtra}
\usepackage{libertine}
\usepackage{amsmath}
\usepackage{amsthm}
\usepackage{amsfonts}
\usepackage{amssymb}
\usepackage{enumitem}
\usepackage{xcolor}
\usepackage[left=2.3cm, right=2.3cm, top=2.7cm, bottom=2.7cm, bindingoffset=0cm, headheight=15pt]{geometry}
\usepackage{fancyhdr}
\usepackage[russian]{babel}
% \usepackage{parindent}

\pagestyle{fancy}
\lfoot{M3137y2019}
\rhead{\thepage\ из \pageref{LastPage}}

\newcommand{\R}{\mathbb{R}}
\newcommand{\Q}{\mathbb{Q}}
\newcommand{\C}{\mathbb{C}}
\newcommand{\Z}{\mathbb{Z}}
\newcommand{\B}{\mathbb{B}}
\newcommand{\N}{\mathbb{N}}

\DeclareMathOperator*{\xor}{\oplus}
\DeclareMathOperator*{\equ}{\sim}
\DeclareMathOperator{\Ln}{\text{Ln}}
\DeclareMathOperator{\sign}{\text{sign}}
\DeclareMathOperator{\Sym}{\text{Sym}}
\DeclareMathOperator{\Asym}{\text{Asym}}
% \DeclareMathOperator{\sh}{\text{sh}}
% \DeclareMathOperator{\tg}{\text{tg}}
% \DeclareMathOperator{\arctg}{\text{arctg}}
% \DeclareMathOperator{\ch}{\text{ch}}

\DeclarePairedDelimiter{\ceil}{\lceil}{\rceil}

\setmainfont{Linux Libertine}

\theoremstyle{plain}
\newtheorem{theorem}{Теорема}
\newtheorem{axiom}{Аксиома}
\newtheorem{lemma}{Лемма}

\theoremstyle{remark}
\newtheorem*{remark}{Примечание}
\newtheorem*{exercise}{Упражнение}
\newtheorem*{consequence}{Следствие}
\newtheorem*{example}{Пример}
\newtheorem*{observation}{Наблюдение}

\theoremstyle{definition}
\newtheorem*{definition}{Определение}
\newtheorem*{obozn}{Обозначение}

\lhead{Алгоритмы в математике \textit{(практика)}}
\cfoot{}
\rfoot{6.11.2021}

\begin{document}

\begin{exercise}
    Пусть \(H, K \vartriangleleft G\) --- две нормальные подгруппы в \(G\). Докажите, что тогда коммутатор любых двух элементов из \(H\) и \(K\) принадлежит пересечению \(H \cap K\).
\end{exercise}
\begin{solution}
    \begin{lemma}
        \((ab)^{-1} = b^{-1}a^{-1}\)
    \end{lemma}
    \begin{proof}
        \((ab)(b^{-1}a^{-1}) = e\)
    \end{proof}
    \(\sphericalangle h \in H, k \in K\)
    \[[h, k] = hk(kh)^{-1} = \overbrace{\underbrace{\lefteqn{\overbrace{\phantom{hkh^{-1}}}^{\in K}}h
        \underbrace{kh^{-1}k^{-1}}_{\in H}}_{\in H}}^{\in K}\]
    \([k, h]\) аналогично.
\end{solution}

\begin{exercise}
    Показать, что коммутант \([H, K]\) двух нормальных подгрупп \(H, K \vartriangleleft G\) есть подгруппа в пересечении \([H, K] \subset H \cap K\). Всегда ли \([H, K] = H \cap K\)?
\end{exercise}
\begin{solution}
    Т.к. коммутатор любых двух элементов \(H\) и \(K\) принадлежит и \(H\), и \(K\), то по замкнутости \(H\) и \(K\) произведение коммутаторов также принадлежит и \(H\) и \(K\). Кроме того, \(1 = [1, 1] \in [H, K]\), следовательно, \([H, K] \subset H \cap K\), т.к. это \([H, K]\) это в точности все коммутаторы вида \([hk], h \in H, k \in K\).

    \([H, K]\) не всегда \( = H \cap K\), например если \(G\) абелева, то \([H, K] = \{e\}\), но очевидно не для каждых \(H\) и \(K\) выполнено \(H \cap K = \{e\}\), например для \(H = K = G\).
\end{solution}

\begin{exercise}
    Пусть \(H, K\) --- две произвольные подгруппы. Рассмотреть отображение \(\psi : H \times K \to HK\):
    \[\psi(h, k) = hk\]
    Найти \(\psi^{-1}(x) = \{(h, k) \mid \psi(h, k) = x\}\) в явном виде. Получить из этого, что:
    \[|HK| = \frac{|H| \cdot |K|}{|H \cap K|}\]
\end{exercise}
\begin{solution}
    Пусть \(h_1k_1 = x\) и \(h_2k_2 = x\).
    \[x^{-1} = k_2^{-1}h_2^{-1}\]
    \[e = x x^{-1} = h_1k_1k_2^{-1}h_2^{-1}\]
    \[h_2h_1^{-1} = k_1k_2^{-1}\]
    Т.к. \(h_2h_1^{-1} \in H, k_1k_2^{-1} \in K, k_1k_2^{-1} = h_2h_1^{-1} \eqqcolon a \in K \cap H\). Тогда:
    \[h_2 = ih_1 \Rightarrow h_1 = i^{-1}h_2 \quad k_1 = ik_2\]
    И, следовательно, любое \((h_1, k_1) : h_1k_1 = x\) записывается в виде \((i^{-1}h_2, ik_2)\), где \(h_2, k_2\) --- произвольное решение уравнения \(hk = x\). Таким образом:
    \[\psi^{-1}(x) = \{(i^{-1}h_2 , ik_2) \mid i \in K \cap H\} \Rightarrow |\psi^{-1}(x)|= |K \cap H|\]
    \[|H| \cdot |K| = \sum_{x \in HK} |\psi^{-1}(x)| = |HK| \cdot |H \cap K|\]
\end{solution}

\begin{exercise}
    Показать, что среди 5 подгрупп порядков \(483, 1309, 3059, 2783, 3451\) есть хотя бы две абелевы.
\end{exercise}
\begin{solution}
    Разложим размеры групп на простые делители.
    \begin{center}
        \begin{tabular}{CC}\toprule
            n    & p_1^{a_1} \dots p_n^{a_n} \\ \midrule
            483  & 3 \cdot 7 \cdot 23        \\
            1309 & 7 \cdot 11 \cdot 17       \\
            3059 & 7 \cdot 19 \cdot 23       \\
            2783 & 11^2 \cdot 23             \\
            3451 & 7 \cdot 11 \cdot 29       \\
            \bottomrule
        \end{tabular}
    \end{center}

    Группу размера \(2783\) больше рассматривать не будем. Размеры всех остальных групп разбиваются на простые числа в первой степени, следовательно, по первой теореме Силова у каждой группы есть силовские \(p\)--подгруппы порядка каждого такого простого числа. Кроме того, они циклические.

    \begin{statement}
        \(G \cong H \times K \Leftrightarrow
        \begin{cases}
            G = HK           \\
            H \cap K = \{e\} \\
            H, K \vartriangleleft G
        \end{cases}\), аналогично для большего числа подгрупп.
    \end{statement}

    Найдём все такие группы, что их силовские подгруппы \(H, K, L\) нормальны, тогда
    \[|HKL| = \frac{|H| \cdot |K| \cdot |L|}{|H \cap K \cap L|} = |G|\]
    и, следовательно, каждому элементу \(g\) можно взаимно-однозначно сопоставить элемент из \(HKL\), т.е. \(G \cong HKL\) и по утверждению \(G \cong H \times K \times L\), а прямое произведение является абелевой группой.

    Из соображений предыдущего домашнего задания для нормальности силовской \(p_i\)-подгруппы \(\mathcal{P}_{p_i}\) достаточно, чтобы \(p_i \not\equiv 1 \mod p_j \ \ \forall j\).

    \begin{center}
        \begin{tabular}{L|CCC}
            483 & 3 & 7     & 23 \\ \hline
            3   & 0 & 3     & 3  \\
            7   & 1 & \dots      \\
            \dots
        \end{tabular} \quad
        \begin{tabular}{L|CCC}
            1309 & 7 & 11 & 17 \\ \hline
            7    & 0 & 7  & 7  \\
            11   & 4 & 0  & 11 \\
            17   & 3 & 6  & 0
        \end{tabular} \quad
        \begin{tabular}{L|CCC}
            3059 & 7 & 19 & 23 \\ \hline
            7    & 0 & 7  & 7  \\
            19   & 5 & 0  & 19 \\
            23   & 2 & 4  & 0
        \end{tabular}
    \end{center}

    Дальше нет смысла перебирать, т.к. для групп размера \(1309\) и \(3059\) искомое доказано.
\end{solution}

\end{document}
