\documentclass[12pt, a4paper]{article}

\usepackage{lastpage}
\usepackage{mathtools}
\usepackage{xltxtra}
\usepackage{libertine}
\usepackage{amsmath}
\usepackage{amsthm}
\usepackage{amsfonts}
\usepackage{amssymb}
\usepackage{enumitem}
\usepackage{xcolor}
\usepackage[left=2.3cm, right=2.3cm, top=2.7cm, bottom=2.7cm, bindingoffset=0cm, headheight=15pt]{geometry}
\usepackage{fancyhdr}
\usepackage[russian]{babel}
% \usepackage{parindent}

\pagestyle{fancy}
\lfoot{M3137y2019}
\rhead{\thepage\ из \pageref{LastPage}}

\newcommand{\R}{\mathbb{R}}
\newcommand{\Q}{\mathbb{Q}}
\newcommand{\C}{\mathbb{C}}
\newcommand{\Z}{\mathbb{Z}}
\newcommand{\B}{\mathbb{B}}
\newcommand{\N}{\mathbb{N}}

\DeclareMathOperator*{\xor}{\oplus}
\DeclareMathOperator*{\equ}{\sim}
\DeclareMathOperator{\Ln}{\text{Ln}}
\DeclareMathOperator{\sign}{\text{sign}}
\DeclareMathOperator{\Sym}{\text{Sym}}
\DeclareMathOperator{\Asym}{\text{Asym}}
% \DeclareMathOperator{\sh}{\text{sh}}
% \DeclareMathOperator{\tg}{\text{tg}}
% \DeclareMathOperator{\arctg}{\text{arctg}}
% \DeclareMathOperator{\ch}{\text{ch}}

\DeclarePairedDelimiter{\ceil}{\lceil}{\rceil}

\setmainfont{Linux Libertine}

\theoremstyle{plain}
\newtheorem{theorem}{Теорема}
\newtheorem{axiom}{Аксиома}
\newtheorem{lemma}{Лемма}

\theoremstyle{remark}
\newtheorem*{remark}{Примечание}
\newtheorem*{exercise}{Упражнение}
\newtheorem*{consequence}{Следствие}
\newtheorem*{example}{Пример}
\newtheorem*{observation}{Наблюдение}

\theoremstyle{definition}
\newtheorem*{definition}{Определение}
\newtheorem*{obozn}{Обозначение}

\lhead{Алгоритмы в математике \textit{(практика)}}
\lfoot{Михайлов Максим}
\cfoot{}
\rfoot{11.12.2021}

\begin{document}

\begin{exercise}
    Пусть \(d = \sqrt[3]{2}\). Рассмотрим кольцо порожденное элементами \(1, d\). Показать, что данное кольцо представимо в виде
    \[R = \{a + bd + cd^2 \mid a, b, c \in \Z\}.\]
    Разрешимо ли в рамках кольца \(R\) уравнение:
    \[(1 - 3d + 5d^2)x =- 18 + 10d + 20d^2\]
    Присутствуют ли в данном кольце делители нуля?
\end{exercise}
\begin{solution}
    Очевидно, что \(\ev{1} = \Z, \ev{d} = d\Z\). В \(\ev{1, d}\) лежит их сумма, т.е. \(\Z + d\Z\). \(\Z \cdot (d\Z) = d\Z\), что не добавляет новых элементов. \((d\Z) \cdot (d\Z) = d^2 \Z\) и тогда промежуточный результат это \(\Z + d\Z + d^2\Z\). \(\Z \cdot d^2\Z = d^2\Z, d\Z \cdot d^2\Z = \Z, d^2\Z \cdot d^2\Z = d\Z\), поэтому больше нечего добавлять.

    \begin{align*}
        (1 - 3d + 5d^2)(a + bd + cd^2)                        & = - 18 + 10d + 20d^2 \\
        a + bd + cd^2 - 3ad - 3bd^2 - 6c + 5ad^2 + 10b + 10cd & = - 18 + 10d + 20d^2
    \end{align*}
    \[\begin{cases}
            a - 6c + 10b = - 18 \\
            b - 3a + 10c = 10   \\
            c - 3b + 5a = 20
        \end{cases}\]
    Система не вырождена, решение есть.

    Делители нуля:
    \begin{align*}
        xy                                                                                            & = 0 \\
        (a_1 + b_1d + c_1d^2)(a_2 + b_2d + c_2d^2)                                                    & = 0 \\
        a_1a_2 + a_1b_2d + a_1c_2d^2 + b_1a_2d + b_1b_2d^2 + 2b_1c_2 + c_1a_2d^2 + 2c_1b_2 + 2c_1c_2d & = 0 \\
    \end{align*}
    \[\begin{cases}
            a_1a_2 + 2b_1c_2 + 2c_1b_2 = 0 \\
            a_1b_2 + b_1a_2 + 2c_1c_2 = 0  \\
            a_1c_2 + b_1b_2 + c_1a_2 = 0
        \end{cases}\]
    Что делать с этой системой нелинейных диофантовых уравнений не очень понятно.

    % Сведение этой системы нелинейных диофантовых уравнений к 2-SAT показывает, что решений (ненулевых) нет.
\end{solution}

\begin{exercise}
    Рассмотрим кольцо многочленов \(R = \R[x]\) и множество:
    \[J = \{p \mid p \divided x^2 + 1\}\]
    Показать, что \(J\) есть идеал. Построить \(\faktor{R}{J}\). Существуют ли в \(\faktor{R}{J}\) делители нуля?
\end{exercise}
\begin{solution}
    То, что \(J\) является подкольцом, очевидно.

    \(\sphericalangle a \in R, p \cdot (x^2 + 1) \in J\).
    \[a \cdot p \cdot (x^2 + 1) \in J\]
    Таким образом, \(J\) --- идеал.

    В каждом классе из \(\faktor{R}{J}\) есть ровно один элемент вида \(ax + b\), потому что если коэффициент при \(x^{n + 2}\) ненулевой и \(n \geq 0\), то такой многочлен можно представить как \((x^2 + 1) \cdot x^n \cdot a + p\) и тогда любой многочлен лежит в \([ax + b]\) для каких-то \(a\) и \(b\).

    Заметим, что в \(\faktor{R}{J}\) \([x^2 + 1] = [0]\), следовательно, \([x^2] = [- 1]\). Итого \(\faktor{R}{J} = \{ax + b \mid a, b \in \Z\}\) со стандартным сложением и умножением таким, что \(x^2 = - 1\). Несложно также заметить, что \(\faktor{R}{J} \cong \mathbb{C}\) по гомоморфизму \([ax + b] \mapsto b + ia\).

    В \(\mathbb{C}\) нет делителей нуля, так что и в \(\faktor{R}{J}\) их нет.
\end{solution}

\begin{exercise}
    Вычислить
    \begin{enumerate}
        \item \(\varphi(360)\)
        \item \(\varphi(125)\)
        \item \(\varphi(\varphi(12))\)
    \end{enumerate}
\end{exercise}
\begin{solution}\itemfix
    \begin{enumerate}
        \item \(\varphi(360) = \varphi(8) \cdot \varphi(9) \cdot \varphi(5) = 4 \cdot (2 - 1) \cdot 3 \cdot (3 - 1) \cdot 4 = 96\)
        \item \(\varphi(125) = \varphi(5^3) = 5^2 \cdot (5 - 1) = 100\)
        \item \(\varphi(\varphi(12)) = \varphi(|\{1,5,7,11\}|) = \varphi(4) = |\{1,3\}| = 2\)
    \end{enumerate}
\end{solution}

\begin{exercise}
    Пусть \(a, n \in \Z\) два взаимно простых числа \((a, n) = 1\). Показать, что:
    \[a^{\varphi(n)} \equiv 1 \pmod n\]
\end{exercise}
\begin{solution}
    Рассмотрим мультипликативную группу \(A\) взаимно простых с \(n\) чисел по модулю \(n\). Очевидно это действительно группа. \(|A| = \varphi(n)\). По теореме Лагранжа \(|A| \divided |\ev{a}|\), т.е. \(|\ev{a}| \cdot k = \varphi(n)\).
    \[a^{\varphi(n)} = a^{|\ev{a}| \cdot k}\]
    Из структуры \(A\) понятно, что \(\ev{a}\) это простой цикл и тогда \(a^{|\ev{a}|} \equiv 1 \pmod n\).
    \[a^{\varphi(n)} = 1 \pmod n\]
\end{solution}

\end{document}
