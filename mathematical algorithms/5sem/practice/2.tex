\documentclass[12pt, a4paper]{article}

\usepackage{lastpage}
\usepackage{mathtools}
\usepackage{xltxtra}
\usepackage{libertine}
\usepackage{amsmath}
\usepackage{amsthm}
\usepackage{amsfonts}
\usepackage{amssymb}
\usepackage{enumitem}
\usepackage{xcolor}
\usepackage[left=2.3cm, right=2.3cm, top=2.7cm, bottom=2.7cm, bindingoffset=0cm, headheight=15pt]{geometry}
\usepackage{fancyhdr}
\usepackage[russian]{babel}
% \usepackage{parindent}

\pagestyle{fancy}
\lfoot{M3137y2019}
\rhead{\thepage\ из \pageref{LastPage}}

\newcommand{\R}{\mathbb{R}}
\newcommand{\Q}{\mathbb{Q}}
\newcommand{\C}{\mathbb{C}}
\newcommand{\Z}{\mathbb{Z}}
\newcommand{\B}{\mathbb{B}}
\newcommand{\N}{\mathbb{N}}

\DeclareMathOperator*{\xor}{\oplus}
\DeclareMathOperator*{\equ}{\sim}
\DeclareMathOperator{\Ln}{\text{Ln}}
\DeclareMathOperator{\sign}{\text{sign}}
\DeclareMathOperator{\Sym}{\text{Sym}}
\DeclareMathOperator{\Asym}{\text{Asym}}
% \DeclareMathOperator{\sh}{\text{sh}}
% \DeclareMathOperator{\tg}{\text{tg}}
% \DeclareMathOperator{\arctg}{\text{arctg}}
% \DeclareMathOperator{\ch}{\text{ch}}

\DeclarePairedDelimiter{\ceil}{\lceil}{\rceil}

\setmainfont{Linux Libertine}

\theoremstyle{plain}
\newtheorem{theorem}{Теорема}
\newtheorem{axiom}{Аксиома}
\newtheorem{lemma}{Лемма}

\theoremstyle{remark}
\newtheorem*{remark}{Примечание}
\newtheorem*{exercise}{Упражнение}
\newtheorem*{consequence}{Следствие}
\newtheorem*{example}{Пример}
\newtheorem*{observation}{Наблюдение}

\theoremstyle{definition}
\newtheorem*{definition}{Определение}
\newtheorem*{obozn}{Обозначение}

\lhead{Алгоритмы в математике \textit{(практика)}}
\cfoot{}
\rfoot{18.9.2021}

\begin{document}

\section*{Подгруппы в полугруппах}

На прошлой практике мы остановились на моноиде, считающем число строк:
\[S \coloneqq \{(l, n, r) \mid l, r \in \{0, 1\}, n \in \N_0\}\]
\[(l_1, n_1, r_1) \cdot (l_2, n_2, r_2) = (l_1, n_1 + n_2 + r_1l_2, r_2)\]

Идемпотенты:
\begin{enumerate}
    \item \(x_1 = (0,0,0)\)
    \item \(x_2 = (1,0,0)\)
    \item \(x_3 = (0,0,1)\)
\end{enumerate}

Рассмотрим \(H_2 = x_2 \cdot S \cdot x_2 \defeq \{x_2 \cdot y \cdot x_2 \mid y \in S\} = \{(0, n, 1) \mid n \in \N_0\}\). В \(H_2\) выполняется \(zx = zy \Rightarrow x = y\), т.к. \(H \sim (\N_0, +)\) c изоморфизмом \((0,n,1) \mapsto n\).

Аналогично можно построить \(H_3 \coloneqq x_3 S x_3, H_1 \coloneqq x_1 S x_1\). Такие подмножества являются подполугруппами \(S\):
\[y, z \in H_1 \quad yz = x_1 \hat{y} x_1 x_1 \hat{z} x_1 = x_1 \underbrace{(\hat{y} x_1 x_1 \hat{z})}_{\in S} x_1\]

В \(H_1\) есть единица --- \(x_1\), т.к. \(y x_1 = x_1 \hat{y} x_1 x_1 x_1 x_1 = x_1 \hat{y} x_1 = y\)
\[H_1 \cap H_2 = \emptyset\]

Рассмотрим полугруппу \(S\) и некоторый \(a \in S\). Построим подмножество \(H\), называемое \textbf{генератором} \(a\), обозначаемое \(\ev{a}\):
\[H = \ev{a} \coloneqq \{a, a^2, a^3 \dots \}\]

\begin{example}
    \(\sphericalangle (\Z, +), a = 2\). Тогда \(H = \{2n \mid n \geq 1\}\).
\end{example}

Может быть \(|H| < \infty\). Тогда \(a^n = a^m\). Выберем наименьшее такое \(n\). Тогда у нас есть некоторый ``хвост'' \(a^1 \dots a^n\) и цикл \(a^n, a^{n + 1} \dots a^m\). \(n\) называется \textbf{индексом} \textit{(также обозначается \(i\))} \(a\), \(d = m - n\) --- \textbf{периодом}.

\begin{statement}
    Среди \(\ev{a}\) есть идемпотент, если \(|\ev{a}| < \infty \)
\end{statement}
\begin{proof}
    \[a^k = a^{k + \alpha d} \ \ \forall \alpha \in \N_1\]
    \[a^k \cdot a^k = a^k \Rightarrow 2k = k + \alpha d \Rightarrow k = \alpha d\]
\end{proof}

Пусть \(e = a^k, e^2 = e, G = eHe\)

\begin{example}
    \(i = 5, d = 3\)

    \(G = \{a^6, a^5, a^7\}\).
\end{example}

\(G\) является подгруппой, т.к. оно содержит все элементы цикла и обратное к \(a^j\) есть \(a^{k - j + d}\)

\section*{Отношения}

\(X = \R\), \(S = \mathbb{B}(X^2) = \{R \mid R \subseteq X \times X\}\)

\(\rho, \tau \in S \quad R = \rho \circ \tau : x R z \Leftrightarrow \exists y : x \rho y, y \tau z\)

Это определение согласовано с композицией функций.

Подполугруппы \(S\):
\begin{itemize}
    \item \(H_1 = X \to X\) --- функции
    \item \(H_2\) --- отношения эквивалентности --- не работает, т.к. есть контрпример:

          \(X = \{a, b, c\}, \rho = \{(a, c), (c, a), (a, a), (b, b), (c, c)\}, \tau = \{(b, c), (c, b), (b, b), (a, a), (c, c)\}\)
\end{itemize}

\[\rho^{ - 1} \coloneqq \{(b, a) \mid (a, b) \in \rho\}\]
\[\rho \circ \rho^{ - 1} = R \quad \rho^{-1} \circ \rho = L\]

\(\sphericalangle \rho(x) = x^2, x\rho^{-1}y \Leftrightarrow y = \pm \sqrt{x}\)

\(\rho \circ \rho^{-1} = R \quad aRb \Leftrightarrow a\rho a^2, a^2\rho^{-1}b \Leftrightarrow a = \pm b\)

\section*{Внешние законы}

Рассмотрим плоскость \(\mathbb{E}^2\) и множество трансляций \(T = \{\vec{v}\}\). \(T\) действует на \(\mathbb{E}^2\). Добавим в \(T\) закон сложения\footnote{Обычное сложение векторов.}.

Рассмотрим \(R\) --- множество поворотов плоскости относительно точки \(P\). \(R\) тоже действует на \(\mathbb{E}^2\). Определим композицию в \(R\) --- последовательное действие поворотов. Обозначим эту композицию ``\( \cdot \)''.

\(\sphericalangle \rho \in R, \tau \in T\). Можно делать \(\rho(\tau(P))\), можно \(\tau(\rho(P))\). Это не одно и то же \textit{(пример очевиден)}. Определим \(\hat{\tau}\) как \(\rho(\tau) =\hat{\tau}\). Все эти внешние законы согласованы:
\[\rho(\tau(P_0)) = \hat{\tau}(\rho(P_0)) = \hat{\tau}(P_0)\]
\[\rho(\tau_1 + \tau_2) = \rho(\tau_1) + \rho(\tau_2)\]

\end{document}
