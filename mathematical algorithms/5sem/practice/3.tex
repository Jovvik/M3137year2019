\documentclass[12pt, a4paper]{article}

\usepackage{lastpage}
\usepackage{mathtools}
\usepackage{xltxtra}
\usepackage{libertine}
\usepackage{amsmath}
\usepackage{amsthm}
\usepackage{amsfonts}
\usepackage{amssymb}
\usepackage{enumitem}
\usepackage{xcolor}
\usepackage[left=2.3cm, right=2.3cm, top=2.7cm, bottom=2.7cm, bindingoffset=0cm, headheight=15pt]{geometry}
\usepackage{fancyhdr}
\usepackage[russian]{babel}
% \usepackage{parindent}

\pagestyle{fancy}
\lfoot{M3137y2019}
\rhead{\thepage\ из \pageref{LastPage}}

\newcommand{\R}{\mathbb{R}}
\newcommand{\Q}{\mathbb{Q}}
\newcommand{\C}{\mathbb{C}}
\newcommand{\Z}{\mathbb{Z}}
\newcommand{\B}{\mathbb{B}}
\newcommand{\N}{\mathbb{N}}

\DeclareMathOperator*{\xor}{\oplus}
\DeclareMathOperator*{\equ}{\sim}
\DeclareMathOperator{\Ln}{\text{Ln}}
\DeclareMathOperator{\sign}{\text{sign}}
\DeclareMathOperator{\Sym}{\text{Sym}}
\DeclareMathOperator{\Asym}{\text{Asym}}
% \DeclareMathOperator{\sh}{\text{sh}}
% \DeclareMathOperator{\tg}{\text{tg}}
% \DeclareMathOperator{\arctg}{\text{arctg}}
% \DeclareMathOperator{\ch}{\text{ch}}

\DeclarePairedDelimiter{\ceil}{\lceil}{\rceil}

\setmainfont{Linux Libertine}

\theoremstyle{plain}
\newtheorem{theorem}{Теорема}
\newtheorem{axiom}{Аксиома}
\newtheorem{lemma}{Лемма}

\theoremstyle{remark}
\newtheorem*{remark}{Примечание}
\newtheorem*{exercise}{Упражнение}
\newtheorem*{consequence}{Следствие}
\newtheorem*{example}{Пример}
\newtheorem*{observation}{Наблюдение}

\theoremstyle{definition}
\newtheorem*{definition}{Определение}
\newtheorem*{obozn}{Обозначение}

\lhead{Алгоритмы в математике \textit{(практика)}}
\cfoot{}
\rfoot{25.9.2021}

\begin{document}

\section*{Теория делимости}

Будем рассматривать \(\Z\).

\begin{definition}
    \(q \in \Z\) \textbf{делит} \(n \in \Z\), если \(\exists t \in \Z : n = qt\)
\end{definition}
\begin{obozn}
    \(q \mid n, n \divided q\).
\end{obozn}

\begin{example}
    \(m^5 - m \divided 5\)
\end{example}
\begin{solution}
    \begin{caseof}
        \case{\(m = 5k\)}{Тривиально.}
        \case{\(m = 5k + 1\)}{
            \begin{align*}
                (5k + 1)^5 - (5k + 1) & = (5k + 1)((5k + 1)^2 - 1)((5k + 1)^2 + 1)                    \\
                                      & = (5k + 1)(5k + 1 - 1)(5k + 1 + 1)((5k + 1)^2 + 1) \divided 5 \\
            \end{align*}
        }
        \case{\(m = 5k + 2\)}{
            \begin{align*}
                (5k + 2)^5 - (5k + 2) & = (5k + 2)(5k + 2 - 1)(5k + 3)((5k + 2)^2 + 1) \\
                                      & = \dots (25k + 20k + 4 + 1) \divided 5         \\
            \end{align*}
        }
    \end{caseof}
    Остальные случаи опущены.
\end{solution}
\begin{definition}
    \(n, m \in Z, d : n \divided d, m \divided d\)

    \(d\) называется \textbf{общим делителем} \(n, m\).
\end{definition}
\begin{definition}
    \(n, m\) \textbf{взаимно простые}, если:
    \[n \divided d, m \divided d \Rightarrow d = \pm 1\]
\end{definition}
\begin{theorem}
    \(n \divided ab\) \(\Leftrightarrow n \divided a, n \divided b\) и \(a, b\) взаимно простые.
\end{theorem}

\begin{exercise*}
    \[m(m + 1)(2m + 1) \divided 6\]
\end{exercise*}
\begin{solution}
    \[m(m + 1) \divided 2\]
    Докажем \(m(m + 1)(2m + 1) \divided 3\)
    \begin{caseof}
        \case{\(m = 3k\)}{Тривиально.}
        \case{\(m = 3k + 1\)}{
            \(2m + 1 = 6k + 3 \divided 3\)
        }
        \case{\(m = 3k + 2\)}{Тривиально.}
    \end{caseof}
\end{solution}

\begin{exercise*}
    \(\forall n \ \ \exists k : n^2 + (n + 1)^2 = 4k + 1\)
\end{exercise*}
\begin{solution}
    \begin{align*}
        n^2 + (n + 1)^2                    & = 4k + 1 \\
        2n^2 + 2n + 1                      & = 4k + 1 \\
        2n^2 + 2n                          & = 4k     \\
        n^2 + n                            & = 2k     \\
        \underbrace{n(n + 1)}_{\divided 2} & = 2k     \\
    \end{align*}
\end{solution}

\begin{exercise*}
    \[n^3(n^2 + 3) \divided 4\]
\end{exercise*}
\begin{solution}
    Для чётных \(n\) \(n^3 \divided 4\). Для \(n = 2k + 1\) \((2k + 1)^2 + 3 = 4k^2 + 4k + 4 \divided 4\).
\end{solution}

\begin{definition}
    \(a, b \in \Z\) \textbf{сравнимы} по модулю \(n\), если \(a - b \divided n\).
\end{definition}
\begin{obozn}
    \(a \equiv b \pmod n\)
\end{obozn}
\begin{example}
    \(4 \equiv 1 \pmod 3, 8 \equiv 2 \pmod 3, 151 \equiv 11 \pmod{10}\)
\end{example}

\[\begin{rcases}
        a - c \divided n \\
        b - d \divided n
    \end{rcases} \Rightarrow \begin{cases}
        a = nk + c \\
        b = nj + d
    \end{cases}\]
\[\sphericalangle ab = \underbrace{n^2 kj + nkd + njc}_{\divided n} + cd\]

\begin{enumerate}
    \item \(a \equiv c \pmod n, b \equiv d \pmod n \Rightarrow a + b \equiv c + d \pmod n, ab \equiv cd \pmod n\)
\end{enumerate}

\begin{exercise*}
    \(a^7 - a + 56 \divided 7\)
\end{exercise*}
\begin{solution}
    \begin{caseof}
        \case{\(a \equiv 0\)}{\(0 + 0 + 56 \equiv 0\)}
        \case{\(a \equiv 1\)}{\(1 - 1 + 56 \equiv 0\)}
        \case{\(a \equiv 2\)}{\(128 - 2 + 56 \equiv 70 + 56 \equiv 0\)}
    \end{caseof}

    Остальные случаи опущены.
\end{solution}

\begin{exercise*}
    \[m^2 + n^2 \divided 7 \Rightarrow n \divided 7, m \divided 7\]
\end{exercise*}

\begin{solution}\itemfix
    \begin{center}
        \begin{tabular}{CC}
            \toprule
            m \equiv & m^2 \equiv \\ \midrule
            0        & 0          \\
            1        & 1          \\
            2        & 4          \\
            3        & 2          \\
            4        & 2          \\
            5        & 4          \\
            6        & 1          \\
            \bottomrule
        \end{tabular}
    \end{center}
\end{solution}

\begin{definition}
    \(\{a_1 \dots a_n\}\) называется \textbf{полной системой вычетов} \(\mod n\), если \(\forall a \in \Z \ \ \exists j : a \equiv a_j \mod n\)
\end{definition}
\begin{theorem}\itemfix
    \begin{itemize}
        \item \(\{a_1 \dots a_n\}\) --- полная система вычетов \(\mod n\)
        \item \(k\) взаимно просто с \(n\)
    \end{itemize}

    Тогда \(\{ka_1 \dots k a_n\}\) --- полная система вычетов \(\mod n\).
\end{theorem}

\end{document}
