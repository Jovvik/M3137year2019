\chapter{29 марта}

\begin{lemma}\itemize
	\begin{itemize}
		\item $u^2 = $
	\end{itemize}
	\?
\end{lemma}

\begin{proof}
	\[\R \ni uv + vu \in \mathbb{I}\]
	Мы доказывали, что \?

	Мы доказывали, что $z \in \mathbb{I} \Rightarrow z^{-1} \in \mathbb{I}$

	По другой лемме $ab \in \R,\ u, v \in \mathbb{I} \Rightarrow au + bv \in \mathbb{I}$

	Тогда $uv + vu = 0$ и $uv = -vu$.

	$\sphericalangle \omega^2 = uv uv = uv (-vu) = -u^2 = -1$
\end{proof}

Остальная часть лекции рассказана повторно на пятой лекции.
