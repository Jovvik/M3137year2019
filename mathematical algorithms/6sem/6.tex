\chapter{9 марта}

\subsection{Напоминание о кватернионах}

Кватернионы записываются как:
\[q = q_0 + q_1 i + q_2 j + q_3 k \defeq q_0 + \tilde{q}\]

\begin{table}[h]
	\centering
	\begin{tabular}{C|CCCC}
		\cdot & 1 & i  & j  & k  \\ \hline
		1     & 1 & i  & j  & k  \\
		i     & i & -1 & k  & -j \\
		j     & j & -k & -1 & i  \\
		k     & k & j  & -i & -1
	\end{tabular}
	\caption{Таблица Кэли для кватернионов}
\end{table}

Пространство кватернионов есть прямое произведение действительных чисел и чистых кватернионов:
\[\mathbb{K} = \mathbb{R} \oplus \underbrace{\mathbb{P}_E}_{\cong \R^3}\]

\begin{table}[h]
	\centering
	\begin{tabular}{C|CCCC}
		\ev{\cdot, \cdot} & i & j & k \\ \hline
		i                 & 1 & 0 & 0 \\
		k                 & 0 & 1 & 0 \\
		j                 & 0 & 0 & 1
	\end{tabular}
	\caption{Cкалярное произведение кватернионов}
\end{table}

Пусть \(q \in \mathbb{K}, \vec{v} \in \{0\} \oplus \mathbb{P}_E\).
\(\sphericalangle \mathcal{L}_q(v) = q v q^{-1}\).

\(|q| \coloneqq \sqrt{\norm{q}}, \norm{q} \coloneqq q q^*, q^{-1} \coloneqq \frac{q^*}{\norm{q}}\)

Если \(|q| = 1\), то  \(q = \cos \frac{\varphi}{2} + \tilde{u} \sin \frac{\varphi}{2}\),
где \(\tilde{u} = \frac{\tilde{q}}{|\tilde{q}|}\).
В этом случае \(q v q^{-1}\) есть поворот на угол \(\varphi\).

\section{Кватернионы и \(SU(2)\)}

Рассмотрим матрицы \(A \in \mathbb{C}^{2\times 2}\), которым соответствуют операторы \(\hat{A} : \mathbb{C}^2 \to \mathbb{C}^2\).

Тогда \(A A^* = E\), где \(A^*\) --- \textbf{эрмитово сопряжение},
т.е. транспонируем матрицу и каждый элемент комплексно сопрягаем.

Т.к. \(\det (A B) = \det A \cdot \det B\), следовательно
\(1 = \det E = \det (A A^*) = \det A \cdot \det A^* \implies \det A = \frac{1}{\det A^*}\).
Кроме того, \(\det A = \det A\tran\) и \(\prod_i a_i^* = \left( \prod_i a_i \right)^*\) и \(\sum_i b_i^* = \left( \sum_i b_i \right)^*\) и тогда:
\[\sum_j \prod_i a_{ij}^k = \sum_j \left( \prod_i a_{ij} \right)^* = \left( \sum_j \prod_i a_{ij}^* \right)\]
И следовательно \(\det A^* = (\det A)^*\)

\begin{definition}
	\(SU(2)\) --- группа \(A\) таких, что \(A A^* = E\) и \(\det A = 1\)
\end{definition}

\(\sphericalangle A \in SU(2), A = \begin{pmatrix} \alpha & \beta \\ \gamma & \delta \end{pmatrix}\).

\[\det A = 1 \implies \alpha \delta - \beta \gamma = 1\]

Тогда:
\[\begin{cases}
		E_{11} = \alpha \overline{\alpha} + \beta \overline{\beta} = 1   \\
		E_{12} = \alpha \overline{\gamma} + \beta \overline{\delta} = 0  \\
		E_{21} = \gamma \overline{\alpha} + \delta \overline{\beta} = 0  \\
		E_{22} = \gamma \overline{\gamma} + \delta \overline{\delta} = 1 \\
	\end{cases}\]

Из условий для \(E_{11}\) и \(E_{22}\):
\[|\alpha|^2 + |\beta|^2 = 1 \implies \begin{cases}
		\alpha = e^{i \varphi_1} \cos\theta \\
		\beta = e^{i \varphi_2} \sin\theta
	\end{cases}\]
\[|\gamma|^2 + |\delta|^2 = 1 \implies \begin{cases}
		\gamma = e^{i \psi_1} \cos \chi \\
		\delta = e^{i \psi_2} \sin \chi
	\end{cases}\]
Из \(E_{12}\) и \(E_{21}\):
\[e^{i \varphi_1} \cos\theta e^{-i \psi_1} \cos \chi + e^{i \varphi_2} \sin\theta e^{-i\psi_2} \sin\chi = 0\]
\[e^{i (\varphi_1 - \psi_1)} \cos\theta \cos\chi + e^{i (\varphi_2 - \psi_2)} \sin\theta \sin\chi = 0\]
\[2 \cos(\varphi_1 - \psi_1) \cos\theta \cos\chi + 2i \sin(\varphi_2 - \chi_2) \sin\theta \sin\chi = 0\]
\[\begin{cases}
		\cos(\varphi_1 - \psi_1) \cos \theta \cos \chi = 0 \\
		\sin(\varphi_2 - \psi_2) \sin \theta \sin \chi = 0
	\end{cases}\]
\[A^{-1} = \begin{pmatrix} \delta & -\beta \\ -\gamma & \alpha \end{pmatrix}
	= \begin{pmatrix} \overline{\alpha} & \overline{\gamma} \\ \overline{\beta} & \overline{\delta} \end{pmatrix} \implies \delta = \overline{\alpha}, \gamma = -\overline{\beta}\]
Пусть \(\alpha = a + ic, \beta = b + id\), где  \(a, b, c, d \in \R\).
\begin{align*}
	A & = \begin{pmatrix} a + ic & b + id \\ -b + id & a - ic \end{pmatrix}                                           \\
	  & = a \begin{pmatrix} 1 & 0 \\ 0 & 1 \end{pmatrix} + \begin{pmatrix} ic & b + id \\ -b + id & -ic \end{pmatrix} \\
	  & = a  \underbrace{\begin{pmatrix} 1 & 0 \\ 0 & 1 \end{pmatrix}}_{\Xi_0}
	{}+ b \underbrace{\begin{pmatrix} 0 & 1 \\ -1 & 0 \end{pmatrix}}_{\Xi_1}
	{}+ c \underbrace{\begin{pmatrix} i & 0 \\ 0 & i \end{pmatrix}}_{\Xi_2}
	{}+ d \underbrace{\begin{pmatrix} 0 & i \\ i & 0 \end{pmatrix}}_{\Xi_3}
\end{align*}

\begin{table}[h]
	\centering
	\begin{tabular}{C|CCCC}
		\cdot & \Xi_0 & \Xi_1  & \Xi_2  & \Xi_3  \\ \hline
		\Xi_0 & \Xi_0 & \Xi_1  & \Xi_2  & \Xi_3  \\
		\Xi_1 & \Xi_1 & -\Xi_0 & \Xi_3  & -\Xi_2 \\
		\Xi_2 & \Xi_2 & -\Xi_3 & -\Xi_0 & \Xi_1  \\
		\Xi_3 & \Xi_3 & \Xi_2  & -\Xi_1 & -\Xi_0
	\end{tabular}
	\caption{Таблица Кэли для матриц \(\Xi_i\)}
\end{table}

% \(\sphericalangle B = \begin{pmatrix} \xi & \zeta \\ \omega & \chi \end{pmatrix}, C \coloneqq A \times B\)

Таким образом, у нас есть соответствие \(SU(2)\) и \(\mathbb{K}\): \(\Xi_0 \Leftrightarrow 1, \Xi_1 \Leftrightarrow i, \Xi_2 \Leftrightarrow j, \Xi_3 \Leftrightarrow k\).
Но это не изоморфизм --- ограничение на \(\det A = 1\) не позволяет любому кватерниону сопоставить элемент \(SU(2)\).
Найдем, чему \(SU(2)\) изоморфно.

\begin{definition}
	Множество \textbf{нормированных кватернионов} \(\mathbb{N}\mathbb{K} = \{|q| = 1 \mid q \in \mathbb{K}\}\)
\end{definition}
\begin{definition}
	\(\mathbb{N}\mathbb{K}\) --- подгруппа (по умножению) \(\mathbb{K}\)
\end{definition}

\[S^3 \sim \mathbb{N}\mathbb{K} \subset \mathbb{K}\]
Здесь и далее \(\conggr\) обозначает \underline{групповой} изоморфизм.
\[\mathbb{N}\mathbb{K} \conggr SU(2) \quad \{0\} \oplus \mathbb{P}_E \conggr SO(3)\]

Резюмируя: мы взяли подгруппу кватернионов \(\mathbb{N}\mathbb{K}\)
и построили изоморфизм между этой подгруппой и \(SU(2)\).
На прошлом занятии мы построили изоморфизм между группой вращений \(SO(3)\) и \(\{0\} \oplus \mathbb{P}_E\).
Эти изоморфизмы по умножению.
Также есть изоморфизм по сложению между \(\mathbb{P}_E\) и \(\R^3\)

\section{\(SU(2)\) и \(SO(3)\)}

Любое вращение трёхмерного пространства можно рассматривать как переход произвольной точки сферы
в другую произвольную точку сферы.
Спроектируем сферу на плоскость, которую будем считать \(\mathbb{C}\) c осями \(\xi\) и \(\eta\). Оси сферы - \(xyz\).

Упражнение читателю --- показать, что:
\[\xi = \frac{x}{\frac{1}{2} - z} \quad \eta = \frac{y}{\frac{1}{2} - z}\]
Тогда введём комплексное число \(\zeta = \xi + i\eta\). Не более сложно заметить, что:
\[\zeta = \frac{\frac{1}{2} + z}{x - iy}\]
, что следует из \(x^2+y^2+z^2=1\)

Нам надо научиться вращать вокруг оси \(z\) и \(x\), тогда композицией этих двух действий мы сможем получить любое вращение.

Вращение вокруг оси \(z\), т.е. в комплексной плоскости тривиально:
\[\zeta' = e^{i\theta} \zeta\]

Поворот вокруг оси \(x\) на угол \(\chi\) (без доказательства):
\[\zeta'' = \frac{\zeta \cos \frac{\chi}{2} + i \sin \frac{\chi}{2}}{i \xi \sin \frac{\chi}{2} + \cos \frac{\chi}{2}}\]

Тогда композиция этих преобразований имеет вид:
\[A = \frac{a\xi + b}{c\xi + d}\]

Будем кодировать вращения как вектора \(\vec{K}\), где направление определяет ось,
относительно которой происходит вращение, и модуль определяет угол поворота (\(|K| \le \pi\)).

Тогда мы можем рассмотреть многообразие таких векторов. В нём отождествлены противоположные точки.
Есть проблема: оно связно, но оно не односвязно, т.е. у него нетривиальна фундаментальная группа.
По теореме каждое многообразие можно достроить до односвязного, такое многообразие называется \textbf{накрытием}.
Для \(SO(3)\) такое многообразие это \(SU(2)\).

\begin{example}
    Рассмотрим накрытие для окружности.
    Т.к. единственное другое многообразие с размерностью \(1\) это прямая, то будем строить накрытие из прямой.
    С помощью преобразования \(e^{it}\) будем накручивать прямую на окружность.
    Если в какой-либо точке прекратить накручивать, то точка конца помешает.
    Число слоев в накрытии, соттветствующих одной точке, называется \textbf{кратностью накрытия}.
    Кратность накрытия окружности --- \(\infty\).
\end{example}

Кратность накрытия для \(SO(3)\) --- \(2\).
