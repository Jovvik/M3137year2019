\chapter{16 апреля}

\section{Алгебраические топологические тела}

Что общего у поля вещественных чисел, поля комплексных чисел и тела кватернионов?
Разумеется, много чего, но нас интересует тот факт, что они являются евклидовыми пространствами.
Для удобства будем обозначать \(\R, \mathbb{C}, \mathbb{K}\) как \(K\).

\begin{definition}
	Элементы последовательности \(a_1, a_2 \ldots a_n \ldots\), где \(a_i \in K\),
	\textbf{сходятся} к \(a \in K\), если \(\rho(a_n, a) \to 0\).
	Тогда обозначаем \(\lim_{n\to \infty} a_n = a\).
\end{definition}

\begin{definition}
	Если на теле введено понятие сходимости, то такое тело называется \textbf{топологическим}.
\end{definition}

На топологическом теле операции непрерывны, т.е. если \(\lim_{n\to \infty} a_n = a\) и \(\lim_{n \to \infty} b_n = b\), то:
\[\lim_{n \to \infty} a_n + b_n = a + b \quad \lim_{n \to \infty} a_n b_n = a b\]
Из этих двух равенств также следует непрерывность вычитания и умножения,
т.к. \(\lim_{n \to \infty} -a_n = -a\) и \(\lim_{n \to \infty} a_n^{-1} = a^{-1}\).

\begin{definition}
	Топологическое тело с операциями \(+, \cdot, -, {}^{-1}\) называется \textbf{алгебраически-топологическим}
\end{definition}

Норма для \(K\):
\begin{enumerate}
	\item \(\R: \norm{r} = \sqrt{r \cdot r}\)
	\item \(\mathbb{C}: \norm{z} = \sqrt{z\cdot z^*}\)
	\item \(\mathbb{K}: \norm{q} = \sqrt{q \cdot q^*}\)
\end{enumerate}

Тогда метрика на \(K\) это \(\rho(z_1, z_2) = \norm{z_1 - z_2}\).

\begin{definition}
	\textbf{Топология} на \(K\) это \(\tau \subset 2^K\) такое, что:
	\begin{enumerate}
		\item \(\{0\}, K \in \tau\)
		\item \(\bigcup_i T_i \in \tau\)
		\item \(\bigcap_{\text{кон.}} T_i \in \tau\)
	\end{enumerate}
	Элементы \(\tau\) называются \textbf{открытыми}.
\end{definition}

\begin{example}
	\[T_0 \coloneqq \{z \in K \mid \norm{z - z_p} < B\}, B \in \R_+\]
\end{example}

О непрерывности некоторого отображения \(f : A \to B\) можно говорить только если \(A, B \in \mathrm{Top}\)\footnote{Категория топологических пространств}

% \begin{definition}[шар]
%      \[U_n \coloneqq \left\{x \mid x \in K, |x| < \frac{1}{n}\right\}\]  
% \end{definition}

% С помощью шаров можно также определить сходимость: \((a - a_p) \in U_n\)

% \begin{definition}
%     \textbf{Предельный переход} есть на некотором множестве \(K\), если \((a - a_p) \in U_n \ \ \forall n\) для достаточно большого \(p\)
% \end{definition}

\begin{notation}
	Для тела \(L\) и \(X, Y \subset L\):
	\begin{itemize}
		\item \(X+Y \coloneqq \{x + y \mid x \in X, y \in Y\}\)
		\item \(X-Y \coloneqq \{x - y \mid x \in X, y \in Y\}\)
		\item \(XY \coloneqq \{xy \mid x \in X, y \in Y\}\)
		\item \(XY^{-1} \coloneqq \{xy^{-1} \mid x \in X, y \in Y\}\)
	\end{itemize}
\end{notation}

\begin{definition}
	Последовательность \(U_1, U_2 \ldots U_n \ldots\),
	где \(U_n \subset L\) и \(0 \in U_{n + 1} \subset U_n\) называется
	\textbf{системой окрестностей нуля топологического тела \(L\)},
	если \(\forall n \in \mathbb{N} \ \ \exists p:\)
	\begin{enumerate}
		\item \((U_p + U_p) \subset U_n\)
		\item \(U_p U_p \subset U_n\)
		\item \(- U_p \subset U_n\)
		\item \((e + U_p)^{-1} \subset e + U_n\)
		      , где \(e\) --- единица тела \(L\).
		\item \(\forall a \in L \ \ a U_p \subset U_n, U_p a \subset U_n\)
	\end{enumerate}
\end{definition}

\begin{definition}
	Последовательность \(a_1, a_2 \ldots a_n \ldots\), где \(\forall i \ \ a_i \in L\),
	сходится к \(a \in L\), если:
	\[\forall n \ \ \exists r \ \ \forall p > n \ \ (a_p - a) \in U_n\]
\end{definition}

\begin{theorem}
	Если на теле\footnote{Не обязательно топологическом} \(L\) определена сходимость и \(\lim_{n \to \infty} a_n = a, \lim_{n \to \infty} b_n = b\), то:
	\begin{align*}
		\lim_{n \to \infty} a_n + b_n     & = a + b     \\
		\lim_{n \to \infty} a_n \cdot b_n & = a \cdot b \\
		\lim_{n \to \infty} -a_n          & = -a        \\
		\lim_{n \to \infty} a_n^{-1}      & = a^{-1}
	\end{align*}
\end{theorem}
\begin{proof}
	По условию теоремы \(\exists p_1 : a_{p_1} \in a + U_n, \exists p_2 : b_{p_2} \in b + U_n\).
	Пусть \(p = \max(p_1, p_2)\).\footnote{На лекции этого не было, но мне кажется это необходимым.}
	\[a_p + b_p \in a + b + U_n + U_n\]
	По определению системы окрестностей нуля:
	\[\exists p_3 : U_{p_3} + U_{p_3} \subset U_n\]
	\[(a_p + b_p) - (a + b) \in U_n\]

	Аналогично остальные пункты.
\end{proof}
