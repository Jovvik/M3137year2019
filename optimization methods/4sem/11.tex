\chapter{14 апреля}

Прямые методы основаны на разложениях матрицы \(L\), например:
\begin{itemize}
    \item \(LU\), где \(L\) --- нижнетреугольная матрица, а \(U\) --- верхнетреугольная матрица.
    \item \(LL^T\) --- метод квадратного корня
    \item \(LDL^T\), где \(L_{ii} = 1\), \(D\) --- диагональная матрица.
\end{itemize}

Мы рассмотрим первый.

\section{\(LU\)-метод}

\begin{align}
    A X & = b                    \\
    LUx & = b                    \\
    y : = Ux   \label{y через x} \\
    Ly  & = b \label{LUmain}
\end{align}

Таким образом, решение задачи сводится к трём этапам:
\begin{enumerate}
    \item По \(A\) получить \(L, U\).
    \item Решить \eqref{LUmain} прямым ходом метода Гаусса, тем самым найти \(y\).
    \item Решить \eqref{y через x} обратным ходом метода Гаусса, тем самым найти \(x\).
\end{enumerate}

Основные временные затраты происходят на первом этапе метода.

\[L = \begin{bmatrix}
        L_{11} & 0      & 0      & \cdots \\
        L_{21} & L_{22} & 0      & \cdots \\
        L_{31} & L_{32} & L_{33} & \cdots \\
        \vdots & \vdots & \vdots & \ddots
    \end{bmatrix}
    \quad U = \begin{bmatrix}
        1      & U_{12} & U_{13} & \cdots \\
        0      & 1      & U_{23} & \cdots \\
        0      & 0      & 1      & \cdots \\
        \vdots & \vdots & \vdots & \ddots
    \end{bmatrix}  \]
\begin{example}
    Красным отмечены вычисляемые на данной итерации элементы:
    \begin{align*}
        A_{11} & = \textcolor{red}{L_{11}}                                             \\
        A_{21} & = \textcolor{red}{L_{21}}                                             \\
        A_{12} & = L_{11} \cdot \textcolor{red}{U_{12}}                                \\
        A_{22} & = L_{21} \cdot U_{12} + \textcolor{red}{L_{22}}                       \\
        A_{31} & = \textcolor{red}{L_{31}}                                             \\
        A_{13} & = L_{11} \cdot \textcolor{red}{U_{13}}                                \\
        A_{32} & = L_{31} \cdot U_{12} + \textcolor{red}{L_{32}}                       \\
        A_{23} & = L_{21} \cdot U_{13} + L_{22} \cdot \textcolor{red}{U_{23}}          \\
        A_{33} & = L_{31} \cdot U_{13} + L_{32} \cdot U_{23} + \textcolor{red}{L_{33}} \\
    \end{align*}
\end{example}

\subsection{Алгоритм разложения}
\(L_{11} = A_{11}\), для \(i\) от 2 до \(n\):
\begin{equation}
    L_{ij} = A_{ij} - \sum_{k = 1}^{j - 1} L_{ik} \cdot U_{kj} \quad j \in \overline{1, i - 1}
    \label{Lij}
\end{equation}
\begin{equation}
    U_{ji} = \frac{1}{L_{jj}} \left( A_{ji} - \sum_{k = 1}^{j - 1} L_{jk} \cdot U_{ki}\right) \quad j \in \overline{1, i - 1}
    \label{Uij}
\end{equation}
\begin{equation}
    L_{ii} = A_{ii} - \sum_{k = 1}^{i - 1} L_{ik} \cdot U_{ki}
    \label{Lii}
\end{equation}
\begin{equation}
    U_{ii} = 1
    \label{Uii}
\end{equation}

\section{Дополнительные рассуждения о точности получаемого численного решения}

\subsection{Близкие к нулю главные элементы}

\begin{example}
    \[\begin{pmatrix}
            10  & - 7   & 0 \\
            - 3 & 2.099 & 6 \\
            5   & - 1   & 5
        \end{pmatrix} \begin{pmatrix}
            x_1 \\ x_2 \\ x_3
        \end{pmatrix} = \begin{pmatrix}
            7 \\ 3.901 \\ 6
        \end{pmatrix} \]

    Точное решение: \(x = (0, - 1, 1)^T\).

    Предположим, что мы решаем эту задачу на ЭВМ с десятичной пятиразрядной арифметикой с плавающей точкой.

    Решим обычным методом Гаусса без модификаций.
    \[\begin{pmatrix}
            10 & - 7              & 0 \\
            0  & - 1.0 \cdot 10^3 & 6 \\
            0  & 2.5              & 5
        \end{pmatrix} \begin{pmatrix}
            x_1 \\ x_2 \\ x_3
        \end{pmatrix} = \begin{pmatrix}
            7 \\ 6.001 \\ 2.5 \end{pmatrix}\]

    \[
        6.001 \cdot 2.5  \cdot 10^3 = 1.5002 \cdot 10^4 \approx 1.5003 \cdot 10^4
    \]
    \[1.5005 \cdot 10^4 \cdot x_3 = 1.5004 \cdot 10^4\]
    \[x_3 = \frac{1.5004 \cdot 10^4}{1.5005 \cdot 10^4} = 0.99993\]
    \[x_2 = \frac{1.5 \cdot 10^{ - 3}}{ - 1.0 \cdot 10^{ - 3}} = - 1.5\]
    \[x_1 = - 0.35\]

    Итого ошибка очень крупная, \(0.5\) для одного из элементов. Ошибка возникла на шаге исключения, т.к. не использовалась модификация метода.
\end{example}

\subsection{Вектор ошибки и невязка}

\[\begin{pmatrix}
        0.78  & 0.563 \\
        0.457 & 0.330
    \end{pmatrix}
    \begin{pmatrix}
        x_1 \\
        x_2
    \end{pmatrix} =
    \begin{pmatrix}
        0.217 \\
        0.127
    \end{pmatrix} \]
\[x^* \]

Арифметика трёхразрядная.

\textcolor{red}{Вычисления опущены.}

\[x = (1.71, - 1.98)^T\]

\begin{definition}
    \textbf{Невязка} \(r = b - Ax\).
\end{definition}
\[r = ( - 0.00206, - 0.00107)^T\]

\unfinished