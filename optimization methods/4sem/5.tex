\chapter{10 марта}

\section{Минимизация функций многих переменных}

\subsection{Постановка задачи}

Необходимо найти \(x^* = \begin{pmatrix} x_1 & x_2 & \dots & x_n \end{pmatrix}^T \in U \subset E_n\), где \(U\) --- множество допустимых значений, а \(E_n\) --- евклидово пространство размера \(n\), при этом \(f(x^*) = \min_{x\in U} f(x)\).

\begin{remark}\itemfix
    \begin{enumerate}
        \item Как и в одномерном случае, задача минимизации эквивалентна задачи максимизации и в общем случае называется задачей поиска экстремума.
        \item Если \(U\) задается ограничениями на вектор \(x\), то такая задача оптимизации называется задачей поиска условного экстремума.
        \item Если \(U = E_n\), т.е. не имеет ограничений, то такая задача оптимизации называется задачей поиска безусловного экстремума.
        \item Решением задачи поиска экстремума называется пара \((x^*, f(x^*))\).
    \end{enumerate}
\end{remark}

\begin{definition}
    Если \(f(x^*) \leq f(x) \ \ \forall x\in U\), то \(x^*\) называется \textbf{глобальным минимумом}.
\end{definition}
\begin{definition}
    Если \(\exists \varepsilon > 0 : ||x - x^*|| < \varepsilon \Rightarrow f(x^*) \geq f(x)\), то \(x^*\) называется \textbf{локальным минимумом}.
\end{definition}
\begin{remark}
    \[||x|| = \sqrt{\sum_i x_i}\]
\end{remark}

\begin{definition}
    \textbf{Поверхностью уровня} функции \(f(x)\) называется множество точек, в которых функция принимает постоянное значение.
\end{definition}

\begin{definition}
    \textbf{Градиентом} \(\nabla f(x)\) непрерывно дифференцируемой функции \(f(x)\) в \(x\) называется:
    \[\nabla f(x) = \begin{pmatrix} \frac{\partial f(x)}{\partial x_1} \\ \frac{\partial f(x)}{\partial x_2} \\ \vdots \\ \frac{\partial f(x)}{\partial x_n} \end{pmatrix} \]
\end{definition}

\begin{remark}
    Градиент направлен по нормали к поверхности уровня, т.е. перпендикулярно к касательной плоскости, проведенной в точке \(x\) в сторону наибольшего возрастания функции.
\end{remark}

\begin{definition}
    Матрица Гессе \(\mathbf H(x)\) дважды непрерывно дифференцируемой в точке \(x\) функции \(f(x)\) называется матрица частных производных производных второго порядка, вычисленных в данной точке.

    \[\mathbf H(x) = \begin{pmatrix}
            \frac{\partial^2 f(x)}{\partial x_1^2}            & \frac{\partial^2 f(x)}{\partial x_1 \partial x_2} & \dots  & \frac{\partial^2 f(x)}{\partial x_1 \partial x_n} \\
            \vdots                                            & \vdots                                            & \ddots & \vdots                                            \\
            \frac{\partial^2 f(x)}{\partial x_n \partial x_1} & \frac{\partial^2 f(x)}{\partial x_n\partial x_2}  & \dots  & \frac{\partial^2 f(x)}{\partial x_n^2}
        \end{pmatrix} = \begin{pmatrix}
            h_{11} & h_{12} & \dots  & h_{1n} \\
            \vdots & \vdots & \ddots & \vdots \\
            h_{n1} & h_{n2} & \dots  & h_{nn}
        \end{pmatrix}  \]
\end{definition}

\begin{enumerate}
    \item \(\mathbf H(x)\) симметрична, имеет размер \(n \times n\).
    \item Можно определить антиградиент --- вектор, равный по модулю градиенту и направленный противоположно. Антиградиент указывает в сторону наибольшего убывания \(f(x)\).
    \item \(\Delta f(x) = f(x + \Delta x) - f(x) = \nabla f(x)^T \Delta x + \frac{1}{2} \Delta x^T \mathbf H(x) \Delta x + \smallO(||\Delta x||^2)\), где \(\smallO(||\Delta x||^2)\) есть сумма всех членов разложения, имеющих порядок выше второго. Можем заметить, что \(\Delta x^T \mathbf H(x) \Delta x\) --- квадратичная форма.
\end{enumerate}

\begin{definition}
    \textbf{Квадратичная форма} \(\Delta x^T \mathbf H(x) \Delta x\)\footnote{и соответствующая ей матрица \(\mathbf H(x)\)} называется:
    \begin{itemize}
        \item Положительно определенной, если \(\forall \Delta x \neq 0 \ \ \Delta x^T \mathbf H(x) \Delta x > 0\)
        \item Отрицательно определенной, если \(\forall \Delta x \neq 0 \ \ \Delta x^T \mathbf H(x) \Delta x < 0\)
        \item Положительно полуопределенной, если \(\forall \Delta x \ \ \Delta x^T \mathbf H(x) \Delta x \geq 0\) и имеется \(\Delta x \neq 0 : \Delta x^T \mathbf H(x) \Delta x = 0\)
        \item Отрицательно полуопределенной, если \(\forall \Delta x \ \ \Delta x^T \mathbf H(x) \Delta x \leq 0\) и имеется \(\Delta x \neq 0 : \Delta x^T \mathbf H(x) \Delta x = 0\)
        \item Неопределенной, если \(\exists \Delta x, \Delta \tilde x : \Delta x^T \mathbf H(x) \Delta x > 0, \Delta \tilde x^T \mathbf H(x) \Delta \tilde x < 0\)
        \item Тождественно равной нулю, если \(\forall \Delta x \ \ \Delta x^T \mathbf H(x) \Delta x = 0\)
    \end{itemize}
\end{definition}

\subsection{Свойства выпуклых множеств и выпуклых функций}

\begin{definition}
    Пусть \(x, y \in E_n\), множество точек вида \(\{z\} \subset E_n : z = \alpha x + (1 - \alpha) y\), т.е. \(z\) это отрезок \([x, y]\).
\end{definition}

\begin{definition}
    \(U \subset E_n\) выпуклое, если вместе с точками \(x, y \in U\) оно содержит весь отрезок \(z\).
\end{definition}

\begin{definition}
    Функция \(f(x)\), заданная на выпуклом множестве \(U \subset E_n\), называется:
    \begin{itemize}
        \item \textbf{выпуклой}, если:
              \[\forall x, y \in U, \alpha \in [0, 1] \ \ f(\alpha x + (1 - \alpha) y) \leq \alpha f(x) + (1 - \alpha) f(y)\]
        \item \textbf{строго выпуклой}, если:
              \[\forall x, y \in U, \alpha \in (0, 1) \ \ f(\alpha x + (1 - \alpha) y) < \alpha f(x) + (1 - \alpha) f(y)\]
        \item \textbf{сильно выпуклой} с константой \(l > 0\), если:
              \[\forall x, y \in U, \alpha \in [0, 1] \ \ f(\alpha x + (1 - \alpha) y) \leq \alpha f(x) + (1 - \alpha) f(y) - \frac{l}{2} \alpha (1 - \alpha) ||x - y||^2\]
    \end{itemize}
\end{definition}

\begin{prop}\itemfix
    \begin{enumerate}
        \item Функция \(f(x)\) выпуклая, если её график целиком лежит не выше отрезка, соединяющего две её произвольные точки.
        \item Функция \(f(x)\) строго выпуклая, если её график целиком лежит ниже отрезка, соединяющего две её произвольные, но не совпадающие точки.\footnote{Пример будет на следующей лекции}.
        \item Если функция сильно выпуклая, то она одновременно строго выпуклая и выпуклая.
        \item Если функция строго выпуклая, то она выпуклая.
        \item Выпуклость функции можно определить по \(\mathbf H(x)\):
              \begin{itemize}
                  \item Если \(\mathbf H(x) \geq 0 \ \ \forall x \in E_n\), то \(f(x)\) выпуклая.
                  \item Если \(\mathbf H(x) > 0 \ \ \forall x \in E_n\), то \(f(x)\) строго выпуклая.
                  \item Если \(\mathbf H(x) \geq lE\footnote{единичная матрица} \ \ \forall x \in E_n\), то \(f(x)\) сильно выпуклая.
              \end{itemize}
    \end{enumerate}
\end{prop}

\begin{prop}[выпуклых функций]\itemfix
    \begin{enumerate}
        \item Если \(f(x)\) --- выпуклая функция на множестве \(U\), то всякая точка локального минимума --- глобальный минимум на \(U\).
        \item Если выпуклая функция достигает своего минимума в двух различных точках, то она достигает минимума во всех точках отрезка, соединяющего эти точки.
        \item Если \(f(x)\) строго выпуклая функция на множестве \(U\), то она может достигать своего глобального минимума на \(U\) не более чем в одной точке.
    \end{enumerate}
\end{prop}

\subsection{Необходимое и достаточное условие безусловного экстремума}

\subsubsection{Необходимое условие экстремума первого порядка}

Пусть \(x^* \in E_n\) --- точка локального минимума\footnote{или максимума} \(f(x)\) на \(E_n\) и \(f(x)\) дифференцируема в точке \(x^*\). Тогда \(\nabla f(x)\) в точке \(x^*\) равен нулю: \(\nabla f(x^*) = 0\) или \(\frac{\partial f(x^*)}{\partial x_i} = 0 \ \ \forall i \in 1 \dots n \). Точка \(x^*\) называется \textbf{стационарной}.

\subsubsection{Необходимое условие экстремума второго порядка}

Пусть \(x^* \in E_n\) --- точка локального минимума\footnote{или максимума} \(f(x)\) на \(E_n\) и \(f(x)\) дважды дифференцируема в точке \(x^*\). Тогда \(\mathbf H(x^*)\) положительно полуопределена или отрицательно полуопределена.

\subsubsection{Достаточное условие экстремума}

Пусть \(f(x)\) в \(x^* \in E_n\) дважды дифференцируема, \(\nabla f(x^*) = 0\) и \(\mathbf H(x) > 0\) (или \(\mathbf H(x) < 0\)). Тогда \(x^*\) --- точка локального минимума\footnote{или максимума} \(f(x)\) на \(E_n\).

\subsubsection{Проверка выполнений условий экстремума}

\begin{itemize}
    \item Вычисление угловых миноров \(\mathbf H(x)\)
    \item Вычисление главных миноров \(\mathbf H(x)\)
\end{itemize}

Есть два способа это сделать:
\begin{enumerate}
    \item Исследование положительной или отрицательной определенности угловых и главных миноров \(\mathbf H(x)\).
    \item Анализ собственных значений \(\mathbf H(x)\).
\end{enumerate}
