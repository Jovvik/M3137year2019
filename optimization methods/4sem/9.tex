\chapter{31 марта}

\section{Форматы хранения матриц}

\begin{definition}
    Матрица с большим количеством нулевых элементов называется \textbf{разреженной}. Такие матрицы хранятся особым образом.
\end{definition}
\begin{definition}
    Не разреженная матрица называется \textbf{плотной}.
\end{definition}

Формат записи матриц зависит от алгоритма, который будет использовать данную матрицу. Наиболее распространены следующие 4 разреженных формата:
\begin{enumerate}
    \item Диагональный
    \item Ленточный
    \item Профильный
    \item Разреженный
\end{enumerate}

Мы будем рассматривать эти форматы только в применении к квадратным матрицам. Первые два формата используются реже и мы не будем их рассматривать в лабораторных работах.

Характеристики форматов:
\begin{itemize}
    \item Учитывается ли симметрия матрицы.
    \item Используются ли отдельно верхний и нижний треугольники матрицы.
    \item Требуется ли ускоренный доступ к строкам или столбцам матрицы.
\end{itemize}

\subsection{Диагональный формат}

Этот формат используется, когда все ненулевые элементы матрицы находятся на относительно небольшом числе диагоналей.

\begin{figure}[h]
    \centering
    \includesvg[]{\detokenize{images/диагональный_формат.svg}}
    \caption{Матрица с ненулевыми элементами на диагоналях, \(m = 5\)}
\end{figure}

Матрица хранится в виде плотной матрицы \(n \times m\), где \(n\) --- размерность исходной матрицы, а \(m\) --- количество ненулевых диагоналей. Также необходимо хранить одномерный массив размерностью \(m - 1\), где для каждой диагонали указан сдвиг относительно главной диагонали. В примере значения этого массива \(( - i_2, - i_1, i_3, i_4, i_5)\).

\subsection{Ленточный формат}

Этот формат используется, когда все ненулевые элементы матрицы расположены на диагоналях, прилегающих к главной диагонали, т.е. \(a_{ij} = 0\), если \(|i - j| > k\). При этом \(k\) называется \textbf{полушириной}, а ширина ленты \(m = 2k + 1\).

\begin{figure}[h]
    \centering
    \includesvg[]{\detokenize{images/ленточный_формат.svg}}
    \caption{Матрица ленточного типа}
\end{figure}

Хранить такие матрицы в виде \(m\) массивов различных длин не представляется возможным, т.к. требуется быстрый доступ к элементам матрицы.

\begin{figure}[h]
    \begin{subfigure}[t]{.5\textwidth}
        \centering
        \includesvg[]{\detokenize{images/строчное.svg}}
        \caption{Строчное хранение}
    \end{subfigure}
    \hfill
    \begin{subfigure}[t]{.5\textwidth}
        \centering
        \includesvg[]{\detokenize{images/столбцовое.svg}}
        \caption{Столбцовое хранение}
    \end{subfigure}
    \hfill

    \medskip

    \begin{subfigure}[t]{.5\textwidth}
        \centering
        \includesvg[]{\detokenize{images/строчно-столбцовое.svg}}
        \caption{Строчно-столбцовое хранение}
    \end{subfigure}
    \hfill
    \begin{subfigure}[t]{.5\textwidth}
        \centering
        \includesvg[]{\detokenize{images/столбцово-строчное.svg}}
        \caption{Столбцово-строчное хранение}
    \end{subfigure}
    \hfill
    \caption{Способы хранения матрицы в ленточном формате}
\end{figure}

Иногда главную диагональ хранят отдельно, в зависимости от алгоритма.

\pagebreak

\subsection{Профильный формат}

\begin{wrapfigure}{R}{0.5\textwidth}
    \centering
    \includesvg[scale=0.85]{\detokenize{images/профиль.svg}}
    \caption{Матрица профильного типа. Зеленым --- профиль строки \(i\)}
\end{wrapfigure}
Профильные форматы хранения матриц используется, когда матрица не
обладает определенной структурой и ненулевые элементы расположены в произвольном порядке, но при этом они сосредоточены у главной диагонали, так что в строке можно выделить \textbf{профиль} --- часть строки от первого ненулевого элемента в строке до диагонального элемента.

Матрицы ленточного формата --- матрицы профильного формата с фиксированным профилем.

\pagebreak

Используемые структуры:
\begin{itemize}
    \item Вещественный массив \texttt{di[n]}
    \item Вещественные массивы:
          \begin{itemize}
              \item \texttt{al} --- элементы нижнего треугольника по строкам
              \item \texttt{au} --- элементы верхнего треугольника по столбцам.
          \end{itemize}
    \item Целочисленный массив \texttt{ia} --- информация о профиле: \texttt{ia[k]} = индекс \textcolor{red}{\textit{(в нумерации с 1)}}, с которого начинаются элементы \(k\)-той строки или столбца в массивах \texttt{al} или \texttt{au}.
\end{itemize}

\texttt{ia[n+1]} = индекс первого незанятого элемента в массивах \texttt{al} и \texttt{au}.

\texttt{ia[i+1] - ia[i]} --- значение профиля \(i\)-той строки \textit{(столбца)} нижнего \textit{(верхнего)} треугольника.

\texttt{ia[1] = ia[2] = 1}.

\begin{remark}
    Если матрица симметрична по значениям, то \texttt{al = au}.
\end{remark}

\begin{example}
    \[\begin{bmatrix}
            a_{11} &                                                                       \\
                   & a_{22} & a_{23} & a_{24}                                              \\
                   & a_{32} & a_{33} & 0      & a_{35} & a_{36}                            \\
                   & a_{42} & 0      & a_{44} & a_{45} & 0      & a_{47}                   \\
                   &        & a_{53} & a_{54} & a_{55} & a_{56} & 0      & a_{58} & a_{59} \\
                   &        & a_{63} & 0      & a_{65} & a_{66} & 0      & a_{68} & 0      \\
                   &        &        & a_{74} & 0      & 0      & a_{77} & 0      & a_{79} \\
                   &        &        &        & a_{85} & a_{86} & 0      & a_{88} & 0      \\
                   &        &        &        & a_{95} & 0      & a_{97} & 0      & a_{99}
        \end{bmatrix} \]
    \[di = \{a_{11}, a_{22}, a_{33}, a_{44}, a_{55}, a_{66}, a_{77}, a_{88}, a_{99}\}\]
    \[ia = \{1, 1, 1, 2, 4, 6, 9, 12, 15, 19\}\]
    \[al = \{a_{32}, a_{42}, 0, a_{53}, a_{54}, a_{63}, 0, a_{65}, a_{74}, 0, 0, a_{85}, a_{86}, 0, a_{95},  0, a_{97}, 0\}\]
    \[au = \{a_{23}, a_{24}, 0, a_{35}, a_{45}, a_{36}, 0, a_{56}, a_{47}, 0, 0, a_{58}, a_{68}, 0, a_{59}, 0, a_{79}, 0\} \]

    Первый элемент для 6-ой строки: \(al[ia[6]] = al[6] = a_{63}\)

    Профиль 6-ой строки: \(ia[7] - ia[6] = 9 - 6 = 3\).
\end{example}

\subsection{Разреженный формат}

Этот формат бывает:
\begin{itemize}
    \item строчным
    \item столбцовым
    \item смешанным: строчно-столбцовым или столбцово-строчным.
\end{itemize}
