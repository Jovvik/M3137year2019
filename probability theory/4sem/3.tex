\chapter{27 февраля}

\section{Условная вероятность}

\begin{obozn}
    \(P(A|B)\) --- вероятность наступления события \(A\), вычисленная в предположении, что событие \(B\) уже произошло.
\end{obozn}

\begin{example}
    Кубик подбрасывается один раз. Известно, что выпало больше \(3\) очков. Какова вероятность, что выпало чётное число очков?

    Пусть \(A\) --- чётное число очков, \(B\) --- больше \(3\) очков.

    \[n = 3 \quad (4, 5, 6) \quad m = 2 \quad (4, 6)\]
    \[P(A | B) = \frac{m}{n} = \frac{2}{3} = \frac{\frac{2}{6}}{\frac{3}{6}} = \frac{P(A \cdot B)}{P(B)}  \]
\end{example}

\begin{definition}
    \textbf{Условной вероятностью} события \(A\) при условии, что имело место событие \(B\), называется величина \(P(A|B) = \frac{P(AB)}{P(B)}\)
\end{definition}

\begin{theorem}
    \(P(A_1 \dots A_n) = P(A_1)\cdot P(A_2 | A_1)\cdot P(A_3| A_2A_1) \dots P(A_n | A_{n - 1}\dots A_1)\)
\end{theorem}
\begin{proof}
    По индукии.
    \begin{itemize}
        \item [\textbf{База}] \(n = 2\) --- по определению полной вероятности.
        \item [\textbf{Переход}] \[P(A_1 \dots A_n) = P(A_1\dots A_{n - 1}) P(A_n | A_1 \dots A_{n - 1}) = P(A_1)\cdot P(A_2 | A_1) \dots P(A_n | A_{n - 1}\dots A_1)\]
    \end{itemize}
\end{proof}

\begin{definition}
    События \(A\) и \(B\) независимы, если \(P(A|B) = P(A)\), что равносильно \(P(AB) = P(A)P(B)\) --- прошлому определению.
\end{definition}
\begin{proof}
    \[P(A|B) = \frac{P(AB)}{P(B)} = P(A) \Leftrightarrow P(AB) = P(A)P(B)\]
\end{proof}

\subsection{Полная группа событий}

\begin{definition}
    События \(H_1\dots H_n\dots \) образуют \textbf{полную группу событий}, если они попарно несовместны и содержат все элементарные исходы.
\end{definition}

\begin{theorem}
    Пусть \(H_1 \dots H_n\) --- полная группа событий. Тогда \(P(A) = \sum_{k = 1}^{+\infty} P(H_i)P(A | H_i)\)
\end{theorem}
\begin{proof}
    \begin{align*}
        P(A) & = P(\Omega A)                                                              \\
             & = P((H_1 + \dots H_n + \dots ) A)                                          \\
             & = P\left(\sum_{k = 1}^{+\infty} H_k A\right)                               \\
             & \symrefeq{по счетной аддитивности} \sum_{k = 1}^{+\infty} P(H_i)P(A | H_i) \\
    \end{align*}

    \ref{по счетной аддитивности}: По лемме о счётной аддитивности и т.к. \(H_iA\) и \(H_jA\) несовместны.
\end{proof}

\subsection{Формула Байеса}

Эта формула также называется формулой проверки гипотезы.

\begin{theorem}
    Пусть \(H_1 \dots H_n\) --- полная группа событий и известно, что \(A\) произошло. Тогда \[P(H_i | A) = \frac{P(H_i)P(A | H_i)}{\sum_{k = 1}^{+\infty} P(H_k)P(A|H_k)} \]
\end{theorem}
\begin{proof}
    \[P(H_i | A) = \frac{P(H_i A)}{P(A)} = \frac{P(H_i)P(A | H_i)}{\sum_{k = 1}^{n} P(H_k)P(A|H_k)}\]
\end{proof}

