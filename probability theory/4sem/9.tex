\chapter{10 апреля}

\subsection{Сингулярное распределение}

% Помимо уже рассмотренных нам распределений, возникает сингулярное распределение, когда случайная величина распределена на множестве меры \(0\) и вероятность попадания в конкретную точку этого множества \(0\).

\begin{definition}
    \(\xi\) имеет \textbf{сингулярное распределение}, если существует борелевское множество \(B \in \mathfrak{B}\) с нулевой мерой Лебега, такое что \(P(\xi \in B) = 1\), но \(\forall x \in B \ \ P(\xi = x) = 0\).
\end{definition}

\begin{remark}
    Такое борелевское множество состоит из несчётного числа точек. В противном случае по счётной
\end{remark}

\begin{remark}
    Функция распределения \(F(x)\) --- непрерывная функция по свойству 7 функции распределения\footnote{В каждой точке функция распределения имеет скачок, равный вероятности попадания в эту точку.}.
\end{remark}

\begin{example}
    \(\xi\) задана функцией распределения ``лестница Кантора'':
    \[F_\xi(x) = \begin{cases}
            0,                                   & x = 0                            \\
            \frac{1}{2} F(3x),                   & 0 < x \leq \frac{1}{3}           \\
            \frac{1}{2},                         & \frac{1}{3} < x \leq \frac{2}{3} \\
            \frac{1}{2} + \frac{1}{2} F(3x - 2), & \frac{2}{3} < x \leq 1           \\
            1,                                   & x > 1
        \end{cases} \]
\end{example}

\begin{theorem}[Лебега]
    Пусть \(F_\xi(x)\) --- функция распределения произвольной случайной величины \(\xi\). Тогда \(\xi(x) = p_1 F_1(x) + p_2 F_2(x) + p_3 F_3(x)\), где \(p_1 + p_2 + p_3 = 1\) и \(F_1(x)\) --- функция дискретного распределения, \(F_2(x)\) --- функция абсолютно непрерывного распределения и \(F_3(x)\) --- функция сингулярного распределения. Другими словами, все распределения делятся на дискретное, абсолютно непрерывное, сингулярное и их смеси.
\end{theorem}

\subsection{Общий взгляд на математическое ожидание}

\begin{definition}
    \textbf{Математическим ожиданием} случайной величины \(\xi\) называется:
    \begin{equation}
        \mathbb{E}\xi = \int_\Omega \xi(w) dP(w)
        \label{матожидание лебега}
    \end{equation}
    При условии, что данный интеграл существует.
\end{definition}

\begin{remark}
    Используя интеграл Стилтьеса эту формулу можно записать в виде:
    \begin{equation}
        \mathbb{E}\xi = \int_{ -\infty}^{+\infty} x dF_\xi(x)
        \label{матожидание стильтеса}
    \end{equation}

    Из определения интеграла Стилтьеса можно получить геометрическую интерпретацию математического ожидания: разность площади подграфика на \(x < 0\) и площади надграфика, ограниченной \(y = 1\), для \(x > 0\):
    \[\mathbb{E}\xi = \int_0^{+\infty} (1 - F(x)) dx - \int_{-\infty}^0 F(x)dx\]
\end{remark}

Рассмотрим две ситуации:
\begin{enumerate}
    \item Вероятностное пространство \((\Omega, \mathfrak{F}, P)\) --- дискретное, т.е. \(\Omega\) состоит из не более, чем счётного числа точек. Тогда из формулы \eqref{матожидание лебега} получаем:
          \[\mathbb{E}\xi = \sum_{i = 1}^{+\infty} \xi(w_i) P(w_i)\]

          \begin{example}[``орлянка'']
              \(\Omega = \{\text{Г}, \text{Р}\}, P(\text{Г}) = \frac{1}{2} = P(\text{Р}), \xi(\text{Г}) = 1, \xi(\text{Р}) =- 1\)
              \[\mathbb{E}\xi = \xi(\text{Г}) \cdot P(\text{Г}) + \xi(\text{Р}) \cdot P(\text{Р})= \frac{1}{2} - \frac{1}{2} = 0\]
          \end{example}

    \item Вероятностное пространство абсолютно непрерывное. \(\Omega \subset \R^n, w = (x_1 \dots x_n)\). Тогда из формулы \eqref{матожидание стильтеса} получаем:
          \[\mathbb{E}\xi = \idotsint\limits_\Omega \xi(x_1 \dots x_n) \rho(x_1 \dots p_n) dx_1 \dots dx_n\]

          \begin{example}
              Круг радиуса 3, наугад бросается точка. \(\xi =\) расстояние от центра круга до выбранной точки.

              \[\Omega = \{(x, y)\ |\ x^2 + y^2 \leq 9\} \quad \xi = \sqrt{x^2 + y^2}\]
              Т.к. бросаем наугад, то плотность \(\rho(x, y) = \const\). Из условия нормировки:
              \[\int_\Omega dP(w) = 1 \quad \iint_\Omega \rho dx dy = 1 \Rightarrow \rho = \frac{1}{S_\Omega} = \frac{1}{9\pi}\]
              \[\mathbb{E}\xi = \iint_\Omega \xi(x, y) \rho dx dy = \iint_\Omega \sqrt{x^2 + y^2} \frac{1}{9\pi}dx dy = \frac{1}{9\pi} \int_0^{2\pi} d\varphi \int_0^3 \rho \rho d\rho = 2\]
          \end{example}
\end{enumerate}

\subsection{Преобразование случайных величин}

\begin{definition}
    \(g : \R \to \R\) --- \textbf{борелевская функция}, если \(\forall B \in \mathfrak{B} \ \ g^{ - 1}(B) \in \mathfrak{B}\).
\end{definition}

\begin{theorem}
    Если \(g(x)\) борелевская функция и \(\xi\) --- случайная величина на \((\Omega, \mathfrak{F}, P)\), то \(g(\xi)\) --- тоже случайная величина на этом же пространстве.
\end{theorem}
\begin{proof}
    Очевидно из определения борелевской функции.
\end{proof}

\begin{remark}
    Если \(\xi\) --- дискретная случайная величина, то её закон распределения находится просто из определения. Поэтому в дальнейшем будем считать, что \(\xi\) имеет абсолютно непрерывное распределение.
\end{remark}

\subsubsection{Стандартизация случайных величин}

\begin{definition}
    \textbf{Стандартной} случайной величиной, соответствующей случайной величине \(\xi\), называется величина \(\frac{\xi - \mathbb{E}\xi}{\sigma_\xi} \)
\end{definition}
\begin{prop}\itemfix
    \begin{itemize}
        \item \(\mathbb{E}\xi = 0\)
        \item \(\mathbb{D}\xi = 1\)
    \end{itemize}
\end{prop}
\begin{proof}
    Было доказано раннее.
\end{proof}

\begin{remark}
    При стандартизации тип распределения не всегда сохраняется.
\end{remark}

\subsubsection{Линейное преобразование}

\begin{theorem}
    Пусть случайная величина \(\xi\) имеет плотность \(f_\xi(x)\). Тогда случайная величина \(\eta = a\xi + b\) при \(a \neq 0\) имеет плотность \(f_\eta = \frac{1}{|a|} f_\xi\left( \frac{x - b}{a} \right)\)
\end{theorem}
\begin{proof}\itemfix
    \begin{enumerate}
        \item \(a > 0\). Тогда
              \begin{align*}
                  F_\xi(x) & = P(a\xi + b < x)                                                                                                                               \\
                           & = P\left( \xi < \frac{x - b}{a} \right)                                                                                                         \\
                           & = \int_{ -\infty}^{\frac{x - b}{a}} f_\xi(t) dt                                                                                                 \\
                           & = \begin{bmatrix}
                      t = \frac{y - b}{a}   & dt = \frac{1}{a}dy                  & y = at + b \\
                      y( -\infty) = -\infty & y\left( \frac{x - b}{a} \right) = x
                  \end{bmatrix}                                                                                                                    \\
                           & = \int_{ -\infty}^{x} \frac{1}{a} f_\xi\left( \frac{y - b}{a} \right) dy \Rightarrow f_\eta = \frac{1}{|a|} f_\xi\left( \frac{y - b}{a} \right)
              \end{align*}

        \item \(a < 0\). Тогда
              \begin{align*}
                  F_\xi(x) & = P(a\xi + b < x)                                                        \\
                           & = P\left( \xi < \frac{x - b}{a} \right)                                  \\
                           & = P\left( \xi > \frac{x - b}{a} \right)                                  \\
                           & = \int_{\frac{x - b}{a}}^{+\infty} f_\xi(t) dt                           \\
                           & = \begin{bmatrix}
                      t = \frac{y - b}{a}   & dt = \frac{1}{a}dy                  & y = at + b \\
                      y( +\infty) = -\infty & y\left( \frac{x - b}{a} \right) = x
                  \end{bmatrix}                                             \\
                           & = \int_{x}^{-\infty} \frac{1}{a} f_\xi\left( \frac{y - b}{a} \right) dy  \\
                           & = \int_{-\infty}^{x} \frac{1}{-a} f_\xi\left( \frac{y - b}{a} \right) dy
                  \Rightarrow f_\eta = \frac{1}{|a|} f_\xi\left( \frac{y - b}{a} \right)
              \end{align*}
    \end{enumerate}
\end{proof}

\begin{enumerate}
    \item Если \(\xi \in N(0, 1)\), то \(\eta = \sigma \xi + a \in N(a, \sigma^2)\).
          \begin{prop}
              \[f_\xi = \frac{1}{\sqrt{2\pi}} e^{ - \frac{x^2}{2}}\]
              \[f_\eta = \frac{1}{\sigma \sqrt{2\pi} e^{ - \frac{x - a}{2\sigma}}}\]
          \end{prop}
    \item Если \(\eta \in N(a, \sigma^2)\), то \(\xi = \frac{\eta - a}{\sigma} \in N(0, 1)\)
    \item Если \(\eta \in N(a, \sigma^2)\), то \(\xi = \gamma \eta + b \in N(a\gamma + b, \gamma^2 \sigma^2)\)
    \item Если \(\xi \in N(0, 1)\), то \(\eta = a \xi + b \in N(b, a + b)\) при \(a > 0\)
    \item Если \(\xi \in E_\alpha\), то \(\eta = \alpha \xi \in E_1\)
\end{enumerate}

\begin{theorem}
    \(f_\xi\) --- плотность случайной величины \(\xi\) и функция \(g(x)\) монотонна. Тогда \(\exists \) обратная функция \(h(x) = g^{ - 1}(x)\) и случайная величина \(\eta = g(\xi)\):
    \[f_\eta(x) = \frac{1}{|h'(x)|} \cdot f_\xi(h(x))\]
\end{theorem}

\begin{theorem}[квантильное преобразование]
    Пусть функция распределения \(F(x)\) случайной величины \(\xi\) непрерывна. Тогда случайная величина \(\eta = F(\xi)\) имеет стандартное равномерное распределение, т.е. \(U(0, 1)\)
\end{theorem}
\begin{proof}
    Ясно, что \(0 \leq \eta \leq 1\), т.к. \(0 \leq F(x) \leq 1\).
    \begin{enumerate}
        \item Предположим, что \(F(x)\) --- строго возрастающая функция. Тогда она имеет обратную функцию \(F^{ - 1}(x)\) и:
              \[F_\eta(x) = P(F(\xi) < x) = P(\xi < F^{ - 1}(x)) \begin{cases}
                      0,                 & x \leq 0  \\
                      F(F^{ - 1}(x)) = x & 0 < x < 1 \\
                      1,                 & x \geq 1
                  \end{cases} \Rightarrow \eta \in U(0, 1)\]
        \item Пусть функция не является строго возрастающей, т.е. есть интервалы постоянства. В этом случае через \(F^{ - 1}(x)\) обозначим самую левую точку такого интервала: \(F^{ - 1}(x) = \min_t \{t\ |\ F(t) = x\}\)\footnote{Можно писать \(\min\), а не \(\inf\), т.к. \(F(x)\) непрерывна слева}.
              \[F_\eta(x) = P(F(\xi) < x) = P(\xi < F^{ - 1}(x)) = F(F^{ - 1}(x)) = x, \quad 0 \leq x \leq 1\]
    \end{enumerate}
\end{proof}

\begin{theorem}[обратная]
    \label{обратная}
    Пусть функция распределения \(F(x)\) случайной величины \(\xi\), причём не обязательно непрерывная. Обозначим через \(F^{- 1}(x) = \inf \{t\ |\ F(t) \geq x\} \). Пусть случайная величина \(\eta \in U(0, 1)\), \(F(x)\) --- произвольная функция распределения. Тогда случайная величина \(\xi = F^{ - 1}(\beta)\) имеет функцию распределения \(F(x)\)
\end{theorem}
\begin{remark}
    \(F^{ - 1}(\eta)\) называется \textbf{квантильным преобразованием} над случайной величиной \(\eta\).
\end{remark}
\begin{remark}
    Датчики случайных чисел имеют обычно стандартное равномерное распределение.
\end{remark}

Из теоремы \ref{обратная} следует, что из датчика случайных чисел и квантильного преобразования можно смоделировать любое желаемое распределение, в том числе и дискретное.

\begin{example}\itemfix
    \begin{enumerate}
        \item Смоделировать \(E_\alpha\).
              \[F(x) = \begin{cases}
                      0                     & x < 0    \\
                      1 - e^{ - \alpha x} , & x \geq 0
                  \end{cases}\]

            \[\eta = 1 - e^{ - \alpha x} \quad e^{ - \alpha x} = 1 - \eta \quad x =- \frac{1}{\alpha} \ln(1 - \eta)\]
            Если \(\eta \in U(0, 1)\), то \(\xi =- \frac{1}{\alpha} \ln(1 - \eta) \in E_\alpha\)
        \item Смоделировать \(N(0, 1)\):
                  \[\Phi_0(x) = \frac{1}{\sqrt{2\pi}} \int_{-\infty}^{x} e^{ - \frac{z^2}{2}} dz\]

                  \[\Phi_0(\eta) \in N(0, 1)\]
    \end{enumerate}
\end{example}