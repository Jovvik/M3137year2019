\documentclass[12pt, a4paper]{article}

\usepackage{lastpage}
\usepackage{mathtools}
\usepackage{xltxtra}
\usepackage{libertine}
\usepackage{amsmath}
\usepackage{amsthm}
\usepackage{amsfonts}
\usepackage{amssymb}
\usepackage{enumitem}
\usepackage{xcolor}
\usepackage[left=2.3cm, right=2.3cm, top=2.7cm, bottom=2.7cm, bindingoffset=0cm, headheight=15pt]{geometry}
\usepackage{fancyhdr}
\usepackage[russian]{babel}
% \usepackage{parindent}

\pagestyle{fancy}
\lfoot{M3137y2019}
\rhead{\thepage\ из \pageref{LastPage}}

\newcommand{\R}{\mathbb{R}}
\newcommand{\Q}{\mathbb{Q}}
\newcommand{\C}{\mathbb{C}}
\newcommand{\Z}{\mathbb{Z}}
\newcommand{\B}{\mathbb{B}}
\newcommand{\N}{\mathbb{N}}

\DeclareMathOperator*{\xor}{\oplus}
\DeclareMathOperator*{\equ}{\sim}
\DeclareMathOperator{\Ln}{\text{Ln}}
\DeclareMathOperator{\sign}{\text{sign}}
\DeclareMathOperator{\Sym}{\text{Sym}}
\DeclareMathOperator{\Asym}{\text{Asym}}
% \DeclareMathOperator{\sh}{\text{sh}}
% \DeclareMathOperator{\tg}{\text{tg}}
% \DeclareMathOperator{\arctg}{\text{arctg}}
% \DeclareMathOperator{\ch}{\text{ch}}

\DeclarePairedDelimiter{\ceil}{\lceil}{\rceil}

\setmainfont{Linux Libertine}

\theoremstyle{plain}
\newtheorem{theorem}{Теорема}
\newtheorem{axiom}{Аксиома}
\newtheorem{lemma}{Лемма}

\theoremstyle{remark}
\newtheorem*{remark}{Примечание}
\newtheorem*{exercise}{Упражнение}
\newtheorem*{consequence}{Следствие}
\newtheorem*{example}{Пример}
\newtheorem*{observation}{Наблюдение}

\theoremstyle{definition}
\newtheorem*{definition}{Определение}
\newtheorem*{obozn}{Обозначение}

\lhead{Теория вероятности \textit{(практика)}}
\cfoot{}
\rfoot{16.2.2021}

\begin{document}

\begin{exercise}
    Трамвай ходит с интервалом строго 15 минут. Найти вероятность того, что, прийдя на остановку, вам придется ждать не более 5 минут.

    Мы можем равновероятно прийти в любой момент интервала 15 минут. Тогда \(\Omega = [0,15]\), \(\mu = l\), \(\mu(\Omega) = 15\), \(\mu(A) = 5\)

    \textbf{Ответ}: \(\frac{1}{3}\)
\end{exercise}

\begin{exercise}
    На прямой расставляются мины с интервалом \(10\) метров. Найти вероятность того, что танк шириной \(3\) метра подорвется на мине.

    \[\Omega = [0, 10] \quad \mu(\Omega) = 10 \quad \mu(A) = 7\]

    \textbf{Ответ}: \(\frac{7}{10}\)
\end{exercise}

\begin{exercise}
    \textcolor{red}{Скипнуто}
\end{exercise}

\begin{exercise}
    В круге радиуса \(1\) наугад нарисовали хорду. Найти вероятность того, что её длина будет больше стороны вписанного правильного треугольника.

    Зафиксируем точку начала хорды. Нарисуем вписанный правильный треугольник с одной из вершин в этой точке. Вторая точка выбирается равновероятно и лежит на одной из трёх дуг, которые замыкаются треугольником. Т.к. это дуги равны, вероятность попасть в противоположную дугу --- \(\frac{1}{3}\). Искомое эквивалентно попаданию в эту дугу.

    \textbf{Ответ}: \(\frac{1}{3}\)

    Другое рассуждение: будем проводить хорды перпендикулярно ``диагонали'' прямоугольника, тогда \(P = \frac{1}{2}\)

    Ещё одним рассуждением можно получить вероятность \(\frac{1}{4}\). Это называется парадокс Бертрана.

    Вывод: слова наподобие ``наугад'' не конкретны и их можно по-разному воспринимать. В формулировке задачи нужно указывать пространство элементарных исходов.
\end{exercise}

\end{document}