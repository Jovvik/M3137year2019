\documentclass[12pt, a4paper]{article}

\usepackage{lastpage}
\usepackage{mathtools}
\usepackage{xltxtra}
\usepackage{libertine}
\usepackage{amsmath}
\usepackage{amsthm}
\usepackage{amsfonts}
\usepackage{amssymb}
\usepackage{enumitem}
\usepackage{xcolor}
\usepackage[left=2.3cm, right=2.3cm, top=2.7cm, bottom=2.7cm, bindingoffset=0cm, headheight=15pt]{geometry}
\usepackage{fancyhdr}
\usepackage[russian]{babel}
% \usepackage{parindent}

\pagestyle{fancy}
\lfoot{M3137y2019}
\rhead{\thepage\ из \pageref{LastPage}}

\newcommand{\R}{\mathbb{R}}
\newcommand{\Q}{\mathbb{Q}}
\newcommand{\C}{\mathbb{C}}
\newcommand{\Z}{\mathbb{Z}}
\newcommand{\B}{\mathbb{B}}
\newcommand{\N}{\mathbb{N}}

\DeclareMathOperator*{\xor}{\oplus}
\DeclareMathOperator*{\equ}{\sim}
\DeclareMathOperator{\Ln}{\text{Ln}}
\DeclareMathOperator{\sign}{\text{sign}}
\DeclareMathOperator{\Sym}{\text{Sym}}
\DeclareMathOperator{\Asym}{\text{Asym}}
% \DeclareMathOperator{\sh}{\text{sh}}
% \DeclareMathOperator{\tg}{\text{tg}}
% \DeclareMathOperator{\arctg}{\text{arctg}}
% \DeclareMathOperator{\ch}{\text{ch}}

\DeclarePairedDelimiter{\ceil}{\lceil}{\rceil}

\setmainfont{Linux Libertine}

\theoremstyle{plain}
\newtheorem{theorem}{Теорема}
\newtheorem{axiom}{Аксиома}
\newtheorem{lemma}{Лемма}

\theoremstyle{remark}
\newtheorem*{remark}{Примечание}
\newtheorem*{exercise}{Упражнение}
\newtheorem*{consequence}{Следствие}
\newtheorem*{example}{Пример}
\newtheorem*{observation}{Наблюдение}

\theoremstyle{definition}
\newtheorem*{definition}{Определение}
\newtheorem*{obozn}{Обозначение}

\usepackage{sectsty}

\allsectionsfont{\raggedright}
\sectionfont{\fontsize{14}{15}\selectfont}
\subsectionfont{\fontsize{14}{15}\selectfont}

\lhead{Матстат, билеты}
\rfoot{}
\lfoot{}

\settoggle{useproofs}{true}

\renewcommand{\import}[3]{
    \section{#1}
    \getwithproof{#2}{#3}
}

\newcommand{\notenough}{
    \textcolor{red}{Кажется, тут недостаточно написано.}
}

\begin{document}

\import{Точечные оценки. Их свойства: состоятельность, несмещенность, эффективность.}{2}{1}

\import{Точечные оценки моментов. Свойства оценок математического ожидания и дисперсии.}{2}{2.1}
\get{2}{2.2}

\import{Метод моментов. Пример.}{2}{3}

\import{Метод максимального правдоподобия. Пример.}{3}{4}
\get{3}{4.1}
\get{3}{4.2}

\import{Информация Фишера. Неравенство Рао-Крамера (без док--ва).}{3}{5}

\import{Основные распределения математической статистики: хи--квадрат, Стьюдента, Фишера-Снедекора. Их свойства.}{4}{6}

\import{Линейные преобразования нормальных выборок. Теорема об ортогональном преобразовании. }{4}{7}

\import{Лемма Фишера.}{4}{8}

\import{Основная теорема о связи точечных оценок нормального распределения и основных распределений статистики.}{4}{9}

\import{Квантили распределений (оба определения). Функции для их вычисления в EXCEL.}{5}{10}

\import{Интервальные оценки. Определения, смысл, терминология.}{5}{11}
\notenough

\import{Доверительный интервал для математического ожидания нормального распределения при известном \(\sigma\).}{5}{12}

\import{Доверительный интервал для математического ожидания нормального распределения при неизвестном \(\sigma\).}{5}{13}

\import{Доверительный интервал для дисперсии нормального распределения при неизвестном \(a\).}{5}{14}

\import{Доверительный интервал для дисперсии нормального распределения при известном \(a\).}{5}{15}

\import{Проверка статистических гипотез. Определения, терминология. Уровень значимости и мощность критерия.}{6}{16}
\get{6}{16.2}

\import{Способы сравнения критериев проверки гипотез.}{6}{17}

\import{Построение критериев согласия (основные принципы).}{6}{18}

\import{Гипотеза о среднем нормальной совокупности с известной дисперсией.}{6}{19}

\import{Гипотеза о среднем нормальной совокупности с неизвестной дисперсией.}{6}{20}

\import{Доверительные интервалы как критерии гипотез о параметрах распределения.}{6}{21}

\import{Критерий хи--квадрат для параметрической гипотезы.}{7}{22}

\import{Критерий хи--квадрат для гипотезы о распределении.}{7}{23}

Дальше все аналогично с прошлым билетом.

\import{Критерий Колмогорова для гипотезы о распределении.}{7}{24}

\import{Критерий Колмогорова-Смирнова.}{7}{25}

\import{Критерий Фишера.}{7}{26}

\import{Критерий Стьюдента.}{7}{27}

\import{Понятие статистической зависимости. Корреляционное облако и корреляционная таблица. Первоначальные выводы по ним.}{8}{28}

\import{Критерий хи--квадрат для проверки независимости.}{8}{29}

\import{Однофакторный дисперсионный анализ. Общая, межгрупповая и внутригрупповая дисперсии. Теорема о разложении дисперсии.}{8}{30}

\import{Однофакторный дисперсионный анализ. Проверка гипотезы о влиянии фактора.}{8}{31}

\import{Математическая модель регрессии. Основные понятия и определения. Метод наименьших квадратов.}{9}{32}

\import{Вывод уравнения линейной парной регрессии. Геометрический смысл прямой регрессии.}{9}{33}

\import{Выборочный коэффициент линейной корреляции. Проверка гипотезы о его значимости.}{9}{34}

\import{Выборочное корреляционное отношение, его свойства.}{9}{35}

\import{Свойства ошибок в модели линейной парной регрессии. Анализ дисперсии фактора--результата. Коэффициент детерминации, его свойства.}{10}{36}

\import{Проверка гипотезы о значимости уравнения линейной регрессии. Связь между коэффициентом детерминации и коэффициентом линейной корреляции.}{10}{37}

\import{Теорема Гаусса-Маркова.}{10}{38}

\import{Стандартные ошибки коэффициентов регрессии. Их доверительные интервалы.}{10}{39}

\import{Прогнозирование в модели линейной парной регрессии. Стандартная ошибка прогноза, доверительный интервал прогноза.}{10}{40}

\import{Общая модель линейной регрессии. Вывод нормального уравнения (свойство существования квадратного корня симметрической матрицы без доказательства).}{11}{41}

\import{Свойства ОНМК в уравнении общей линейной регрессии.}{11}{42}

\import{Основная теорема об ОМНК (п.2 без доказательства).}{11}{43}

\import{Мультиколлинеарность, ее неприятные последствия. Основные принципы отбора факторов в модель общей линейной регрессии.}{12}{44}

\import{Стандартная ошибка общей линейной регрессии и стандартные ошибки коэффициентов регрессии. Проверка гипотезы о значимости отдельного коэффициента регрессии.}{12}{45.1}
\get{12}{45.2}

\import{Уравнение регрессии в стандартных масштабах. Смысл стандартизованных коэффициентов и частных коэффициентов эластичности. Разложение влияния фактора на прямое и косвенное.}{12}{46}

\import{Коэффициенты детерминации и множественной корреляции, их свойства. Проверка гипотезы о значимости уравнения регрессии в целом.}{12}{47}

\import{Взвешенный МНК.}{13}{48}

\import{Приемы сведения нелинейных регрессий к линейным.}{13}{49}

\import{Математические датчики случайных чисел.}{14}{50}

\import{Моделирование случайных величин методом обратной функции (включая дискретный случай).}{14}{51.1}
\get{14}{51.2}
\label{51конец}

\import{Моделирование нормальной случайной величины.}{14}{52}

\import{Быстрый показательный датчик.}{14}{53}

\import{Моделирование дискретных случайных величин.}{14}{54}
\textcolor{red}{Возможно, ещё нужно рассказать про моделирование через обратную функцию, см. конец билета~\ref{51конец}, стр. \pageref{51конец}.}

\import{Метод Монте-Карло. Общая постановка, оценка погрешности.}{15}{55}

\import{Вычисление определенного и кратного интегралов методом Монте-Карло. Метод расслоенной выборки.}{15}{56.1}
\get{15}{56.2}
\get{15}{56.3}

\end{document}
