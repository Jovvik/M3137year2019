\documentclass[12pt, a4paper]{article}

\usepackage{lastpage}
\usepackage{mathtools}
\usepackage{xltxtra}
\usepackage{libertine}
\usepackage{amsmath}
\usepackage{amsthm}
\usepackage{amsfonts}
\usepackage{amssymb}
\usepackage{enumitem}
\usepackage{xcolor}
\usepackage[left=2.3cm, right=2.3cm, top=2.7cm, bottom=2.7cm, bindingoffset=0cm, headheight=15pt]{geometry}
\usepackage{fancyhdr}
\usepackage[russian]{babel}
% \usepackage{parindent}

\pagestyle{fancy}
\lfoot{M3137y2019}
\rhead{\thepage\ из \pageref{LastPage}}

\newcommand{\R}{\mathbb{R}}
\newcommand{\Q}{\mathbb{Q}}
\newcommand{\C}{\mathbb{C}}
\newcommand{\Z}{\mathbb{Z}}
\newcommand{\B}{\mathbb{B}}
\newcommand{\N}{\mathbb{N}}

\DeclareMathOperator*{\xor}{\oplus}
\DeclareMathOperator*{\equ}{\sim}
\DeclareMathOperator{\Ln}{\text{Ln}}
\DeclareMathOperator{\sign}{\text{sign}}
\DeclareMathOperator{\Sym}{\text{Sym}}
\DeclareMathOperator{\Asym}{\text{Asym}}
% \DeclareMathOperator{\sh}{\text{sh}}
% \DeclareMathOperator{\tg}{\text{tg}}
% \DeclareMathOperator{\arctg}{\text{arctg}}
% \DeclareMathOperator{\ch}{\text{ch}}

\DeclarePairedDelimiter{\ceil}{\lceil}{\rceil}

\setmainfont{Linux Libertine}

\theoremstyle{plain}
\newtheorem{theorem}{Теорема}
\newtheorem{axiom}{Аксиома}
\newtheorem{lemma}{Лемма}

\theoremstyle{remark}
\newtheorem*{remark}{Примечание}
\newtheorem*{exercise}{Упражнение}
\newtheorem*{consequence}{Следствие}
\newtheorem*{example}{Пример}
\newtheorem*{observation}{Наблюдение}

\theoremstyle{definition}
\newtheorem*{definition}{Определение}
\newtheorem*{obozn}{Обозначение}

\lhead{Матстат \textit{(практика)}}
\cfoot{}
\rfoot{22.9.2021}

\begin{document}

\[\ln L (\vec{x}, a, \sigma^2) = - n \ln \sigma - \frac{n}{2} \ln (2\pi) - \frac{1}{2\sigma^2} \sum (x_i - a)^2\]
\[\frac{\partial}{\partial a} = \frac{1}{\sigma^2} (n \overline{x} - na) \quad \frac{\partial}{\partial \sigma} = \frac{1}{\sigma^3} \sum (x_i - a)^2 - \frac{n}{\sigma}\]
\[\begin{cases}
        \hat{a} = \overline{x} \\
        \hat{\sigma^2} = D_B
    \end{cases} \quad M \coloneqq (\overline{x}, D_B)\]
\[\frac{\partial^2}{\partial a^2} = -\frac{n}{\sigma^2} = - \frac{n}{D_B}\]
\[\frac{\partial^2}{\partial \sigma^2} = - \frac{3}{\sigma^4} \sum (x_i - a)^2 + \frac{n}{\sigma^2} = - \frac{3}{D_B^2} \sum (x_i - \overline{x})^2 + \frac{n}{D_B} = - \frac{3n}{D_B} + \frac{n}{D_B} = - \frac{2n}{D_B}\]
\[\frac{\partial^2}{\partial a \partial \sigma} = - \frac{2}{\sigma^3} (n \overline{x} - na) = 0\]
\[d^2 L(M) = - \frac{n}{D_B} (da)^2 + 2 \cdot 0 \cdot dad\sigma - \frac{2n}{D_B} (d\sigma)^2 = -\frac{n (da)^2 + 2n (d\sigma)^2}{D_B} < 0\]
Таким образом, \(M\) --- точка максимума.

\begin{remark}
    Можно было посчитать по теореме Сильвестра.
\end{remark}

\begin{exercise}
    \(X \in B_p\). Найти оценку параметра \(p\) методом максимального правдоподобия.
\end{exercise}

\begin{solution}
    \[L(\vec{X}, p) = \prod_{i=1}^{n} (1 - p)^{n - n \overline{x}} \cdot p^{n \overline{x}}\]
    \[\ln L(\vec{X}, p) = \sum_{i=1}^{n} \ln(1 - p) \cdot \ln p \cdot (n - n \overline{x}) \cdot n \overline{x} = \ln(1 - p) \cdot (n - n \overline{x}) + \ln p \cdot n \overline{x}\]
    \[\frac{\partial \ln L}{\partial p} = - \frac{n - n \overline{x}}{1 - p} + \frac{n \overline{x}}{p} = 0\]
    \[n \overline{x} (1 - p) + (n \overline{x} - n)p = 0\]
    \[p = \overline{x}\]
\end{solution}

\begin{exercise}
    \(X \in E_\alpha\). \(E_\alpha\) --- регулярное ли семейство? Найти \(I(\alpha)\).
\end{exercise}
\begin{solution}
    \(f_\alpha(x) = \begin{cases}
        0,                      & x < 0    \\
        \alpha e^{ -\alpha x} , & x \geq 0
    \end{cases}\)

    Носитель \(C = (0, \infty)\). Можно выкинуть точку \(0\), т.к. она имеет меру \(0\).

    \[\frac{\partial}{\partial \alpha} \ln f_\alpha(x) = \frac{\partial}{\partial \alpha} (\ln \alpha - \alpha x) = \frac{1}{\alpha} - x\]
    Эта функция непрерывна \(\forall \alpha \in C\).

    \[I(\alpha) = \E\left( \frac{\partial}{\partial \alpha} \ln(f_\alpha(X)) \right) = \E \left(X - \frac{1}{\alpha}\right) = \E\left(X - \E X\right) = \D X = \frac{1}{\alpha^2}\]
\end{solution}

\begin{exercise}
    То же самое, но для \(X \in E_{\frac{1}{\alpha}}\)
\end{exercise}
\begin{solution}
    \[f_\alpha(x) = \begin{cases}
            0,                                       & x < 0 \\
            \frac{1}{\alpha} e^{ -\frac{x}{\alpha}}, & x > 0
        \end{cases}\]
    \[\frac{\partial}{\partial \alpha} \ln f_\alpha(x) = \frac{\partial}{\partial \alpha} \left(\ln\frac{1}{\alpha} - \frac{x}{\alpha}\right) = - \frac{1}{\alpha} + \frac{x}{\alpha^2}\]
    \[I(\alpha) = \E\left(\frac{\partial}{\partial a} \ln f_\alpha(x)\right)^2 = \E\left(\frac{x}{\alpha^2} - \frac{1}{\alpha}\right)^2 = \frac{1}{\alpha^4} \E(X - \alpha)^2 = \frac{1}{\alpha^4} \D X = \frac{1}{\alpha^2}\]

    \[\alpha^* = \overline{x}\]
    \[\D \alpha^* = \D \overline{x} = \frac{\D x}{n} = \frac{\alpha^2}{n}\]
    \[\D \alpha^* \geq \frac{1}{n I(\alpha)}\]
    \[\frac{\alpha^2}{n} = \frac{\alpha^2}{n}\]
    Таким образом, оценка эффективная.
\end{solution}

\begin{exercise}
    Для \(X \in U(0, \theta)\) найти информацию фишера, проверить регулярность.

    \(\theta^* = 2 \overline{X}^*\) --- по м. моменты, \(\tilde \theta = \frac{n + 1}{n} X_{(n)}\) --- ОМП.
\end{exercise}
% \begin{solution}
%     \[f_\theta(x) = \begin{cases}
%             \frac{1}{\theta}, & x \in [0, \theta] \\
%             0,                & \text{иначе}
%         \end{cases}\]
%     \[C = (0, + \infty)\]
%     \[\frac{\partial}{\partial \theta} \ln f_\theta(x) = \]
% \end{solution}

\end{document}
