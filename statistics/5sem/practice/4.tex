\documentclass[12pt, a4paper]{article}

\usepackage{lastpage}
\usepackage{mathtools}
\usepackage{xltxtra}
\usepackage{libertine}
\usepackage{amsmath}
\usepackage{amsthm}
\usepackage{amsfonts}
\usepackage{amssymb}
\usepackage{enumitem}
\usepackage{xcolor}
\usepackage[left=2.3cm, right=2.3cm, top=2.7cm, bottom=2.7cm, bindingoffset=0cm, headheight=15pt]{geometry}
\usepackage{fancyhdr}
\usepackage[russian]{babel}
% \usepackage{parindent}

\pagestyle{fancy}
\lfoot{M3137y2019}
\rhead{\thepage\ из \pageref{LastPage}}

\newcommand{\R}{\mathbb{R}}
\newcommand{\Q}{\mathbb{Q}}
\newcommand{\C}{\mathbb{C}}
\newcommand{\Z}{\mathbb{Z}}
\newcommand{\B}{\mathbb{B}}
\newcommand{\N}{\mathbb{N}}

\DeclareMathOperator*{\xor}{\oplus}
\DeclareMathOperator*{\equ}{\sim}
\DeclareMathOperator{\Ln}{\text{Ln}}
\DeclareMathOperator{\sign}{\text{sign}}
\DeclareMathOperator{\Sym}{\text{Sym}}
\DeclareMathOperator{\Asym}{\text{Asym}}
% \DeclareMathOperator{\sh}{\text{sh}}
% \DeclareMathOperator{\tg}{\text{tg}}
% \DeclareMathOperator{\arctg}{\text{arctg}}
% \DeclareMathOperator{\ch}{\text{ch}}

\DeclarePairedDelimiter{\ceil}{\lceil}{\rceil}

\setmainfont{Linux Libertine}

\theoremstyle{plain}
\newtheorem{theorem}{Теорема}
\newtheorem{axiom}{Аксиома}
\newtheorem{lemma}{Лемма}

\theoremstyle{remark}
\newtheorem*{remark}{Примечание}
\newtheorem*{exercise}{Упражнение}
\newtheorem*{consequence}{Следствие}
\newtheorem*{example}{Пример}
\newtheorem*{observation}{Наблюдение}

\theoremstyle{definition}
\newtheorem*{definition}{Определение}
\newtheorem*{obozn}{Обозначение}

\lhead{Матстат \textit{(практика)}}
\cfoot{}
\rfoot{29.9.2021}

\begin{document}

\begin{exercise}
    Найти \(I(p)\) для \(B_p\), \(p^* = \overline{X}, \D p^*\) и \?
\end{exercise}
\begin{solution}
    \[f_p(x) = \begin{cases}
        1 - p, & x = 0 \\
        p, & x = 1
    \end{cases}\]
    \(C = \{0, 1\}\)
    \[I(p) = \E \left(\frac{\partial \ln f_p(x)}{\partial p}\right)^2 = \begin{cases}
        \E \left(\frac{1}{p - 1}\right)^2, & x = 0 \\
        \E \left(\frac{1}{p}\right)^2, & x = 1
    \end{cases} = \begin{cases}
        \frac{1}{(p - 1)^2}, & x = 0 \\
        \frac{1}{p^2}, & x = 1
    \end{cases}\]
    
    \(p\) выносится за \(\E\), т.к. не зависит от \(x\).
\end{solution}

\begin{exercise}
    Величины \(\xi, \eta\) имеют стандартное нормальное распределение и независимы. Будут ли независимы величины \(\xi + \eta\) и \(\xi - \eta\)?
\end{exercise}
\begin{solution}
    Да, т.к. матрица \(\begin{pmatrix}
        \frac{1}{\sqrt{2}} & \frac{1}{\sqrt{2}} \\
        \frac{1}{\sqrt{2}} & - \frac{1}{\sqrt{2}}
    \end{pmatrix}\) переводит \((\xi, \eta)\) в \((\frac{1}{\sqrt{2}}(\xi + \eta), \frac{1}{\sqrt{2}}(\xi - \eta))\) и по теореме эти величины независимы.
\end{solution}

\begin{exercise}
    Доказать, что если \(A\) --- симметричная положительно определенная матрица, то существует матрица \(B = \sqrt{A}\), т.е. \(B^2 = A\).
\end{exercise}
\begin{solution}
    Повернем \(A\) таким образом, что мы получим диагональную матрицу:
    \[A' = C^T A C = \begin{pmatrix}
        \lambda_1 & 0 & 0 & \cdots & 0 \\
        0 & \lambda_2 & 0 & \cdots & 0 \\
        0 & 0 & \lambda_3 & \cdots & 0 \\
        \vdots & \vdots & \vdots & \ddots & \vdots \\
        0 & 0 & 0 & \cdots & \lambda_n \\
    \end{pmatrix}\]
    
    Тогда
    \[\sqrt{A'} = \begin{pmatrix}
        \sqrt{\lambda_1} & 0 & 0 & \cdots & 0 \\
        0 & \sqrt{\lambda_2}  & 0 & \cdots & 0 \\
        0 & 0 & \sqrt{\lambda_3}  & \cdots & 0 \\
        \vdots & \vdots & \vdots & \ddots & \vdots \\
        0 & 0 & 0 & \cdots & \sqrt{\lambda_n}  \\
    \end{pmatrix} = B'\]
    
    И \(B = C B' C^T\)
    
    Проверим искомое:
    \[B^2 = (CB'C^T)(CB'C^T) = CB'B'C^T = CA'C^T = A\]
\end{solution}

\begin{exercise}
    Доказать, что правило трёх сигм \(P(|\xi - a| < 3\sigma) \geq \frac{8}{9}\) неулучшаемо.
\end{exercise}

\end{document}
