\documentclass[12pt, a4paper]{article}

\usepackage{lastpage}
\usepackage{mathtools}
\usepackage{xltxtra}
\usepackage{libertine}
\usepackage{amsmath}
\usepackage{amsthm}
\usepackage{amsfonts}
\usepackage{amssymb}
\usepackage{enumitem}
\usepackage{xcolor}
\usepackage[left=2.3cm, right=2.3cm, top=2.7cm, bottom=2.7cm, bindingoffset=0cm, headheight=15pt]{geometry}
\usepackage{fancyhdr}
\usepackage[russian]{babel}
% \usepackage{parindent}

\pagestyle{fancy}
\lfoot{M3137y2019}
\rhead{\thepage\ из \pageref{LastPage}}

\newcommand{\R}{\mathbb{R}}
\newcommand{\Q}{\mathbb{Q}}
\newcommand{\C}{\mathbb{C}}
\newcommand{\Z}{\mathbb{Z}}
\newcommand{\B}{\mathbb{B}}
\newcommand{\N}{\mathbb{N}}

\DeclareMathOperator*{\xor}{\oplus}
\DeclareMathOperator*{\equ}{\sim}
\DeclareMathOperator{\Ln}{\text{Ln}}
\DeclareMathOperator{\sign}{\text{sign}}
\DeclareMathOperator{\Sym}{\text{Sym}}
\DeclareMathOperator{\Asym}{\text{Asym}}
% \DeclareMathOperator{\sh}{\text{sh}}
% \DeclareMathOperator{\tg}{\text{tg}}
% \DeclareMathOperator{\arctg}{\text{arctg}}
% \DeclareMathOperator{\ch}{\text{ch}}

\DeclarePairedDelimiter{\ceil}{\lceil}{\rceil}

\setmainfont{Linux Libertine}

\theoremstyle{plain}
\newtheorem{theorem}{Теорема}
\newtheorem{axiom}{Аксиома}
\newtheorem{lemma}{Лемма}

\theoremstyle{remark}
\newtheorem*{remark}{Примечание}
\newtheorem*{exercise}{Упражнение}
\newtheorem*{consequence}{Следствие}
\newtheorem*{example}{Пример}
\newtheorem*{observation}{Наблюдение}

\theoremstyle{definition}
\newtheorem*{definition}{Определение}
\newtheorem*{obozn}{Обозначение}

\lhead{Матстат \textit{(практика)}}
\cfoot{}
\rfoot{6.10.2021}

\begin{document}

\begin{exercise}
    Для набора прибыли найти оценку параметра \(a\) при известном \(\sigma = 7.9\) с уверенностью \(\gamma = 0.99\)
\end{exercise}
\begin{solution}
    \(t_\gamma = 2.58\) \textit{(из таблицы)}.
    \[\overline{X} - t_\gamma \cdot \frac{\sigma}{\sqrt{n}} < a < \overline{X} + t_\gamma \cdot \frac{\sigma}{\sqrt{n}}\]
    \[97.6 - 2.58 \cdot \frac{7.9}{\sqrt{50}} < a < 97.6 + 2.58 \cdot \frac{7.9}{\sqrt{50}}\]
    \[94.718 < a < 100.482\]
\end{solution}

\begin{exercise}
    То же самое, но без известного \(\sigma\).
\end{exercise}
\begin{solution}
    Степеней свободы \(n - 1 = 49\).

    По таблице \(t_\gamma = 2.680\), по Excel СТЬЮДЕНТ.ОБР.2Х\((1-0,99;  49) = 2.679\)

    \[\overline{X} - t_\gamma \cdot \frac{S}{\sqrt{n}} < a < \overline{X} + t_\gamma \cdot \frac{S}{\sqrt{n}}\]
    \[97.6 - 2.68 \cdot \frac{7.89}{\sqrt{50}} < a < 97.6 + 2.68 \cdot \frac{7.89}{\sqrt{50}}\]
    \[94.609 < a < 100.590\]
\end{solution}

\begin{exercise}
    Найти доверительный интервал для \(\sigma\) при неизвестном значении параметра \(a\), \(\gamma = 0.99\).
\end{exercise}
\begin{solution}
    Степеней свободы \(k = n - 1 = 49, 1 - \frac{\gamma}{2} = 0.505, 1 + \frac{\gamma}{2} = 1.495\).

    По Excel: \(\chi_1^2 =\)ХИ2.ОБР\(\left(\frac{1 - \gamma}{2}; 49\right) = 27.249\) и \(\chi_2^2 =\)ХИ2.ОБР\(\left(\frac{1 + \gamma}{2}; 49\right) = 78.231\).

    \[\frac{(n - 1)S^2}{\chi_2^2} < \sigma^2 < \frac{(n - 1)S}{\chi_1^2}\]
    \[\frac{49 \cdot 62.245}{78.231} < \sigma^2 < \frac{49 \cdot 62.245}{27.249}\]
    \[78.99 < \sigma^2 < 111.93\]
    \[8.89 < \sigma^2 < 10.58\]
\end{solution}

\begin{exercise}
    То же самое, но при известном \(\sigma^{2*}\)
\end{exercise}
\begin{solution}
    \(k = n = 50\)

    \(\chi_1^2 =\)ХИ2.ОБР\(\left(\frac{1 - \gamma}{2}; 50\right) = 28.991.249\) и \(\chi_2^2 =\)ХИ2.ОБР\(\left(\frac{1 + \gamma}{2}; 49\right) = 79.49\).
    \[\sigma^{2*} = \frac{1}{n} \sum_{i=1}^{n} (X_i - a)^2\]
    \[\sigma^{2*} = \frac{1}{n} \sum_{i=1}^{7} (c_i - 98)^2 \cdot m_i = 58.42 \quad \text{(из Excel)}\]
    Дальше считается тривиально.
\end{solution}

\begin{solution}[Решение (задачи при правило трёх сигм)]\itemfix
    \begin{center}
        \begin{tabular}{CCCC}
            \toprule
            \xi_i & -1           & 0           & 1            \\ \midrule
            p_i   & \frac{1}{18} & \frac{8}{9} & \frac{1}{18} \\
            \bottomrule
        \end{tabular}
    \end{center}

    \[\E \xi = 0, \D \xi = \sum_{i=1}^{n} x_i^2 p_i - (\E \xi)^2 = \frac{1}{18} + \frac{1}{18} = \frac{1}{9}, \sigma = \frac{1}{3}, P(|\xi| < 3 \cdot \frac{1}{3} = 1) = \frac{8}{9}\]
\end{solution}

\end{document}
