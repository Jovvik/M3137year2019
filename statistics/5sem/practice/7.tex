\documentclass[12pt, a4paper]{article}

\usepackage{lastpage}
\usepackage{mathtools}
\usepackage{xltxtra}
\usepackage{libertine}
\usepackage{amsmath}
\usepackage{amsthm}
\usepackage{amsfonts}
\usepackage{amssymb}
\usepackage{enumitem}
\usepackage{xcolor}
\usepackage[left=2.3cm, right=2.3cm, top=2.7cm, bottom=2.7cm, bindingoffset=0cm, headheight=15pt]{geometry}
\usepackage{fancyhdr}
\usepackage[russian]{babel}
% \usepackage{parindent}

\pagestyle{fancy}
\lfoot{M3137y2019}
\rhead{\thepage\ из \pageref{LastPage}}

\newcommand{\R}{\mathbb{R}}
\newcommand{\Q}{\mathbb{Q}}
\newcommand{\C}{\mathbb{C}}
\newcommand{\Z}{\mathbb{Z}}
\newcommand{\B}{\mathbb{B}}
\newcommand{\N}{\mathbb{N}}

\DeclareMathOperator*{\xor}{\oplus}
\DeclareMathOperator*{\equ}{\sim}
\DeclareMathOperator{\Ln}{\text{Ln}}
\DeclareMathOperator{\sign}{\text{sign}}
\DeclareMathOperator{\Sym}{\text{Sym}}
\DeclareMathOperator{\Asym}{\text{Asym}}
% \DeclareMathOperator{\sh}{\text{sh}}
% \DeclareMathOperator{\tg}{\text{tg}}
% \DeclareMathOperator{\arctg}{\text{arctg}}
% \DeclareMathOperator{\ch}{\text{ch}}

\DeclarePairedDelimiter{\ceil}{\lceil}{\rceil}

\setmainfont{Linux Libertine}

\theoremstyle{plain}
\newtheorem{theorem}{Теорема}
\newtheorem{axiom}{Аксиома}
\newtheorem{lemma}{Лемма}

\theoremstyle{remark}
\newtheorem*{remark}{Примечание}
\newtheorem*{exercise}{Упражнение}
\newtheorem*{consequence}{Следствие}
\newtheorem*{example}{Пример}
\newtheorem*{observation}{Наблюдение}

\theoremstyle{definition}
\newtheorem*{definition}{Определение}
\newtheorem*{obozn}{Обозначение}

\lhead{Матстат \textit{(практика)}}
\cfoot{}
\rfoot{22.10.2021}

\begin{document}

Все вычисления доступны по \href{https://docs.google.com/spreadsheets/d/1lU-0FnVcjXLhWT8c0K_v2xk-lVkF9Ary3zV-o6R1Was/edit?usp=sharing}{ссылке}, листы ``7.1'' и ``7.2''.

\begin{exercise}
    Проверить, что распределение второй величины нормально с уверенностью \(\alpha = 0.05\).
\end{exercise}

Мы объединяем интервалы таким образом, чтобы в каждом интервале было хотя бы \(5\) точек.

\[P(a_i < \xi < a_{i+1}) = F(a_{i+1}) - F(a_i) = \Phi\left(\frac{a_{i+1} - a}{\sigma}\right) - \Phi\left(\frac{a_i - a}{\sigma}\right) = \Phi\left(\frac{a_{i+1} - \overline{X}}{S}\right) - \Phi(\frac{a_i - \overline{X}}{S})\]
Число степеней свободы \(s = k\) \textit{(число интервалов)} \(- m\) \textit{(число параметров распределения)} \(- 1 = 1\)
Т.к. \(\chi\) практическое \( = 0.49 < \chi\) теоретическое \( = \text{ХИ2.ОБР.2Х}(\alpha, s) = 3.84\), гипотеза принимается.

\begin{exercise}
    Проверить, что распределение первой величины экспоненциально с уверенностью \(\alpha = 0.05\).
\end{exercise}

Все так же, но \(F(a_i) = \exp( - \alpha a_i)\) и \(\alpha^* = \frac{1}{\overline{X}}\). \(\chi_{\text{практ.}} = 23.45 > \chi_{\text{теор.}} = 7.81\), гипотеза отклоняется.

\begin{exercise}
    Среди населения \(1\%\) воров. В комнате из \(10\) человек пропал кошелек. Какова вероятность того, что случайно выбранный из комнаты человек --- вор?
\end{exercise}

\end{document}
