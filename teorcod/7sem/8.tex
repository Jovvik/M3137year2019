\chapter{21 октября}

\begin{theorem}
    Минимальное расстояние полярного кода длины \(n = 2^m\)
    с замороженным множеством \(\mathcal{F}\) равно
    \(\min_{i \notin \mathcal{F}} 2^{\wt(i)} = \min_{i \notin \mathcal{F}} \wt(A_m^{i})\)
    
    , где \(A_m^i\) --- \(i\)--тая строка матрицы \(A_m\).
\end{theorem}

Минимальное расстояние полярного кода сильно хуже,
чем минимальное расстояние расширенного БЧХ или кода Рида-Маллера, например.
Но для этих кодов нет быстрого алгоритма декодирования,
а для полярного кода есть.

Когда алгоритм последовательного исключения доходит до какого-то
информационного символа, то при принятии решения об этом символе
никак не учитываются последующие замороженные символы.
Кроме того, если алгоритм допустил ошибку, то он не может
ее исправить.
Будем эти проблемы чинить.

\subsubsection{Списочное декодирование}

Не будем принимать окончательное решение о \(u_i\) на фазе \(i\),
вместо этого на каждой фазе \(i\) будем рассматривать \(L\) (константа)
векторов \(u^{i-1}_0\), строить возможные продолжения для них и выбирать
из них \(L\) наиболее вероятных.

Тогда вектора \(u_0^i\) задают пути в кодовом дереве.

\unfinished