\chapter{28 октября}

\section{Циклические коды}

\unfinished

\begin{theorem}
    Подмножество \(\mathcal{C} \in \faktor{\mathbb{F}[x]}{(x^n - 1)}\)
    образует циклический код тогда и только тогда, когда:
    \begin{enumerate}
        \item \(\mathcal{C}\) образует группу по сложению
        \item Если \(c(x) \in \mathcal{C}\) и \unfinished
    \end{enumerate}
\end{theorem}

\begin{definition}
    \textbf{Порождающий многочлен} циклического кода
    --- ненулевой кодовый многочлен \(g(x) \in \mathcal{C}\)
    наименьшей степени с коэффициентом при старшем члене 1.
\end{definition}

\begin{lemma}
    Все кодовые слова \(c(x)\) в циклическом коде делятся на \(g(x)\).
\end{lemma}
\begin{proof}
    Предположим противное, т.е. \(c(x) = a(x) g(x) + r(x), r(x) \in \mathcal{C}\).
    Но \(\deg r(x) < \deg g(x)\), что противоречит предположению о минимальности степени \(g(x)\).
\end{proof}

Порождающий многочлен циклического кода единственнен. % TODO: почему?

\begin{lemma}
    Циклический код длины \(n\) с порождающим многочленом \(g(x)\)
    существует тогда и только тогда, когда \(g(x) \mid (x^n - 1)\)
\end{lemma}
\begin{proof}\itemfix
    \begin{itemize}
        \item [``\(\Rightarrow\)''] Пусть \(g(x)\) не делится на \((x^n - 1)\).
        Тогда \(x^n - 1 = a(x)g(x) + r(x), \deg r < \deg g\)
        \unfinished
        \item [``\(\Leftarrow\)'']  В качестве порождающего многочлена можно выбрать любой делитель \(x^n - 1\).
    \end{itemize}
\end{proof}

\begin{definition}
    Т.к. \(g(x) \mid (x^n - 1)\), то \((x^n - 1) = g(x)h(x)\).
    \(h(x)\) называется \textbf{проверочным многочленом}.
\end{definition}

Несистематическое кодирование \unfinished

\begin{definition}
    Систематическое кодирование информационных символов
    \(a_0 \dots a_{k-1}\) в \(c_{n-k} \dots c_{n-1}\):
    \[c(x) = x^{n - k} a(x) - r(x)\]
    Определим \(r(x)\) как:
    \[r(x) \equiv x^{n - k} a(x) \mod g(x), \quad \deg r < \deg g\]
\end{definition}

Каждый метод кодирования можно представить в матричном виде,
и каждому из них соответствует своя порождающая матрица.
Матрицы различных методов выражаются друг через друга как \(G' = QG\),
где \(Q\) --- некоторая обратимая матрица.

Какой бы метод кодирования не использовался, корректирующая способность остается одинаковой.

\section{Элементы общей алгебры}

\begin{definition}
    Поле --- \unfinished.
\end{definition}

\begin{definition}
    Поля, содержащие конечное число элементов \(q\) называются
    \textbf{конечными полями} или \textbf{полями Галуа} \(GF(q)\).
\end{definition}

\begin{definition}
    Бесконечные поля называются полями \textbf{характеристики 0}.
\end{definition}

\begin{definition}
    Поле является \textbf{областью целостности}, т.к. если допустить
    \(\exists a, b \neq 0 : ab = 0\), то \(0 = (((ab)b^{-1})a^{-1}) = 1\).
\end{definition}

Поле \(GF(p)\), где \(p\) простое, суть арифметика по модулю \(p\).

Характеристика поля --- \unfinished.

\begin{definition}
    Группа называется \textbf{циклической}, если \(\exists y \in G\) такой,
    что любой ее элемент может быть получен как \(x = y^i\)
    для некоторого \(i \in \N\). 
\end{definition}

\begin{definition}
    \textbf{Порядок элемента} группы \(x \in G\) ---
    минимальное число \(i \in \N_+\), такое что \(x^i = 1\).
\end{definition}

\begin{definition}
    \textbf{Порядок группы} --- порядок образующего элемента группы.    
\end{definition}

\begin{theorem}[Лагранж]
    Порядок любой конечной группы делится на порядок любой ее подгруппы.
\end{theorem}

\begin{theorem}
    Пусть \(\mathcal{G} = (G, \cdot)\) --- конечная группа и элементы
    \(g, h \in G\) имеют порядок \(r, s\) соответственно,
    причем \(\gcd(r, s) = 1\). Тогда \(gh\) имеет порядок \(rs\).
\end{theorem}
\begin{proof}
    Очевидно, что \((gh)^{rs} = 1\).
    Следовательно, порядок \(p\) элемента \(gh\) --- делитель числа \(rs\).
    Таким образом, \(p \mid rs\) и \((gh)^p = 1\),
    следовательно, \((gh)^{pr} = 1 \cdot h^{pr} = 1\).
    Следовательно, \(s \mid ps \Rightarrow s \mid p\). Аналогично \(r \mid p\).
    
    Т.к. \(\gcd(r, s) = 1\), получаем \(rs \mid p \Rightarrow p = rs\)
\end{proof}

\begin{theorem}
    Пусть \(\mathbb{F}\) --- поле из \(q\) элементов.
    Тогда \(q = p^m\), где \(p\) --- простое, а \(m \in \N\). 
\end{theorem}
\begin{proof}
    Элемент \(1 \in \mathbb{F}\) образует аддитивную циклическую подгруппу
    простого порядка поля \(\mathbb{F} \Rightarrow p \mid q \xRightarrow{\text{Лагранж}} GF(p) \subset \mathbb{F}\).
    
    Будем называть элементы \(\alpha_1 \dots \alpha_m\) линейно независимыми
    с коэффициентами из \(GF(p)\), если:
    \[\left\{(x_1 \dots x_m) \in GF(p^m) \mid \sum_{i = 1}^m x_i \alpha_i = 0\right\} = \{(0 \dots 0)\}\]
    
    Среди всех ЛНЗ подмножеств \(\mathbb{F}\) выделим подмножество
    \(\{\alpha_1 \dots \alpha_m\} \subset \mathbb{F}\) с максимальным числом элементов, тогда:
    \[\forall \alpha_0 \in \mathbb{F} \ \ \exists x_1 \dots x_m \in GF(p) : \alpha_0 = \sum_{i = 1}^m x_i \alpha_i\]
    При этом различные \(x_1 \dots x_m\) приводят к различным \(\alpha_0 \in \mathbb{F} \Rightarrow |\mathbb{F}| = p^m\)
\end{proof}

\begin{itemize}
    \item \(GF(q^m)\) образует \(m\)-мерное линейное пространство над полем \(GF(q)\)
    \item \(GF(p^m), m > 1\) --- расширенное конечное поле, где \(m\) --- степень расширение.
    \item \(GF(p^m) \neq \Z_{p^m}\)
\end{itemize}

\begin{theorem}
    Ненулевые элементы \(GF(q)\) образуют конечную циклическую группу
    по умножению.
\end{theorem}
\begin{proof}
    По аксиомам поля \(\mathbb{F} \setminus \{0\}\) образует
    конечную группу по умножению.
    
    Выберем в \(\mathbb{F} \setminus \{0\}\) элемент \(\alpha\),
    называемый \textbf{примитивным}, с наибольшим порядком \(r\).
    Пусть \(l\) --- порядок некоторого элемента \(\beta \neq 0\).
    
    \unfinished
\end{proof}

\begin{prop}[Конечные поля]\itemfix
    \begin{itemize}
        \item Для всякого ненулевого \(\beta \in GF(q)\) выполняется \(\beta^{q - 1} = 1\)
        \item Все элементы поля \(GF(q)\) удовлетворяют уравнению \(x^q - q = 0\)
        \item Порядок любого ненулевого \(\beta \in GF(q)\) делит \(q - 1\).
        \item В поле характеристики \(p \geq 1\) справедливо
        \[(x + y)^p = x^p + y^p\]
        \[(x + y)^p = \sum_{i = 0}^p C_p^i x^i y^{p - i} \quad C_p^i = \frac{p!}{i!(p - i)!} \equiv 0 \mod p, 0 < i < p\]
    \end{itemize}
\end{prop}

\subsection{Минимальные многочлены}

\begin{definition}
    \textbf{Минимальным многочленом} элемента \(\beta \in GF(p^m)\) над \(GF(p)\)
    называется нормированный многочлен \(M_\beta(x) \in GF(p)[x]\)
    наименьшей степени такой, что \(M_\beta(\beta) = 0\)
\end{definition}

\begin{theorem}
    \(M_\beta(x)\) неприводим над \(GF(p)\).
\end{theorem}
\begin{proof}
    Пусть \(M_\beta(x)\) приводим, т.е. \(M_\beta(x) = M_1(x)M_2(x)\) и \(\deg M_i < \deg M\), \unfinished
\end{proof}

\begin{theorem}
    \unfinished
\end{theorem}

\begin{theorem}
    \unfinished
\end{theorem}

\begin{theorem}
    Если \(\alpha\) --- примитивный элемент \(GF(p^m)\), то степень его минимального многочлена равна \(m\).
\end{theorem}
\begin{proof}
    \unfinished
\end{proof}

\begin{theorem}
    Все конечные поля \(GF(p^m)\) изоморфны.
\end{theorem}
\begin{proof}
    Пусть \(F\) и \(G\) --- поля, содержащие \(p^m\) элементов, пусть \(\alpha\)
    --- примитивный элемент \(F\) с минимальным многочленом \(\pi(x)\).
    \[\pi(x) \mid (x^{p^m} - x) \Rightarrow \exists \beta \in G : \pi(\beta) = 0\]
    Тогда можно рассматривать \(F\) как множество многочленов от \(\alpha\)
    степени не более \(m - 1\), а \(G\) --- как множество многочленов от
    \(\beta\) степени не более \(m - 1\).
    Тогда соответствие \(\alpha \leftrightarrow \beta\) задает изоморфизм
    полей \(F\) и \(G\).
\end{proof}

\unfinished

\subsection{Построение минимальных многочленов}

\begin{theorem}
    \[\forall \beta \in GF(p^m) \ \ M_{\beta}(x) = M_{\beta^{p}}(x)\]
\end{theorem}
\begin{proof}
    \[0 = M_\beta(\beta) = \sum_{i = 0}^d M_{\beta,i}\beta^i, M_{\beta,i} \in GF(p)\]
    \[0 = (M_\beta(\beta))^p = \sum_{i = 0}^d M_{\beta,i}^p \beta^{pi}
    = \sum_{i = 0}^d M_{\beta,i}\beta^{pi} = M_\beta(\beta^p) \Rightarrow
    M_{\beta^p}(x) \mid M_\beta(x)\]
    Т.к. минимальные многочлены неприводимы, из делимости следует
    \(M_{\beta^p}(x) = M_\beta(x)\).
\end{proof}

\unfinished