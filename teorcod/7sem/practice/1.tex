\documentclass[12pt, a4paper]{article}

\usepackage{lastpage}
\usepackage{mathtools}
\usepackage{xltxtra}
\usepackage{libertine}
\usepackage{amsmath}
\usepackage{amsthm}
\usepackage{amsfonts}
\usepackage{amssymb}
\usepackage{enumitem}
\usepackage{xcolor}
\usepackage[left=2.3cm, right=2.3cm, top=2.7cm, bottom=2.7cm, bindingoffset=0cm, headheight=15pt]{geometry}
\usepackage{fancyhdr}
\usepackage[russian]{babel}
% \usepackage{parindent}

\pagestyle{fancy}
\lfoot{M3137y2019}
\rhead{\thepage\ из \pageref{LastPage}}

\newcommand{\R}{\mathbb{R}}
\newcommand{\Q}{\mathbb{Q}}
\newcommand{\C}{\mathbb{C}}
\newcommand{\Z}{\mathbb{Z}}
\newcommand{\B}{\mathbb{B}}
\newcommand{\N}{\mathbb{N}}

\DeclareMathOperator*{\xor}{\oplus}
\DeclareMathOperator*{\equ}{\sim}
\DeclareMathOperator{\Ln}{\text{Ln}}
\DeclareMathOperator{\sign}{\text{sign}}
\DeclareMathOperator{\Sym}{\text{Sym}}
\DeclareMathOperator{\Asym}{\text{Asym}}
% \DeclareMathOperator{\sh}{\text{sh}}
% \DeclareMathOperator{\tg}{\text{tg}}
% \DeclareMathOperator{\arctg}{\text{arctg}}
% \DeclareMathOperator{\ch}{\text{ch}}

\DeclarePairedDelimiter{\ceil}{\lceil}{\rceil}

\setmainfont{Linux Libertine}

\theoremstyle{plain}
\newtheorem{theorem}{Теорема}
\newtheorem{axiom}{Аксиома}
\newtheorem{lemma}{Лемма}

\theoremstyle{remark}
\newtheorem*{remark}{Примечание}
\newtheorem*{exercise}{Упражнение}
\newtheorem*{consequence}{Следствие}
\newtheorem*{example}{Пример}
\newtheorem*{observation}{Наблюдение}

\theoremstyle{definition}
\newtheorem*{definition}{Определение}
\newtheorem*{obozn}{Обозначение}

\lhead{Теория кодирования \textit{(практика)}}
\cfoot{}
\rfoot{2.9.2022}
\lfoot{M343$\frac{6}{7}$y2019}

\makeatletter
\renewcommand*\env@matrix[1][*\c@MaxMatrixCols c]{%
  \hskip -\arraycolsep
  \let\@ifnextchar\new@ifnextchar
  \array{#1}}
\makeatother

\begin{document}

Скорость кода --- отношение исходных данных к передаваемым данным.

\begin{example}
    Решить систему линейных уравнений:
    \[
        \begin{pmatrix} 
            1 & 2 & 3 & 4 \\
            5 & 6 & 7 & 8 \\
            9 & 10 & 11 & 12 \\
            13 & 14 & 15 & 16
            \end{pmatrix} x = \begin{pmatrix} 1 \\ 1 \\ 1 \\ 1 \end{pmatrix}\]
        \begin{align*}
            \begin{pmatrix}[cccc|c] 
                1 & 2 & 3 & 4 & 1\\
                5 & 6 & 7 & 8 & 1\\
                9 & 10 & 11 & 12 & 1 \\
                13 & 14 & 15 & 16 & 1
            \end{pmatrix}
            & = 
            \begin{pmatrix}[cccc|c] 
                1 & 2 & 3 & 4 & 1 \\
                0 & -4 & -8 & -12 & -4 \\
                0 & -8 & -16 & -24 & -8 \\
                0 & -12 & -24 & -36 & -12
            \end{pmatrix} \\
            & = 
            \begin{pmatrix}[cccc|c] 
                1 & 2 & 3 & 4 & 1 \\
                0 & -4 & -8 & -12 & -4 \\
                0 & 0 & 0 & 0 & 0 \\ 
                0 & 0 & 0 & 0 & 0 
            \end{pmatrix} \\
        .\end{align*}
    \[\begin{cases}
        x_1 + 2x_2 + 3x_3 + 4x_4 = 0 \\
        x_2 + 2x_3 + 3x_4 = 0
    \end{cases} \implies \begin{cases}
        x_1 = x_3 + 2x_4 \\
        x_2 = -2x_2 - 3x_3
    \end{cases}\] 
    Общее решение:
    \[
        C_1 \begin{pmatrix} 1 \\ -2 \\ 1 \\ 0 \end{pmatrix}
        + C_2 \begin{pmatrix} 2 \\ -3 \\ 0 \\ 1 \end{pmatrix} 
        + \begin{pmatrix} -1 \\ 1 \\ 0 \\ 0 \end{pmatrix} 
    \] 
\end{example}

Мы часто будем рассматривать линейные коды, а не произвольные отображения.

\begin{notation}
    \[u_a^b = (u_a, u_{a + 1}, \ldots, u_b)\] 
\end{notation}

Линейный код задается \textbf{порождающей} матрицей \(G\) размера \(n \times k\) и образуется как \(c_1^n = u_1^k G\).

По матрице \(G\) находится матрица \(H\), называемая \textbf{проверочной}, такая, что \(G H\tran = 0\). Тогда \(C H\tran = 0\), таким образом можно проверять, принадлежит ли полученная последовательность коду.

\begin{example}
    Найти проверучную матрицу для кода, который задан порождающей матрицей:
    \[G = \begin{pmatrix} 1111 & 1111 \\ 1111 & 0000 \\ 1100 & 1100 \\ 1010 & 1010 \end{pmatrix}\] 

\end{example}

\begin{definition}
    \textbf{Дуальный код} --- код, в котором \(G\) и \(H\) совпадают.
\end{definition}

Рассмотрим канал с белым гауссовским шумом \(y_i = x_i + \eta_i, \eta_i \sim \mathcal{N}(0, \sigma^2)\). 
Не любой канал является таким, но это хороший benchmark. 
Шум мжоно сводить к примерно белому гауссову различными эквалайзерами и т.д.
Пусть канал передает \(c_i \in \{0, 1\} \to x_i \in \{-1, 1\}\).
\[P(c_i \mid y) = \frac{P(c_i \cap y)}{P(y)} = \frac{P(y \mid c_i) P(y)}{P(c_i)}\] 
\[W(y \mid x) = \frac{1}{\sqrt{2 \pi \sigma^2}} e^{-\frac{(y-x)^2}{2\sigma^2}}\] 
\[L
= \ln \frac{P(c_i = 0 \mid y_i)}{P(c_i = 1 \mid y_i)}
= \ln \frac{P(y_i \mid c_i = 0)}{P(y_i \mid c_i = 1)}
= \ln \frac{e^{-\frac{(y+1)^2}{2\sigma^2}}}{e^{-\frac{(y-1)^2}{2\sigma^2}}}
= -\frac{(y+1)^2}{2\sigma^2} + \frac{(y-1)^2}{2\sigma^2}
= \frac{-2y}{\sigma^2}
\] 

\end{document}
