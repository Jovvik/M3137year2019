\chapter{16 сентября}

\begin{example}
    Напишем парсер для арифметических выражений.

    \begin{minted}{cpp}
E()
    res = Node(E)
    switch (token)
        case 'n':
        case 'c':
            res.add(T())
            res.add(Eprime())
    return res

Eprime()
    res = Node(Eprime)
    switch (token)
        case '+':
            assert(token = '+')
            res.add(Node(+))
            nextToken()
            res.add(T())
            res.add(Eprime())
        case '$':
        case ')':
            pass
    return res
    
T()
    res = Node(T)
    switch (token)
        case 'n':
            assert(token = 'n')
            res.add(Node(n))
            nextToken()
        case '('
            assert(token = '(')
            res.add(Node('('))
            nextToken()
            res.add(E())
            assert(token = ')')
            res.add(Node(')'))
            nextToken()
    return res
    \end{minted}

    Вспомним код, который мы сдавали Георгию Александровичу:
    \begin{minted}{python}
int E():
    res = T()
    while token = '+':
        nextToken()
        res += T()
    return res
    \end{minted}

    Он мощнее, потому что написан для другой грамматики \textit{(без \(E'\))}, которая не \(LL(1)\).
\end{example}

\begin{example}
    \(\sphericalangle A \to A\alpha, A \to \beta_1|\beta_2\dots\), после устранения левой рекурсии мы получим \(A \to \beta A', A' \to \alpha A', A' \to \varepsilon\). Запишем рекурсивный спуск для \(A\) без спуска для \(A'\):
    \begin{minted}[escapeinside=||,mathescape=true]{cpp}
A()
    res = Node(A)
    switch (token)
        case FIRST1(|$\beta_1$|)
            ...
        case FIRST1(|$\beta_2$|)
            ...
        cur = Node(Aprime)
        res.add(cur)
        while token |$\in$| FIRST1(|$\alpha_i$|)
            switch (token)
                case FIRST(|$\alpha_1$|)
                    cur.add(...)
            next = Node(Aprime)
            cur.add(next)
            cur = next
        // assert token $\in$ FOLLOW(A)
    \end{minted}
    Таким образом, мы можем не устранять левую рекурсию в грамматике, а делать это при написании парсера.
\end{example}

Писать парсеры так, как мы сейчас это делали, неудобно. Будем писать по-другому.

\(\sphericalangle \Gamma \in LL(1)\). Запишем информацию в таблицу, называемую \textbf{управляющей таблицей}:

\begin{notation}\itemfix
    \begin{itemize}
        \item \( \to \) --- \texttt{nextToken()}
        \item \(\bot\) --- дно стека.
    \end{itemize}
\end{notation}

Для каждого \(A \in N\) пусть есть правила \(A \to \alpha_i\). Тогда для каждого \(t \in \mathrm{FIRST1}(\alpha_i)\) впишем в ячейку \((A, t)\) число \(i\):
\begin{center}
    \begin{tabular}{llccccc}
        \toprule
                                    &       & \multicolumn{4}{c}{\(\Sigma\)}                                & \$                                                                                             \\
        \midrule
        \multirow{4}{*}{\(N\)}                                                                                                                                                                               \\
                                    & \(A\) & \multicolumn{4}{c}{\(\dots i \dots i \dots i \dots i \dots\)}                                                                                                  \\
        \\
        \\
        \cmidrule{3-6}
        \multirow{4}{*}{\(\Sigma\)} &       & \(\to\)                                                       &         & \multicolumn{2}{c}{ошибка} & \multirow{4}{*}{\rotatebox[origin=c]{-90}{ошибка}}      \\
                                    &       &                                                               & \(\to\) &                            &                                                         \\
                                    &       &                                                               &         & \(\ddots\)                 &                                                         \\
                                    &       & \multicolumn{2}{c}{ошибка}                                    &         & \( \to \)                                                                            \\
        \cmidrule{3-6}
        \(\bot\)                    &       &                                                               &         &                            &                                                    & ок \\
        \bottomrule
    \end{tabular}
\end{center}

Как разборщик пользуется таблицей? На каждой итерации он смотрит на вершину стека\footnote{В котором лежат разбираемые нетерминалы/терминалы.} и на управляющую таблицу.
\begin{enumerate}
    \item Если на вершине стека терминал:
          \begin{enumerate}
              \item Если терминал совпал с токеном, то с вершины стека снимается терминал и происходит переход к следующему токену.
              \item Если терминал не совпал с токеном, то ошибка.
          \end{enumerate}
    \item Если на вершине стека нетерминал, то в таблице ищется номер правила грамматики, соответствующий данному нетерминалу и токену.
          \begin{enumerate}
              \item Если ячейка не пуста, то из вершины стека удаляется элемент и в стек кладется правая часть соответствующего правила.
              \item Если ячейка пуста, то ошибка.
          \end{enumerate}
\end{enumerate}

\begin{example}[арифметические выражения]\itemfix
    \begin{center}
        \begin{tabular}{cCCCCCC}
            \toprule
            \diagbox{Стек}{Ввод} & n   & +   & *   & (   & )   & \$          \\
            \midrule
            \(E\)                & 1   &     &     & 1   &     &             \\
            \(E'\)               &     & 2   &     &     & 3   & 3           \\
            \(T\)                & 4   &     &     & 4   &     &             \\
            \(T'\)               &     & 6   & 5   &     & 6   & 6           \\
            \(F\)                & 7   &     &     & 8   &     &             \\
            \(n\)                & \to &     &     &     &     &             \\
            \(+\)                &     & \to &     &     &     &             \\
            \(*\)                &     &     & \to &     &     &             \\
            \((\)                &     &     &     & \to &     &             \\
            \()\)                &     &     &     &     & \to &             \\
            \(\bot\)             &     &     &     &     &     & \mathrm{ОК} \\
            \bottomrule
        \end{tabular}
    \end{center}
\end{example}
