\chapter{9 ноября}

\section{Равенство}

Напоминание: \(a = b\) по определению выполнено тогда и только тогда когда существует путь \(a \rightsquigarrow b\). Путь можно определить так:

\begin{definition}
    Пусть \(I\) --- интервальный тип. Если существует непрерывная функция \(f : I \to A\), такая что \(f\ \mathrm{left} =_\beta a, f\ \mathrm{right} =_\beta b\), то \(a \rightsquigarrow b\).
\end{definition}

То же самое в рамках Arend:
\begin{minted}{haskell}
\data Path
    path (f : I -> A) : f left = f right
\end{minted}

Когда мы говорим о любом типе данных, у нас есть две конструкции:
\begin{enumerate}
    \item Построение
    \item Удаление
\end{enumerate}

Для \texttt{Path} есть построение --- конструктор. Мы хотим получить ещё и элиминатор.

\begin{example}[элиминатор для \(\lor\)]\itemfix
    \begin{minted}[escapeinside=||,mathescape=true]{haskell}
case (f : L -> |$\theta$|) (g : R -> |$\theta$|) (v : L |$\lor$| R) : |$\theta$|
    \end{minted}
\end{example}

\begin{example}[элиминатор для \(I\)]\itemfix
    \begin{minted}{haskell}
\func coe (P : I -> \Type)
        (a : P left)
        (i : I) : P i \elim i
    | left => a
            \end{minted}

    Т.к. все значения на пути равны, то мы возвращаем значение на \texttt{left}.

    \verb!\elim! это паттерн матчинг. Т.к. \(I\) --- особый тип, нам не нужно расписывать все случаи.
\end{example}

Это все весьма странно, но это единственная конструкция подобного рода.

\begin{minted}{haskell}
\func transport {A : \Type} (B : A -> \Type) {a a' : A} (p : a = a') (b : B a)
    : B a'
    => coe (\lam i => B (p @ i)) b right
\end{minted}

\begin{notation}
    Если \verb!p : a = a'!, то:
    \begin{enumerate}
        \item \verb!p @ left! это \verb!a!
        \item \verb!p @ right! это \verb!a'!
    \end{enumerate}
\end{notation}

\begin{remark}
    Фигурные скобки означает типы, которые компилятор сам выведет из контекста. Рассчитывать на него нельзя, т.к. выведение типов неразрешимо в общем случае.
\end{remark}

\begin{example}[доказательство равенства]\itemfix
    \begin{minted}{haskell}
\func inv (A : \Type) {a a' A} (p : a = a') : a' = a
    => transport (\lam x => x) p idp

\func idp {A : \Type} {a : A} : a = a
    => path (\lam _ => a)
        \end{minted}
\end{example}

\begin{example}[конгруэнтность]\itemfix
    \begin{minted}{haskell}
\func pmap (A B : \Type) (f : A -> B) (a a' : A) {p : a = a'} : f a = f a'
    => transport (\lam x => f a = f x) p idp
    \end{minted}
\end{example}

\begin{example}[натуральные числа]\itemfix
    \begin{minted}{haskell}
\data Nat
    | zero
    | suc (k : Nat)
    
\data Empty

Not (A : \Type) : A -> Empty

Not (zero = suc zero)

\func proof_ne (a : Nat) : \Type \elim a
    | zero => 0 = 0
    | suc x => Empty
    
\func zne (x : 0 = 1) : Empty
    => transport proof_ne {0} {1} x idp
    \end{minted}
\end{example}

\begin{remark}
    В стандартной библиотеке это доказано немного по-другому, вместо \verb!0 = 0! используется \verb!\Sigma! тип кортежа из нуля элементов, то есть \verb!()! из Haskell.
\end{remark}
