\documentclass[12pt, a4paper]{article}

\usepackage{lastpage}
\usepackage{mathtools}
\usepackage{xltxtra}
\usepackage{libertine}
\usepackage{amsmath}
\usepackage{amsthm}
\usepackage{amsfonts}
\usepackage{amssymb}
\usepackage{enumitem}
\usepackage{xcolor}
\usepackage[left=2.3cm, right=2.3cm, top=2.7cm, bottom=2.7cm, bindingoffset=0cm, headheight=15pt]{geometry}
\usepackage{fancyhdr}
\usepackage[russian]{babel}
% \usepackage{parindent}

\pagestyle{fancy}
\lfoot{M3137y2019}
\rhead{\thepage\ из \pageref{LastPage}}

\newcommand{\R}{\mathbb{R}}
\newcommand{\Q}{\mathbb{Q}}
\newcommand{\C}{\mathbb{C}}
\newcommand{\Z}{\mathbb{Z}}
\newcommand{\B}{\mathbb{B}}
\newcommand{\N}{\mathbb{N}}

\DeclareMathOperator*{\xor}{\oplus}
\DeclareMathOperator*{\equ}{\sim}
\DeclareMathOperator{\Ln}{\text{Ln}}
\DeclareMathOperator{\sign}{\text{sign}}
\DeclareMathOperator{\Sym}{\text{Sym}}
\DeclareMathOperator{\Asym}{\text{Asym}}
% \DeclareMathOperator{\sh}{\text{sh}}
% \DeclareMathOperator{\tg}{\text{tg}}
% \DeclareMathOperator{\arctg}{\text{arctg}}
% \DeclareMathOperator{\ch}{\text{ch}}

\DeclarePairedDelimiter{\ceil}{\lceil}{\rceil}

\setmainfont{Linux Libertine}

\theoremstyle{plain}
\newtheorem{theorem}{Теорема}
\newtheorem{axiom}{Аксиома}
\newtheorem{lemma}{Лемма}

\theoremstyle{remark}
\newtheorem*{remark}{Примечание}
\newtheorem*{exercise}{Упражнение}
\newtheorem*{consequence}{Следствие}
\newtheorem*{example}{Пример}
\newtheorem*{observation}{Наблюдение}

\theoremstyle{definition}
\newtheorem*{definition}{Определение}
\newtheorem*{obozn}{Обозначение}

\lhead{Теория типов \textit{(практика)}}
\cfoot{}
\rfoot{21.9.2021}

\begin{document}

\section*{Домашнее задание №2: <<формализация лямбда-исчисления>>}

\begin{enumerate}
    \item Придумайте грамматику для лямбда-выражений, однозначно разбирающую любое выражение
          (в частности, учитывающую все сокращения скобок в записи).
          \begin{solution}\itemfix
              \begin{itemize}
                  \item \(S \to \lambda T.S\)
                  \item \(S \to S\ S\)
                  \item \(S \to T\)
                  \item \(S \to (S)\)
                  \item \(T \to \Sigma\)
              \end{itemize}
          \end{solution}
    \item Приведите пример лямбда-выражения, корректная бета-редукция которого невозможна без переименования
          связанных переменных. Возможно ли, чтобы в этом выражении все переменные в лямбда-абстракциях
          были различными?
          \begin{solution}
              \((\lambda y.\lambda x.y\ x)\ (\lambda x.x)\)
          \end{solution}
    \item Два выражения $A$ и $B$ назовём родственными, если существует $C$, что
          $A \twoheadrightarrow_\beta C$ и $B \twoheadrightarrow_\beta C$.
          Как соотносится родственность и бета-эквивалентность?
          \begin{solution}
              \begin{notation}
                  \(A\) родственн. \(B\) \(\Leftrightarrow\) \(A \sim B\)
              \end{notation}

              \[A \sim B \Rightarrow \begin{cases}
                      A \twoheadrightarrow_\beta C \\
                      B \twoheadrightarrow_\beta C
                  \end{cases} \Rightarrow \begin{cases}
                      A =_\beta C \\
                      B =_\beta C
                  \end{cases} \Rightarrow A =_\beta B\]

              В обратную сторону тоже верно по ромбовидному свойству.
          \end{solution}

    \item Рассмотрим представление лямбда-выражений де Брауна (de Bruijn): вместо имени связанной переменной будем
          указывать число промежуточных лямбда-абстракций между связывающей абстрацией и переменной.
          Например, $\lambda x.\lambda y.y\ x$ превратится в $\lambda.\lambda.0\ 1$.

          Докажите, что $A =_\alpha B$ тогда и только тогда, когда представления де Брауна для $A$ и $B$ совпадают.
          Сформулируйте правила (алгоритмы) для подстановки термов и бета-редукции для этого представления.

    \item Как мы знаем, $\Omega \rightarrow_\beta \Omega$. А существуют ли такие лямбда-выражения
          $A$ и $B$ ($A \ne_\alpha B$), что $A \rightarrow_\beta B$ и $B \rightarrow_\beta A$?

          \begin{solution}
              \(A = \omega\ (\lambda x.\omega\ x), B = (\lambda x.\omega\ x)\ (\lambda x.\omega\ x)\)
          \end{solution}

    \item Рассмотрим следующие лямбда-выражения для задания алгебраических типов:

          \begin{tabular}{lll}
              Обозначение & лямбда-терм                             & название                 \\\hline
              $Case$      & $\lambda l.\lambda r.\lambda c.c\ l\ r$ & сопоставление с образцом \\
              $InL$       & $\lambda l.(\lambda x.\lambda y.x\ l)$  & левая инъекция           \\
              $InR$       & $\lambda r.(\lambda x.\lambda y.y\ r)$  & правая инъекция          \\
          \end{tabular}

          Сопоставление с образцом --- это функция от значения алгебраического типа и двух действий $l$ и $r$,
          которая выполняет действие $l$, если значение создано <<левым>> конструктором, и $r$ в случае
          <<правого>> конструктора. Иными словами, $Case\ l\ r\ c$ --- это аналог
          \verb!case c { InL x -> l x; InR x -> r x }!.

          Заметим, что список (например, целых чисел) — это алгебраический тип:

          \verb!List = Nil | Cons Integer List!.

          Можно сконструировать значение данного типа: \verb!Cons 3 (Cons 5 Nil)!.
          Можно, например, вычислить его длину:
          \begin{verbatim}
length Nil = 0
length (Cons _ tail) = length tail + 1
\end{verbatim}

          Определим $Nil = InL\ 0$, а $Cons\ a\ b = InR\ (MkPair\ a\ b)$. Заметим, что теперь списки могут быть впрямую
          перенесены в лямбда выражения.

          Определите следующие функции в лямбда-исчислении для списков:
          \begin{enumerate}
              \item вычисление длины списка;
                    \[Y\ \lambda f.\lambda l.\mathrm{Case}\ (\lambda x.0)\ (\lambda p.( + 1)\ (f\ (\mathrm{PrR}\ p)))\ l\]
              \item построение списка длины $n$ из элементов $0, 1, 2, \dots, n-1$;
                    \[\lambda n.(Y\ \lambda f.(\lambda n'.\lambda m.(\mathrm{Eq}\ n'\ m)\ \mathrm{Nil}\ (\mathrm{Cons}\ m\ (f\ n'\ (\mathrm{inc}\ m)))))\ n\ 0\]
              \item разворот списка: из списка $a_1, a_2, \dots, a_n$ сделать список $a_n, a_{n-1}, \dots, a_1$;
                    \[\mathrm{Add} = \lambda e.Y\ \lambda f.\mathrm{Case}\ (\lambda x.\mathrm{Cons}\ e\ \mathrm{Nil})\ (\lambda p.\mathrm{Cons}\ (\mathrm{PrL}\ p)\ (f\ (\mathrm{PrR}\ p)))\]
                    \[\mathrm{Reverse} = Y\ \lambda f.\mathrm{Case}\ (\lambda x.\mathrm{Nil})\ (\lambda p.\mathrm{Add}\ (\mathrm{PrR}\ p)\ (f\
                        (\mathrm{PrL}\ p)))\]
              \item функцию высшего порядка $map$, которая по функции $f$ и списку $a_1, a_2, \dots, a_n$
                    строит список $f(a_1), f(a_2), \dots, f(a_n)$.
                    \[\lambda g.Y\ \lambda f.\mathrm{Case}\ (\lambda x.\mathrm{Nil})\ (\lambda p.\mathrm{Cons}\ (g\ (\mathrm{PrL}\ p))\ (f\ (\mathrm{PrR}\ p)))\]
          \end{enumerate}

          Решением задачи является полный текст соответствующего лямбда-выражения с объяснениями механизма его работы.
          Используйте интерпретатор лямбда-выражений $lci$ или аналогичный для демонстрации результата.

    \item Чёрчевские нумералы соответствуют натуральным числам в аксиоматике Пеано.
          \begin{enumerate}
              \item Предложите <<двоичные нумералы>> --- способ кодирования чисел, аналогичный двоичной системе
                    (такой, при котором длина записи числа соответствует логарифму числового значения).
                    \[\lambda n.\mathrm{Reverse}\ ((Y\ \lambda f.\lambda r.(\mathrm{IsZero}\ r)\ \mathrm{Nil}\ (\mathrm{Cons}\ ((\mathrm{IsEven}\ r)\ 0\ 1)\ (f\ (\mathrm{DivBy2}\ r))))\ n)\]
              \item Предложите реализацию функции (+1) в данном представлении.
                    \begin{verbatim}
BinaryPlus1 = \n.Reverse(
    (Y \f.\l.\r.Case 
        (\x.
            (IsZero r)
                Nil
                (Cons 1 Nil))
        (\p.
            (IsZero r)
                (Cons (PrL p) (f (PrR p) 0))
                ((IsZero (PrL p))
                    (Cons 1 (f (PrR p) 0))
                    (Cons 0 (f (PrR p) 1))))
        l)
    (Reverse n)
    1);
                    \end{verbatim}
              \item Предложите реализацию лямбда-выражения преобразования числа из двоичного нумерала в чёрчевский.
                    \begin{verbatim}
ToChurch = \b.
(Y \f.\l.\s.Case
    (\x.s)
    (\p.f
        (PrR p)
        (Plus (Plus s s) (PrL p)))
    l)
b 0;
                    \end{verbatim}
          \end{enumerate}

          Аналогично прошлому заданию, решение должно содержать полный код лямбда-выражения вместе с объяснением механизма его работы.

\end{enumerate}

\end{document}
