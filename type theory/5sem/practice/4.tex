\documentclass[12pt, a4paper]{article}

\usepackage{lastpage}
\usepackage{mathtools}
\usepackage{xltxtra}
\usepackage{libertine}
\usepackage{amsmath}
\usepackage{amsthm}
\usepackage{amsfonts}
\usepackage{amssymb}
\usepackage{enumitem}
\usepackage{xcolor}
\usepackage[left=2.3cm, right=2.3cm, top=2.7cm, bottom=2.7cm, bindingoffset=0cm, headheight=15pt]{geometry}
\usepackage{fancyhdr}
\usepackage[russian]{babel}
% \usepackage{parindent}

\pagestyle{fancy}
\lfoot{M3137y2019}
\rhead{\thepage\ из \pageref{LastPage}}

\newcommand{\R}{\mathbb{R}}
\newcommand{\Q}{\mathbb{Q}}
\newcommand{\C}{\mathbb{C}}
\newcommand{\Z}{\mathbb{Z}}
\newcommand{\B}{\mathbb{B}}
\newcommand{\N}{\mathbb{N}}

\DeclareMathOperator*{\xor}{\oplus}
\DeclareMathOperator*{\equ}{\sim}
\DeclareMathOperator{\Ln}{\text{Ln}}
\DeclareMathOperator{\sign}{\text{sign}}
\DeclareMathOperator{\Sym}{\text{Sym}}
\DeclareMathOperator{\Asym}{\text{Asym}}
% \DeclareMathOperator{\sh}{\text{sh}}
% \DeclareMathOperator{\tg}{\text{tg}}
% \DeclareMathOperator{\arctg}{\text{arctg}}
% \DeclareMathOperator{\ch}{\text{ch}}

\DeclarePairedDelimiter{\ceil}{\lceil}{\rceil}

\setmainfont{Linux Libertine}

\theoremstyle{plain}
\newtheorem{theorem}{Теорема}
\newtheorem{axiom}{Аксиома}
\newtheorem{lemma}{Лемма}

\theoremstyle{remark}
\newtheorem*{remark}{Примечание}
\newtheorem*{exercise}{Упражнение}
\newtheorem*{consequence}{Следствие}
\newtheorem*{example}{Пример}
\newtheorem*{observation}{Наблюдение}

\theoremstyle{definition}
\newtheorem*{definition}{Определение}
\newtheorem*{obozn}{Обозначение}

\lhead{Теория типов \textit{(практика)}}
\cfoot{}
\rfoot{5.10.2021}

\begin{document}

\section*{Домашнее задание №4: <<просто-типизированное лямбда исчисление>>}
\begin{enumerate}
      \item Сформулируйте аксиомы для просто типизированного исчисления по Чёрчу.
            \emph{Указание:} аксиомы должны быть согласованы с типами аргументов
            лямбда-абстракций.
            \begin{solution}\itemfix
                  \begin{enumerate}
                        \item \(\begin{prooftree}
                                    \infer0{\Gamma, x : \tau \vdash x : \tau}
                              \end{prooftree}\)
                        \item \(\begin{prooftree}
                                    \hypo{\Gamma \vdash A : \sigma \to \tau}
                                    \hypo{\Gamma \vdash B : \sigma}
                                    \infer2{\Gamma \vdash A\ B : \tau}
                              \end{prooftree}\)
                        \item \(\begin{prooftree}
                                    \hypo{\Gamma, x : \tau \vdash A : \sigma}
                                    \infer1{\Gamma \vdash \lambda x^\tau.A : \tau \to \sigma}
                              \end{prooftree}\)
                  \end{enumerate}
            \end{solution}
      \item Рассмотрим типизацию по Чёрчу. Определим стирающее преобразование $|\cdot|: \Lambda \rightarrow \Lambda_\textrm{ч}$:

            $$|A| = \left\{ \begin{array}{ll} \alpha,   & A = \alpha           \\
             |P| |Q|,       & A = P Q              \\
             \lambda x.|P|, & A = \lambda x^\tau.P\end{array}  \right.$$

            Верно ли следующее: если $P \rightarrow_\beta Q$ и $|P'|=P$, $|Q'|=Q$, то $P'\rightarrow_\beta Q'$.
            \begin{solution}
                  Кажется, преобразование \(| \cdot | : \Lambda_{\mathrm{ч}} \to \Lambda_{\mathrm{к}}\).

                  Нет. Пусть \(Q' = \lambda x^\tau.x, P' = \lambda x^\sigma.(\lambda x^\sigma.x)\ x\). \(|Q'| \equiv \lambda x.x, |P'| \equiv \lambda x.(\lambda x.x)\ x\), тогда \(|P'| \to_\beta \lambda x.x \equiv |Q'|\). Но \(P' \not\to_\beta Q'\), т.к. единственный возможный шаг это \(P' \to_\beta \lambda x^\sigma.x \neq_\alpha \lambda x^\tau.x\)

                  Иначе переберём как было сделано \(P \to_\beta Q\):
                  \begin{enumerate}
                        \item \(P \equiv A\ B, Q = C\ D\) и либо \(A \to_\beta C\) и \(B =_\alpha D\) либо \(A =_\alpha C\) и \(B \to_\beta D\). Тогда индукция даёт \(P' \equiv A'\ B'\)
                  \end{enumerate}
            \end{solution}
      \item Покажите, что если $A =_\alpha B$ и $\Gamma\vdash A:\tau$, то $\Gamma\vdash B:\tau$ (или, иными словами, доказательство не зависит от выбора пред-лямбда-терма).
            \begin{solution}\itemfix
                  \begin{enumerate}
                        \item \(A \equiv x, B \equiv x\). Тогда искомое очевидно.
                        \item \(A \equiv P\ Q, B \equiv R\ S, P =_\alpha R, Q =_\alpha S\) --- индукция:

                              \(\Gamma \vdash P\ Q : \tau\), тогда \(\Gamma \vdash P : \sigma \to \tau, \Gamma \vdash Q : \sigma\). По индукционному предположению будет \(\Gamma \vdash R : \sigma \to \tau, \Gamma \vdash S : \sigma\) и следовательно по второму правилу вывода искомое верно.
                        \item \(A \equiv \lambda x.P, B \equiv \lambda y.Q\) и существует \(t\) --- новая переменная, такая что \(P[x \coloneqq t] =_\alpha Q[y \coloneqq t]\) --- опять индукция:

                              \(\Gamma \vdash \lambda x.P : \tau \to \sigma\), тогда \(\Gamma, x : \tau \vdash P : \sigma\), но тогда и \(\Gamma, t : \tau \vdash P[x \coloneqq t] : \sigma\), т.к. это просто переименование и следовательно \(\Gamma, t : \tau \vdash Q[y \coloneqq t] : \sigma\) по индукционному предположению. Опять же \(\Gamma, y : \tau \vdash Q : \sigma\) и тогда по правилу вывода \(\Gamma \vdash \lambda y.Q : \tau \to \sigma\).
                  \end{enumerate}
            \end{solution}
\end{enumerate}

\end{document}
