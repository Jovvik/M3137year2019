\documentclass[12pt, a4paper]{article}

\usepackage{lastpage}
\usepackage{mathtools}
\usepackage{xltxtra}
\usepackage{libertine}
\usepackage{amsmath}
\usepackage{amsthm}
\usepackage{amsfonts}
\usepackage{amssymb}
\usepackage{enumitem}
\usepackage{xcolor}
\usepackage[left=2.3cm, right=2.3cm, top=2.7cm, bottom=2.7cm, bindingoffset=0cm, headheight=15pt]{geometry}
\usepackage{fancyhdr}
\usepackage[russian]{babel}
% \usepackage{parindent}

\pagestyle{fancy}
\lfoot{M3137y2019}
\rhead{\thepage\ из \pageref{LastPage}}

\newcommand{\R}{\mathbb{R}}
\newcommand{\Q}{\mathbb{Q}}
\newcommand{\C}{\mathbb{C}}
\newcommand{\Z}{\mathbb{Z}}
\newcommand{\B}{\mathbb{B}}
\newcommand{\N}{\mathbb{N}}

\DeclareMathOperator*{\xor}{\oplus}
\DeclareMathOperator*{\equ}{\sim}
\DeclareMathOperator{\Ln}{\text{Ln}}
\DeclareMathOperator{\sign}{\text{sign}}
\DeclareMathOperator{\Sym}{\text{Sym}}
\DeclareMathOperator{\Asym}{\text{Asym}}
% \DeclareMathOperator{\sh}{\text{sh}}
% \DeclareMathOperator{\tg}{\text{tg}}
% \DeclareMathOperator{\arctg}{\text{arctg}}
% \DeclareMathOperator{\ch}{\text{ch}}

\DeclarePairedDelimiter{\ceil}{\lceil}{\rceil}

\setmainfont{Linux Libertine}

\theoremstyle{plain}
\newtheorem{theorem}{Теорема}
\newtheorem{axiom}{Аксиома}
\newtheorem{lemma}{Лемма}

\theoremstyle{remark}
\newtheorem*{remark}{Примечание}
\newtheorem*{exercise}{Упражнение}
\newtheorem*{consequence}{Следствие}
\newtheorem*{example}{Пример}
\newtheorem*{observation}{Наблюдение}

\theoremstyle{definition}
\newtheorem*{definition}{Определение}
\newtheorem*{obozn}{Обозначение}

\lhead{Теория типов \textit{(практика)}}
\cfoot{}
\rfoot{26.10.2021}

\begin{document}

\section*{Домашнее задание №7: <<типовая система Хиндли-Милнера>>}
\begin{enumerate}
    \item \emph{О выразительной силе HM.} Заметим, что список --- это <<параметризованные>> числа в
          аксиоматике Пеано. Число --- это длина списка, а к каждому штриху мы присоединяем какое-то значение.
          Операции добавления и удаления элемента из списка --- это операции прибавления и вычитания
          единицы к числу.

          Рассмотрим тип <<бинарного списка>> (расширение Окамля):

          \begin{minted}{OCaml}
type 'a bin_list = Nil | Zero of (('a*'a) bin_list) |
    One of 'a * (('a*'a) bin_list);;
\end{minted}

          и операцию добавления элемента к списку:

          \begin{minted}{OCaml}
let rec add elem lst = match lst wi th
    Nil -> One (elem,Nil)
  | Zero tl -> One (elem,tl)
  | One (hd,tl) -> Zero (add (elem,hd) tl)
\end{minted}

          \begin{enumerate}
              \item Какой тип имеет \verb!add! (обратите внимание на ключевое слово \verb!rec!: для
                    точного указания соответствующего лямбда-выражения и вывода типа необходимо использовать Y-комбинатор)?
                    Считайте, что семейство типов \verb!bin_list 'a! предопределено и обозначается как $\tau_\alpha$.
                    Выразим ли этот тип в системе Хиндли-Милнера?
              \item Реализуйте предложенный тип и функцию \verb!add! на Хаскеле (используйте опцию \verb!RankNTypes!).
                    Также реализуйте функцию для удаления элемента списка (головы).
              \item Предложите функцию для эффективного соединения двух списков (источник для
                    вдохновения --- сложение двух чисел в столбик).
              \item Предложите функцию для эффективного выделения $n$-го элемента из списка.
          \end{enumerate}
    \item На занятии мы рассмотрели функцию \verb!strange_pair x = (x 1, x "a")!.
          Покажите, что данную функцию невозможно типизировать в типовой системе Хиндли-Милнера.
          Указания: (а) ограничение мономорфизма отношения к делу не имеет;
          (б) ограничение на правило введения квантора всеобщности может оказаться существенным.
    \item Покажем, что алгоритм $W$ действительно находит корректный тип для лямбда-выражения
          (доказательство, что он находит наиболее общий тип, мы оставим в стороне).
          Для этого докажем по индукции, что $W(\Gamma,X)$ действительно находит такие тип $\tau$ и подстановку $S$,
          что $S\Gamma \vdash X:\tau$:
          \begin{enumerate}
              \item покажите базу индукции: $W(\Gamma,x)$;
                    \[\begin{prooftree}
                            \hypo{\Gamma, x : \forall \alpha_1 \dots \alpha_n.\tau : x \vdash \forall \alpha_1 \dots \alpha_n.\tau}
                            \infer1{\Gamma \vdash x : \forall \beta_1 \dots \beta_n.\tau}
                        \end{prooftree}\]
              \item покажите случай аппликации: $W(\Gamma,P\ Q)$;
                    \[\begin{prooftree}
                            \hypo{W(\Gamma, P) = (\tau_p, S_1)}
                            \infer1{S_1\Gamma \vdash P : \tau_p}
                            \infer1{S_{21}\Gamma \vdash P : S_2 \tau_p}
                            \infer1{S_{321}\Gamma \vdash P : S_3(S_2 \tau_p)}
                            \infer1{S_{321}\Gamma \vdash P : S_3(\tau_q \to \gamma)}
                            \infer1{S_{321}\Gamma \vdash P : (S_3 \tau_q) \to S_3 \gamma}
                            \hypo{W(S_1\Gamma, Q) = (\tau_q, S_2)}
                            \infer1{S_{21}\Gamma \vdash Q : \tau_q}
                            \infer1{S_{321}\Gamma \vdash Q : S_3 \tau_q}
                            \infer2{(S_3 \circ S_2 \circ S_1)\Gamma \vdash P\ Q : S_3 \gamma}
                        \end{prooftree}\]
              \item покажите случай лямбда-абстракции: $W(\Gamma,\lambda x.P)$;
                    \[\begin{prooftree}
                            \hypo{W(\Gamma \cup \{x : \tau_x\}, P) =  (\tau_p, S_1)}
                            \infer1{S_1(\Gamma \cup \{x : \tau_x\}) \vdash P : S_1\tau_p}
                            \infer1{S_1\Gamma, x : S_1\tau_x \vdash P : S_1\tau_p}
                            \infer1{S_1\Gamma \vdash \lambda x. P : S_1\tau_x \to S_1\tau_p}
                            \infer1{S_1\Gamma \vdash \lambda x. P : S_1(\tau_x \to \tau_p)}
                        \end{prooftree}\]
              \item покажите случай \verb!let!-выражения: $W(\Gamma,\texttt{let}\ x=P\ \texttt{in}\ Q)$.
                    \[\begin{prooftree}
                            \hypo{W(\Gamma, P) =  (\tau_p, S_1)}
                            \infer1{S_{1}\Gamma \vdash P : S_{1}\tau_p}
                            \infer1{S_{21}\Gamma \vdash P : S_{21}\tau_p}
                            \hypo{W(S_1\Gamma \cup \{x : \forall \{\alpha_i\}.\tau_p\}, Q) =  (\tau_q, S_2), \ \{\alpha_i\} \in FV(\tau_p)}
                            \infer1{S_{2}(S_1\Gamma \cup \{x : \forall \{\alpha_i\}.\tau_p\}) \vdash Q : \tau_q, \ \{\alpha_i\} \in FV(\tau_p)}
                            \infer1{\dots}
                            \infer1{S_{2}(S_1\Gamma \cup \{x : S_{1}\tau_p\}) \vdash Q : \tau_q}
                            \infer1{S_{21}\Gamma, x : S_{21}\tau_p \vdash Q : \tau_q}
                            \infer2{S_{21}\Gamma \vdash let\ x = P\ in\ Q : \tau_q}
                        \end{prooftree}\]
          \end{enumerate}
    \item Покажите, что в Хаскеле выражается $Y: \forall \alpha.(\alpha\rightarrow\alpha)\rightarrow\alpha$ и
          правило исключённого третьего $E: \alpha\vee\neg\alpha$.

          \begin{minted}{haskell}
magic :: a
magic = magic

Y :: (a -> a) -> a
Y = magic

E :: Either a (a -> Void)
E = magic
          \end{minted}
    \item Возможно ли в C++ построить выражения с типами ранга два и выше (включая конструкции с темплейтами)?
          Приведите пример, если да.
\end{enumerate}

\end{document}
