\documentclass[12pt, a4paper]{article}

\usepackage{lastpage}
\usepackage{mathtools}
\usepackage{xltxtra}
\usepackage{libertine}
\usepackage{amsmath}
\usepackage{amsthm}
\usepackage{amsfonts}
\usepackage{amssymb}
\usepackage{enumitem}
\usepackage{xcolor}
\usepackage[left=2.3cm, right=2.3cm, top=2.7cm, bottom=2.7cm, bindingoffset=0cm, headheight=15pt]{geometry}
\usepackage{fancyhdr}
\usepackage[russian]{babel}
% \usepackage{parindent}

\pagestyle{fancy}
\lfoot{M3137y2019}
\rhead{\thepage\ из \pageref{LastPage}}

\newcommand{\R}{\mathbb{R}}
\newcommand{\Q}{\mathbb{Q}}
\newcommand{\C}{\mathbb{C}}
\newcommand{\Z}{\mathbb{Z}}
\newcommand{\B}{\mathbb{B}}
\newcommand{\N}{\mathbb{N}}

\DeclareMathOperator*{\xor}{\oplus}
\DeclareMathOperator*{\equ}{\sim}
\DeclareMathOperator{\Ln}{\text{Ln}}
\DeclareMathOperator{\sign}{\text{sign}}
\DeclareMathOperator{\Sym}{\text{Sym}}
\DeclareMathOperator{\Asym}{\text{Asym}}
% \DeclareMathOperator{\sh}{\text{sh}}
% \DeclareMathOperator{\tg}{\text{tg}}
% \DeclareMathOperator{\arctg}{\text{arctg}}
% \DeclareMathOperator{\ch}{\text{ch}}

\DeclarePairedDelimiter{\ceil}{\lceil}{\rceil}

\setmainfont{Linux Libertine}

\theoremstyle{plain}
\newtheorem{theorem}{Теорема}
\newtheorem{axiom}{Аксиома}
\newtheorem{lemma}{Лемма}

\theoremstyle{remark}
\newtheorem*{remark}{Примечание}
\newtheorem*{exercise}{Упражнение}
\newtheorem*{consequence}{Следствие}
\newtheorem*{example}{Пример}
\newtheorem*{observation}{Наблюдение}

\theoremstyle{definition}
\newtheorem*{definition}{Определение}
\newtheorem*{obozn}{Обозначение}

\lhead{Теория типов \textit{(практика)}}
\cfoot{}
\rfoot{2.11.2021}

\begin{document}

\section{Домашнее задание №8: <<обобщённые типовые системы>>}
\begin{enumerate}
    \item Укажите тип (род) в исчислении конструкций для следующих выражений (при необходимости определите
          типы используемых базовых операций и конструкций самостоятельно) и докажите его:
          \begin{enumerate}
              \item Функция возведения целого числа в квадрат: \verb!sq x = x \star x!
                    \begin{solution}
                        \[\begin{prooftree}
                                \hypo{\vdash \texttt{int} : \star}
                                \hypo{x : \texttt{int} \vdash x \star x : \texttt{int}}
                                \hypo{x : \texttt{int} \vdash \texttt{int} : \star}
                                \infer3{\vdash (\lambda x^{\texttt{int}}.x \star x) : (\Pi x^{\texttt{int}}.\texttt{int})}
                            \end{prooftree}\]

                        \[\begin{prooftree}
                                \hypo{\vdash \texttt{int} : \star}
                                \hypo{x : \texttt{int} \vdash \texttt{int} : \star}
                                \infer2{\vdash (\Pi x^{\texttt{int}}.\texttt{int}) : \star}
                            \end{prooftree}\]
                    \end{solution}
              \item \verb!sizeof!
                    % \begin{solution}
                    %     \[\begin{prooftree}
                    %             \hypo{\vdash \star : \square}
                    %             \hypo{x : \star \vdash F\ x : \star\ \ (\texttt{int})}
                    %             \infer2{ \vdash (\Pi x^\star.F\ x) : \star}
                    %         \end{prooftree}\]
                    % \end{solution}
              \item \verb!std::map!

                    \verb!Compare!, \verb!Allocate! опущены, там то же самое.
                    \begin{solution}
                        \[\begin{prooftree}
                                \hypo{\vdash \star : \square}
                                \hypo{\vdash \star : \square}
                                \hypo{\vdash \star : \square}
                                \hypo{\vdash \star : \square}
                                \infer2[ослабл.]{k : \star, v : \star \vdash \star : \square}
                                \infer2[\(\Pi\)]{k : \star \vdash \Pi v^\star.\star : \square}
                                \infer2[\(\Pi\)]{ \vdash \Pi k^\star.(\Pi v^\star.\star) : \square}
                            \end{prooftree}\]
                    \end{solution}
              \item Монада \verb!ST! из Хаскеля.
                    \begin{solution}
                        Кажется, то же самое, что и \verb!std::map!. Вся магия \verb!ST! в \verb!runST! --- она второго kindа, но нас это не волнует.
                    \end{solution}
              \item Пусть задано выражение рода $\textbf{nonzero}: \star\rightarrow\star$, выбрасывающее нулевой элемент из
                    типа. Например, $\textbf{nonzero}\ \textbf{unsigned}$ --- тип положительных целых чисел.
                    Определите, каков в коде
                    \begin{minted}{cpp}
template<typename T, T x>
struct NonZero { const static std::enable_if_t<x != T(0), T> value = x; };
\end{minted}
                    будет тип (род) поля value.

                    \begin{solution}
                        \(\star \to \mathrm{nonzero}\ \star\)
                        \[\begin{prooftree}
                                \hypo{ \vdash \star : \square}
                                \hypo{x : \star \vdash \mathrm{nonzero}\ x : \star}
                                \infer2{ \vdash (\Pi x^\star.\mathrm{nonzero}\ x) : \star}
                            \end{prooftree}\]
                    \end{solution}
          \end{enumerate}

    \item Приведём следующее странное рассуждение: если мы рассмотрим правый нижний дальний угол лямбда-куба,
          соответствующий $S = \{ \langle\star,\star\rangle, \langle\star,\square\rangle, \langle\square,\star\rangle\}$,
          то можем заметить, что теоретически возможно существование функций, отображающих тип в значение ---
          а потом значение в тип
          (например, по типу вернуть его название в строке, изменить его, а потом по изменённому названию построить другой тип).

          Поясните, почему тем не менее необходимо существование случая $\langle\square,\square\rangle$ в аксиоматике,
          почему всё равно мы не сможем формально построить функции рода $\Pi x^\star.F\ x$ в такой теории.

          \begin{solution}
              По какому правилу мы получим \((\Pi x^\star.F\ x) : \square\)?

              \[\begin{prooftree}
                      \hypo{\Gamma \vdash \star : \star}
                      \hypo{\dots}
                      \infer2[\(\Pi\)-правило]{\Gamma \vdash (\Pi x^\star.F\ x) : \square}
                  \end{prooftree}\]

              Такого не может быть, потому что если \(\Gamma \vdash \star : \star\), то \(\star \to \star : \star\), но \(\star \to \star : \square\).

              \begin{itemize}
                  \item Очевидно по \(\lambda\)-правилу, аксиоме и начальному правилу не может быть.
                  \item Применение откладывает доказательство искомого ``на потом'':

                        \[\begin{prooftree}
                                \hypo{\Gamma \vdash \varphi : (\Pi y^A.B) : s}
                                \hypo{\Gamma \vdash a : A}
                                \infer2[применение]{\Gamma \vdash (\varphi\ a) \equiv \Pi qx^\star.F\ x : (B[y \coloneqq A]) \equiv \square}
                            \end{prooftree}\]
                        \[B \equiv \Pi x^\star.F'\ y\ x\]
                        Нужно найти тип \(B\), что сводится к исходной задаче.

                  \item \(\beta\)-редукция из преобразования не поможет --- не работает на \(S\).
                  \item Для ослабления нужно опять доказать искомое.
              \end{itemize}
          \end{solution}

    \item Предложите выражение на языке C++ (возможно, использующее шаблоны), имеющее следующий род (тип):
          \begin{enumerate}
              \item $\textbf{int}\rightarrow(\star\rightarrow\star)$
                    \begin{minted}{cpp}
#define bruh(n, t) std::array<t, n>
                \end{minted}
              \item $(\star\rightarrow\textbf{int})\rightarrow\star$
                    \begin{minted}[linenos, breaklines]{cpp}
template<template <typename> typename T>
    requires std::is_same_v<
        std::decay_t<decltype(T<std::nullptr_t>::value)>, // no way to check template
        int>
class answer
{};

template<typename T>
struct sizeof_v
{
    static constexpr int value = sizeof(T);
};
              \end{minted}
              \item $\Pi x^\star.n^\textbf{int}.F(n,x)$, где $$F(n,x) = \left\{\begin{array}{ll}\textbf{int},    & n = 0 \\
             x\rightarrow F(n,x), & n > 0\end{array}\right.$$
                    \begin{solution}\itemfix
                        \begin{minted}{cpp}
template<typename x, unsigned n>
struct answer
{
    static constexpr auto get(x const&)
    {
        return answer<x, n-1>::get;
    }
};

template<typename x>
struct answer<x, 0>
{
    static constexpr int get()
    {
        return 0;
    }
};

#include <iostream>

int main()
{
    std::cout << answer<int, 2>::get(0)(1)() << std::endl; // prints 0
}
                 \end{minted}
                    \end{solution}
          \end{enumerate}

    \item Аналогично типу $\Pi$, мы можем ввести тип $\Sigma$, соответствующий квантору существования
          в смысле изоморфизма Карри-Ховарда.
          \begin{enumerate}
              \item Определите правила вывода для $\Sigma$ в обобщённой типовой системе (воспользуйтесь правилами
                    для экзистенциальных типов в системе $F$).
              \item Укажите способ выразить $\Sigma$ через $\Pi$ (также воспользуйтесь идеями для системы $F$).
                    \[\operatorname{pack} \tau, M \text { to } \Sigma \alpha . \sigma=\lambda \beta^{*} . \lambda x^{\Pi \alpha^{*} . \sigma \rightarrow \beta} . x\ \tau\ M: \Pi \beta^{*} .\left(\Pi \alpha^{*} . \sigma \rightarrow \beta\right) \rightarrow \beta\]
          \end{enumerate}

    \item Рассмотрим классы типов в Хаскеле (например, \verb!Num!). Каким образом их можно представить в обобщённой
          типовой системе? Как формализовать запись типа функции \verb!f :: Num a => a -> a!?
\end{enumerate}

\end{document}
